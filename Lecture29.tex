 \stepcounter{lecture}
 \setcounter{lecture}{29}
 \sektion{Lecture 29}

 Homework(12):  For the last problem, you may
 assume that $k$ is algebraically closed of characteristic
 different from 2 and 3.

 Comment on last homework(11): $Z'$ should be noetherian.

 Comment of homework 10: $X\subseteq \P^n$, then there may be line
 sheaves on $X$ which do not come from line sheaves on $\P^n$.
 For example, take the 2-uple embedding $i:\P^1\to \P^2$.  Then
 $i^*\O(1) = \O(2)$, so you can only get the $\O(2n)$'s on $\P^1$.
 Thus, $\pic \P^n \to \pic X$ may not be onto.

 A reference for what we're doing on smoothness:  Bosch,
 Lutkebohnert, and Raynand, Neron Models \S 2.2.

 \underline{Last time:} If $(R,\m)$ is a local noetherian
 $k$-algebra such that $\m^2=0$ and $k'=R/\m$ is finite separable
 over $k$, then
 \[
 0\to \underbrace{\m/\m^2}_{\m} \to \Omega_{R/k}\otimes k' \to
 \underbrace{\Omega_{k'/k}}_0 \to 0
 \]
 is exact.

 \begin{corollary}
 If $(R,\m)$ is a local noetherian
 $k$-algebra such with $k'=R/\m$ is finite separable
 over $k$, then $\m/\m^2 \to \Omega_{R/k}\otimes k'$ is an
 isomorphism (cf. II.8.7)
 \end{corollary}
 \begin{proof}
   Apply the earlier lemma to $R/\m^2$.  Note that $\m^2/\m^4 \to
   \Omega_{R/k}\otimes (R/\m^2) \to \Omega_{(R/\m)/k} \to 0$ is
   exact.  This gives
   \[
    (\m/\m^2)\otimes k' \xrightarrow{0} \Omega_{R/k}\otimes k'
    \xrightarrow{\sim} \underbrace{\Omega_{(R/\m^2)/k}}_{\m/\m^2}\otimes k'
    \to 0
   \]
 \end{proof}

 For today, all schemes are of finite type over $k$.  Recall that
 \'etale means smooth of relative dimension 0.  $f:X\to Y$ is
 unramified if for all $x\in X$, writing $y=f(x)$, we have $k(x)$
 is separable algebraic over $k(y)$ and $\m_y\cdot \O_{X,x}=\m_x$
 ($X$ of finite type implies $k(x)$ is finite separable over
 $k(y)$).

 \underline{Exercise III.10.3} {\it Let $f:X\to Y$ be a morphism,
 then TFAE:
 \begin{itemize}
 \item[(i)] $f$ is \'etale
 \item[(ii)] $f$ is flat and $\Omega_{X/Y}=0$
 \item[(iii)] $f$ is flat and unramified
 \end{itemize}
 }
 \begin{proof}
 (i$\Rightarrow$ ii) flat is obvious.  Also, (3) in smoothness
 condition implies that $\Omega_{X/Y}\otimes k(x)=0$ for all $x$.,
 so by Nakayama, we have $\Omega_{X/Y}=0$.

 (ii$\Rightarrow$ iii) Let $x\in X$ and $y=f(x)$.  By the second
 exact sequence,
 \[
    \m_x/\m_x^2 \to \Omega_{X/Y}\otimes k(x) \to
    \Omega_{k(x)/k(y)}\to 0
 \]
 and $\Omega_{X/Y}=0$ implies that $\Omega_{k(x)/k(y)}=0$ which
 implies that $k(x)$ is separated and algebraic over $k(y)$
 (II.8.6A)

 Assume $\m_y\cdot \O_{X,x}\not=\m_x$ (i.e. $\subsetneqq$).  We have $\O_{X_y,x}
 = \O_{X,x}\otimes k(y) = \O_{X,x}/\m_y\O_{X,x}$.  This is a
 noetherian local $k$-algebra as in the corollary, with maximal
 ideal not zero.  Thus, $\Omega_{\O_{X_y,x}/k(y)}\otimes k(x)
 \not=0$.  Thus, $0 =_{ii} \Omega_{X/Y}\otimes k(x) =
 \Omega_{X_y/k(y)}\otimes k(x)\not=0$. contradiction.

 (iii$\Rightarrow$ i)Check (1),(2'),and (3') in the definition of
 smoothness.  (1) is obvious.  For (2'), (3'), note that
 $\O_{X_y,x}=\O_{X,x}/\m_y\O_{X,x} =_{iii} \O_{X,x}/\m_x = k(x)$.
 (2') holds because $\dim_x X_y = \dim \O_{X_y,x} = \dim k(x_=0$.
 (3') holds because $\Omega_{X_y/k(y)}\otimes k(x) =
 \Omega_{\O_{X_y,x}/k(y)} \otimes k(x) = \Omega_{k(x)/k(y)}\otimes
 k(x)=_{sep\ alg}0$
 \end{proof}

 \underline{Eisenbud 4.4:}{\it Let $R$ be a ring.  If $n$ elements of
 the $R$-module $R^n$ generate it, then they form a basis.}

 \begin{corollary}
 Let $R$ be a ring, and let
 \[
     M' \xrightarrow{\alpha} M\to M''\to 0
 \]
 be and exact sequence of $R$-modules.  If $M'$ and $M''$ can be
 generated by $n$ and $m$ elements, resp, and if $M$ is free of
 rank $n+m$, then $\alpha$ is injective, and the given generating
 sets are bases of $M'$ and $M''$.
 \end{corollary}
 \begin{proof}
 exercise.
 \end{proof}

 \begin{proposition}
 If $f:X\to Y$ is smooth of relative dimension $n$, then $\Omega_{X/Y}$ is locally
 free of rank $n$.
 \end{proposition}
 \begin{proof}
 The question is local on $X$, so we may assume that $X=\spec A$.
 Pick a closed immersion $i:X\hookrightarrow \A^N_Y$.  The second
 exact sequence gives
 \[
    \I/\I^2 \xrightarrow{\delta} i^*\Omega_{\A^N_Y/Y} \to \Omega_{X/Y}\to 0
 \]
 where $\I$ is the sheaf of ideals defining $X$ in $\A^N_Y$.
 Taking stalks, we have
 \[
    (\I/\I^2)_x \xrightarrow{\delta_x} (i^*\Omega_{\A^N_Y/Y})_x \to (\Omega_{X/Y})_x\to 0
 \]
 The middle thing is free of rank $N-n$, and the two other things
 can be generated by $N-n$ and $n$ elements, respectively.  By the
 corollary, $(\Omega_{X/Y})_x$ is free, so $\Omega_{X/Y}$ is
 locally free of rank $n$.
 \end{proof}

 \begin{lemma}
 Let $f:X\to Y$ be a smooth $S$-morphism of smooth $S$-schemes.
 \marginpar{\xymatrix{X \ar[r]\ar[dr] & Y\ar[d]\\ & S}}
 Then the canonical sequence
 \[
    0\to f^*\Omega_{Y/S} \to \Omega_{X/S} \to \Omega_{X/Y}\to 0
 \]
 is exact and locally split.
 \end{lemma}
 \begin{proof}
 For all $x\in X$, the first exact sequence gives an exact
 sequence
 \[
    (f^*\Omega_{Y/S})_x\xrightarrow{\alpha_x} (\Omega_{X/S})_x \to (\Omega_{X/Y})_x
    \to 0
 \]
 in which the terms are locally free and the ranks add up, so by
 the corollary, $\alpha_x$ is injective, and the short exact
 sequence is split.
 \end{proof}

 \begin{proposition}[Another Jacobian Criterion]
 Let $f:X\to Y$ be an $S$-morphism of smooth $S$-schemes.
 Consider the conditions
 \begin{itemize}
 \item[(i)] $f$ is smooth
 \item[(ii)] the canonical map $f^*\Omega_{Y/S} \to \Omega_{X/S}$
 is locally left invertible
 \item[(iii)] for all $x\in X$, the map $(f^*\Omega_{Y/S})\otimes
 k(x) \to \Omega_{X/S}\otimes k(x)$ is injective
 \end{itemize}
 Then (i)$\Rightarrow$(ii)$\Rightarrow$(iii).  If $Y=\A^N_S$, then
 (iii)$\Rightarrow$(i) (actually, this assumption is unnecessary).
 \end{proposition}
 \begin{proof}
 (i$\Rightarrow$ii) follows from the most recent lemma.

 (ii$\Rightarrow$iii) obvious.

 (iii$\Rightarrow$i) Since $Y=\A^N_S$, $f$ is given by coordinates
 $\bar f_1,\dots, \bar f_N\in \Gamma(X,\O_X)$.  Then (iii) is equivalent to
 the $d\bar f_i$ being linearly independent in $\Omega_{X/S}\otimes
 k(x)$ for all $x\in X$.  We may assume that $X$ is affine.
 Embed $i:X\hookrightarrow \A^m_S$, let $\I$ be the sheaf of
 ideals, and let $r$ be the relative dimension of $X$ over $S$.
 Pick $x\in X$.

 Lift $\bar f_1,\dots,\bar f_N$ to functions $f_1,\dots, f_N$ on a
 neighborhood of $x$ in $\A^m_S$.  Also, let $h_1,\dots, h_{m-r}$
 be a generating set for $\I$ near $x$ (by the Jacobian
 Criterion).

 Consider the composite morphism
 \[\xymatrix{
    X\ar@{^(->}[r]^{\Gamma_f}& X\times_S Y \ar@{^(->}[r] &
    \A^m_S\times_S Y = \A^m_Y
 }\]
 Here $Y$ has relative dimension $N$ over $S$ and $X$ has
 codimension $m-r$ in $\P^m_S$, so the image of the composition
 has codimension $N+m-r$.

 Let $t_1,\dots, t_N$ be the coordinate functions on $Y=\A^N_S$.
 Then (locally) $X\subseteq \A^m_Y$ has local defining equations
 $\underbrace{h_1,\dots, h_{m-r}}_{\text{first factor}}$ and
 $\underbrace{t_1-f_1,\dots,t_N-f_N}_{\text{graph morphism}}$.
 \begin{claim}
 These functions satisfy the earlier Jacobian criterion for
 smoothness at the image, $x'$, of $x$ in $\A^m_Y$.
 \end{claim}
 \begin{proof}
 The images of $dh_1,\dots, dh_{m-r}$ form a basis for the kernel
 of $\Omega_{\A^m_S/S}\otimes k(x)\to \Omega_{X/S}\otimes k(x)$,
 which is isomorphic to the kernel of $\Omega_{\A^m_Y/Y}\otimes
 k(x') \to \Omega_{X\times_S Y/Y}\otimes k(x')$.  So we need to
 know whether $d(t_i-f_i)$ are linearly independent in
 $\Omega_{X\times Y/Y}\otimes k(x')$.  The $dt_i$ are zero ($t_i$
 are functions on $Y$), so we are looking at $-df_i$, which are
 linearly independent in $\Omega_{X/S}\otimes k(x)$ by (iii).
 \renewcommand{\qedsymbol}{$\square_{\text{Claim}}$}
 \end{proof}
 \end{proof}

 \begin{corollary}\label{lec29cor}
 Let $X$ be a smooth scheme over $S$, and let $f:X\to Y$ be an
 $S$-morphism, where $Y$ is smooth over $S$.  If $f$ is \'etale,
 then the natural map
 \[
    f^*\Omega_{Y/S}\to \Omega_{X/S}
 \]
 is an isomorphism, and the converse holds if $Y=\A^N_S$ (actually, it
 always holds).
 \end{corollary}
 \begin{proof}
 If $f$ is \'etale, then it is smooth, so
 \[
 0\to f^*\Omega_{Y/S}\to \Omega_{X/S}\to \Omega_{X/Y}\to 0
 \]
 is exact, but $\Omega_{X/Y}=0$, so we get our isomorphism.

 For the converse, use (ii)$\Rightarrow$(i) to get that $f$ is
 smooth, and then $\Omega_{X/Y}=0$ from the first exact sequence
 to get relative dimension 0.
 \end{proof}

 \begin{theorem}[``\'etale over affine'']\label{lec29etale}
 A morphism $f:X\to Y$ is smooth of relative dimension $n$ at a point
 $x\in X$ if an only if there is an open neighborhood $x\in
 U\subseteq X$ and an \'etale morphism $g:U\to \A^n_Y$ such that
 \[\xymatrix{
  U\ar[r]^g \ar[rd]_{f|_U} & \A^n_Y \ar[d]\\
  & Y
 }\]
 commutes
 \end{theorem}
 We are saying that locally, near $X$, it looks like the affine
 fibers.  previous corollary is like implicit function theorem.
