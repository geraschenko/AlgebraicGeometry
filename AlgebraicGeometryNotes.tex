\documentclass[11pt,reqno]{amsart}

\usepackage{latexsym}
\usepackage{amssymb}
\usepackage{graphicx}
\usepackage{multicol}
\usepackage{mathrsfs}
\usepackage[all]{xy}
    \SelectTips{cm}{10}     %use the nicer arrowheads
\usepackage{fancyhdr}      %%%%%%%%%%%% Pagestyle stuff %%%%%%%%%%%%%%%
    \setlength{\hoffset}{0in}
    \addtolength{\textwidth}{0in}
    \setlength{\voffset}{0in}
    \addtolength{\textheight}{.5in}

    \newcommand{\sektion}[1]{\def\sectitle{#1} \section*{#1}}
    \def\sectitle{} %This is the empty section title, before any section title is set

    \pagestyle{fancy}
    \fancyhf{} %delete the current section for header and footer
    \fancyhead[LE,RO]{\thepage}
    \fancyhead[LO]{\sectitle}
    \fancyhead[RE]{Math 256B Algebraic Geometry Lecture Notes}
   % \renewcommand{\headrulewidth}{0in}
%%%%%%%%%%%%%%%%%%%%% End page style stuff %%%%%%%%%%%%%%%%%%%%%%

 \setlength{\hoffset}{0in}
 \addtolength{\textwidth}{0in}
 \setlength{\voffset}{0in}
 \addtolength{\textheight}{0in}

\newcounter{lecture}

\theoremstyle{plain}
%\newtheorem*{theorem}{Theorem}
\newtheorem{theorem}{Theorem}[lecture]
\newtheorem*{claim}{Claim}
\newtheorem*{lemma*}{Lemma}
\newtheorem{lemma}[theorem]{Lemma}
%\newtheorem*{corollary}{Corollary}
\newtheorem{corollary}[theorem]{Corollary}
%\newtheorem*{proposition}{Proposition}
\newtheorem{proposition}[theorem]{Proposition}

\theoremstyle{definition}
\newtheorem*{definition}{Definition}

\theoremstyle{remark}
\newtheorem*{remark}{Remark}

 \def\a{\mathfrak{a}}
 \def\A{\mathbb{A}}
 \def\Ab{\mathcal{A}b}
 \def\ann{\text{Ann}}
 \def\stack#1#2{\genfrac{}{}{0pt}{2}{#1}{#2}}
 \def\b{\mathfrak{b}}
 \def\C{\mathscr{C}}
 \def\cl{\text{\rm Cl}}
 \def\coim{\text{\rm coim}}
 \def\coker{\text{\rm coker}}
 \def\codim{\text{\rm codim}\,}
 \def\depth{\text{\rm depth}\,}
 \def\div{\text{\rm Div}}
 \def\E{\mathscr{E}}
 \def\ext{\text{\rm Ext}}
 \def\Ext{\mathscr{E}\!xt}
 \def\F{\mathscr{F}}
 \def\G{\mathscr{G}}
 \def\H{\mathscr{H}}
 \def\hom{\text{\rm Hom}}
 \def\Hom{\mathscr{H}\!om}
 \def\height{\text{\rm ht}\,}
 \def\id{\text{id}}
 \def\im{\text{im}}
 \def\I{\mathscr{I}}
 \def\J{\mathscr{J}}
 \def\K{\mathscr{K}}
 \def\L{\mathscr{L}}
 \def\m{\mathfrak{m}}
 \def\Mod{\mathcal{M}od}
 \def\O{\mathscr{O}}
 \def\P{\mathbb{P}}
 \def\p{\mathfrak{p}}
 \def\proj{\text{\rm Proj\,}}
 \def\pic{\text{\rm Pic\,}}
 \def\q{\mathfrak{q}}
 \def\Q{\mathbb{Q}}
 \def\R{\mathscr{R}}
 \def\Sch{\mathfrak{Sch}}
 \def\spec{\text{\rm Spec\,}}
 \def\supp{\text{\rm Supp}}
 \def\tor{\text{\rm Tor}}
 \def\U{\mathcal{U}}
 \def\Z{\mathbb{Z}}


\begin{document}
 \title{Math 256B - Algebraic Geometry \\ Lecture Notes}
 \author{Anton}
 \thanks{
   The first part of this document is based on Tony's notes from
   Professor Vojta's lectures. Two (I think) of the lectures were taken from Dave Brown.
   Send comments and corrections to \texttt{geraschenko@gmail.com}
 }
 \maketitle

 { \stepcounter{lecture}
 \setcounter{lecture}{1}
 \sektion{Lecture 1}

\def\A{\mathcal{A}}

\marginpar{Why Cohomology?}

Why study cohomology?  Well, it is useful for determining
$\F(X)=\Gamma(X,\F)$ where $X$ is a scheme and $\F$ is a sheaf of
abelian groups on $X$.

\marginpar{Key Tool}

Our key tool (so far) is Exercise II.1.8: If $0\to \F' \to \F \to
\F''\to 0$ is an exact sequence of sheaves (of abelian groups),
then
\[
    0\to \Gamma(X,\F') \to \Gamma(X,\F) \to \Gamma(X,\F'')
\]
is exact.  It is not always exact on the right.  For example,
Exercise II.1.21c: let $k$ be a field, $X=\P^1_k$,
$P=[1,0],Q=[0,1],Y=\{P,Q\}$ with reduced induced subscheme
structure, and let $\I_Y$ be the sheaf of ideals of $Y$.  Then
\[
    0\to \I_Y \to \O_X\to \underbrace{\O_X/\I_Y}_{i_*\O_X, i:Y\hookrightarrow X} \to 0
\]
is exact, but applying the $\Gamma(X,-)$ functor, we have
\[
    0\to \underbrace{\Gamma(X,\I_Y)}_0 \to
         \underbrace{\Gamma(X,\O_X)}_k \to
         \underbrace{\Gamma(X,\O_X/\I_Y)}_{k^2}
\]
where the right arrow cannot be surjective.

\marginpar{Application 1}

Given an exact sequence
\[
    0\to \Gamma(X,\F') \to \Gamma(X,\F) \to \Gamma(X,\F'')
\]
we get a long exact sequence in cohomology
\begin{align*}
0 &\to\ \Gamma(X,\F')\ \to\ \Gamma(X,\F) \ \to\ \Gamma(X,\F'')\ \to \\
  &\to H^1(X,\F') \to H^1(X,\F) \to H^1(X,\F'') \to \\
  &\to H^2(X,\F') \to \cdots \\
\end{align*}
Sometimes you can show $H^1(X,\F')=0$.  Notation: $H^0(X,\F) =
\Gamma(X,F)$.

\marginpar{Application 2}

If $X$ is a scheme over a field $k$, then $H^i(X,\F)$ is a
$k$-vector space (where $\F$ is a sheaf of $\O_X$-modules). Define
\[
    h^i(X,\F) = \dim_k H^i(X,\F).
\]
Then the Riemann-Roch theorem gives a formula for
$\displaystyle\sum_{i=0}^{\infty}(-1)^i h^i(X,\F)$

\marginpar{How should it look?}
    \begin{list}{}{}
    \item 1) We want it to have a long exact sequence (LES).
    \item 2) The LES should be functorial in the short exact
    sequence (SES).  That is, given a morphism of short exact
    sequences, i.e. a commutative diagram
    \begin{equation}\label{D:*}\xymatrix{
        0\ar[r] & \F' \ar[r] \ar[d] & \F \ar[r] \ar[d] & \F''
        \ar[r] \ar[d] & 0\\
        0\ar[r] & \G' \ar[r]& \G \ar[r] & \G''
        \ar[r] & 0\\
    }\end{equation}
    there should be an induced morphism of long exact sequences, i.e. a
    commutative diagram
    \begin{equation}\label{D:**}\xymatrix@C=4mm{
        0\ar[r] & H^0(X,\F') \ar[r] \ar[d] & H^0(X,\F) \ar[r] \ar[d] & H^0(X,\F'')
        \ar[r] \ar[d] & H^1(X,\F')\ar[r]\ar[d] & \cdots \\
        0\ar[r] & H^0(X,\G') \ar[r] & H^0(X,\G) \ar[r]& H^0(X,\G'')
        \ar[r] & H^1(X,\G')\ar[r] & \cdots \\
    }\end{equation}
    in a functorial way.
    \end{list}
So lets require that $\F\mapsto H^i(X,\F)$ be a functor
$\Ab(X)\xrightarrow{H^i}\Ab$, where $\Ab(X)$ is the category of
sheaves of abelian groups on $X$ and $\Ab$ is the category of
abelian groups.

Then we get
    \begin{list}{}{}
    \item[-] $2/3$ of the maps in the LES and all the vertical maps
    in (\ref{D:**})
    \item[-] commutativity of $2/3$ of the squares in (\ref{D:**})
    \item[-] SES$\mapsto$ LES is functorial
    \end{list}
So we need
    \begin{list}{}{}
    \item[-]to find functors $\F\mapsto H^i(X,\F)$ for each $i\in
    \mathbb{N}$
    \item[-] for all short exact sequences, $0\to \F' \to \F \to \F''\to
    0$, and for all $i\in \mathbb{N}$, maps
    \[
        \delta^i:H^i(X,\F'')\to H^{i+1}(X,\F')
    \]
    such that the LES is exact and so that for all diagrams
    (\ref{D:*}), the diagram commutes:
    \[\xymatrix{
    H^i(X,\F'')\ar[r]^{\delta^i}\ar[d] & H^{i+1}(X,\F')\ar[d] \\
    H^i(X,\G'')\ar[r]^{\delta^i} & H^{i+1}(X,\G') \\
    }\]

    \item - an isomorphism of functors $H^0(X,\F)\cong \Gamma(X,\F)$
    \end{list}

\marginpar{Some Definitions}

\begin{definition}
 An \emph{abelian category} is a category $\A$ together with
\begin{list}{}{}
 \item[(i)] a structure of an abelian group on $Hom(A,B$ for all
objects $A,B\in \A$.
 \item[(ii)] fore every morphism $A\to B$, a kernel $A'\to A$ and a
 cokernel $B\to B'$.
\end{list}
such that
\begin{list}{}{}
 \item[(1)] $Hom(A,B)\times Hom(B,C) \to Hom(A,C)$ is a bilinear
map.
 \item[(2)] Finite sums and products exist.
 \item[(3)] Every monomorphism\footnote{$A\to B$ monomorphism if
    for all $C$, $Hom(C,A)\to Hom(C,B)$ is injective.} is the kernel of its
    cokernel.
 \item[(4)] Every epimorphism\footnote{$A\to B$ epimorphism if for all $C$, $Hom(B,C)\to
    Hom(A,C)$ is injective.} is the cokernel of its kernel.
 \item[(5)] Every morphism can be factored into an epimorphism
 followed by a monomorphism.
\end{list}
\end{definition}

Some examples:
\begin{list}{}{}
 \item $\Ab$
 \item $\Ab(X)$, where $X$ is a topological space
 \item $\Mod(X)$, the category of $\O_X$-modules where $X$ is a
 scheme
\end{list}

\begin{remark}\begin{list}{}{}
 \item[-] All kernels are monomorphisms and all cokernels are epimorphisms.
 \item[-] If $\A$ is an abelian category, the so is
 $\A^{\text{op}}$.
 \item[-] The empty product (resp. coproduct) is an initial
 (final) object; call it 0 (0').  There is a (doubly unique)
 morphism $0\to 0'$, which is an isomorphism (exercise).
\end{list}\end{remark}

\begin{definition}
A covariant functor $F:\A \to \mathcal{B}$ of abelian categories
is \emph{additive} if for all $A,A'\in \A$, the induced map
$Hom(A,A')\to Hom(FA,FA')$ is a homomorphism of abelian groups.
Similarly for contravariant functors.
\end{definition}
We will want $H^i(X,-)$ to be additive.

\begin{definition}
A sequence $A\xrightarrow{f}B\xrightarrow{g} C$ is exact at $B$ if
$g\circ f=0$ and the homology of the sequence (at $B$) is 0.
\end{definition}

\begin{definition}
A \emph{complex} $A^{\cdot}$ in an abelian category $\A$ is a set
of objects $A^i$ and morphisms $d^i:A^i\to A^{i+1}$  for all $i\in
\mathbb{Z}$ such that $d^{i+1}\circ d^i = 0$.  By convention, if
$A^i$ is only given for $i\in I\subset \mathbb{Z}$, we assume
$A^i=0$ for $i\in \mathbb{Z}\smallsetminus I$.
\end{definition}

\begin{theorem}[Freyd's Embedding Theorem]
Every abelian can be embedded as a full subcategory of $\Ab$
\end{theorem}

\begin{definition}
An object in an abelian category $I\in \A$ is \emph{injective} if
the functor $Hom(-,I)$ is exact\footnote{The contravariant functor
$Hom(-,I)$ is always left exact.}.
\end{definition}

\begin{definition}
A \emph{resolution} of an object $A$ is a complex $I^{\cdot}$ and
a morphism $A\to I^0$ such that
\[
    0\to A\to I^0\to I^1\to \cdots
\]
is exact.  An \emph{injective resolution} is a resolution where
each $I^i$ is injective.
\end{definition}
}
 { \stepcounter{lecture}
 \setcounter{lecture}{2}
 \sektion{Lecture 2}

\def\A{\mathcal{A}}

\begin{definition}
\marginpar{$\xymatrix@C=4mm @R=4mm{
 \ker f \ar[d] & \im f \ar[d] \\
 A \ar[r]^f \ar[d] & B\ar[d]\\
 \coim f & \coker f}
$} Given a morphism $f:A\to B$, the \emph{image} of $f$ is the
kernel of its cokernel, and the \emph{coimage} of $f$ is the
cokernel of its kernel.
\end{definition}

\begin{remark}\begin{list}{}{}
\item[(1)] If $f_i:A_i\to B$ for $i=1,2$ are monomorphisms, then
there is at most one $g:A_1\to A_2$ such that $f_2\circ g=f_1$
($\Hom(A_1,A_2)\hookrightarrow \Hom(A_1,B)$ since $f_2$ mono).
Thus, there is a partial ordering on monomorphisms to $B$.
 \item[(2)] If $A\xrightarrow{f}B\xrightarrow{g}C$ with $g\circ
 f=0$, then $\im f \subseteq \ker g$.  We define the
 \emph{homology} at $B$ to be $\ker g/\im f = \ker (\coker f \to \coim g)$.
\end{list}\end{remark}

\begin{definition}
An abelian category has \emph{enough injectives} if every object
is a subobject of an injective object.  That is, there is a
monomorphism from any object to an injective object.
\end{definition}

\begin{lemma}
 In an abelian category with enough injectives, every object has
 an injective resolution.
\end{lemma}
\begin{proof}
Given $A\in \A$, there is a monomorphism to some injective object
$I^0$.  Assume inductively that we have $0\to A \to I^0\to \cdots
\to I^{n-1}\xrightarrow{d^{n-1}} I^n$ exact.  Then there must be a
monomorphism $\coker d^{n-1}\to I^{n+1}$ with $I^{n+1}$ injective.
Then the sequence $0\to A \to I^0\to \cdots \to I^n\to I^{n+1}$
\end{proof}

\begin{remark}
\begin{itemize}
 \item[]
 \item[(1)] We haven't really used injectivity yet.
 \item[(2)] In the following diagram, exactness of the horizontal
 sequence is equivalent to exactness of the diagonal sequences,
 where $K^0=\coker (\epsilon)$ and $K^{i+1}=\coker (d^i)$
 \[\xymatrix @C=-1mm @R=3mm {
 & 0\ar[dr]&   &   &   & 0\ar[dr] &   & 0 &   &   \\
 &   & A\ar[dr] &   &   &   &K^1\ar[dr]\ar[ur]&   &   &   \\
0\longrightarrow A\ar[rrr]_(.65){\epsilon} &  &   &I^0\ar[dr]\ar[rr]^{d^0}&   &I^1\ar[ur]\ar[rr]_{d^1}&   &I^2\ar[dr]\ar[rr]^{d^2}&   &\cdots   \\
 &   &   &   &K^0\ar[dr]\ar[ur]&   &   &   &K^2\ar[dr]\ar[ur]&   \\
 &   &   & 0\ar[ur] &   & 0 &   & 0\ar[ur] &   & 0 \\
 }\]
%If we have a left exact functor, $F$, then we have that
%$\ker(FK^i\to FI^{i+1}) = \ker(FI^i\to FI^{i+1})$
\end{itemize}\end{remark}

Let $X$ be a scheme (or a topological space), and supposed we have
a cohomology theory for $\Ab(X)$.

\begin{definition}
A sheaf $\F\in \Ab(X)$ is \emph{acyclic} if $H^i(X,\F)=0$ for all
$i>0$.
\end{definition}

\begin{theorem}
\marginpar{Acyclic objects can be used to compute cohomology.}

Suppose $X$ as above with a cohomology theory.  If a sheaf $\F\in
\Ab(X)$ has an acyclic resolution
\[
    0\to \F\to \I^0 \to \I^1 \to \cdots
\]
then there is a natural isomorphism $H^i(X,\F)\cong
h^i(H^0(X,\I^{\cdot}))$.
\end{theorem}

\begin{proof}
We have that
\[
    0\to \F \to \I^0\to K^0 \to 0
\]
is exact, so we get an exact sequence in cohomology
\[
    0\to H^0(X,\F) \to H^0(X,\I^0)\to H^0(X,K^0)\to H^1(X,\F) \to
    \underbrace{H^1(X,\I^0)}_{0\quad \I^0\text{ acyclic}}
\]
from which we can say that
\begin{align*}
H^0(X,\F) &=\ker(H^0(X,\I^0)\to H^0(X,K^0))\\
    &= \ker(H^0(X,\I^0)\to H^0(X,\I^1)) & \text{($K\hookrightarrow \I^1$ and $H^0$ left exact)}\\
    &= h^0(H^0(X,\I^{\cdot}))
\end{align*}
and
\begin{align}
H^1(X,\F) &=H^0(X,K^0)/\im(H^0(X,\I^0)\to H^0(X,K^0)) \nonumber \\
    &= \frac{\ker(H^0(X,\I^0)\to H^0(X,\I^1))}{\im(H^0(X,\I^0)\to H^0(X,\I^1))} & \text{($H^0(X,-)$ left exact)} \nonumber \\
    &= h^1(H^0(X,\I^{\cdot})) \label{E:hom}
\end{align}
We also have that
\[
    \underbrace{H^i(X,\I^0)}_0 \to H^i(X,K^0) \stackrel{\sim}{\to} H^{i+1}(X,\F)\to
    \underbrace{H^{i+1}(X,\I^0)}_0
\]
for all $i>0$. The exact sequences
\[
    0\to K^i\to \I^{i+1}\to K^{i+1} \to 0
\]
yield the isomorphisms $H^j(X,K^{i+1})\cong H^{j+1}(X,K^i)$ for
all $j>0$ (since $\I^{i+1}$ is acyclic).  Thus,
\[
    H^i(X,\F) \cong H^{i-1}(X,K^0) \cong \cdots \cong
    H^1(X,K^{i-2}).
\]
But $K^{i-2}$ has the acyclic resolution $K^{i-2}\to \I^{i-1}\to
\I^i\to \cdots$, so we may apply formula (\ref{E:hom}) to get
\[
H^1(X,K^{i-2}) \cong h^1(H^0(X,\I^{\cdot+(i-1)})) \cong
h^i(H^0(X,\I^{\cdot})))\cdot
\]
as desired.
\end{proof}

\marginpar{Derived Functors} Let $\A$ and $\mathcal{B}$ be abelian
categories and let $F:\A \to \mathcal{B}$ be a left exact functor.
Given $A\in \A$ with an injective resolution $0\to A\to
I^{\cdot}$, define
\[
    R^i_{I^{\cdot}}F(A) = h^i(F(I^{\cdot}))
\]
for each $i\in \mathbb{N}$.

\begin{lemma}\label{L:lec2lemma1} Let $A$ and $B$ be objects with injective
resolutions $I^{\cdot}$ and $J^{\cdot}$ together with a map
$f:A\to B$
 \begin{equation}
 \xymatrix{
 0\ar[r] & A \ar[r]^{\delta}\ar[d]^f & I^0\ar[r]^{d^0} & I^1 \ar[r]^{d^1}&\cdots \\
 0\ar[r] & B \ar[r]^{\epsilon} & J^0\ar[r]^{e^0} & J^1\ar[r]^{e^1} & \cdots
} \label{E:data*} \end{equation}
 , then there are functions $f^i:I^i\to J^i$ for each $i\in
\mathbb{N}$ such that this diagram commutes:
 \[\xymatrix{
 0\ar[r] & A \ar[r]^{\delta}\ar[d]^f & I^0\ar[r]^{d^0}\ar[d]^{f^0} & I^1 \ar[r]^{d^1}\ar[d]^{f^1}&\cdots \\
 0\ar[r] & B \ar[r]^{\epsilon} & J^0\ar[r]^{e^0} & J^1\ar[r]^{e^1} & \cdots
}\]
\end{lemma}
\begin{proof}
Applying injectivity of $J^0$ to $\epsilon\circ f:A\to J^0$, we
obtain $f^0$.  Now assume inductively that we have $f^0,\dots,
f^n$.  Let $K^0=\coker (\epsilon)$ and $K^i = \coker(d^{i-1})$ for
$i>0$, then by the universal property of cokernels, we have a map
$K^n\to J^{n+1}$.
\[\xymatrix{
 I^{n-1} \ar[r]^{d^{n-1}}\ar[d]^{f^{n-1}} & I^n \ar[r]\ar[d]^{f^n} & K^n \ar[r]\ar@{.>}[d]^{\exists !} & 0\\
 J^{n-1} \ar[r] & J^n \ar[r] & J^{n+1}
}\]
 And $K^n\hookrightarrow I^{n+1}$, so injectivity of $J^{n+1}$
 gives us $f^{n+1}$.
\end{proof}

\begin{lemma}\label{L:lec2lemma2} Any two choices of $f^{\cdot} =
\{f^0,f^1,\dots\}$ for the same data (\ref{E:data*}) are
\emph{homotopic}.  That is, there is a morphism, $k^{\cdot}$, of
chain complexes of degree -1 such that
\[
    e^{n-1}k^n + k^{n+1}d^n = f^n-g^n
\]
where $e^{-1}=\epsilon$.
\end{lemma}
\begin{proof}
Arrow chasing.
\end{proof}

\begin{corollary}
If $f^{\cdot}$ and $g^{\cdot}$ are two maps of complexes as in
Lemma (\ref{L:lec2lemma1}), then the induced maps
$R^iF(f^{\cdot}), R^iF(g^{\cdot}): R_{I^{\cdot}}^iF(A) \to
R_{J^{\cdot}}^iF(B)$ are the same, so we can call them
$R_{I^{\cdot},J^{\cdot}}^iF(f)$.
\end{corollary}

\begin{corollary}
If $B=A$ and $f=Id_A$, then these maps are isomorphisms.
\end{corollary}
\begin{proof}
For two injective resolutions $I^{\cdot}$ and $J^{\cdot}$ of $A$,
Lemma (\ref{L:lec2lemma1}) tells us that the identity induces
$f^{\cdot}$ and $g^{\cdot}$ so that
$I^{\cdot}\xrightarrow{f^{\cdot}} J^{\cdot}
\xrightarrow{g^{\cdot}} I^{\cdot}$. Commutativity of the diagrams
and functoriality tell us that \[\xymatrix{
 0 \ar[r] & A \ar[r]\ar@{=}[d] & I^{\cdot}\ar[d]^{f^{\cdot}}\\
 0 \ar[r] & A \ar[r]\ar@{=}[d] & J^{\cdot}\ar[d]^{g^{\cdot}}\\
 0 \ar[r] & A \ar[r] & I^{\cdot}
 } \qquad Id =
R_{I^{\cdot},I^{\cdot}}^iF(Id_A) =
\underbrace{R_{I^{\cdot},J^{\cdot}}^iF(Id_A)}_{g^{\cdot}} \circ
\underbrace{R_{J^{\cdot},I^{\cdot}}^iF(Id_A)}_{f^{\cdot}}
\]
and likewise with $I^{\cdot}$ and $J^{\cdot}$ interchanged.\
\end{proof}

\begin{corollary}
$R_{I^{\cdot}}^iF(A)$ is independent of $I^{\cdot}$ up to unique
isomorphism.  These isomorphisms are compatible with
$R_{I^{\cdot},J^{\cdot}}^iF(f)$ at both ends.  Thus, we have well
defined functors $R^iF:\A\to \mathcal{B}$ for all $i\in
\mathbb{N}$.
\end{corollary}

\begin{remark}
\begin{itemize}
 \item[]\hspace{-1.3cm} If $I$ is injective, then it is
$F$-acyclic (i.e. $R^iF(I)=0$ for all $i>0$).
\begin{proof}
$0\to I\to I \to 0$ is an injective resolution, so $R^iF(I) =
h^i(0\to \underbrace{F(I)}_{\text{deg 0}}\to 0\to 0\to \cdots) =0$
for $i>0$.
\end{proof}
\end{itemize}
\end{remark}

\begin{theorem}\label{T:lec2thm}
If $F:\A \to \mathcal{B}$ is as above, then $\{R^iF\}_{i>0}$ is a
$\delta$-functor with $R^0F\cong F$.
\end{theorem}
\begin{proof}[Proof(sketch)]
 Let $A\in \A$ and let $0\to A\to I^{\cdot}$ be an injective
 resolution, then $0\to F(A)\hookrightarrow F(I^0)\to F(I^1)$, so $F(A)\cong
 \ker(F(I^0)\to F(I^1)) = h^0(F(I^{\cdot}))=R^0F(A)$.  To get the
 morphisms, chase diagrams until you catch one.

 Given a short exact sequence $0\to A\to B \to C\to 0$ in $\A$
 with injective resolutions $I^{\cdot}$ and $J^{\cdot}$ for $A$
 and $C$, we have
 \[\xymatrix{
 & 0\ar[d] & 0\ar[d] & 0\ar[d]\\
 0\ar[r] &A\ar[r]\ar[d] & B\ar[r]\ar@{.>}[d]\ar@{.>}[dl]|{I^0\text{
 inj}} \ar[dr] & C\ar[r]\ar[d] & 0\\
 0\ar[r]& I^0\ar[r]\ar[d] & I^0\times J^0 \ar[r] & J^0 \ar[r]
 \ar[d] & 0\\
 & I^1 & & J^1
} \]
 Then by the Snake Lemma, we have the exact sequence
 \[
    0\to \coker(A\to I^0)\to \coker(B\to I^0\times J^0) \to
    \coker(C\to J^0)\to 0
 \]
 and $\coker(A\to I^0) \hookrightarrow I^1$ and $\coker(C\to
 J^0)\hookrightarrow J^1$, so repeat.  Then remove the top row and
 apply $F$.  Note that the rows are still (split) short exact, and
 apply the Snake Lemma to get the long exact sequence\footnote{Anton
 doesn't see how to do this.} Verify commutativity of
 \[\xymatrix{
    R^iF(C)\ar[d]\ar[r]^{\delta} & R^{i+1}F{A}\ar[d]\\
    R^iF(C)\ar[r]^{\delta} & R^{i+1}F{A'}\\
 }\]
\end{proof}
The functors $R^iF$ are called the \emph{right derived functors}
of $F$.
}
 { \stepcounter{lecture}
 \setcounter{lecture}{3}
 \sektion{Lecture 3}

\def\A{\mathcal{A}}

For effaceable functors, see the book.

Recap of what we've done:
\begin{list}{}{}
 \item[(1)] Given a covariant left exact functor from an abelian
category with enough injectives to an abelian category, there are
right derived functors.
 \item[(2)] A Cohomology should have some desirable properties.
\end{list}

\begin{definition}
An abelian group $A$ is \emph{divisible} if for all $a\in A$ and
non-zero $n\in \Z$, there is some $a'\in A$ such that $na'=a$.
\end{definition}
\marginpar{Injective\\ $A$-modules} For example, 0,
$\mathbb{Q},\mathbb{Q/Z}, \mathbb{R}, \mathbb{R/Z}, \dots$ are
divisible groups.

\begin{lemma} An abelian group is injective if and only if it is
divisible.
\end{lemma}
\begin{proof}
($\Rightarrow$) Let $A$ be an injective group, and let $a\in A,
0\not=n\in \Z$, then
\[\xymatrix{
    0 \ar[r] & \Z \ar[r]^{\cdot n} \ar[d]^{\phi} & \Z \ar@{.>}[dl]^{\psi} \\
    & A
}\]
 where $\phi(1)=a$.  Then by injectivity, $\psi$ exists, and
 $n\cdot \psi(1)=a$.  Thus, $A$ is divisible.

($\Leftarrow$) Let $M'\subseteq M$ be abelian groups, and let
$\phi:M'\to A$ be a homomorphism, where $A$ is divisible.
\[\xymatrix{
    0 \ar[r] & M' \ar[r] \ar[d]^{\phi} & M  \\
    & A
}\]
 Pick $x\in M\smallsetminus M'$, and let $d$ be a non-negative
 integer generator of the ideal $\{n\in \Z| nx\in M'\}$.  If
 $d=0$, then $\langle M',x \rangle \cong M'\oplus \Z$ and we can
 extend $\phi$ to $\langle M',x \rangle \to A$ by setting
 $\phi(x)=0$.  If $d\not=0$, then pick $a\in A$ such that
 $da=\phi(dx)$.  Then we can extend $\phi$ to $\langle M',x
 \rangle$ by setting $\phi(x)=a$.  By the standard Zorn's Lemma argument, we
 can extend $\phi$ to a map $M\to A$, so $A$ is injective.
\end{proof}

\begin{lemma}
The category $\Ab$ of abelian groups has enough injectives.
\end{lemma}
\begin{proof}
For an abelian group $A$, we define the \emph{dual} $\hat
A=\Hom(A,\mathbb{Q/Z})$.  Then a homomorphism $f:A\to B$ has a
dual $\hat f: \hat B\to \hat A$.  We have a natural map $A\to
\hat{\hat A}= \Hom(\Hom(A,\mathbb{Q/Z},\mathbb{Q/Z}), a\mapsto
[\phi\mapsto \phi(a)]$
\begin{itemize}
\item[\underline{Claim:}] This map is one to one.

\noindent Indeed, if it were not, then there would be a non-zero
$a\in A$ such that $\phi(a)=0$ for all $\phi \in \hat A$.  But
since $\mathbb{Q/Z}$ is injective, we have
\[\xymatrix{
 0\ar[r] & \langle a \rangle \ar[r] \ar[d]_{\phi_0} & A\ar@{.>}[dl]\\
 & \mathbb{Q/Z}
}\] For any $\phi_0:\langle a \rangle \to \mathbb{Q/Z}$ we have an
extension $\phi: A\to \mathbb{Q/Z}$ so it suffices to find
$\phi_0:\langle a \rangle \to \mathbb{Q/Z}$ with
$\phi_0(a)\not=0$.  But either $\langle a \rangle \cong \Z$, in
which case we can send $a\mapsto 1/2 \in \mathbb{Q/Z}$, or
$\langle a \rangle \cong \Z/n\Z$, in which case we can send
$a\mapsto 1/n\in \mathbb{Q/Z}$.  Thus, the claim is true.
\end{itemize}
So $A\hookrightarrow \hat{\hat A}$.  Next, there is a surjection
$\bigoplus_{i\in I} \Z \to \hat A \to 0$ for some index set $I$.
By taking duals, we get a map
 \begin{align*}
 \hat{\hat A} \to \widehat{\bigoplus_{i\in I}\Z} &= \Hom(\bigoplus \Z,
 \mathbb{Z/Q})\\
 &= \prod \Hom(\Z,\mathbb{Q/Z})\\
 &= \prod \mathbb{Q/Z}
 \end{align*}
 which is divisible, and therefore injective.  Also, the map $\hat{\hat A} \to
 \widehat{\bigoplus_{i\in I}\Z}$ is one to one because the dual of
 a surjective map is one to one.  Thus, $A\hookrightarrow
 \hat{\hat A} \hookrightarrow \prod \mathbb{Q/Z}$.
\end{proof}

If $A$ is a ring, $T$ is an abelian group, and $X$ is an
$A$-module, then we can put an $A$-module structure on
$\Hom_{\Z}(X,T)$ by setting $a\cdot \phi = (x\mapsto \phi(a\cdot
x))$ for each $a\in A$

\begin{lemma}
If $A$ is a ring, $X$ is an $A$-module, and $T$ is an abelian
group, then
\begin{align*}
\Hom_{\Z}(X,T) & \stackrel{\sim}{\rightarrow}
\Hom_A(X,\Hom(A,T))\\
\phi &\mapsto \ (x\mapsto (a\mapsto \phi(ax)))
\end{align*}
as abelian groups (also as $A$-modules).
\end{lemma}
\begin{proof}
exercise.
\end{proof}

\begin{lemma}
If $T$ is and injective abelian group, then $\Hom_{\Z}(A,T)$ is an
injective $A$-module.
\end{lemma}
\begin{proof}
Given $A$-modules $0\to X\to Y$, we have \marginpar{Exercise: show
that the diagram commutes}\[\xymatrix{
    \Hom_A(Y,\Hom_{\Z}(A,T)) \ar[r] \ar[d]^{\wr} & \Hom_A(X,\Hom_{\Z}(A,T))
    \ar[r] \ar[d]^{\wr} & 0\\
    \Hom_{\Z}(Y,T)\ar[r] & \Hom_{\Z}(X,T) \ar[r] & 0
}\] with the top row exact.
\end{proof}

\begin{theorem}
The category of $A$-modules has enough injectives.
\end{theorem}
\begin{proof}
Let $M$ be an $A$-module.  Embed it into an injective abelian
group $T$, so $f:M\hookrightarrow T$.  Define $g:M\to
\Hom_{\Z}(A,T)$ by $g(m)=(a\mapsto f(am))$.  If $m\not=0$, then
$g(m)(1)=f(1\cdot m)\not=0$, so $g$ is one to one.
\end{proof}

\begin{theorem}
If $(X,\O_X)$ is a ringed space, then $\Mod(X)$ has enough
injectives.
\end{theorem}
\begin{proof}
Let $\F$ be a sheaf of $\O_X$-modules.  For each $x\in X$, $\F_x$
is an $\O_{X,x}$-module, and can therefore be embedded into an
injective module $I_x$.  Define $\J = \prod_{x\in X}j_*\I_x$,
where $\I_x=I_x$ as a sheaf at $x$ and $j:\{x\}\hookrightarrow X$
is the inclusion, so $j_*\I_x$ is a skyscraper sheaf.

Then we have that
\[
 \Hom(\F,\J) = \prod_{x\in X} \Hom(\F,j_*\I_x) = \prod_{x\in
 X}\Hom_{\O_{X,x}}(\F_x,I_x).
\]
By taking an injection from each factor, we get a map $\F\to \J$
which must be an injection because it is an injection on all
stalks.

Finally, we must show injectivity of $\J$.  Given any $\G_1\to
\G_2$, we have that
\begin{align*}
 \Hom_{\O_{X,x}}(\G_{2,x},I_x) &\twoheadrightarrow
 \Hom_{\O_{X,x}}(\G_{1,x},I_x) \\
 \Hom(\G_{2,x},j_*\I_x) &\twoheadrightarrow
 \Hom(\G_{1,x},j_*\I_x) \\
 \Hom(\G_{2,x},\J) &\twoheadrightarrow
 \Hom(\G_{1,x},\J) & \text{(prod of surj is surj)} \\
\end{align*}
\end{proof}

\begin{corollary}
If $X$ is a topological space, then $\Ab(X)$ has enough
injectives.
\end{corollary}
\begin{proof}
Make $X$ into a ringed space by setting $\O_X$ to the constant
sheaf $\Z$.  Then $\Mod((X,\O_X))=\Ab(X)$.
\end{proof}

\begin{definition}
Let $X$ be a topological space, then the cohomology functors
$H^{\cdot}(X,-)$ are the right derived functors of
$\Gamma(X,-):\Ab(X)\to \Ab$. \marginpar{$H^{\cdot}(X,-)$ defined}
\end{definition}

Note: We use injective filtrations in $\Ab(X)$ rather than
$\Mod(X)$ because sometimes we will want to consider sheaves of
abelian groups which are not $\O_X$-modules.  The following
theorem says that we get the same cohomology as we would have if
we used $\Mod(X)$ injectives.

\begin{theorem}[*] If $(X,\O_X)$ is a ringed space (e.g. a
scheme), then the right derived functors $\Mod(X)\to \Ab$
associated to $\Gamma(X,-)$ coincide with $H^{\cdot}(X,-)$,
restricted to $\Mod(X)$.
\end{theorem}

Recall that if $X$ is a topological space and $\F$ is a sheaf (of
abelian groups) on $X$, then we say $\F$ is flasque if
$\rho_{UV}:\F(U)\to \F(V)$ is surjective for all open sets
$V\subseteq U$.

\begin{lemma}
Let $(X,\O_X)$ be a ringed space.  Then any injective element of
$\Mod(X)$ is flasque. \marginpar{Injective sheaves are flasque}
\end{lemma}
\begin{proof}
Let $V\subseteq U$ be open.  Regard $\O_U$ and $\O_V$ as sheaves
on $X$ by extending by zero (see exercise II.1.19b), so $W\mapsto
\O_U(W)$ if $W\subseteq U$ and $W\mapsto 0$ otherwise.

We have a map $\O_V\to \O_U$ as $\O_X$-modules, which is injective
on stalks, and therefore injective.  Now look at
\[
 \Hom_{\O_X}(\O_U,\F) \to \Hom_{\O_X}(\O_V,\F) \to 0.
\]
The sequence is exact by injectivity of $\F$.  But we have a
natural isomorphism $\Hom_{\O_X}(\O_U,\F)\cong \F(U)$ (since a
morphism is determined by the image of $1\in \O_U(U)$), so
$\F(U)\to \F(V)$ is surjective.
\end{proof}

\begin{proposition}
Let $X$ be a topological space and $\F$ a flasque sheaf on $X$,
the $\F$ is acyclic.
\end{proposition}
\begin{proof}
 Embed $\F$ into an injective sheaf and let $\G$ be the quotient.
 \[
    0\to \F\to \I \to G\to 0.
 \]
Then $\F$ and $\I$ are flasque, so $\G$ must also be flasque
(Exercise in chapter II), and we get
\[
    0\to \Gamma(X,\F)\to \Gamma(X,\I)\to \Gamma(X,\G)\to 0
\]
from the same exercise.  Therefore, we get the long exact sequence
\[\begin{tabular}{lllll}
 0 &$\to \Gamma(X,\F) $&$\to \Gamma(X,\I) $&$\twoheadrightarrow
 \Gamma(X,\G) $&$\xrightarrow{0}$\\
 &$\xrightarrow{0} H^1(X,\F) $&$\to 0 $&$\to H^1(X,\G)$&$ \to$ \\
 &$\xrightarrow{0} H^2(X,\F) $&$\to 0 $&$\to H^2(X,\G)$&$ \to \cdots$ \\
\end{tabular}\]
where the middle column is 0 because $\I$ is injective.  So we
have that $H^1(X,F)=0$ and $H^i(X,\F)=H^{i-1}(X,\G)$.  But since
$\G$ is also flasque, we have that $H^{i-1}(X,\G)=0$ by induction.
\end{proof}

So flasque sheaves can be used to compute cohomology.

\begin{proof}[Proof of (*)]
Let $\F\in \Mod(X)$ and let $0\to \F\to \I^{\cdot}$ be an
injective resolution (in $\Mod(X)$).  Then
$h^i(\Gamma(X,\I^{\cdot}))$ computes the right derived functors of
$\Gamma(X,-):\Mod(X)\to \Ab$.  But the $\I^i$ are flasque and
therefore acyclic, and so $h^i(\Gamma(X,\I^{\cdot})) = H^i(X,\F)$.
\end{proof}

For next time, read page 209.
}
 { \stepcounter{lecture}
 \setcounter{lecture}{4}
 \sektion{Lecture 4}

\def\A{\mathcal{A}}

Last time, we showed that injective objects in $\Mod(X)$ are
acyclic in $H^{\cdot}(X,-)$.  In particular, the right derived
functors of $\Gamma(X,-):\Mod(X)\to \Ab$ coincide with
$H^{\cdot}(X,-)|_{\Mod(X)}$.

\begin{corollary}
Let $(X,\O_X)$ be a ringed space with $B=\Gamma(X,\O_X)$.  Then
for all $\F\in \Mod(X)$ and $i\in \mathbb{N}$, $H^i(X,\F)$ has a
natural $B$-module structure.  Thus, if $X$ is a scheme over
$\spec A$, then the groups $H^i(X,\F)$ are $A$-modules.
\end{corollary}

\begin{remark}
Let $\A$ be an abelian category with enough injectives and
$\mathcal{B}$ and $\mathcal{C}$ abelian categories.  If $F:\A\to
\mathcal{B}$ is left exact and $G:\mathcal{B}\to \mathcal{C}$ is
exact, then the $\delta$-functor $\{R^i(G\circ F)\}$ is isomorphic
to $\{G\circ R^iF\}$.
\end{remark}

\begin{theorem}
Let $X$ be a noetherian topological space of dimension $n$.  Then
$H^i(X,\F)=0$ for all $i>n$ and for all $\F\in \Ab(X)$.
\marginpar{Grothendieck Vanishing Theorem}
\end{theorem}
Note: For the rest of this lecture, $X$ is a noetherian
topological space.

\begin{lemma}
The direct limit of flasque sheaves is flasque.
\end{lemma}
\begin{proof}
Let $\{\F_{\alpha}\}_{\alpha\in A}$ be a directed system of
flasque sheaves and let $V\subseteq U\subseteq X$ be open sets.
Then we have that
\[\xymatrix{
 (\varinjlim \F_{\alpha})(U) \ar[r]\ar[d]_{\wr} & (\varinjlim
 \F_{\alpha})(V) \ar[d]_{\wr}\\
 \varinjlim \F_{\alpha}(U) \ar@{->>}[r] & \varinjlim
 \F_{\alpha}(V)
}\]
 Where the bottom arrow is surjective because the $\F_{\alpha}$
 are flasque and $\varinjlim$ is an exact functor on abelian
 groups, and the vertical arrows are isomorphisms by exercises
 II.1.10 and II.1.11.
\end{proof}

\begin{lemma} \label{L:CohomologyLimit}
Let $\{\F_{\alpha}\}$ be a directed system in $\Ab(X)$.  Then
there are natural isomorphisms for all $i\in \mathbb{N}$
\[
 \varinjlim H^i(X,\F_{\alpha}) \xrightarrow{\sim} H^i(X,\varinjlim
 \F_{\alpha}).
\]
\end{lemma}
\begin{proof}
Let $\mathcal{C}$ be the category of $A$-directed systems in
$\Ab(X)$.  Then $(\F_{\alpha})\in \mathcal{C}$.  For each
$\alpha$, inject $\F_{\alpha}$ into its sheaf of discontinuous
sections $\G^0_{\alpha}$.  Then inject the cokernel of this map
into \emph{its} sheaf of discontinuous sections, $\G^1_{\alpha}$,
etc. This yields a flasque resolution $0\to \F_{\alpha} \to
\G^{\cdot}_{\alpha}$ which is functorial in $\F_{\alpha}$, so we
get an $A$-directed system of complexes.  In particular,
$(\G^i_{\alpha})_{\alpha}\in \mathcal{C}$.

Then we have that \begin{align*}
 \varinjlim_{\alpha} H^i(X,\F_{\alpha}) &\cong \varinjlim_{\alpha}
 h^i(\Gamma(X,\G^{\cdot}_{\alpha})) & \text{(definition)} \\
  &\cong h^i(\varinjlim_{\alpha} \Gamma(X,\G^{\cdot}_{\alpha})) & \text{($\varinjlim$ exact in
  $\Ab$)}\\
  &\cong h^i(\Gamma(X,\underbrace{\varinjlim_{\alpha} \G^{\cdot}_{\alpha}}_{\text{flasque}})) &
  \text{(nontrivial, need $X$ noetherian)}
\end{align*}
If $0\to \varinjlim \F_{\alpha} \to \varinjlim \G^0_{\alpha} \to
\cdots$ is exact, then we are done.  To see that it is exact,
observe that $\varinjlim$ commutes with taking stalks:
\[
    (\varinjlim \F_{\alpha})_P = \varinjlim_{P\in U}
    \varinjlim_{\alpha} \F_{\alpha}(U) = \varinjlim_{\alpha}
    \varinjlim_{P\in U} \F_{\alpha}(U) = \varinjlim
    (\F_{\alpha})_P
\]
\end{proof}

\begin{lemma} \label{L:ClosedImmersion}
Let $j:Y\hookrightarrow X$ be a closed immersion of topological
spaces and let $\F\in \Ab(Y)$.  Then
\[
    H^i(X,j_*\F)\cong H^i(Y,\F)
\]
for all $i$.
\end{lemma}
\begin{proof}
$H^0$ coincides for all $j$ by definition of $j_*$.  If $0\to \F
\to \I^{\cdot}$ is a flasque resolution of $\F$ on $Y$, then $0\to
j_*\F \to j_*\I^{\cdot}$ is exact (look at the stalks), so the
computation is the same
\end{proof}

If $Y\subseteq X$ is a closed subset, $U=X\smallsetminus Y$ and
$\F\in \Ab(X)$, then define
\[\F_Y = j_*(\F|_Y)\]
where $j:Y\hookrightarrow X$ and $\F|_Y = j^{-1}\F$.  We also
define
\[ \F_U = i_!(\F|_U)\]
where $i:U\hookrightarrow X$ is the inclusion.  Then by exercise
II.1.19, we have that
\[
    0 \to \F_U \to \F \to \F_Y \to 0.
\]

\begin{proof}[Proof of Theorem]\def\R{\mathscr{R}}  Use induction on $n=\dim X$.
\begin{itemize}
\item[\underline{Step 1}:] Reduce to the case where $X$ is
irreducible by induction on the number of components.  If the
number of components is 0, then we are done (since vacuously).  If
it is 1, then $X$ is irreducible.  If there is more than one
irreducible component, then let $Y$ be one component and let
$U=X\smallsetminus Y$.  Then
\[
     0 \to \F_U \to \F \to \F_Y \to 0
\]
is exact, so by the long exact sequence in cohomology, we have the
exact sequence
\[
    H^i(X,\F_U)\to H^i(X,\F) \to H^i(X,\F_Y)
\]
for each $i>n$, but $H^i(X,\F_Y) = H^i(Y,\F|_Y)=0$ by the
inductive hypothesis (and Lemma (\ref{L:ClosedImmersion})).  Then
\[
    0\to \underbrace{(\F_U)_{X\smallsetminus \overline{U}}}_0 \to
    \F_U\xrightarrow{\sim} (\F_U)_{\overline{U}}\to 0
\]
Hence $H^i(X,\F_U)\cong H^i(X,(\F_U)_{\overline{U}} =
H^i(\overline{U},(\F_U)_{\overline{U}}) = 0$ by induction.  Thus,
$H^i(X,\F)=0$ for all $i>n$.

\item[\underline{Step 2}:] If $X$ is irreducible of dimension 0,
then the open subsets are $\varnothing$ and $X$, so $\F$ is
flasque and therefore acyclic.  So $H^i(X,\F)=0$ for all $i>0$, as
desired.

\item[\underline{Step 3}:] Assume $X$ is irreducible (with $n=\dim
X>0$).  Let $B$ be a generating set for $\F$ (e.g.
$\coprod_{U\subseteq X} \F(U)$).  Let $A=\{\text{finite subsets of
} B\}$, then $A$ is a directed system.  For $\alpha\in A$, let
$\F_{\alpha}$ be the subsheaf of $\F$ generated by $\alpha$. That
is, $\F_{\alpha}$ is the set of sections that can be obtained from
$\alpha$ by restrictions and gluing.

Since $X$ is noetherian, all open subsets are quasi-compact, so
all gluings are finite. Thus, any section of $\F$ comes from a
finite subset of $B$, so $\F=\varinjlim \F_{\alpha}$.

By Lemma (\ref{L:CohomologyLimit}), it is enough to show the
theorem for the $\F_{\alpha}$.

Finally, we do induction on the number of elements in $\alpha$. If
the number of elements in $\alpha$ is greater than 1, choose a
proper subset $\alpha'$. Then
\[
    0\to \F_{\alpha'} \to \F_{\alpha} \to
    \underbrace{\text{quotient}}_{\text{gen by images of
    $\alpha\smallsetminus \alpha'$}}\to 0.
\]
By the long exact sequence, we have
\[
    H^i(X,\F_{\alpha'})\to H^i(X,\F_{\alpha})\to
    H^i(X,\text{quotient}).
\]
By induction on $|\alpha|$, we have that the side terms are zero,
and so the middle is zero.  Thus, we may assume that $\F$ is
generated by a single element (over some open set $U$).  Then we
have the exact sequence
\[
    0\to \R \to \Z_U\to \F \to 0
\]
which gives us the exact sequence
\[
    H^i(X,\Z_U)\to H^i(X,\F)\to H^{i+1}(X,\R).
\]
So it is enough to show the result for $\Z_U$ and its subsheaves.
\begin{itemize}
\item[\underline{Proof for $\Z_U$}:] Let $Y=X\smallsetminus U$. We
have that
\[
    0\to \Z_U \to \Z \to \Z_Y\to 0,
\]
and $\Z$ is flasque since $X$ is irreducible, so for all $i>n$,
\[
    H^{i-1}(X,\Z_Y)\to H^i(X,\Z_U)\to \underbrace{H^i(X,\Z)}_0.
\]
The leftmost term is $H^{i-1}(X,\Z_Y)\cong H^{i-1}(Y,\Z|_Y)=0$ by
induction on dimension ($\dim Y<n$).  Thus, $H^i(X,\Z_U)=0$ for
all $i>n$.

\item[\underline{Proof for $\R$}:]
%We may assume that for all $x\in U$, $\R_x\not=0$ since
For all $x\in U$, $\R_x$ is a subgroup of $(\Z_U)_x = \Z$.  Let
$d$ be a minimal positive element in $\Z$ such that $d\in \R_x$
for some $x\in U$.  Then $d\in \R(V)$ for some open set $x\in
V\subseteq U$.  Since $\Z$ is a PID, we have that $\R|_V\subseteq
d\cdot \Z_V$.  But we also have that $d\cdot \Z_V\subseteq \R|_V$,
so $\R|_V\cong \Z$ on $V$.  Thus, we have
\[
    0\to \Z_V \to \R \to \text{quotient} \to 0.
\]
We have already shown that $H^i(X,\Z_V)=0$ for $i>n$, and the
quotient is a sheaf on the lower dimensional $X\smallsetminus V$,
so $H^i(X,\text{quotient})=0$ for $i>n$ by induction on dimension.
Therefore, $H^i(X,\R)=0$ for all $i>n$.
\end{itemize}
\end{itemize}
\end{proof}

Example: Exercise III.2.1a.  Let $X=\mathbb{A}^1_k$ for an
infinite field $k$.  Let $P,Q$ be distinct closed points.  Let
$Y=\{P,Q\}$, and let $U=X\smallsetminus Y$.  Then
$H^1(X,\Z_U)\not=0$ (i.e. the bound given by the theorem cannot be
improved).
\begin{proof}
From
\[
    0\to \Z_U\to \Z\to \Z_Y\to 0
\]
we get
\[
    \underbrace{H^0(X,\Z)}_{\Z} \xrightarrow{\text{not onto}}
    \underbrace{H^0(X,\Z_Y)}_{\Z\times\Z} \to H^1(X,\Z_U).
\]
Therefore, $H^1(X,\Z_U)\not=0$.
\end{proof}
Part (b)*: Show for all $n>0$ that if $X=\mathbb{A}^n_k$, $Y=$
union of $n+1$ hyperplanes with empty intersection (in $\P^n_k$)
and $U=X\smallsetminus Y$, then $H^n(X,\Z_U)\not=0$.

In the case $n=2$, we have
\[
 0\to \Z_U\to \underbrace{\Z}_{\text{flasque}}\to \Z_Y\to 0
\]
\marginpar{
 \begin{xy}<1cm,0cm>:
  \ar@{-} (-.7,1.5);(1.5,-.7)
  \ar@{-} (0,1.5);(0,-.7)
  \ar@{-} (-.7,0);(1.5,0)
 \end{xy}
} so
\[
    \underbrace{H^{n-1}(X,\Z)}_0 \to H^{n-1}(X,\Z_Y)
    \xrightarrow{\sim} H^n(X,\Z_U) \to \underbrace{H^n(X,\Z)}_0
\]
so it suffices to show that $H^{n-1}(Y,\Z_Y)\not=0$.  Let $Y'=$
the three points of intersection and let $U'=Y\smallsetminus Y'$.
Then
\[
    0\to \Z_{U'} \to \Z \to \Z_{Y'}\to 0
\]
so
\[
    \underbrace{H^0(Y,\Z)}_{\Z} \to
    \underbrace{H^0(Y,\Z_{Y'})}_{\Z^3} \xrightarrow{\text{not
    onto}} \underbrace{H^1(Y,\Z_{U'})}_{\Z^3} \to H^1(Y,\Z).
\]
To see that $H^1(Y,\Z_{U'})=\Z^3$, observe that $U'=U_1'\cup U_2'
\cup U_3'$, where each $U_i'\cong\mathbb{A}^1_k\smallsetminus $ 2
points.  So $H^1(Y,\Z_{U'}) = \bigoplus_{i=1}^3 H^1(Y,\Z_{U_i'}) =
\Z^3$ by the Mayer-Vietoris sequence (exercise III.2.4).
Therefore, $H^1(Y,\Z)\not=0$, as desired.
}
 { \stepcounter{lecture}
 \setcounter{lecture}{5}
 \sektion{Lecture 5}

Note on Freyd's Theorem:  every abelian category can be embedded
as a full subcategory of $\Ab$.  This does \emph{not} imply that
$\varinjlim$ is always exact (just because it is exact in $\Ab$).
For example, $\varprojlim$ (which is the $\varinjlim$ in
$\Ab^{\text{op}}$) is not exact.

\marginpar{III \S 3}

\begin{theorem}[Goal]\label{T:lec5}
Let $\F$ be a quasi-coherent sheaf on a noetherian affine scheme
$X=\spec A$, then $\F$ is acyclic. \marginpar{Qco sheaves on
noetherian affine schemes are acyclic}
\end{theorem}

Note: $A$ is noetherian (Proposition II.3.2) and $\F= \tilde{M}$
for some $A$-module $M$.

\begin{lemma}[Key Lemma]\label{L:lec5keylemma}
If $I$ is an injective module over a noetherian ring $A$, then the
sheaf $\tilde I$ on $\spec A$ is flasque.
\end{lemma}

\begin{proof}[Proof assuming Key Lemma]
Let $0\to M\to I^{\cdot}$ be an injective resolution of $M$ (in
$\Mod{A}$.  Then $0\to \tilde M\to \tilde I^{\cdot}$ is exact
($\tilde{}$ is exact).  By the Key Lemma, this is an acyclic
resolution, so we have $H^i(X,\F) = h^i(\Gamma(X,I^{\cdot})) =
h^i(I^{\cdot})=0$ for all $i>0$.
\end{proof}

\begin{lemma}\label{L:lec5IdealsEnoughInjectivity}
Let $A$ be a ring and $J$ an $A$-module.  Then $J$ is injective if
and only if for all ideals $\a\subseteq A$, all maps $\phi:\a\to
J$ extend to $A\to J$.
\end{lemma}
\begin{proof}
($\Rightarrow$) Trivial from definition of injectivity:$\xymatrix
@C=4mm @R=4mm{0\ar[r] & \a \ar[d]_(.4){\phi} \ar[r] &
A\ar@{.>}[dl]^(.4){\text{$J$ inj}} \\ & J}$

($\Leftarrow$) Let $M'\subseteq M$ be $A$-modules and let
$\phi:M'\to J$.  By the usual Zorn's Lemma argument, it suffices
to extend $\phi$ to a map $\langle M',x\rangle \to J$ for some
$x\in M\smallsetminus M'$.  We have the obvious surjection
$M'\oplus A \to \langle M',x\rangle$.  Let $\a=\{a\in A|ax\in
M'\}$ be the kernel
\begin{align*}
    0\to &\a \to M'\oplus A \to
    \langle M',x\rangle \to 0\\
    &a\mapsto (ax,-a)
\end{align*}
To extend $\phi$, we need a map $\psi:A\to J$ such that $\phi+
\psi: M'\oplus A \to J$ vanishes on $\a$.  This is exactly what we
get from the assumption.
\end{proof}

\begin{lemma}\label{L:lec5lemma1}
Let $I$ be an injective module over a noetherian ring $A$ and let
$f\in A$.  Then the localization map $\theta:I\to I_f$ is
surjective.
\end{lemma}
\begin{proof}
For all $i\in \mathbb{N}$, let $\b_i=\ann(f^i)$.  Then
$0=\b_0\subseteq \b_1\subseteq \cdots$.  By the noetherian
hypothesis, there is some $r$ such that $\b_r=\b_{r+1}=\cdots$.
Pick $x\in I_f$ and write $x=y/f^n$ with $y\in I$ and $n\in
\mathbb{N}$.  Define $\phi:(f^{n+r})\to I$ by $f^{n+r}\mapsto
f^ry$ [if $af^{n+r}=0$, then $a\in \b_{n+r}=\b_r$, so $af^r=0$ and
so the map is well defined]. \marginpar{$\xymatrix @C=4mm
@R=4mm{0\ar[r] & (f^{n+r}) \ar[r]\ar[d]_{\phi} &
A\ar@{.>}[dl]^{\psi} \\ & I}$}By injectivity, there is some
$\psi:A\to I$ such that $\psi|_{(f^{n+r})}=\phi$. Let $z=\psi(1)$,
then $\theta(z)=f^{-n}y=x$.
\end{proof}

\begin{definition}[Exercise II.5.6] Let $A$ be a ring, let $\a$ be a
principal ideal in $A$, and let $M$ be an $A$-module. Then
$J=\Gamma_{\a}(M) = \{x\in M|\a^nx=0 \text{ for some } n\in
\mathbb{N}\}$.
\end{definition}

\begin{lemma}
Let $X=\spec A$, $I$ be an $A$-module, $\a=(f)\subseteq A$ a
principal ideal, $J=\Gamma_{\a}(I)$, $U\subseteq X$ open, and
$Z=Z(f)=V_f$.  Then
\[
    J=\Gamma_Z(X,\tilde I) = \{s\in \Gamma(X,\tilde I)|s_P=0 \text{ for all } P\not\in Z\}
\]
and
\[
 \Gamma(U,\tilde J) = \Gamma_Z(U,\tilde I).
\]
\end{lemma}
\begin{proof}
\begin{align*}
\Gamma_Z(X,\tilde I) &= \{x\in I: x\mapsto 0 \in I_P \text{ for
all } P\not\in Z\}\\
    &= \{x\in I| x\mapsto 0\in I_f\} \\
    &= \ker(I\to I_f) \\
    &=\{x\in I| f^nx=0 \text{ for some } n\in \mathbb{N}\}\\
    &=\{x\in I| \a^nx=0 \text{ for some } n\in \mathbb{N}\}\\
    &= \Gamma_{\a}(I) = J.
\end{align*}

For the other claim, start with the special case $U=D(g)$, so
$\Gamma(U,\tilde J)=J_g=(\ker(I\to I_f))_g$.  So $x\in
\Gamma(U,\tilde J)$ if and only if there are $n,m\in \mathbb{N}$
such that $x=g^{-n}y$, $y\in I$ and $f^my=0$. On the other hand,
$\Gamma_Z(U,\tilde I)=\ker(I_g\to (I_g)_f)$,  So $x\in
\Gamma_Z(U,\tilde I)$ if and only if there are $n,n',m\in
\mathbb{N}$ such that $x=g^{-n}y$, $y\in I$ and $g^{n'}f^my=0$.
Thus, $\Gamma_Z(U,\tilde I)=\Gamma(U,\tilde J)$ in this case. For
the general case, cover $U$ with open affine sets an glue.
\end{proof}

\begin{lemma}\label{L:lec5lemma2}
 Let $A$ be a noetherian ring, $\a\in A$ an ideal, $I$
 an $A$-module, and $J=\Gamma_{\a}(I)$.  If $I$ is
 injective, then so is $J$.
\end{lemma}
\begin{proof}
By Lemma (\ref{L:lec5IdealsEnoughInjectivity}), it suffices to
complete the diagram
\[\xymatrix{
 0\ar[r] & \b \ar[r] \ar[d]_{\phi} & A \ar@{.>}[dl]\\
 & J
 }\]
for any ideal $\b$ and any $\phi: \b\to J$. For all $b\in \b$,
there is some $n$ such that $\phi(\a^nb) = \a^n\phi(b)=0$.  Since
$\b$ is finitely generated, we may choose one $n$ that works for
all $b$. Thus, $\a^n\b\subseteq \ker \phi$.

Recall \underline{Krull's Theorem}: Let $\a\subseteq A$ be an
ideal in a noetherian ring, and let $M\subseteq N$ be finitely
generated $A$-modules.  Then the $\a$-adic topology on $M$ is
induced by the $\a$-adic topology on $N$.  That is, for all $n$,
there is some $n'$ such that $\a^nM\supseteq M\cap \a^{n'}N$.
Thus, taking $M=\b$ and $N=A$, $\a^n\b \supseteq \b\cap \a^{n'}$.
So we have
\[\xymatrix@C=13mm{
 A\ar@{->>}[r] & A/\a^{n'}\ar@{.>}@/^/[dr] \ar@{.>}@/^/[drr]^{\psi}\\
 \b\ar@{^(->}[u]\ar[r] \ar@/_3ex/[rr]_{\phi} & \b/\b\cap \a^{n'}
 \ar@{^(->}[u] \ar[r]^(.6){\text{Krull}} & J \ar@{^(->}[r] & I
 }\]
By injectivity of $I$, $\psi$ exists such that the diagram
commutes.  If we can show $\im \psi \subseteq J$, then we're done.
This is true because the image is killed by $\a^{n'}$.
\end{proof}

\begin{proof}[Proof of Key Lemma]
Statement:if $X=\spec A$ is noetherian, $I\in \Mod(A)$ injective,
then $\tilde I$ is flasque.

We want to show that for all $U\subseteq V$, $\Gamma(V,\tilde
I)\to \Gamma(U,\tilde I)$ is
surjective.\marginpar{$\xymatrix@C=3mm
@R=3mm{\Gamma(V,\tilde I)\ar[r] & \Gamma(U,\tilde I)\\
\Gamma(X,\tilde I)\ar[u]\ar[ur]}$}  It suffices to check the case
where $V=X$.  Let $Y=\supp \tilde I = \{P\in X| I_P\not=0\}$.  Now
we apply noetherian induction on $Y$.  If $Y\cap U=\varnothing$,
then $\Gamma(U,\tilde I)=0$ and there is nothing to show.  So we
may assume that $Y$ meets $U$.  Then there is some $f\in A$ such
that $D(f)\subseteq U$ and $D(f)\cap Y\not=\varnothing$.  Let
$Z=X\smallsetminus D(f) = Z(f)$ and consider the commutative
diagram
\[\xymatrix{
 I=\Gamma(X,\tilde I) \ar[r] & \Gamma(U,\tilde I) \ar[r] & \Gamma(D(f),\tilde
 I)=I_f & \text{(row \emph{not} exact)}\\
 \Gamma_Z(X,\tilde I)\ar@{^(->}[u] \ar[r] & \Gamma_Z(U,\tilde
 I)\ar@{^(->}[u]\\
 J=\Gamma(X,\tilde J) \ar@{}[u]|{\parallel} \ar[r] & \Gamma(U,\tilde J)\ar@{}[u]|{\parallel}
 }\]
Let $s\in \Gamma(U,\tilde I)$, and let $s'\in \Gamma(D(f),\tilde
I)$ be its image.  By Lemma (\ref{L:lec5lemma1}) \marginpar{Lem
\ref{L:lec5lemma1}: $I\to I_f$ surjective} there is some $t\in
\Gamma(X,\tilde I)$ mapping to $s'$. Let $t'=t|_U\in
\Gamma(U,\tilde I)$, so that $s-t'\in \Gamma_Z(U,\tilde I)$ [$s$
and $t'$ agree on $D(f)=Z^c$].  So it suffices to check that the
bottom map is surjective.

By Lemma (\ref{L:lec5lemma2}), $J$ is injective \marginpar{Lem
\ref{L:lec5lemma2}: $I$ inj $\Rightarrow
J=\Gamma_{\mathfrak{a}}(I)$ inj}.  Also, $\supp J \subseteq Y\cap
Z \subsetneq Y$, so we can apply induction on the dimension of
$Y=\supp I$.
\end{proof}

\begin{corollary}[to the Theorem]\label{C:lec5}
Let $\F$ be a quasi-coherent sheaf on a noetherian scheme $X$.
Then $\F$ can be embedded into a flasque, quasi-coherent sheaf.
\end{corollary}
\begin{proof}
Cover $X$ with finitely many open affine sets $U_i=\spec A_i$,
$i=1\dots n$.  For each $i$, $\F|_{U_i} \cong \tilde M_i$ for some
$A_i$-module $M_i$.  Embed $M_i$ into an injective $A_i$-module
$I_i$, and let
\[
    \G = \bigoplus_{i=1}^n f_{i*}(\tilde I_i)
\]
where $f_i:U_i\to X$ is the inclusion map.  We have $\F|_{U_i}\to
\tilde I_i$, which is injective for each $i$, and we have
\[\F\to
\underbrace{f_{i*}(\F|_{U_i})}_{V\mapsto \F(V\cap U_i)} \to
f_{i*}(\tilde I_i).\] Composing and adding these maps, we get an
injection $\F\to \G$ [check on stalks that $\F\to f_{i*}(\tilde
I_i)$ is injective].  By the Key Lemma (\ref{L:lec5keylemma}),
$\tilde I_i$ is flasque for each $i$, so $f_{i*}(\tilde I_i)$ is
flasque by exercise II.1.16d, so $\G$ is flasque.  Similarly, $\G$
is quasi-coherent.
\end{proof}
}
 { \stepcounter{lecture}
 \setcounter{lecture}{6}
 \sektion{Lecture 6}

Application to Corollary (\ref{C:lec5}): Every such sheaf has a
quasi-coherent flasque resolution which can therefore be used to
compute cohomology.

Also (Exercise III.3.6) $\mathfrak{Qco}(X)$ has enough injectives,
and the resulting derived functor cohomology theory agrees with
the usual one.

\begin{theorem}[Serre] Let $X$ be a noetherian scheme.  Then the
following are equivalent:
\begin{itemize}
 \item[(1)] $X$ is affine
 \item[(2)] All quasi-coherent sheaves on $X$ are acyclic
 \item[(3)] For all (quasi-)coherent sheaves of ideals $\I$ on
 $X$, $H^1(X,\I)=0$.
\end{itemize}
\end{theorem}
\begin{proof}
$(1)\Rightarrow (2)$ is Theorem (\ref{T:lec5}).

$(2)\Rightarrow (3)$ is trivial.

$(3)\Rightarrow (1)$:  Assume (3) and let $A=\Gamma(X,\O_X)$.
\vspace{-5mm}
\begin{itemize}
\item[] \begin{claim} Every closed point $P\in X$ has an open
\underline{affine} neighborhood of the form $X_f$ for some $f\in
A$.
\end{claim}
\begin{proof}[Proof of Claim]
Let $U$ be any open affine neighborhood of $P$, and let
$Y=X\smallsetminus U$.  Consider
\[
    0\to \I_{Y\cup \{P\}} \to \I_Y \to k(P) \to 0
\]
where $k(P)$ is the residue field skyscraper sheaf at $P$, and $Y$
and $Y\cup \{P\}$ are taken as reduced subschemes of $X$.  By the
long exact sequence, we have
\[
    H^0(X,\I_Y)\to \underbrace{H^0(X,k(P))}_{k(P)} \to
    \underbrace{H^1(X,\I_{Y\cup\{P\}})}_{0 \text{ by }(3)}.
\]
Therefore, there is some $f\in \Gamma(X,\I_Y)$ mapping to $1\in
k(P)$.  In particular, $f\not\in \mathfrak{m}_P$, so $P\in X_f$
[$f\in H^0(X,\I_Y)\subseteq H^0(X,\O_X)=A$].  Also, for all $Q\in
Y$, $f\in \I_Y \subseteq \mathfrak{m}_Q$, so $Q\not\in X_f$. Thus,
$X_f\subseteq U$.  Let $B=\Gamma(U,\O_X)$ and $\bar f=f|_U\in B$.
Then $X_f=X_f\cap U=D(\bar f) = \spec B_{\bar f}$ (in $U$).  So
$X_f$ is affine.
\renewcommand{\qedsymbol}{$\square_{\text{Claim}}$}
\end{proof}
\end{itemize}
Since $X$ is noetherian, the sets $X_f$ cover $X$.  Take a finite
subcover, $X_{f_1},\dots,X_{f_r}$.  By the handout (Exercises
II.2.16 and II.2.17), it will suffice to check that the ideal
$(f_1,\dots,f_n)\subseteq A$ is all of $A$.  Define a map of
sheaves $\O_X^r\to \O_X$ by $(a_1,\dots,a_r)\mapsto \sum a_if_i$.
We may assume it is onto, because if $P\in X$, then $P\in X_{f_i}$
for some $i$ and $e_i = (0,\dots, 1,\dots, 0)\mapsto f_i\in \O_X$,
and $f_i\not\in \mathfrak{m}_P$.

Now blah blah \marginpar{Start:Fast forward}
\end{proof}

Application: Exercise III.3.1 (neat filtration trick)

Exercise III.3.8: Lemma (\ref{L:lec5lemma1}) (Hartshorne Lemma
III.3.3) and Lemma (\ref{L:lec5keylemma}) (Hartshorne Proposition
III.3.4) are false without the noetherian hypothesis.

\marginpar{Chapter II, \S 4:\\ Separated and Proper morphisms}

\begin{definition} Let $f:X\to Y$ be a morphism of schemes.  Then
the diagonal map is the morphism $\Delta:X\to X\times_Y X$ defined
by the diagram
\[\xymatrix{
 X\ar@/_4ex/[ddr]_{\text{id}_X}\ar@{.>}[dr]^{\Delta}
 \ar@/^4ex/[drr]_{\text{id}_X} \\
 & X\times_Y X \ar[r] \ar[d] & X\ar[d]^f\\
 & X\ar[r]^f & Y
}\]
 We say that $f$ is \emph{separated} if $\Delta$ is a closed
 immersion.  In this case, we say that $X$ is \emph{separated
 over} $Y$.  A \emph{separated scheme} is a scheme $X$ which is
 separated over $\spec \Z$.
\end{definition}

Notation:
\begin{itemize}
 \item[1)] In EGA, prescheme is the same as scheme in Hartshorne,
and scheme means separated scheme.
 \item[2)] If $f_1:X\to Y_1$ and $f_2:X\to Y_2$ are morphisms in
 $\mathfrak{Sch}(S)$, then $(f_1,f_2)$ is the $S$-morphism defined
 by the diagram
\[\xymatrix{
 X\ar@/_4ex/[ddr]_{f_1}\ar@{.>}[dr]^{(f_1,f_2)}
 \ar@/^4ex/[drr]_(.6){f_2} \\
 & Y_1\times_S Y_2 \ar[r] \ar[d] & Y_2\ar[d]\\
 & Y_1\ar[r] & S
}\] Thus, $\Delta=(\text{id}_X,\text{id}_X)$.

 On the other hand, if $f_i:X_i\to Y_i$ are morphisms for $i=1,2$,
 then $f_1\times f_2 = (f_1\circ pr_1, f_2\circ pr_2)$.
\end{itemize}

Example: The affine line with two origins is not separated over
$k$ because $\Delta\subseteq X\times_k X=\mathbb{A}^2$ with double
axes and quadruple origin contains only two of the origins, but
$\overline{\Delta}$ contains all four.  Thus, $\Delta$ is not a
closed immersion.
}
 { \stepcounter{lecture}
 \setcounter{lecture}{7}
 \sektion{Lecture 7}

Recall that $f:X\to Y$ is separated if $\Delta:X\to X\times_Y X$ is a closed
immersion.

\begin{proposition}
If $f:X\to Y$ is a morphism of affine schemes, then it is
separated.
\end{proposition}
\begin{proof}
Let $X=\spec A$ and $Y=\spec B$.  Then $X\times_Y X = \spec
(A\otimes_B A)$ and $\Delta$ corresponds to the morphism
$A\otimes_B A\to A$, $a\otimes a'\mapsto aa'$, which is onto.
Thus, $\Delta$ is a closed immersion.
\end{proof}

\begin{corollary}
All affine schemes are separated.
\end{corollary}
\begin{corollary}
A morphism $f:X\to Y$ is separated if and only if $\Delta(X)$ is a
closed subset of $X\times_Y X$
\end{corollary}
\begin{proof}
($\Rightarrow$) obvious.

($\Leftarrow$) We need to show that
\begin{itemize}
 \item[(1)] $\Delta$ is a homeomorphism onto $\Delta(X)$
 \item[(2)] $\O_{X\times_Y X} \to \Delta_* \O_X$ is onto
\end{itemize}

(1) We have
\[\xymatrix{
 X\ar@/_4ex/[ddr]_{\text{id}_X}\ar@{.>}[dr]^{\Delta}
 \ar@/^4ex/[drr]_{\text{id}_X} \\
 & X\times_Y X \ar[r]^(.6){\pi_2} \ar[d]^{\pi_1} & X\ar[d]^f\\
 & X\ar[r]^f & Y
}\]
 Since $\pi_1\circ \Delta:X\to \Delta(X)\to X$ is the identity,
 $\Delta$ must be one to one, and $\pi_1$ is a continuous inverse,
 so $\Delta$ is a homeomorphism $X\to \Delta(X)$.

(2) It is enough to look at stalks.  Let $P=\Delta(P')\in
\Delta(X)$ for some $P'\in X$.  Restrict to an open affine
neighborhood $V$ of $f(P')$ in $Y$, and an open affine
neighborhood $U$ of $P'$ in $f^{-1}(V)$.  Then we've reduced to
the affine case, where we know the result holds.
\end{proof}

\begin{theorem}[Valuative Criterion of Separatedness]
\marginpar{Valuative Criterion for Separatedness}
A morphism $f:X\to Y$ of noetherian schemes is separated if and
only if the following criterion holds: for any field $K$ and any
valuation ring $R$ of $K$ (Frac($R)=K$), and for any diagram
\[\xymatrix{
 U:= \spec K \ar[r] \ar[d]^i & X\ar[d]^f\\
 T:= \spec R \ar[r] \ar@{.>}[ur]^{\exists \le 1} & Y
}\] there exists at most one $h:T\to X$ such that the diagram
commutes.
\end{theorem}

\begin{remark}
\begin{itemize}
 \item[(1)] You really only need $X$ noetherian.
 \item[(2)] Criterion fails for the affine line with the doubled
 origin.
 \item[(3)] Have to use valuation rings rather than curves.
\end{itemize}
\end{remark}

\begin{lemma}
Let $R,K,U,T$ be as above, and let $X$ be a scheme, then
\begin{itemize}
 \item[(a)] To give a map $U\to X$ is equivalent to giving a point
 $x\in X$ and a field extension $k(x)\hookrightarrow K$, where
 $k(x)=\O_x/\mathfrak{m}_x$.
 \item[(b)] Giving a map $T=\spec R \to X$ is equivalent to giving
 points $x_0,x_1\in X$ with $x_1\rightsquigarrow x_0$
 and\footnote{$x_1\rightsquigarrow x_0$ means that $x_0$ is a
 specialization of $x_1$.  i.e. $x_0$ is in the closure of
 $\{x_1\}$} an inclusion $k(x_1)\hookrightarrow K$ such that if
 you let $z=\overline{\{x_1\}}\subseteq X$ (with reduced induced
 subscheme structure), then $R$ dominates the local ring
 $\O_{X,z}$
\end{itemize}
\end{lemma}
\begin{proof}
in the works.
\end{proof}

\begin{lemma}
Let $f:X\to Y$ be a quasi-compact morphism of schemes, and let
$Z=f(X)\subseteq Y$.  Then $Z$ is closed if and only if it is
stable under specialization.
\end{lemma}
\begin{proof}
in the works.
\end{proof}

\begin{remark}
Since $X$ is noetherian, $\Delta: X\to X\times_Y X$ is
quasi-compact.
\end{remark}

\begin{proof}[Proof of Valuative Criterion]
in the works

\renewcommand{\qedsymbol}{\text{\tiny proof continued in next lecture}}
\end{proof}
}
 { \stepcounter{lecture}
 \setcounter{lecture}{8}
 \sektion{Lecture 8}

\begin{proof}[Valuative Criterion proof continued]
still in the works.
\end{proof}

\begin{corollary}
When working with noetherian schemes: \marginpar{Separatedness on
noetherian schemes}
\begin{itemize}
 \item[(a)] Open and closed immersions are separated.
 \item[(b)] Compositions of separated morphisms are separated.
 \item[(c)] Separatedness is stable under base extension. i.e.  if
 $f:X\to Y$ is separated, then $f':X\times_Y Y' \to Y'$ is
 separated.
 \[\xymatrix{
 X\times_Y Y' \ar[r] \ar[d]^{f'} & X\ar[d]^f\\
 Y' \ar[r] & Y
}\]
 \item[(d)] If $f:X\to Y$ and $f':X'\to Y'$ are separated
 $S$-morphisms, then so is $f\times f':X\times_S X' \to Y\times_S
 Y$.
 \item[(e)] If $X\xrightarrow{f} Y \xrightarrow{g} Z$ are
 morphisms and $g\circ f$ is separated, then so is $f$.
 \item[(f)] Separatedness is local on the base.  i.e. $f:X\to Y$
 is separated if and only if there is an open cover $\{U_i\}$ of
 $Y$ such that $f^{-1}(U_i)\to U_i$ is separated for all $i$.
 \item[(g)] $f:X\to Y$ is separated if and only if
 $f_{\text{red}}:X_{\text{red}}\to Y_{\text{red}}$ is separated.
\end{itemize}
\end{corollary}
\begin{proof}[Proof of (e)]
in the works.
\end{proof}
\begin{proof}[Proof of (g)]
in the works.
\end{proof}

\begin{remark}
All of these things are true without the noetherian hypothesis.
\end{remark}

\begin{corollary}[of the Corollary]
A morphism of separated schemes is separated.  More precisely, if
$f:X\to Y$ with $X$ separated (over $\Z$), then $f$ is separated.
\end{corollary}
\begin{proof}
Use (e) of the Corollary above.
\end{proof}

\begin{corollary}[of the Corollary]
Most schemes you work with will be separated.  An exception:
\underline{gluing}.
\end{corollary}

\marginpar{Properness} In topology, a map is proper if the inverse
image of a compact set is compact.  We want $\P^1_k\to \spec k$ to
be proper, but $\mathbb{A}^1_k\to \spec k$ not to be proper.

\begin{definition}
A morphism $f:X\to Y$ is \emph{closed} if for all closed
subschemes $Z\subseteq X$, $f(Z)$ is closed in $Y$.  We say that
$f$ is \emph{universally closed} if for all morphisms $Y'\to Y$,
the pullback of $f$ is closed.
\[\xymatrix{
 X'\ar[d]^{f'} \ar[r] & X\ar[d]^f\\
 Y'\ar[r] & Y
}\]
\end{definition}

Example: $\mathbb{A}^1_k\to \spec k$ is closed, but not
universally closed.  Take $Y'= \mathbb{A}^1_k$, so $X' =
\mathbb{A}^2_k$, and $f'$ is a projection.  Then look at
$Z=\{xy=1\}$.

\begin{definition}
A morphism is \emph{proper} if it is separated, of finite type,
and universally closed.
\end{definition}

\begin{theorem}[Valuative Criterion for Properness]
\marginpar{Valuative Criterion for Properness} Let $f:X\to Y$ be a
finite type morphism of noetherian schemes, then $f$ is proper if
and only if
\[\xymatrix{
 U:= \spec K \ar[r] \ar[d]^i & X\ar[d]^f\\
 T:= \spec R \ar[r] \ar@{.>}[ur]^{\exists !} & Y
}\]
\end{theorem}
\begin{proof}
in the works.
\end{proof}
}
 { \stepcounter{lecture}
 \setcounter{lecture}{9}
 \sektion{Lecture 9}

A question from last time: If $f:\spec K\to X\times_Y X$ sends the
point to $\xi \in \Delta$, then $f$ factors though $\Delta$ as a
morphism of schemes.
\[\xymatrix{
 X\ar@/_4ex/[ddr]_{\text{id}_X}\ar@{->}[dr]^{\Delta}
 \ar@/^4ex/[drr]_{\text{id}_X} \\
 & X\times_Y X \ar[r] \ar[d] & X\ar[d]^f\\
 & X\ar[r]^f & Y
}\hspace{2.4cm}\xymatrix{
 k(x_1)\ar@{<-}@/_4ex/[ddr]_{\text{id}}\ar@{<-}[dr]
 \ar@{<-}@/^4ex/[drr]_{\text{id}} \\
 & k(\xi) \ar@{<-}[r] \ar@{<-}[d] & k(x_1)\\
 & k(x_1)
}\] Since $\xi \in \Delta$, there is some $x_1\in X$ such that
$\Delta(x_1)=\xi$.  So we get the diagram on the right, in which
every arrow must be an isomorphism (since all the morphisms are
between fields).  Now $\spec K \to X\times_Y X$ gives $k(\xi)\cong
k(x_1)\to K$, so $f$ factors through $\Delta$.

\begin{corollary}[of the Valuative Criterion for Properness]\
\begin{itemize}
 \item[(a)] Closed immersions are proper (but not open immersions).
 \item[(b)] Compositions of proper morphisms are proper.
 \item[(c)] Properness is stable under base extension. i.e.  if
 $f:X\to Y$ is proper, then $f':X\times_Y Y' \to Y'$ is
 proper.
 \[\xymatrix{
 X\times_Y Y' \ar[r] \ar[d]^{f'} & X\ar[d]^f\\
 Y' \ar[r] & Y
}\]
 \item[(d)] If $f:X\to Y$ and $f':X'\to Y'$ are proper
 $S$-morphisms, then so is $f\times f':X\times_S X' \to Y\times_S
 Y$.
 \item[(e)] If $X\xrightarrow{f} Y \xrightarrow{g} Z$ are
 morphisms and $g\circ f$ is proper and $g$ is separated, then $f$ is proper.
 \item[(f)] Properness is local on the base.  i.e. $f:X\to Y$
 is proper if and only if there is an open cover $\{U_i\}$ of
 $Y$ such that $f^{-1}(U_i)\to U_i$ is proper for all $i$.
 \item[(g)] If $f:X\to Y$ is proper, then so is
 $f_{\text{red}}:X_{\text{red}}\to Y_{\text{red}}$.  If $f$ is of
 finite type, the the converse also holds.
\end{itemize}
\end{corollary}
\begin{proof}[Partial Proof]
 In each case, we need to check finite-type-ness first.
\end{proof}
Again, this result actually holds without the noetherian
assumption.

The Valuative criterion for properness is \emph{important}, as
illustrated by the following fact.

\marginpar{An important application of the Valuative Criterion}Let
$Y$ be proper over a noetherian scheme $S$. Let $X$ be a
noetherian regular $S$-scheme of dimension 1. Assume $f$ is a
rational map such that the diagram
\[\xymatrix{
X\ar@{-->}[r]^f \ar[dr]&Y\ar[d]\\
& S }\] commutes.  Then $f$ extends (uniquely) to a morphism $X\to
Y$.
\begin{proof}
We have a dense open $U\subseteq X$ and an $S$-morphism $f:U\to
Y$.  Since $X\smallsetminus U$ is zero dimensional and $X$ is
noetherian, $X\smallsetminus U$ has a finite number of points.

For $x\in X\smallsetminus U$, let $R=\O_{X,x}$.  Then $R$ is a
regular noetherian local ring of dimension 1.  By remark
II.6.11.2A on page 142\footnote{A regular local ring is a UFD.},
it is entire (an integral domain), so by theorem I.6.2A on page
40\footnote{For a noetherian local domain of dimension 1, TFAE: i)
is a DVR ii) is integrally closed iii) is regular iv) max'l ideal
is principal.}, it is a DVR. Let $K=\text{Frac}\, R$. Let $T=\spec
R$ with $t=$ the closed point, and $\tau=$ the generic point.  Let
$g:T\to X$ be the canonical map.  Let $\xi=g(\tau)\in X$, then
$\xi\in U$ since it is not a closed point of $X$, so we get the
diagram
\[\xymatrix{
 \spec K \ar[rr] \ar[d] & & Y\ar[d]\\
 T=\spec R \ar[rr]\ar@{^(->}[dr] \ar@{.>}[urr]^{\exists !h}& & S\\
 & X  \ar[ur] \ar@{-->}[uur]^(.25){f}|!{[ur];[ul]}{\hole}
}\] By the valuative criterion, we get the map $h$.  Then we can
extend $f$ by saying $f(x)=h(t)$.

We may assume $S$ is affine, equal to $\spec C$.  Let $\spec B$ be
an open affine neighborhood of $h(t)\in Y$, and let $X'=\spec A$
be an open affine neighborhood of $x\in X$.  Let
$\mathfrak{p}\subseteq A$ correspond to $x$, then we get the
diagram
\[\xymatrix{
 K & & B \ar[ll] \ar[dll]^{h^*} \ar@{}[r]|(.3){=} & C[y_1,\dots,y_r]  \\
 A_{\mathfrak{p}}\ar[u] &  & C\ar[ld]\ar[u]\\
 & A \ar[lu]
 }\]
 There is some $a\not\in \mathfrak{p}$ such that $h^*(y_i)\in A_a$
 for all $i$.  So replace $A$ with $A_a$.  Then we get a map $B\to
 A$ and still have $x\in \spec A$.  From an exercise, we have that
 \[\xymatrix{
  \Hom_S(U\cap X',\spec B) \ar[r]^(.47){\sim} \ar@{}[dr]|{\circlearrowleft} & \Hom_C(B,\Gamma(U\cap
  X',\O_{X'})) \\
  \Hom_S(X',\spec B) \ar[u] \ar[r] & \Hom_C(B,\Gamma(X',\O_{X'}))
  \ar[u]
 }\]
So we have some $f':X'\to \spec B$ restricting to $f$.  Therefore,
$f$ extends to $X$.  Uniqueness follows from Exercise II.4.2,
which says that if two morphisms from a reduced scheme to a
separated scheme agree on a dense open subset, then they agree
everywhere.
\end{proof}

\begin{itemize}
\item[Example 1:] The assumption that $X$ is regular is necessary.
Take $X=\{y^2=x^2+x^3\}\subseteq \mathbb{A}^2_k$ for $k$ an
algebraically closed field of characteristic $\not=2$.  Take
$U=X\smallsetminus \{(0,0)\}$ and $f:U\to \P^1_k$, $(x,y)\mapsto
[x,y]$ the projection from the point $(0,0)$.  Then the map cannot
extend to the point $(0,0)$.

\item[Example 2:] The assumption $\dim X=1$ is necessary.  Let
$X=\P^2_k$, and let $Y$ be the blow-up of $\P^2_k$ at the origin,
$[0,0,1]$.  Then there is an obvious birational equivalence
between $X$ and $Y$, but they are not isomorphic.

\item[Key Application:] (Number Theory)  If $R$ is a Dedekind ring
and $Y=\spec R$, $K=\text{Frac}(R)$, and if $X$ is proper over $Y$
(i.e. $\spec R$), then any $K$-morphism $\spec K \to X$ gives a
rational map from $Y$ to $X$ by stuff from chapter I \S 4.  Thus,
we get a unique $R$-morphism $Y\to X$ (i.e. a section of $X\to
Y$).  Therefore, $X(K)=X(R)$.\footnote{Notation: For $X$ and $Y$
$S$-schemes, $Y(X)$ means $\Hom_S(X,Y)$.  Also, by $X(K)$ we mean
$X(\spec K)$.}
\end{itemize}

Recall that if $A$ is a ring and $n\in \mathbb{N}$, then $\P^n_Y
=\proj A[x_0,\dots, x_n]$.  If $A\to A'$ is a homomorphism, then
$\P^n_{A'} = \P^n_A\times_{\spec A} \spec A'$.

\begin{definition}
If $Y$ is an arbitrary scheme, then $\P^n_Y=\P^n_{\Z}\times Y$.
\end{definition}

\begin{definition}
A morphism $f:X\to Y$ is \emph{projective} if there is a closed
immersion $i:X\to \P^n_Y$ for some $n\in \mathbb{N}$ such that the
diagram commutes:
\[\xymatrix{
 X\ar[r]^i \ar[dr]^f & \P^n_Y\ar[d]\\
& Y }\]
\end{definition}
\begin{itemize}
 \item[Caution:] EGA uses a more general definition.

 \item[Note:] Neither definition is local on the base.  Why not?  For an
 open cover $\{U_i\}$, we could have $f^{-1}(U_i)\to
 \P^{n_i}_{U_i}$ with $n_i$ unbounded.  Or, more to the point, we
 need an $\O(1)$ on $X$ that works globally.

 \item[Example:] If $S$ is a graded ring, generated over $S_0=A$
 by finitely many elements of degree 1.  Then $\proj S \to \spec
 A$ is projective.
 \begin{proof}
 Let $s_0,\dots, s_n\in S_1$ be a generating set.  Then
 $T = A[t_0,\dots,t_n] \twoheadrightarrow S$.  So we have
 \[\xymatrix{
 \proj S \ar[dr] \ar@{^(->}[r]^(.4){\stack{\txt{\tiny closed}}{\txt{\tiny imm}}} & \proj T =
 \P^n_A \ar[d]\\
 & \spec A
 }\]
 \end{proof}
 \end{itemize}

\begin{lemma}
Let $n\in \mathbb{N}$.  Then $\P^n_{\Z}$ is proper over $\Z$.
\end{lemma}
\begin{proof}
in the works.
\renewcommand{\qedsymbol}{\text{\tiny proof continued in next lecture}}
\end{proof}
}
 { \stepcounter{lecture}
 \setcounter{lecture}{10}
 \sektion{Lecture 10}

\begin{proof}[continued proof]
in the works.
\end{proof}
\marginpar{end fast forward} \marginpar{$\xymatrix{ X
\ar@{^(->}[r]^i \ar[dr]^f & \P^n_Y \ar[d]^{\pi} \\  & Y}$} It
follows that $\P^n_Y\to Y$ is proper (it is a base extension). So
given a projective morphism $f:X\to Y$ we have that $f=\pi\circ i$
is a composition of proper morphisms, so it is proper.
\marginpar{Projective morphisms are proper}

The converse is false.  There are proper morphisms (even over
$\spec k$ with $k$ algebraically closed) which are not projective.

\begin{lemma}[Chow's Lemma, ex II.4.10]
Given a proper morphism $f:X\to S$, with $S$ noetherian, there is
a birational morphism\footnote{A \emph{birational morphism} is a
morphism which, as a rational map, is birational.} $g:X'\to X$
such that $f\circ g$ is projective.
\end{lemma}

\begin{definition}
A morphism $f:X\to Y$ is \emph{quasi-projective} if there is an
open immersion $i:X\hookrightarrow X'$ and a projective morphism
$X'\to Y$ such that $f=g\circ i$:
\[\xymatrix{
 X \ar@{^(->}[r]^{\txt{\tiny open}} \ar[dr]_f & X'
 \ar@{^(->}[r]^{\txt{\tiny closed}} \ar[d]^{\txt{proj}} & \P^n_Y
 \ar@/^4ex/[dl]
 \\
 & Y
}\] (therefore, $X$ is isomorphic to a subscheme of $\P^n_Y$ for
some $n$)
\end{definition}

\begin{definition}
A \emph{subscheme} of a scheme $Y$ is an immersion
$X\hookrightarrow Y$.  An \emph{immersion} is a map that can be
written as an open immersion followed by a closed immersion (or
vice versa (exercise)).
\end{definition}

\begin{theorem}
If $f:X\to Y$ is quasi-projective and $Y$ is noetherian, then $f$
is separated and of finite type.
\end{theorem}
\begin{proof}
Let $X\to X'$ be an open immersion with $X'$ projective.  Then
$X\to X' \to Y$ is separated (it is a composition of separated
morphisms) and of finite type (it is a composition of finite type
morphisms - Ex II.3.3c; $X'$ noeth $\Rightarrow X$ noeth
$\Rightarrow X$ quasi-compact).
\end{proof}

\underline{Near Converse(Nagata)}: If $f:X\to Y$ is separated and
of finite type, then there is an open immersion $X\hookrightarrow
X'$ with $X'$ proper over $Y$.

\begin{definition}
Let $k$ be an algebraically closed field.  A \emph{variety} over
$k$ is an integral scheme, separated and of finite type over
$\spec k$.  (i.e. it is a separated, finite type morphism $X\to
\spec k$ with $X$ integral).

It is \emph{projective} (resp. \emph{quasi-projective)} if the
morphism is, and it is \emph{complete} if the morphism is proper.
\end{definition}

Note: not everybody uses this definition.  Some say $X$ is reduced
instead of integral.  Many allow arbitrary $k$ (in which integral
may be replaced by geometrically integral\footnote{$X$ is
\emph{geometrically integral} if $X\times_{\spec k} \bar k$ is
integral.  See exercise II.3.15}).

\vspace{8mm} \marginpar{\S III.4 \v{C}ech Cohomology}

In this section:
\begin{itemize}
 \item[-] $X$ is a topological space
 \item[-] $\U=(U_i)_{i\in I}$ is an open cover of $X$ with a
 \emph{well ordered} index set $I$
 \item[-] $\F$ is a sheaf of abelian groups on $X$
 \item[-] $C^p(\U,\F) := \displaystyle\prod_{i_0<i_1< \cdots <
 i_p}\!\!\!\!\!\!\!\!
 \F(U_{i_0 i_1 \cdots i_p})$, with $U_{i_0 i_1 \cdots i_p}:=
 U_{i_1}\cap U_{i_2}\cap \cdots \cap U_{i_p}$ for any $p\in
 \mathbb{N}$
\end{itemize}

If $\alpha\in C^p(\U,\F)$, we write its components as $\alpha_{i_0
i_1 \cdots i_p}\in \F(U_{i_0 i_1 \cdots i_p})$.  For arbitrary
$(p+1)$-tuples $i_0 i_1 \cdots i_p$ we write
\[
 \alpha_{i_0 i_1 \cdots i_p} = \left\{
 \begin{tabular}{ll}
 0 & if the tuple contains a repeat \\
 $(-1)^{|\sigma|}\alpha_{\sigma (i_0) \sigma( i_1) \cdots \sigma (i_p)}$
 & where $\sigma (i_0) < \cdots < \sigma (i_n)$
 \end{tabular} \right.
 \]

Define $d:C^p(\U,\F) \to C^{p+1}(\U,\F)$ by
\[
    (d\alpha)_{i_0 i_1 \cdots i_{p+1}} = \sum_{j=0}^{p+1} (-1)^j
    \alpha_{i_0 \cdots \hat i_j \cdots i_{p+1}}|_{U_{i_0\cdots i_{p+1}}}.
\]
Observe that the definition is compatible with the convention for
arbitrary $i_0 i_1 \cdots i_{p+1}$.  Also, we have that $d^2=0$.

Thus, we have a complex $C^{\cdot}(\U,\F)$
\begin{definition}
The \v{C}ech Cohomology of $\F$ with respect to $\U$ is
\[
    \check H^p(\U,\F) = h^p(C^{\cdot}(\U,\F)).
\]
\end{definition}


Example: $X,\F$ as above, with $\U=\{X\}$.  Then $C^p(\U,\F)=0$
for all $p\not=0$ and $C^0(\U,\F)=\Gamma(X,\F)$, so
\[
    \check{H}^p(\U,\F) = \left\{
 \begin{tabular}{ll}
 $\Gamma(X,\F)$ & $p=0$ \\
 0 & $p\not=0$.
 \end{tabular} \right.
\]
In particular, \v{C}ech Cohomology might not have a long exact
sequence.  We will eventually show that if all the $U_{i_1\cdots
i_p}$ are acyclic for $\F$, then
\[
    \check H^{\cdot}(\U,\F) \cong H^{\cdot} (X,\F).
\]

Example: Lets find $\check H(\U,\O(1))$ where
\begin{itemize}
 \item[] $X=\P^1_k = \proj k[x,y]$ for $k$ a field
 \item[] $\U = \{U,V\}$ with
 \item[] \quad $U=D_+(x) = \spec k[1/t]$ where $t=x/y$
 \item[] \quad $V=D_+(y) = \spec k[t]$
\end{itemize}
Then \begin{align*}
 C^0 &= \Gamma(U,\O(1))\times \Gamma(V,\O(1)) \\
    &\cong \Gamma(U,\O_U)\times \Gamma(V,\O_V) \\
    &= k[1/t]\times k[t]\\
 C^1 &= \Gamma(U\cap V,\O(1)) \\
    &\cong \Gamma(U\cap V,\O_{U\cap V})\\
    &= k[t]_t = k[t,1/t]
 \end{align*}
What is the map $d$?  Recall that $\O(1) = \widetilde{S(1)}$, so
\begin{align*}
 \O(1)|_{D_+(x)} &= \widetilde{S(1)_{(x)}} = \widetilde{\ xk[1/t]\ }\\
 \O(1)|_{D_+(y)} &= \widetilde{S(1)_{(y)}} = \widetilde{\ yk[t]\ }
\end{align*}
So
\[
 {\begin{array}{rl}
  \Gamma(U,\O(1)) & \hspace{-2.5mm} = xk[1/t]\cong k[1/t]    \\
  \Gamma(V,\O(1)) & \hspace{-2.5mm} = yk[t]\cong k[t]        \\
  \Gamma(U\cap V,\O(1)) & \hspace{-2.5mm} = xk[t]_t \cong k[t,1/t]
 \end{array}}\quad
 \begin{xy}<1cm,0cm>:<0cm,5mm>::
   \ar@{|->} (0,0) *{\ni 1};(1,-1) *{1/t},
   \ar@{|->} (1,1) *{\ni 1};(3,-1) *{1},
 \end{xy}
\]

%\begin{align*}
% \Gamma(U,\O(1)) &= xk[1/t]\cong k[1/t] \hspace{2cm} \rnode{a}{1}\\
% \Gamma(V,\O(1)) &= yk[t]\cong k[t]\hspace{2cm} \rnode{b}{1}\\
% \Gamma(U\cap V,\O(1)) &= xk[t]_t \cong k[t,1/t]\hspace{2cm} \rnode{c}{1/t}
% \hspace{15mm} \rnode{d}{1} \ncline{|->}{a}{d}
% \ncline{|->}{b}{c}
%\end{align*}

Then we have $\check H^1(\U,\O(1)) = C^1/\im(C^0\to C^1)$.  But
\begin{align*}
\im (\Gamma(U,\O(1))\to C^1) &= \bigoplus_{n\le 0} kt^n\\
\im (\Gamma(V,\O(1))\to C^1) &= \bigoplus_{n\ge -1} kt^n\\
\end{align*}
so $\im(C^0\to C^1)$ is all of $C^1$, and therefore $\check
H^1(\U,\O(1))=0$.
\begin{align*}
\check H^0(\U,\O(1)) &= \ker(C^0\to C^1) \\
    &=\im(\Gamma(U,\O(1))\to \Gamma(U\cap V,\O(1)))& \text{(since these maps}\\ &\quad \cap \im(\Gamma(V,\O(1))\to \Gamma(U\cap
    V,\O(1))) &\text{are one to one)}\\
    &= \bigoplus_{n\le 0} kt^n \cap \bigoplus_{n\ge -1} kt^n\\
    &= kt^{-1}\oplus kt^0 = k^2 = \Gamma(X,\O(1))
\end{align*}

More generally, for all $X,\F,\U$, $\check H^0(\U,\F) =
\Gamma(X,\F)$.
\begin{proof}
\[\xymatrix{
 0\ar[r] &\check H^0(\U,\F) \ar[r] & C^0(\U,\F) \ar[r]^f \ar@{}[d]|{\parallel} &
 C^1(\U,\F) \ar@{^(->}[d] \\
 0 \ar[r] & \Gamma{X,\F} \ar[r]  & \prod \F(U_i) \ar[r]_(.4)g \ar@{}[ru]|{\circlearrowleft} &
 \prod_{i,j}\F(U_i\cap U_j)
}\] The top row is exact, and the bottom row is exact by the sheaf
axioms.  Commutativity of the square tells us that $\ker f = \ker
g$, as desired.
\end{proof}
}
 { \stepcounter{lecture}
 \setcounter{lecture}{11}
 \sektion{Lecture 11}

We will show that $\check H^p(\U,\F)\xrightarrow{\sim} H^p(X,\F)$
whenever $\U$ is such that $\F|_{U_{i_0\dots i_p}}$ is acyclic
with respect to derived functor cohomology.

Define a sheaf version of $C^{\cdot}(\U,\F)$ in the following way:
\[
    \C^p(\U,\F) = \prod_{i_0<\cdots <i_p} f_*(\F|_{U_{i_0\dots
    i_p}})
\]
where $f$ has components $f_{i_0\dots i_p}:U_{i_0\dots
i_p}\hookrightarrow X$.  Then we have that
$\Gamma(X,\C^p(\U,\F))=C^p(\U,\F)$.  Also, define
$d:\C^p(\U,\F)\to \C^{p+1}(\U,\F)$ in the obvious way.

\begin{lemma}\label{L:lec11cechexact}
For all $X,\U,\F$, the complex $\C^{\cdot}(\U,\F)$ is a resolution
of $\F$.
\end{lemma}
\begin{proof}
Define $\epsilon: \F \to \C^0(\U,\F)=\prod_i f_*(\F|_{U_i})$ by
$\F(V)\ni s\mapsto (s|_{U_i\cap V})_i$.  Then we need to show that
the sequence
\[
    0\to \F \xrightarrow{\epsilon} \C^0(\U,\F)\xrightarrow{d}
    C^1(\U,\F) \xrightarrow{d} \cdots
\]
is exact.

First, $\epsilon$ is injective because it is injective on $\F(V)$
for all $V$ (since $\F$ is a sheaf).  So view $\F$ as a subsheaf
of $\C^0(\U,\F)$.  Then exactness at $p=0$ is equivalent to
\[
    \ker(\C^0(\U,\F)\to \C^1(\U,\F)) = \F.
\]
But that follows from
\begin{align*}
    \ker(\C^0(\U,\F)(V)\to \C^1(\U,\F)(V)) &= \ker(C^0(\U|_V,\F|_V)\to
    C^1(\U|_V,\F|_V))\\
     &= \check H^0(\U|_V,\F|_V) = \Gamma(V,\F|_V)\\
     &= \F(V)
\end{align*}

To show exactness everywhere else, fix $x\in X$ and $j$ such that
$x\in U_j$.  For each $p\ge 1$, define
\[
    k:\C^p(\U,\F)_x\to \C^{p-1}(\U,\F)_x
\]
in the following way: given $\alpha_x\in \C^p(\U,\F)_x$, lift it
to $\C^p(\U,\F)(V)$ for some open neighborhood $x\in V\subseteq
U_j$.  Then for any $i_0<\cdots<i_{p-1}$, let
\[
    (k\alpha)_{i_0\dots i_{p-1}} = \alpha_{j,i_0\dots i_{p-1}}.
\]
Then $k(\alpha_x) = (k\alpha)_x$.  This is well defined
(independent of $V$).
\[\xymatrix{
    \C^0(\U,\F)\ar[r]^d \ar@<-2pt>[d]_{\text{id}} \ar@<2pt>[d]^0 &
    \C^1(\U,\F)\ar[r]^d \ar[dl]_k \ar@<-2pt>[d]_{\text{id}} \ar@<2pt>[d]^0 &
    \C^1(\U,\F)\ar[r]^d \ar[dl]_k \ar@<-2pt>[d]_{\text{id}} \ar@<2pt>[d]^0 &
    \cdots \ar[dl]_k \\
    \C^0(\U,\F)\ar[r]^d & \C^1(\U,\F)\ar[r]^d & \C^1(\U,\F)\ar[r]^d & \cdots\\
}\]
 \vspace{-5mm}
\begin{itemize}
 \item[] \begin{claim} $(kd+dk)(\alpha_x) = \alpha_x$ for all
 $\alpha_x\in\C^p(\U,\F)_x$.  That is, $k$ is a homotopy between
 the identity map on the complex $\C^{\cdot}(\U,\F)$ and the zero
 map. \end{claim}
 \begin{proof}[Proof of Claim]
  Compute away:
  \begin{align*}
   (d(k(\alpha_x)))_{i_0\dots i_p} &= \sum_{l=0}^p (-1)^l
   (k(\alpha_x))_{i_0\dots \hat i_l \dots i_p}|_{U_{i_0\dots
   i_p}}\\
   &= \sum_{l=0}^p (-1)^l \alpha_{j,i_0\dots \hat i_l \dots
   i_p}|_{U_{i_0\dots i_p}}\\
   (k(d(\alpha_x)))_{i_0\dots i_p} &= (d\alpha)_{j,i_0\dots i_p}\\
   &= \alpha_{i_0\dots i_p} +
   \sum_{l=0}^p (-1)^{l+1}\alpha_{j,i_0\dots \hat i_l \dots
   i_p}|_{U_{i_0\dots i_p}}
  \end{align*}
  Now add the two and you get $\alpha_{i_0\dots i_p}$.
 \renewcommand{\qedsymbol}{$\square_{\text{Claim}}$}
 \end{proof}
\end{itemize}

Since the identity map on the complex $\C^{\cdot}(\U,\F)$ is
homotopic to the identity, the induced maps on homologies are the
same, so the homologies are zero\footnote{Another way to say this:
if $\alpha$ is a cocycle, then we have that $\alpha =
(kd+dk)\alpha = d(k\alpha)$, so it is a coboundary.}. That is, the
sequence is exact for $p\ge 1$, as desired.
\end{proof}

\begin{lemma}\label{L:lec11cohommap}
Let $X,\U,\F$ be as usual.  Then there is a map
\[
    \check H^p(\U,\F) \to H^p(X,\F)
\]
for all $p\in \mathbb{N}$ which is natural in $\F$.
\end{lemma}
\begin{proof}
Let $0\to \F \to \I^{\cdot}$ be an injective resolution of $\F$.
Then by Lemma (\ref{L:lec2lemma1}), there are maps $f^p$ for all
$p\in \mathbb{N}$ such that
\[\xymatrix{
 0 \ar[r] & \F \ar[r]^(.4){\epsilon}\ar@{}[d]|{\parallel} &
 \C^0(\U,\F)\ar[r]^d \ar[d]^{f^0} & \C^1(\U,\F)
 \ar[r]^d\ar[d]^{f^1} & \cdots\\
 0 \ar[r] & \F \ar[r]_{\epsilon} & \I^0\ar[r] & \I^1\ar[r] & \cdots
}\] and the system of maps $f^{\cdot}$ is unique up to homotopy.
Eliminating the first column and taking global sections, we have
\[\xymatrix{
 0\ar[r] & C^0(\U,\F) \ar[r]^d \ar[d] & C^1(\U,\F)\ar[r]^(.6)d \ar[d] & \cdots \\
 0\ar[r] & \Gamma(X,\I^0)\ar[r] & \Gamma(X,\I^1)\ar[r] & \cdots
}\] Taking homologies, we get well defined maps
\[
    \check H^p(\U,\F)\to H^p(X,\F)
\]
for all $p\in \mathbb{N}$.

Finally, we need to show naturality.  That is, we need to show
that for all maps $\F\to \G$, the box
\[\xymatrix{
 \check H^p(\U,\F) \ar[r]\ar[d] & H^p(X,\F)\ar[d]\\
 \check H^p(\U,\G) \ar[r] & H^p(X,\G)
}\] commutes.  To see this, let $\J^{\cdot}$ be an injective
resolution of $\G$, then observe that the diagram of complexes
\[\xymatrix{
 C^{\cdot}(\U,\F) \ar[r]\ar[d] & \I^{\cdot}\ar[d]\\
 C^{\cdot}(\U,\G) \ar[r] & \J^{\cdot}
}\] commutes up to homotopy\footnote{I didn't actually check
this.}. Thus, when we take homologies, we get a commutative
diagram.
\end{proof}

\begin{lemma}
Let $X,\U,\F$ be as usual.  If $\F$ is flasque, then
\[ \check H^p(\U,\F)=0 \] for all $p>0$.
\end{lemma}
\begin{proof}
For all $V\subseteq X$ open and for all $p$, we have
\[\xymatrix{
 \C^p(\U,\F)(X)\ar[r] \ar@{}[d]|{\parallel} &
 \C^p(\U,\F)(V)\ar@{}[d]|{\parallel}\\
 \prod \F(U_{i_0\dots i_p}) \ar[r] & \prod \F(U_{i_0\dots i_p}\cap
 V)
}\]where the bottom arrow is surjective because it is surjective
componentwise.  Thus, the top arrow is surjective.  This shows
that $\C^p(\U,\F)$ is flasque.  Thus, $\C^{\cdot}(\U,\F)$ is a
flasque resolution of $\F$, so we can use it to compute derived
functor cohomology.
\begin{align*}
 H^p(X,\F) &= h^p(\Gamma(X,\C^{\cdot}(\U,\F)))\\
    &= h^p(C^{\cdot}(\U,\F))\\
    &= \check H^p(\U,\F)
\end{align*}
But $H^p(X,\F)=0$ for all $p>0$ because $\F$ is flasque.
\end{proof}

\begin{theorem}[Exercise III.4.11] \label{T:lec11} \marginpar{\v{C}ech Cohomology
agrees with Derived Functor Cohomology for the right $\U$} Let
$X,\U,\F$ be as usual. Assume that $H^l(U_{i_0\dots
i_p},\F|_{U_{i_0\dots i_p}})=0$ for all $p\in \mathbb{N}$,
$i_0<\cdots i_p$, and $l>0$.  Then the maps from Lemma
(\ref{L:lec11cohommap}) are isomorphisms:
\[
    \check H^p(\U,\F) \xrightarrow{\sim} H^p(X,\F).
\]
\end{theorem}
\begin{proof}
We apply induction on $p$.  If $p=0$, then both cohomologies are
isomorphic to $\Gamma(X,\F$.

For $p>0$, assume the result up to $p-1$.  Embed $\F$ into a
flasque sheaf $\G$, and let $\R$ be the quotient:
\[
    0\to \F \to \G \to \R \to 0
\]
Then we get the exact sequence
\[
    0\to \F(U_{i_0\dots i_p}) \to \G(U_{i_0\dots i_p}) \to \R(U_{i_0\dots
    i_p}) \to \underbrace{H^1(U_{i_0\dots i_p},\F|_{U_{i_0\dots
    i_p}})}_{0\text{ by assumption}}
\]
That is,
\begin{equation}\label{S:lec11seq}
    0\to C^{\cdot}(\U,\F)\to C^{\cdot}(\U,\G) \to C^{\cdot}(\U,\R) \to 0
\end{equation}
is exact, so we get a long exact sequence in \v{C}ech cohomology:
\[
    0\to \check H^0(\U,\F) \to  \check H^0(\U,\G) \to  \check H^0(\U,\R)
    \to  \check H^1(\U,\F) \to  \underbrace{\check H^1(\U,\G)}_{0\text{ since $\G$ flasque}}
\]

Let $0\to \F \to \I^{\cdot}$ and $0\to \R \to \J^{\cdot}$ be
injective resolutions.  Then, as in Theorem (\ref{T:lec2thm}), we
see that $\I^{\cdot}\oplus \J^{\cdot}$ is an injective resolution
for $\G$. such that
 \[\xymatrix{
 & 0\ar[d] & 0\ar[d] & 0\ar[d]\\
 0\ar[r] &\F\ar[r]\ar[d] & \G\ar[r]\ar[d] & \R\ar[r]\ar[d] & 0\\
 0\ar[r]& \I^0\ar[r]\ar[d] & \I^0\times \J^0 \ar[r] \ar[d] & \J^0 \ar[r]
 \ar[d] & 0\\
 & \vdots & \vdots & \vdots
} \] commutes.  Then we will get a diagram
\[\xymatrix{
 0\ar[r] & C^{\cdot}(\U,\F)\ar[r] \ar[d] & C^{\cdot}(\U,\G) \ar[r] \ar[d] &
 C^{\cdot}(\U,\R) \ar[r] \ar[d] & 0\\
 0\ar[r] & \Gamma(X,\I^{\cdot}) \ar[r] & \Gamma(X,\I^{\cdot}\oplus
 \J^{\cdot}) \ar[r] & \Gamma(X,\J^{\cdot}) \ar[r] & 0
}\tag{loose end 1}\] such that the rows are exact and it commutes.

Taking long exact sequences, we get
\[\xymatrix{
    0\ar[r]& \check H^0(\U,\F) \ar[r] \ar[d]^{\wr} &  \check H^0(\U,\G) \ar[r] \ar[d]^{\wr}
    &  \check H^0(\U,\R) \ar[r] \ar[d]^{\wr} &  \check H^1(\U,\F) \ar[r] \ar[d] &  0 \\
    0\ar[r]& H^0(\U,\F) \ar[r]& H^0(\U,\G) \ar[r]&  H^0(\U,\R)
    \ar[r]& H^1(\U,\F) \ar[r]&  0
}\] So the map on the right must also be an isomorphism (though we
still need to show it is the same as the map obtained in Lemma
(\ref{L:lec11cohommap}) ... this is loose end 2).

For all $p>1$, the long exact sequence obtained from
(\ref{S:lec11seq}) give us that
\[\xymatrix{
 0\ar@{}[r]|(.3){=} & \check H^{p-1}(\U,\G) \ar[r] &
 \check H^{p-1}(\U,\R) \ar[d]^{\wr\text{ induction}} \ar[r]^{\sim}& \check H^p(\U,\F)\ar[r]\ar[d] & \check H^p(\U,\G)\ar@{}[r]|(.7){=} & 0 \\
 0\ar@{}[r]|(.3){=} & H^{p-1}(X,\G) \ar[r] & H^{p-1}(X,\R) \ar[r]^{\sim} & H^p(X,\F)\ar[r] &
 H^p(X,\G) \ar@{}[r]|(.7){=} & 0
}\]
 To use induction, we need $\R$ to satisfy the hypothesis.  That
 is, we need to check that $H^l(U_{i_0\dots i_p},\R|_{U_{i_0\dots
 i_p}})=0$ for all $l>0$, $p\in \mathbb{N}$, and $i_0<\cdots<i_p$.
 To see this, note that we have
 \[
    0\to \F|_{U_{i_0\dots i_p}} \to \G|_{U_{i_0\dots i_p}} \to
    \R|_ {U_{i_0\dots i_p}}\to 0
 \]
 exact, so the long exact sequence in cohomology tells us that
 \[
    \underbrace{H^l(U_{i_0\dots i_p},\G|_{U_{i_0\dots i_p}})}_{0\text{ since $\G$ flasque}}
    \to H^l(U_{i_0\dots i_p},\R|_{U_{i_0\dots i_p}}) \to \underbrace{H^{l+1}(U_{i_0\dots i_p},\F|_{U_{i_0\dots
    i_p}})}_{0\text{ by assumption}}
 \]
 for all $l>0$, so $H^l(U_{i_0\dots i_p},\R|_{U_{i_0\dots
 i_p}})=0$, as required.

Now only two loose ends remain.
\renewcommand{\qedsymbol}{\text{\tiny proof continued in next lecture}}
\end{proof}
}
 { \stepcounter{lecture}
 \setcounter{lecture}{12}
 \sektion{Lecture 12}

\begin{corollary}[of unfinished theorem] Let $\F$ be a
quasi-coherent sheaf on a noetherian separated scheme $X$, and let
$\U$ be an open affine cover of $X$.  Then the conclusion of the
theorem holds.  The maps from Lemma (\ref{L:lec11cohommap}) are
isomorphisms:
\[
    \check H^p(\U,\F) \xrightarrow{\sim} H^p(X,\F).
\]
\end{corollary}
\begin{proof}
All the $U_{i_0\dots i_p}$ are affine by exercise II.4.3, so use
Theorem (\ref{T:lec5}). \marginpar{Theorem \ref{T:lec5}:qco
sheaves are acyclic on affine schemes}
\end{proof}

\begin{proof}[Continued proof of \ref{T:lec11}] \def\M{\mathscr{M}}
Loose end 1: we need to construct the commutative diagram
\begin{equation}\label{D:lec12loose}\xymatrix{
 0\ar[r] & C^{\cdot}(\U,\F)\ar[r] \ar[d] & C^{\cdot}(\U,\G) \ar[r] \ar[d] &
 C^{\cdot}(\U,\R) \ar[r] \ar[d] & 0\\
 0\ar[r] & \Gamma(X,\I^{\cdot}) \ar[r] & \Gamma(X,\I^{\cdot}\oplus
 \J^{\cdot}) \ar[r] & \Gamma(X,\J^{\cdot}) \ar[r] & 0
}\end{equation}
 To do this, we switch to sheaves:
\[\xymatrix{
 0\ar[r] & \C^{\cdot}(\U,\F)\ar[r] \ar@{.>}[d]^1 & \C^{\cdot}(\U,\G) \ar[r] \ar@{.>}[d]^2 &
 \C^{\cdot}(\U,\R) \ar@{.>}[d]^3 & \txt{(not surjective)}\\
 0\ar[r] & \I^{\cdot} \ar[r] & \I^{\cdot}\oplus
 \J^{\cdot} \ar[r] & \J^{\cdot} \ar[r] & 0
}\]
 Note that the sequence of sheaves is not exact on the right.  We wish
 to construct arrows 1,2, and 3.  Assume we have
 constructed them up to $p-1$.  Then we have the commutative diagram
 \[\hspace{-1cm} \xymatrix@!0 @C=15mm @R=15mm {
  0\ar[rr] &    & \K'\ar[rr]\ar@{^(->}[dr]\ar[dd]|\hole &
    & \K\ar[rr]\ar@{^(->}[dr]\ar[dd]|\hole  &
    & \K''\ar@{^(->}[dr] \ar[dd]|\hole &    &  \txt{(not surjective)} \\
    & 0\ar[rr]  &
    & \C^p(\U,\F) \ar[rr] \ar@{.>}[dd]^(.3)1 &
    & \C^p(\U,\G) \ar[rr]^(.35){\psi} \ar@{.>}[dd]^(.3)2 &
    & \C^p(\U,\R)  \ar@{.>}[dd]^(.3)3 &   &  \txt{(not surjective)} \\
  0\ar[rr] &
    & \M'\ar[rr]|\hole \ar@{^(->}[dr] &
    & \M\ar[rr]|\hole \ar@{^(->}[dr]  &
    & \M''\ar[rr]|\hole \ar@{^(->}[dr] &    &  0 \\
    & 0\ar[rr]  &
    & \I^p \ar[rr] &
    & \I^p \oplus \J^p \ar[rr] &
    & \J^p \ar[rr] &   & 0
 }\]
 where the $\K$'s and $\M$'s are the cokernels of the map from
 $(p-2)$-th to the $(p-1)$-th terms in the complexes.  Since each
 complex is exact (Lemma \ref{L:lec11cechexact}), these cokernels
 inject into the $p$-th terms.  If $p=0$, then $\K'=\M'=\F$,
 $\K=\M=\G$, and $\K''=\M''=\R$, with the downward morphisms
 identity maps.  Note that the rows of $\K$'s and $\M$'s are
 exact (to the degree shown).  The maps from
 the $\K$'s to the $\M$'s are the induced cokernel maps.

 To construct arrow 1, note that $\K'$ maps to $\I^p$, then
 injectivity of $\I^p$ produces the arrow.

 To construct arrow 2, consider the map $\varphi:\C^p(\U,\F)\oplus \K
 \to \C^p(\U,\G)$ given by addition of the images of the
 coordinates.  Then
 \begin{align*}
  \ker \varphi &= \{(x,-x)|x\in \C^p(\U,\F)\cap \K\}\\
    &= \C^p(\U,\F)\cap \K \\
    &= (\ker \psi) \cap \K\\
    &= \ker (\psi|_{\K})\\
    &= \ker(\K\to \K'') = \K'
 \end{align*}
Also, both $\C^p(\U,\F)$ and $\K$ map to $\I^p\oplus \J^p$, so we
get a map $\C^p(\U,\F)\oplus \K\to \I^p\oplus \J^p$ by adding the
images, and the kernel of this map contains $\C^p(\U,\F)\cap\K =
\K'=\ker \varphi$, so we get an induced map from the image of
$\varphi$ to $\I^p\oplus \J^p$:
\[\xymatrix{
 0\ar[r] & \im \varphi \ar[d] \ar[r] & \C^p(\U,\G)\ar@{.>}[dl]^2\\
 & \I^p\oplus J^p
}\] Then we get arrow 2 from injectivity of $\I^p\oplus \J^p$.

To construct arrow 3, note that the existence of arrows 1 and 2
imply that there is a map from the image of $\psi$ to $\J^p$
(since $\psi^{-1}$ followed by 2 followed by projection is well
defined). By injectivity of $\J^p$, we get arrow 3.

Taking global sections of the front face of the diagram, we get
the diagram (\ref{D:lec12loose}) and tie up our loose end (note
that we get surjectivity of the top row).

\vspace{3mm}

Loose end 2:  When we take long exact sequences of the rows in
diagram (\ref{D:lec12loose}), we get induced maps from \v{C}ech
cohomology to derived functor cohomology, and we need to know that
these maps are the same as those obtained in Lemma
(\ref{L:lec11cohommap}).  In the lemma, we constructed a map of
resolutions $\C^{\cdot}(\U,\F)\xrightarrow{f^{\cdot}} \I^{\cdot}$,
took global sections, and looked at the induced maps in homology.
The way we tied up loose end 1 makes it clear that the maps
obtained in the theorem are the same.
\end{proof}

\vspace{5mm} \underline{Exercise III.4.4:}\marginpar{Exercise
III.4.4}
 \def\V{\mathcal{V}}
 \def\W{\mathcal{W}}
Let $X$ be a topological space, and let $\F\in \Ab(X)$, then we
will show that
\[
 \varinjlim_{\U} \check H^1(\U,\F)\ \longrightarrow\ H^1(X,\F)
\]
is an isomorphism.

{\bf (a)} Let $\U=(U_i)_{i\in I}$ and $\V = (V_j)_{j\in J}$ be
open covers of $X$.  Suppose we're also given a function
$\lambda:J\to I$ such that $V_j\subseteq U_{\lambda(j)}$ for all
$j$ (that is, $\V$ is a refinement of $\U$).  Then for all $p$,
there is an induced map $\lambda^p:\check H^p(\U,\F)\to \check
H^p(\V,\F)$.  To see this, define
\begin{align*}
 C^p(\lambda): C^p(\U,\F) & \to C^p(\V,\F)\\
                    \alpha\quad &\mapsto\quad \beta
\end{align*}
where $\beta_{j_0\dots
j_p}=\alpha_{\lambda(j_0)\dots\lambda(j_p)}|_{V_{j_0\dots j_p}}$
(with the usual sign convention).  As $p$ varies, these maps
commute with the coboundary maps of $C^{\cdot}(\U,\F)$ and
$C^{\cdot}(\V,\F)$.  Thus, we get induced maps $\lambda^p:\check
H^p(\U,\F)\to \check H^p(\V,\F)$ for each $p$.  Moreover, given a
refinement $\W = \{W_k\}_{k\in K}$ of $\V$ and $\mu:K\to J$ such
that $W_k\subseteq V_{\mu(k)}$ for all $k$, the following diagram
commutes:
\[\xymatrix{
 \check H^p(\U,\F)\ar[r]^{\lambda^p}\ar[dr]_{(\mu\circ \lambda)^p}
 & \check H^p(\V,\F) \ar[d]^{\mu^p}\\
 & \check H^p(\W,\F).
}\]

So far, $\lambda^p$ depends on $\lambda$.
\begin{lemma*}
$\lambda^p$ is independent of $\lambda$, at least for $p\le 1$.
\end{lemma*}
\begin{proof}
 \underline{If $p=0$}: Let $\alpha\in C^0(\U,\F)$ be a cocycle (i.e.
 $d\alpha = 0$), so
 \[
    0 = (d\alpha)_{i_0,i_1} =
    (\alpha_{i_0}-\alpha_{i_1})|_{U_{i_0,i_1}} \qquad \quad \forall
    i_0<i_1
 \]\marginpar{All sections are restricted termwise to the appropriate open sets.}
 Then
 \[
    (\lambda^0(\alpha)-\mu^0(\alpha))_j = (\alpha_{\lambda(j)} -
    \alpha_{\mu(j)}) = 0
 \] so $\lambda^0(\alpha) = \mu^0(\alpha)$ for any $\lambda$ and
 $\mu$.

 \underline{If $p=1$}: Let $\alpha \in C^1(\U,\F)$ be a cocycle, so
\[
    (d\alpha)_{j_0,j_1,j_2} = \alpha_{j_1,j_2} - \alpha_{j_0,j_2}
    + \alpha_{j_0,j_1} = 0 \qquad\quad \forall j_0<j_1<j_2
\]
 Given
 $j_0,j_1\in J$, let $i_0=\lambda(j_0), i_1=\lambda(j_1), i_0' =
 \mu(j_0), i_1' = \mu(j_1)$, then
 \begin{align*}
 (\lambda^1(\alpha)-\mu^1(\alpha))_{j_0,j_1} &=
 \alpha_{i_0,i_1}-\alpha_{i_0',i_1'} \\
    &= (\alpha_{i_0,i_1}-\alpha_{i_0,i_1'})
        +(\alpha_{i_0,i_1'}-\alpha_{i_0',i_1'})\\
    &= -\alpha_{i_1,i_1'} + \alpha_{i_0,i_0'} & \text{($\alpha$ a
    cocycle)}\\
    &= (d\gamma)_{j_0,j_1}
 \end{align*}
 where $\gamma$ is defined by $\gamma_j =
 -\alpha_{\lambda(j),\mu_j}|_{V_j}$.  Thus, $\lambda^1(\alpha)$ and
 $\mu^1(\alpha)$ are cohomologous.
\end{proof}


More on $X\cong (X\times_S Y)\times_{Y\times_S Y}Y$, and when
fiber products associate. \marginpar{A neat trick for showing
schemes are isomorphic - a glimpse of Yoneda's Lemma}

The following always hold:
\begin{align*}
(A\otimes_S B)\otimes_S C &\cong A\otimes_S (B\otimes_S C) &\text{(for rings)}\\
A\otimes_S B &\cong B\otimes_S A \\
(A\times_S S')\times_{S'} B &\cong A\times_S B & \text{(for \underline{base change})}\\
A\times_S S &\cong A
\end{align*}


Fix a scheme, $S$.  Recall that $\Sch(S)$ is the category of
$S$-schemes, whose objects are morphism $X\to S$ and whose arrows
are commutative diagrams
 \xymatrix@!0 { X\ar[r]\ar[dr] & Y\ar[d] \\ & S}.
 An object $X\in \Sch(S)$ may also be viewed as the representable
 contravariant functor $\Hom_S(-,X):\Sch(S)\to {\bf Sets}$,
 $S'\mapsto \Hom_S(S',X) = X(S')$.  Then an $S$-morphism $f:X\to
 Y$ corresponds to a natural transformation of functors, $\varphi:
 X\to Y$ given by $\varphi(S') = f\circ -$.  Thus, given $S''\to S$, the diagram
 \[\xymatrix{
    X(S')\ar[r]^{\varphi(S')} \ar[d] & Y(S')\ar[d]\\
    X(S'') \ar[r]^{\varphi(S'')} & Y(S'')
 }\]
 commutes.  Conversely, given a natural transformation $\varphi:
 X\to Y$\marginpar{\xymatrix{X(X)\ar[r]^{\varphi(X)} & Y(X)}}, you can define $f =
 \varphi(X)(\id_X)$, which is an element of $Y(X) = \Hom_S(X,Y)$.
 If you start with $f$ and produce a natural transformation
 $\varphi$, then one may verify that $\varphi(X)(\id_X) =f$.
 Likewise, one may verify that the natural transformation
 associated to $\varphi(X)(\id_X)$ is indeed $\varphi$.

 So $S$-morphisms are the same a natural transformations\footnote{There is
 nothing special about $S$-schemes.  Objects in \emph{any} category may be viewed this
 way.  It is an immediate corollary of Yoneda's Lemma.}.  Now
 suppose $X, Y\in \Sch(S)$ give isomorphic functors, then the
 natural isomorphism between the functors induces an isomorphism
 of $S$-schemes.  Thus, to show that the two schemes are
 isomorphic, it is enough to find a natural transformation between
 the functors.

One may verify that $\Hom_S(S',-)$ behaves as
expected\footnote{$\Hom_S(S',-)$ should be the right adjoint to
something like $-\times_S S'$ ... I haven't verified this.}. For
example $(X\times_Z Y)(S') = X(S')\times_{Z(S')} Y(S') =
\{(x,y)\in X(S')\times Y(S')| f\circ x = g\circ y \text{ in }
Z(S')\}$.

Thus, we may calculate \marginpar{\xymatrix{
Y\ar[d]^{\Delta} \\
Y\times_S Y \\
X\times_S Y \ar[u]^{f\times \id_Y}}}
\begin{align*}
 ((X\times_S Y)\times_{Y\times_S Y}Y)(S') \
    &\cong (X(S')\times Y(S'))\times_{Y(S')\times Y(S')}Y(S')\\
    &= \{(\alpha,\beta,\gamma)\in X(S')\times Y(S')\times Y(S')| \\
    &\qquad (f(\alpha),\beta) = (\gamma,\gamma)\in Y(S')\times Y(S')
    \}\\
    &= X(S')\qquad \qquad (\beta=\gamma=f(\alpha))
\end{align*}
So $(X\times_S Y)\times_{Y\times_S Y}Y\cong X$.
}
 { \stepcounter{lecture}
 \setcounter{lecture}{13}
 \sektion{Lecture 13}

\marginpar{Begin Fast Forward}

Some stuff about proof in last lecture. in the works.

More on products: If $X,Y,Z,S'$ are $S$-schemes and $X\to Z,Y\to
Z$ are $S$-morphisms, then \[
    (X\times_Z Y)\times _S S' = X' \times_{Z'} Y'
\] where $\lozenge':=\lozenge\times_S S'$. \begin{proof} For all
$S$-schemes $Y$, we have that $\lozenge'(T) = \lozenge(T)\times
S'(T)$, so \begin{align*}
 (X'\times_{Z'} Y')(T) = \{&(\alpha,\gamma,\beta,\gamma') \in
 X(T)\times S'(T)\times Y(T)\times S'(T) |  \\ &(\alpha,\gamma)\text{
 and }(\beta,\gamma')\text{ both lie over the same element of } Z(T)\times
 S'(T)\} \\
 = \{&(\alpha,\gamma,\beta,\gamma') | \alpha\text{ and }
 \beta\text{ lie over the same element of } Z(T)\times S'(T)\} \\
 = (&X\times_Z Y)(T) \times S'(T)\\
 = (&(X\times_Z Y)\times_S S')(T)
\end{align*} \end{proof}

\begin{corollary} $(X\times_S Y)\times_S S' = X'\times_{S'}Y'$.
\end{corollary}

We were in the middle of doing Exercise III.4.4, which states that
$\varinjlim \check H^p(\U,\F) \to H^p(X,\F)$ is an isomorphism. in
the works.

\marginpar{Ch III \S 5: Cohomology of Projective space}

Let $A$ be a Noetherian ring, and let $X=\P^r_A$ for some $r\in
\mathbb{N}$.  Then we will compute $H^i(X,\O(n))$ for all $i\in
\mathbb{N}$ and $N\in \Z$.  Recall from \S II.5 that if
$Y\subseteq \P^r_A $ is a closed subscheme and $\F$ is a sheaf on
$Y$, then \[
    \Gamma_*(\F):= \bigoplus_{n\in \Z} \Gamma(Y,\F(n))
\] where $\F(n)=\F\otimes_{\O_Y}\O_Y(n)$, $\O_Y(n)=\O_X(n)|_Y =
i^*\O_X(n)$ where $i:Y\hookrightarrow X$.

\begin{theorem} If $\F$ is quasi-coherent, then there is a natural
isomorphism \[
    \widetilde{\Gamma_*(\F)} \xrightarrow{\sim} \F
\] (However, if $M$ is a graded $S$-module (where $Y=\proj S$),
then we may not have $\Gamma_*(\tilde M) \cong M$.). \end{theorem}
\begin{proof}
 Let $S=A[x_0,\dots,x_r]$ and $X=\proj S\ (=\P^r_A)$.
Let $\F=\bigoplus_{n\in \Z}\O(n)$.  This is a quasi-coherent
$\O_X$-module (though not coherent), and since cohomology commutes
with arbitratry direct sums (Lemma \ref{L:CohomologyLimit}),
 \[
    H^i(X,\F) = \bigoplus_{n\in X} H^i(X,\O_X(n)).
 \]
 in the works.

\renewcommand{\qedsymbol}{\text{\tiny proof continued in next
lecture}}
 \end{proof}
}
 { \stepcounter{lecture}
 \setcounter{lecture}{14}
 \sektion{Lecture 14}

 \begin{proof}[Proof Continued]
 in the works.
 \end{proof}

This suggests:
 \begin{theorem}
 The natural pairing
 \[
    H^0(X,\O_X(n))\times H^r(X,\O_X(-n-r-1)) \to H^r(x,\O_X(-r-1))
    \cong A
 \]
 is a perfect pairing of finitely generated free $A$-modules for
 all $n$.
 \end{theorem}
 \begin{proof}
 in the works.
 \end{proof}
 more in the works.

 We've just proved
 \begin{theorem}[III.5.1]
 Let $A$ be a noetherian ring, let $X=\P^r_A$ with $r\in \Z_{>0}$
 and let $n\in \Z$.  Then
 \begin{itemize}
 \item[(a)] The natural map $S_n\to H^0(X,\O_X(n))$ is an isomorphism.
 (where $S=A[x_0,\dots,x_n]$)
 \item[(b)] $H^i(X,\O_X(n))=0$ for all $i\not= 0,r$ and for all $n$.
 \item[(c)] $H^r(X,\O_X(-r-1))\cong A$
 \item[(d)] $H^0(X,\O_X(n)) \times H^r(X,\O_X(-n-r-1))\to
 H^r(X,\O_X(-r-1)) \cong A$ is a perfect pairing of finitely
 generated $A$-modules for all $n$.
 \end{itemize}
 \end{theorem}

 \begin{theorem}[III.5.2, Serre]\label{T:III.5.2}
 Let $A$ be a noetherian ring, $X$ a projective scheme over $A$,
 $\O_X(1)$ a very ample line sheaf (invertible sheaf) on $X$ over
 $A$ and let $\F$ be a coherent sheaf on $X$.  Then
 \begin{itemize}
 \item[(a)] $H^i(X,\F)$ is a finitely generated $A$-module for all
 $i$, and
 \item[(b)] $H^i(X,\F(n))=0$ for all $n\gg 0$ (depending on $\F$)
 and for all $i>0$.
 \end{itemize}
 \end{theorem}
 \begin{proof}
 in the works.
 \renewcommand{\qedsymbol}{\text{\tiny proof continued in next
lecture}}
 \end{proof}
}
 { \stepcounter{lecture}
 \setcounter{lecture}{15}
 \sektion{Lecture 15}

 \begin{proof}[Proof Continued]
 in the works.
 \end{proof}

 \begin{corollary}
 For any coherent sheaf $\F$ on a projective scheme $X$,
 $\Gamma(X,\F)$ is finitely generated (this generalizes Thm
 II.5.19).
 \end{corollary}

 \begin{corollary}
 Let $X$ be a closed subscheme of $\P^r_A=\proj S$ with
 $S=A[x_0,\dots, x_n]$.  Then $S_n\to \Gamma(X,\O_X(n))$ is
 surjective for all $n \gg 0$.
 \end{corollary}
 \begin{proof}
 in the works.
 \end{proof}

 \begin{proposition}[III.5.3]
 Let $A$ be a noetherian ring and let $X$ be a proper scheme over
 $A$.  Then a line sheaf $\L$ on $X$ is ample if and only if:
 \[
    \forall \F\in \mathfrak{Coh}(X), H^i(X,\F\otimes \L^{\otimes n})=0
    \quad\forall i>0, \forall n\gg 0 \text{ (depending on $\F$)} \tag{*}
 \]
 \end{proposition}
 \begin{proof}
 in the works
 \end{proof}

 \begin{definition} if $X$ is a scheme over a field $k$ and $\F$ a
 sheaf on $X$, then $h^i(X,\F):= \dim_k H^i(X,\F)$.
 \end{definition}

 \underline{Exercise (III.5.5)} in the works.
}
 { \stepcounter{lecture}
 \setcounter{lecture}{16}
 \sektion{Lecture 16}

 \marginpar{\S III.6: Ext Groups and Sheaves}
 Let $(X,\O_X)$ be a ringed space.  We'll be working with the
 category of $\O_X$-modules, so $\hom$ means $\hom_{\O_X}$ and
 $\Hom$ means $\Hom_{\O_X}$.  Recall that $\Hom(\F,\G):U\mapsto
 \hom_{\O_U}(\F|_U,\G|_U)$ is a sheaf (Ex. II.1.15).

 For us, $(X,\O_X)$ will be one of
 \begin{itemize}
 \item[(i)] a scheme, or
 \item[(ii)] $X=\{point\}$, $\O_X=$ some ring $A$, so $\Mod(X)=Mod(A)$.
 \end{itemize}

 \begin{definition}
 Let $(X,\O_X)$ be a ringed space, and let $\F$ be a sheaf of
 $\O_X$-modules.  Then \[\ext^i(\F,-)\] are the right derived
 fuctors of $\hom(\F,-)$, and \[\Ext^i(\F,-)\] are the right
 derived functors of $\Hom(\F,-)$. (Note that $\Mod(X)$ has enough
 injectives)
 \end{definition}

 \underline{Motivation:}
 \begin{itemize}
 \item[(a)] $\hom$ and $\Hom$ are basic functors, so it makes sense to
 look at thier derived functors.
 \item[(b)] Used in duality.
 \item[(c)] (Ex. III.6.1) $\ext^1(\F,\G)$ parameterizes \emph{extensions} of
 $\F$ by $\G$.  An extension is a sheaf $\F'$ such that
 \[
    0\to \G\to \F'\to \F\to 0.
 \]
 \end{itemize}

 \begin{lemma}
 If $\I$ is an injective $\O_X$-module and $U\subseteq X$ is open,
 then $\I|_U$ is an injective $\O_U$-module.
 \end{lemma}
 \begin{proof}
 Say we have a diagram of $\O_U$-modules with the top row exact:
 \[\xymatrix{
 0 \ar[r] & \F \ar[r] \ar[d] & \G\\ & \I|_U
 }\]
 Then we have
 \[\xymatrix{
 0 \ar[r] & j_!\F \ar[r] \ar[d] & j_!\G\\ & j_!(\I|_U)
 \ar[r]^-{\tiny \txt{ Ex.\\
 II.1.19}} & \I
 }\]
 where $j:U\to X$ is the inclusion and $j_!$ is as in (Ex.
 II.1.19).  By injectivity of $\I$, there is a map $\phi:j_!\G \to
 \I$ extending this diagram.  Restricting to $U$, we have
 $\phi|_U:(j_!\G)|_U=\G \to \I_U$, as desired.
 \end{proof}

 \begin{proposition} For any $U\subseteq X$, $\Ext^i(\F,\G)|_U =
 \Ext^i(\F|_U,\G|_U)$ naturally.
 \end{proposition}
 \begin{proof}
 Let $0\to \G \to \I^{\cdot}$ be an injective resolution of $\G$.
 Then, by the lemma, $0\to \G|_U\to \I^{\cdot}|_U$ is an injective
 resolution of $\G|_U$, so
 \begin{align*}
    \Ext^i(\F,\G)|_U &= h^i(\Hom(\F,\G))|_U \\
        &= h^i(\Hom(\F|_U,\G|_U))\\
        &= \Ext(\F|_U,\G|_U).
 \end{align*}
 \end{proof}

 \begin{proposition}[III.6.3]\label{P:III.6.3}\begin{itemize}
 \item[]
 \item[(a)] $\Ext^i(\O_X,\G) = \left\{ \begin{tabular}{ll}
  $\G$ & if $i=0$\\
  0 & if $i\not=0$
  \end{tabular} \right.$
 \item[(b)] $\ext^i(\O_x,\G) = H^i(X,\G)$ for all  $i$.
 \end{itemize}
 \end{proposition}
 \begin{proof}
 (a) $\Ext^i(\O_X,-)$ are the right derived functors of
 $\Hom(\O_X,-)$, which is the identity functor, which is exact.

 (b) $\ext^i(\O_X,-)$ are the right derived functors of
 $\hom(\O_X,-)$, which is the functor $\Gamma(X,-)$, whose right
 derived functors are $H^i(X,-)$.
 \end{proof}

 \begin{proposition}[III.6.4]
 If $0\to \F'\to \F\to \F''\to 0$ is a short exact sequence of
 $\O_X$-modules, then we have long exact sequences
 \[\xymatrix{
 0\to \hom(\F'',\G)\to \hom(\F,\G) \to \hom(\F',\G) \to
 \ext^1(\F'',\G)\to \cdots
 }\]
 and
 \[\xymatrix{
 0\to \Hom(\F'',\G)\to \Hom(\F,\G) \to \Hom(\F',\G) \to
 \Ext^1(\F'',\G)\to \cdots
 }\]
 \end{proposition}
 \begin{proof}
 Let $0\to \G\to \I^{\cdot}$ be an injective resolution.  Then by
 injectivity, we the exact sequences
 \[
    0\to \hom(\F'',\I^{\cdot})\to \hom(\F,\I^{\cdot}) \to
    \hom(\F',\I^{\cdot}) \to 0.
 \]
 Applying the Snake Lemma gives the result.  Similarly for the
 second long exact sequence.
 \end{proof}

 \begin{lemma}[III.6.6]
 If $\I$ is an injective $\O_X$-module and $\L$ is locally free of
 finite rank, then $\I\otimes \L$ is also injective.
 \end{lemma}
 \begin{proof}
 Recall from Ex II.5.1 that $\check \L = \Hom(\L,\O_X)$ and
 \begin{itemize}
  \item[(a)] $\check{\check \L} \cong \L$
  \item[(b)]$\Hom(\L,\F) \cong \F\otimes \check \L$ for all
  sheaves $\F$
  \item[(c)]  $\hom(\L\otimes \F,\G) \cong
  \hom(\F,\Hom(\L,\G))$ for all $\F,\G$.  More generally, it is
  true that
  \[\hom(\F_1\otimes \F_2,\G) \cong \hom(\F_1,\Hom(\F_2,\L))\]
  for all $\F_1,\F_2,\G$.
 \end{itemize}
  Thus, $\hom(-,\I\otimes\L)$ is equal to $\hom(-\otimes \check
  \L,\I)$, which is the composite of two exact functors, and is
  therefore exact.
 \end{proof}

 Note that
 \[
    \hom(\F\otimes\L,\G)\stackrel{(b),(c)}{\cong}
    \hom(\F,\G\otimes\check \L) \cong \Hom(\F,\G)\otimes \check
    \L \footnote{why is the second isormophism true?}
 \]

 \begin{proposition}
 If $\L$ is locally free of finite rank and $\F,\G\in \Mod(X)$,
 then
 \begin{itemize}
  \item[(a)] $\ext^i(\F\otimes \L,\G) \cong \ext^i(\F,\G\otimes
  \check \L)$ and
  \item[(b)] $\Ext^i(\F\otimes\L,\G) \cong \Ext^i(\F,\G\otimes
  \check \L) \cong \Ext^i(\F,\G)\otimes \check \L$.
 \end{itemize}
 \end{proposition}
 \begin{proof}
 Let $0\to \G \to \I^{\cdot}$ be an injective resolution. Then
 (a):
 \begin{align*}
 \ext^i(\F\otimes \L,\G) &= h^i(\hom(\F\otimes \L,\I^{\cdot}))\\
    &= h^i(\hom(\F,\I^{\cdot}\otimes \check \L))\\
    &= \ext^i(\F,\G\otimes \check \L) & (\I^{\cdot}\otimes \check \L
        \text{ inj res of } \G\otimes \check \L)
 \end{align*}
 And for (b), the first isomorphism is ``the same''.  We also have
 that
 \begin{align*}
 \Ext^i(\F,\G\otimes\check \L) &=
 h^i(\Hom(\F,\I^{\cdot}\otimes\check \L) \\
 &= h^i(\Hom(\F,\I^{\cdot}))\otimes \check \L\\
 &= \Ext^i(\F,\G)\otimes \check \L
 \end{align*}
 \end{proof}

 \begin{corollary}\label{C:lec16}
 We know $\ext^i$ and $\Ext^i$ when the first argument is locally
 free of finite rank:
 \begin{align*}
    \ext^i(\L,\G) &= \ext^i(\O_X\otimes \L,\G)\\
        &= \ext^i(\O_X,\G\otimes \check \L)\\
        &= H^i(X,\G\otimes \check \L)
 \end{align*}
 and
 \begin{align*}
    \Ext^i(\L,\G) &= \Ext^i(\O_X,\G)\otimes \check \L\\
        &=\left\{\begin{tabular}{ll}
            $\G\otimes \check \L$ & if $i=0$\\
            0 & if $i\not=0$\end{tabular} \right.
 \end{align*}
 \end{corollary}

 \begin{proposition}[III.6.5]
 Suppose we have a locally free resolution of $\F$ (i.e. an exact
 sequence $\cdots\to \L_1\to \L_0\to \F\to 0$, where $\L_i$ are
 locally free of finite rank for all $i$).  Then for all $\G$,
 \[
    \Ext^i(\F,\G)\cong h^i(\Hom(\L_{\cdot},\G)).
 \]
 \end{proposition}
 \begin{proof}
 in the works
 \end{proof}

 \begin{proposition}[III.6.8]
 Let $X$ be a noetherian scheme.  Let $\F$ be a coherent sheaf on
 $X$, and let $\G$ be any sheaf of $\O_X$-modules.  Then for all
 $x\in X$, and $i\in \mathbb{N}$,
 \[
    \Ext^i(\F,\G)_x = \ext^i_{\O_{X,x}}(\F_x,\G_x).
 \]
 \end{proposition}
 \begin{proof}
 This is a local question, so we may assume that $X$ is affine,
 say $X=\spec A$ with $A$ noetherian, and $\F=\tilde M$ where $M$
 is a finitely generated $A$-module.  Then ther is a free
 resolution
 \[
    \cdots \to L_1\to L_0\to M\to 0
 \]
 giving the locally free resolution $\tilde L_{\cdot}\to \F \to
 0$.  So
 \begin{align*}
  \Ext^i_X(\F,\G)_x &= h^i(\Hom(\tilde L_{\cdot},\G))_x\\
    &= h^i(\Hom(\tilde L_{\cdot},\G)_x)\\
    &= h^i((\check{\tilde {L_{\cdot}}}\otimes \G)_x)\\
    &= h^i(((\check{\tilde {L_{\cdot}}})_x\otimes_{\O_{X,x}} \G_x))\\
    &= h^i(\hom_{\O_{X,x}}((L_{\cdot})_x,\G_x))\\
    &= \ext^i_{\O_{X,x}}(\F_x,\G_x).
 \end{align*}
 \end{proof}

 \begin{proposition}[III.6.9] Let $X$ be a projective scheme over a
 noetherian ring $A$, let $\O_X(1)$ be a very ample line sheaf on
 $X$ over $A$; let $\F$ and $\G$ be coherent sheaves on $X$, and
 let $i\in \mathbb{N}$.  Then
 \[
    \Gamma(X,\Ext^i(\F,\G(n))) = \ext^i(\F,\G(n))
 \]
 for all $n\gg 0$ (depending on $i$).
 \end{proposition}
 \begin{proof}
 If $i=0$, then its true for all $\F,\G,n$ by definition of
 $\Hom$, so assume $i>0$.

 If $\F$ is locally free of finite rank, then we compute that the
 left hand side is 0 for all $n$ (Prop \ref{P:III.6.3}), and the right hand side
 is zero for $n\gg 0$ (Cor \ref{C:lec16} and Thm \ref{T:III.5.2}).
  In general, by Corollary (II.5.18), there is a short exact sequence
  \[
    0\to \R \to \E\to \F\to 0
  \]
  where $\E$ is a finite direct sum of twisted structure sheaves.
  Then for $n\gg 0$, $\check \E \otimes \G(n)$ has no higher
  cohomology (by Thm \ref{T:III.5.2}), so $\ext^i(\E,\G(n))=0$ for all $i>0$.
 \end{proof}
}
 { \stepcounter{lecture}
 \setcounter{lecture}{17}
 \sektion{Lecture 17}

 \marginpar{\S II.8: Differentials}
 \begin{definition}[An alternative definition] Let $B$ be an
 $A$-algebra ($A\xrightarrow{f} B$), then $\Omega_{B/A}$ is the
 $B$-module described by generators $db$ for all $b\in B$ and
 relations
 \begin{itemize}
  \item[] $da = 0$
  \item[] $d(b_1+b_2)=db_1+db_2$
  \item[] $d(b_1b_2)=b_1db_2+b_2db_1$
 \end{itemize}
 for all $a\in A$ and $b_i\in B$.  Equivalently, $d$ is $A$-linear
 and satisfies the Leibniz condition\footnote{$\Rightarrow$ trivial. $\Leftarrow:\
 da = ad1 = ad(1\cdot 1) = 2ad1$, so $da=0$.}.
 \end{definition}

 \begin{proposition}[II.8.3A: First Exact Sequence]\label{P:II.8.3A} If $A\to B \to C$
 are ring homomorphisms, then
 \[
 \Omega_{B/A}\otimes_B C   \longrightarrow \Omega_{C/A}   \longrightarrow \Omega_{C/B} \to
 0\]
 \[ db\otimes c  \mapsto cdb,\ dc  \mapsto dc \]
 is an exact sequence of $C$-modules.
 \end{proposition}
 \begin{proof}
 Surjectivity is obvious (you are simply imposing more relations).
 The kernel of the second map is generated by the ``new''
 relations, $\{ db=0| b\in B\}$, which is clearly the image of the
 first term.
 \end{proof}

 \underline{Example:} If $B$ is the polynomial ring $A[x_i]_{i\in
 I}$, then $\Omega_{B/A}$ is the free $B$-module generated by
 $dx_i$ for all $i\in I$\footnote{Anton and Dave say $I$ has to be finite.}.
 \begin{proof}
 By the universal property of $\Omega_{B/A}$, we get
 \[\xymatrix{
  B \ar[r]^d \ar[dr]_{\partial/\partial x_i} & \Omega_{B/A}\ar@{-->}[d]^{\exists\ dx_i\mapsto 1}\\
  & B
 }\]
 This gives a map $\Omega_{B/A}\to \prod_I B$.  The kernel of this
 map is zero (anything mapping to zero must be $d($something w/ no
 $x$'s$) = 0$.  And for any element of the product, it is easy to
 construct an inverse image\footnote{Not if $I$ is infinite.}.
 \end{proof}

 \begin{proposition}[Second Exact Sequence]
 If $A\to B \to C$ is a sequence of ring homomorphisms with $B\to
 C$ surjective, and with kernel $I$, then
 \[
 I/I^2 \xrightarrow{b\mapsto db\otimes 1} \underbrace{\Omega_{B/A} \otimes_B
 C}_{\Omega_{B/A}/I\Omega_{B/A}\footnote{in the works}}
 \xrightarrow{\text{same}} \Omega_{C/A}\to 0
 \]
 is an exact sequence of $C$-modules.
 \end{proposition}
 \begin{proof}
 Since $B\to C$ is surjective, $\Omega_{B/C}=0$, so by the first
 exact sequence, the second map is surjective.  The first map is
 well-defined since
 \[
    d(b_1b_2)\otimes 1 = db_1\otimes b_2 + db_2\otimes b_1 = 0.
 \]
 Now we show exactness in the middle.  It is clear that the
 composition of the two maps is zero.  Conversely, if $\sum
 c_idb_i=0$, then it is a $C$-linear

 in the works
 \end{proof}

 \begin{corollary}
 If $C= A[x_i]_{i\in I}/\a$, then $\Omega_{C/A}$ is the $C$-module
 described by generators $dx_i$ and relations $df=0$ for all $f\in
 \a$ (or for all $f$ in a generating set for $\a$).
 \end{corollary}
 \begin{corollary}[of the Corollary]
 If $A\to B$ and $A\to A'$ are $A$-algebras and $B'=B\otimes_A
 A'$, then
 \[
    \Omega_{B'/A'} \cong \Omega_{B/A}\otimes_B B' \quad ( = \Omega_{B/A}\otimes_A
    A')
 \]
 \end{corollary}
 \begin{proof}
 in the works (for Anton).
 \end{proof}

 \begin{corollary}
  If $S$ is a multiplicative subset of $B$, then
  \[\Omega_{S^{-1}B/A} \cong S^{-1} \Omega_{B/A}.\]
 \end{corollary}
 \begin{proof}
 $\Omega_{S^{-1}B/A}$ has the generators $d(s^-1 b)$.  Then
 $db = d(s\cdot s^{-1} \cdot b) = s^{-1}b\cdot ds + s\cdot d(s^{-1}b)
 $, so
 \[
    d(s^{-1}b) = s^{-1}db - s^{-2}b ds
 \]
 so you don't really get any new relations (exercise).
 \end{proof}

 \marginpar{Sheaves of Differentials} Now we pass to sheaves over
 schemes.  If $X$ is a scheme over $\spec A$, then there is a
 unique sheaf $\Omega_{X/A}$
}
 { \stepcounter{lecture}
 \setcounter{lecture}{18}
 \sektion{Lecture 18}

 Last time: If $Y=\spec A$ and $X=\P^n_Y$, then there is an exact
 sequence
 \[
    0\to \Omega_{X/Y} \to \O(-1)^{n+1} \to \O_X\to 0.
 \]
  By base change from $\spec \Z$, this is true for arbitrary $Y$.
  By Exercise II.5.16b, we have that $\wedge^{n+1}\O(-1)^{n+1} =
  \O(-n-1)\cong \wedge^1 \O(-1)\otimes \wedge^n
  \Omega_{X/Y}$.

  \begin{definition}
  Let $X$ be a non-singular variety over an algebraically closed
  field $k$.  Then the \emph{canonical sheaf} is $\omega_X =
  \wedge^n\Omega_{X/k}$, where $n=\dim X$.
  \end{definition}

 So if $X=\P^n_k$, then $\omega_X=\O(-n-1)$.  Concretely, say
 $X=\proj k[x_0,\dots,x_n]$.  Then on $D_+(x_0)\cong \spec
 k[y_1,\dots,y_n]$ for $y_i=x_i/x_0$, $\Omega_{X/k}|_{D_+(x_0)}$ is
 free with generators $dy_1,\dots, dy_n$.  Thus,
 $\omega_X|_{D_+(x_0)}$ is free of rank 1, generated by
 $dy_1\wedge\cdots\wedge dy_n$.

 On $D_+(x_n) = \spec k[x_0/x_n,\dots,x_{n-1}/x_n]$,
 $\Omega_{X/k}|_{D_+(x_n)}$ is generated by $d(x_0/x_n),\dots,
 d(x_{n-1}/x_n)$, so $\omega_{X}|_{D_+(x_n)}$ is free, generated
 by $d(x_0/x_n)\wedge \cdots \wedge d(x_{n-1}/x_n)$.

 On $D_+(x_0)\cap D_+(x_n)$, this generator is
 \begin{align*}
  d(1/y_n)\wedge &d(y_1/y_n)\wedge\cdots \wedge d(y_{n-1}/y_n) =\\
   &=
    (-y_n^{-2}dy_n)\wedge
    (y_n^{-1}dy_1-y_n^{-2}y_1dy_n)\wedge\cdots\wedge(y_n^{-1}dy_{n-1}-y_n^{-2}y_{n-1}dy_n)\\
   &= -y_n^{-n-1} dy_n\wedge dy_1\wedge \cdots \wedge dy_{n-1}
  \end{align*}
  which has a pole of order $n+1$ at $y_n=0$.  So the divisor of $d(x_0/x_n),\dots,
 d(x_{n-1}/x_n)$ is $-(n+1)\{x_n=0\}$, and
 \[
    \L(-(n+1)\{x_n=0\}) = \O(-n-1).
 \]

 \vspace{3mm}

 \marginpar{\S III.7: Serre Duality}
 In our computations on $\P^n$, we computed $H^i(\P^n_A,\O(q))$
 and came up with the perfect pairing
 \[
    \underbrace{\hom(\O_X,\O(r))}_{\hom(\O(-r-n-1),\omega)}\times
    H^n(\P^n_A what???
 \]
 in the works

 \begin{theorem}[III.7.1]\label{T:III.7.1} Let $k$ be a field and $X=\P^n_k$.  Then
 \begin{itemize}
  \item[(a)] $H^n(X,\omega_X) \cong k$
  \item[(b)] Fix such an isomorphism.  For all $\F\in
  \mathfrak{Coh}(X)$, the pairing
  \[
    \hom(\F,\omega)\times H^n(X,\F) \to H^n(X,\omega)
    \xrightarrow{\sim} k
  \]
  is a perfect pairing of finite dimensional vector spaces over
  $k$, and
  \item[(c)] For all $i\ge 0$, there is a natural isomorphism
  \[
    \ext^i(\F,\omega) \xrightarrow{\sim} H^{n-i}(X,\F)'
  \]
  which for $i=0$ is the isomorphism comming from the pairing (b)
  (and isn't canonical).
 \end{itemize}
 \end{theorem}

 \begin{proof}
 in the works
 \end{proof}

 \begin{definition} \marginpar{dualizing sheaf!}
 Let $X$ be a proper scheme over a field $k$, of
 dimension $n$.  A \emph{dualizing sheaf} for $X$ (over $k$) is a
 coherent sheaf $\omega_X^{\circ}$ on $X$ which represents the
 contravariant functor
 \[
    \mathfrak{Coh}(X)\to \Mod(k)
 \]
 given by $\F\mapsto H^n(X,\F)'$.  That is,
 \[
    H^n(X,-)'\cong \hom (-,\omega_X^{\circ}).
 \]
 \end{definition}

 Given $\alpha: \hom(-,\omega_X^{\circ})\xrightarrow{\sim}
 H^n(X,-)'$, we have
 \begin{align*}
  \alpha(\omega_X^{\circ}):
  \hom(\omega_X^{\circ},\omega_X^{\circ})&\to
  H^n(X,\omega_X^{\circ})'\\
  \id_{\omega_X^{\circ}} &\mapsto t
 \end{align*}
 So $\alpha$ gives us a $t:H^n(X,\omega_X^{\circ})\to k$.
 Conversely, given such a $t$, there is \emph{at most} one
 $\alpha$ inducing it because for all $\F$ and for all $\phi:\F\to
 \omega_X^{\circ}$, the diagram
 \[\xymatrix{
 \omega_X^{\circ}& & \hom(\omega_X^{\circ},\omega_X^{\circ})\ar[r]\ar[d]_{\hom(\phi,\omega_X^{\circ})}
 & H^n(X,\omega_X^{\circ})' \ar[d]^{H^n(X,\phi)} \\
 \F\ar[u]^{\phi}& & \hom(\F,\omega_X^{\circ}) \ar[r]^{\alpha(\F)} & H^n(X,\F)'
 }\]
 commutes.  So
 \[
 \alpha(\F)(\phi) =
 [H^n(X,\F)\xrightarrow{H^n(X,\phi)}H^n(X,\omega_X^{\circ})
 \xrightarrow{t} k]\in H^n(X,\F)' \tag{$\ast$}.
 \]
 If $\alpha$ exists, then it gives an isomorphism
 \[
 \alpha(\F):\hom(\F,\omega_X^{\circ})\xrightarrow{\sim} H^n(X,\F)'
 \]
 for all $\F$, and by $(\ast)$ it is the map associated with the
 pairing
 \[
    \hom(\F,\omega_X^{\circ})\times H^n(X,\F) \to
    H^n(X,\omega_X^{\circ})\xrightarrow{t} k
 \]
 and conversely\footnote{what?}.

 \begin{corollary} The dualizing sheaf, if it exists, is unique up
 to unique isomorphism.
 \end{corollary}

 \begin{lemma} Let $k$ be a field, $P=\P^N_k$, and let $X$ be a
 closed subscheme of $P$ of codimension $r$ (i.e. $r = \inf_{Z\subseteq X\
 irr} \codim Z$).  Then
 \[
    \Ext^i_P(\O_X,\omega_P)=0
 \]
 for all $i<r$.
 \end{lemma}
 \begin{proof}
 in the works
 \end{proof}
 Note that $X$ doesn't have to be equidimensional.
}
 { \stepcounter{lecture}
 \setcounter{lecture}{19}
 \sektion{Lecture 19}

Recall the lemma from last time: if $P=\P^N_k$, and $X\subseteq P$
has codimension $r$, then $\Ext^i_P(\O_X,\omega_P)=0$ for $i<r$.

\begin{lemma}\label{L:lec19} Let $k, N,P, X$ and $r$ be as before,
and let $\omega_X^{\circ}=\Ext^r_P(\O_X,\omega_P)$.  Then for all
$\O_X$-modules $\F$, there is a functorial isomorphism
 \[
    \hom(\F,\omega_X^{\circ})\cong \ext^r_P(\F,\omega_P).
 \]
 \end{lemma}
 \begin{proof}
 in the works
 \end{proof}

 \begin{theorem} Let $X$ be a projective scheme over a field $k$.
 Then $X$ has a dualizing sheaf.
 \end{theorem}
 \begin{proof}
 We may assume $X\not=\varnothing$.  Embed $X\hookrightarrow
 \P^N_k=P$, and let $n=\dim X$, $r=\codim_P X=N-n$.  Let
 $\omega_X^{\circ} = \Ext^r(\O_X,\omega_P)$.  Then for all
 $\O_X$-modules $\F$,
 \[
    \hom_X(\F,\omega_X^{\circ})\stackrel{\ref{L:lec19}}{\cong} \ext^r_P(\F,\omega_P)
    \stackrel{\ref{T:III.7.1}}{\cong} H^{N-r}(P,\F)' \cong H^n(X,\F)'
 \]
 functorially in $\F$.
 \end{proof}

 \begin{theorem}[(part of) Duality]
 Let $X$ be a non-empty projective scheme over a field $k$; let
 $\omega_X^{\circ}$ be a dualixing sheaf for $X$, and let $n=\dim
 X$.  Then for all $i\in \mathbb{N}$ and $\F\in
 \mathfrak{Coh}(X)$, there are natural maps
 \[
    \theta^i:\ext^i_X(\F\omega_X^{\circ})\to H^{n-i}(X,\F)'
 \]
 which for $i=0$ reduces to the isomorphism in the definition of
 the dualizing sheaf.
 \end{theorem}
 \begin{proof}
 Pick a projective embedding $X\subseteq P=\P^N_k$.  We have a
 surjection $\E\to \F\to 0$, where $\E$ is a finite direct sum
 $\bigoplus \O(-q)$ for $q\gg 0$ (Cor II.5.18).  Then
 \[
    \ext^i(\E,\omega_X^{\circ}) = \bigoplus
    \ext^i(\O(-q),\omega_X^{\circ}) = \bigoplus
    H^i(X,\omega_X^{\circ}(q))=0
 \]
 for all $i>0$, so $\Ext^i(-,\omega_X^{\circ})$ is coeffacable for
 $i>0$.  Also, $H^{n-i}(X,\F)'$ are contravariant
 $\delta$-functors, agreeing with
 $\ext^{\cdot}(-,\omega_X^{\circ})$ in degree 0.  Thus, but
 Theorem III.1.3A, there are maps of
 delta functors, as desired, and these are the $\theta^i$ we seek.
 \end{proof}
 \begin{remark}
 We didn't need $k$ to be algebraically closed.
 \end{remark}

 \begin{definition} A non-empty noetherian scheme $X$ is
 \emph{equidimensional} (of dimension $n$) if all of its
 irreducible components have the same dimension ($n$).
 \end{definition}

 \begin{definition}
 A scheme $X$ is \emph{Cohen-Macaulay} if all its local rings are
 Cohen-Macaulay.
 \end{definition}

 A finite dimensional local ring $(A,\m)$ is Cohen-Macaulay if
 depth$A=\dim A$.  Here \emph{depth} of $A$ is the maximal length
 of a regular sequence in $A$.  A \emph{regular sequence} is a
 sequence $x_1,x_2,\dots, x_n\in \m$ such that $x_i$ is not a
 zero divisor in $A/(x_1,\dots,x_{i-1})$ for all $i$.

 Facts about Cohen-Macaulay rings and schemes:
 \begin{itemize}
  \item[(1)] A regular scheme is Cohen-Macaulay (II.8.21Aa).
  \item[(2)] A locally complete intersection inside a regular
  scheme of finite type over a field is Cohen-Macaulay.
  \begin{proof}
   Let $Y\subseteq X$ be a locally complete intersection with $X$
   regular and of finite type over a field.  Let $P\in Y$.  After
   shrinking $X$, we may assume $Y$ is globally a complete
   intersection, cut out by $X_1,\dots, X_r$.  Thus, $\O_{Y,P} =
   \O_{X,P}/(x_1,\dots, x_r)$.  By II.8.21Ac, $x_1,\dots, x_r$ is
   a regular sequence in $\O_{X,P}$, so by II.8.21Ad, $\O_{Y,P}$
   is Cohen-Macaulay.

   To apply II.8.21Ad above, we need to check that
   \[
    r = \dim \O_{X,P} - \dim \O_{Y,P}.
   \]
   We always have the inequality $\ge$.  We may assume that
   $X=\spec A$, and $Y=\spec A/I$, in which case
   \begin{align*}
    \dim \O_{X,P} &= \height_A P \\
        &= \dim A - \depth_A P\\
   \end{align*}
   and
   \begin{align*}
    \dim \O_{Y,P} &= \height_{A/I} P \\
        &= \dim A/I - \depth_{A/I} P\\
        &= \dim A/I - \dim(A/P) & \text{(since $\I\subseteq P$)}
   \end{align*}
   \end{proof}
  \end{itemize}
}
 { \stepcounter{lecture}
 \setcounter{lecture}{20}
 \sektion{Lecture 20}

Last time we showed that if $Y$ is a locally complete intersection
in a regular (or Cohen-Macauley) scheme $X$, then $Y$ is
Cohen-Macaulay.

To find an example of a non-Cohen-Macaulay scheme, we look to
contradict facts we know about Cohen-Macaulay rings, like

\begin{theorem}[Eisenbud Cor 18.10] In a Cohen-Macaulay ring, all
associated primes are minimal. \end{theorem}

If $k$ is a field and $A=k[x,y]/(x^2,xy)$, then is has an embedded
point, so $\spec A$ is not Cohen-Macaulay.  To see this directly,
let $\m=(x,y)\subseteq A$.  Then $A_{\m}$ is not Cohen-Macaulay
since $\dim A_{\m}=1$, but $\depth A_{\m}=0$ since all elements of
the maximal ideal are zero divisors.

By II.8.21Ab, if $A$ is a local Cohen-Macaulay ring, then any
localization of $A$ at a prime ideal is also Cohen-Macaulay. Thus,
a scheme $X$ is Cohen-Macaulay if and only if all of its local
rings \emph{at closed points} are Cohen-Macaulay.

 \begin{remark}
 Let $Y$ be a locally complete intersection in a Cohen-Macaulay
 scheme $X$, and let $\I$ be its ideal sheaf.  Assume that $Y$ is
 equicodimensional (all irreducible components are of the same
 codimension, $r$).  Then $\I/\I^2$ is a locally free sheaf on $Y$
 of rank $r$.
 \begin{proof}
 We may assume $Y$ is globally a complete intersection in $X$, cut
 out by $x_1,\dots, x_r$, and that $X$ is affine, say $X=\spec A$.
  Then $\I=\tilde I$ for $I=(x_1,\dots, x_r)$.  We have that
  \[
    (A/I)^r\to I/I^2\quad,\quad (a_1,\dots,a_r)\mapsto \sum a_ix_i
  \]
  is well defined and onto.  We need to show that it is injective.

  Suppose not.  Then there is a prime $P\in A$ such that
  $(\text{kernel})_P \not =0$.  Clearly $I\subseteq P$.  Replace
  $A$ with $A_P$.  Then $A$ is Cohen-Macaulay and local, and
  $x_1,\dots,x_r$ is a regular sequence.  Let $(b_1,\dots, b_r)$
  be a nonzero element of the kernel in question.  We may assume
  $b_r\not\in I$.  Then $\sum b_ix_i \in I^2$, so $\sum b_ix_i =
  \sum c_ix_i$ with $c_i\in I$ for all $i$.  Then we have that
  $\sum (b_i-c_i)x_i = 0$.  Since $b_r\not\in I=(x_1,\dots, x_r)$,
  it is non-zero in $A/(x_1,\dots, x_{r-1})$.  Thus, we have that
  $x_r$ is a zero divisor (or 0) in $A/(x_1,\dots, x_{r-1})$.
  Contradiction.
  \end{proof}
  \end{remark}

  \begin{theorem}
  Let $X$ be a non-empty projective scheme of dimension $n$ over
  an algebraically closed field $k$, and let $\omega_X^{\circ}$ be
  a dualizing sheaf for $X$ (over $k$).  Then TFAE:
  \begin{itemize}
   \item[(i)] $X$ is equidimensional and Cohen-Macaulay
   \item[(ii)] for all locally free sheaves $\F$ on $X$,
   $H^{n-i}(X,\F(-q))=0$ for all $i>0$ and $q\gg 0$ (depending on
   $\F$).
   \item[(iii)] the maps $\theta^i:\ext^i(\F,\omega_X^{\circ})\to
   H^{n-i}(X,\F)'$ are isomorphisms for all $i$ and for all
   coherent sheaves $\F$.
  \end{itemize}
 \end{theorem}
 \begin{proof}
 in the works
 \end{proof}

 \begin{corollary}
 Let $X$ be an equidimensional Cohen-Macaulay projective scheme of
 dimension $n$ over an algebraically closed field $k$ (e.g. a
 non-singular variety of dimension $n$).  Let $\F$ be a locally
 free sheaf on $X$ and let $\omega_X^{\circ}$ be the dualizing
 sheaf.  Then there is a natural isomorphism for all $i$:
 \[
    H^i(X,\F) \cong H^{n-i}(X,\check \F \otimes
    \omega_X^{\circ})'.
 \]
 \end{corollary}
 \begin{proof}
 \[
    H^{n-i}(X,\F)' \cong \ext_X^i(\F,\omega_X^{\circ}) \cong
    \ext_X^i(\O_X,\check \F\otimes \omega_X^{\circ}) \cong
    H^i(X,\check \F\otimes \omega_X^{\circ}).
 \]
 \end{proof}

 \begin{remark}
 In proving Lemmas III.7.3 and III.7.4, we only used duality for
 $P$, so the proofs actually give
 \begin{theorem} Let $P$ be an equidimensional Cohen-Macaulay
 projective scheme of dimension $N$ over an algebraically closed
 field $k$, and let $X$ be a non-empty closed subscheme of $P$ of
 dimension $n$.  Then
 \[
    \omega_X^{\circ} = \Ext^{N-n}_P(\O_X,\omega_P^{\circ})
 \]
 \end{theorem}
 \end{remark}

 \begin{definition} Let $A$ be a ring, $f_1,\dots, f_r\in A$.
 Then the \emph{Kozul complex} of $A$ is the complex $K_{\cdot} =
 K_{\cdot}(f_1,\dots,f_r)$ defined by $K=$ free $A$-module of rank
 $r$, and $K_i:=\wedge^i K$, and $d:K_p\to K_{p-1}$ is given by
 \[
    e_{i_1}\wedge \cdots \wedge e_{i_p} \mapsto \sum_{j=1}^p
    (-1)^{j-1} f_{i_j} e_{i_1}\wedge\cdots \wedge \hat e_{i_j}
    \wedge \cdots \wedge e_{i_p}
 \]
 for all $i_1<\cdots < i_p$, where $\{e_1,\dots, e_r\}$ is the
 standard basis for $A^r$.  Note that $d^2=0$.  If $M$ is an
 $A$-module, then se define
 \[
    K_{\cdot}(f_1,\dots, f_r, M) = K_{\cdot}(f_1,\dots,
    f_r)\otimes_A M.
 \]
 \end{definition}
}
 { \stepcounter{lecture}
 \setcounter{lecture}{21}
 \sektion{Lecture 21}

Recall the previous definition.

\begin{proposition} If $f_1,\dots, f_r$ is a regular sequence for
an $A$-module $M$, then
 \[
 h_i(K_{\cdot}(f_1,\dots,f_r,M)) = \left\{\begin{array}{ll}
 M/(f_1,\dots,f_r)M & \text{if } i=0\\
 0 & \text{if } i>0
 \end{array} \right.
 \]
 \end{proposition}
 ($i=0$ is trivial ... you don't even need the regular sequence.)
 Thus, if $M$ is free, the Kozul complex is a free resolution of
 $M/(f_1,\dots, f_r)M$.

 \begin{theorem}
 Let $X$ be a locally complete intersection closed subscheme of
 $P=\P^N_k$, with ideal sheaf $\I$.  Then
 \[
     \omega_X^{\circ} = \omega_P \otimes
     \wedge^r(\I/\I^2)\check{}.
 \]
 In particular, the dualizing sheaf is a line sheaf on $X$.
 \end{theorem}
 \begin{proof}
 in the works
 \end{proof}

 \begin{remark} This works for any field.
 \end{remark}

 \underline{Next:} Compare $\omega_X^{\circ}$ with $\omega_X$ when
 $X$ is a non-singular variety.

 We prove half of a theorem from chapter II:
 \begin{theorem}[II.8.17] Let $X$ be a non-singular variety over
 an algebraically closed field $k$.  Let $Y\subseteq X$ be an
 irreducible closed subscheme, and let $\I$ be its sheaf of
 ideals.  Then $Y$ is non-singular if and only if
 \begin{itemize}
  \item[(1)] $\Omega_{X/Y}$ is locally free, and
  \item[(2)] the second exact sequence is a short exact sequence
  \[
    0\to \I/\I^2 \to \Omega_{X/k}\otimes \O_Y\to \Omega{Y/k} \to
    0.
  \]
  \end{itemize}
  In this case, $\I/\I^2$ is locally free (on $Y$) of rank
  $r=\codim Y$ and $Y$ is a locally complete intersection in $X$.
  \end{theorem}
  \begin{proof}[Half Proof]
  in the works
  \end{proof}
}
 { \stepcounter{lecture}
 \setcounter{lecture}{22}
 \sektion{Lecture 22}

\begin{corollary}[Adunction formula] In this situation,
 \[
 \omega_Y \cong \omega_X\otimes \wedge^r (\I/\I^2)\check{}
 \]
 \end{corollary}
 \begin{proof}
 By Exercise II.5.16d, $\wedge^n(\Omega_{X/k} \otimes
 \wedge^r(\I/\I^2)$.  The left hand side is $(\wedge^n
 \Omega_{X/k}) \otimes \O_Y = \omega_X\otimes \O_Y$.  Here $n=\dim
 X$ and $q=\dim Y$.  Also, $\wedge^q\Omega_{Y/k} = \omega_Y$, so
 \[
    \omega_Y \cong \omega_X\otimes \O_Y\otimes
    \wedge^r(\I/\I^2)\check{} \cong \omega_X \otimes
    \wedge^r(\I/\I^2)\check{}.
 \] \end{proof}

\begin{corollary} If $X$ is a non-singular projective variety, the
$\omega_X^{\circ} \cong \omega_X$. \end{corollary}
 \begin{proof}
 Embed $X$ into $P=\P^N_k$ and let $\I$ be the sheaf of ideals.
 Then
 \[
    \omega_X^{\circ} \cong \omega_P \otimes
    \wedge^r(\I/\I^2)\check{} \cong \omega_X.
 \]
 \end{proof}

 \begin{corollary}
 If $Y\subseteq X$ are non-singular varieties over an
 algebraically closed field $k$, and $\I$ is the ideal sheaf of
 $Y$, with $r=\codim Y$ then
 \[
    \omega_Y^{\circ} \cong \omega_X^{\circ} \otimes \wedge^r
    (\I/\I ^2)\check{}.
 \]
 \end{corollary}
 \begin{proof}
 $\omega_X^{\circ} \cong \omega_Y^{\circ}$, $\omega_Y^{\circ}
 \cong \omega_Y$.  Actually, if $X$ is a non-singular projective
 variety over $k$ ($k=\bar k$ for now) and $Y$ is a locally
 complete intersection closed subvariety with ideal sheaf $\I$.
 Then $\omega_Y\cong \omega_X\otimes \wedge^r(\I/\I^2)\check{}$.
 Same proof as before.
 \end{proof}

in the works

%\stepcounter{lecture} \setcounter{lecture}{23} \section*{Lecture
%23}
%
%\begin{theorem}[III.8.8] Let $f:X\to Y$ be a projective morphism
%of noetherian schemes, let $\O(1)$ be a very ample sheaf on $X$
%over $Y$, and let $\F$ be a coherent sheaf on $X$.  Then
% \begin{itemize}
%  \item[(a)] The natural map $f^*f_*\F(n)\to \F(n)$ is surjective
%  for $n\gg 0$
%  \item[(b)] $R^if_*\F$ is coherent for all $i$
%  \item[(c)] $r^if_*\F(n) = 0$ for all $i>0$ and $n\gg 0$.
% \end{itemize}
%\end{theorem}
% \begin{proof}
% The question is local on $Y$ (since $Y$ is quasi-compact), so we
% may assume that $Y$ is affine, say $Y=\spec A$.
%
% (a)
% \end{proof}
}
 { \stepcounter{lecture}
 \setcounter{lecture}{23}
 \sektion{Lecture 23}

\textbf{Homework comments:}\\

1st problem: Ampleness. Given a coherent sheaf $\F$ on a
Noetherian scheme $X$, Supp $\F$ is defined as a \textbf{set}, and
is closed. Then $\F$ can be viewed as a sheaf on $Y$, but it is
not a sheaf of $\O_Y$ modules (therefore not coherent).

Example: $X$ = $\spec{\Z}$, $\F = \widetilde{\Z/4\Z}$. Then $Y = $
Supp $\F$ = $\{2\}$, but $\F$ is not a sheaf of $O_Y$ modules if
we take $Y = \spec{\Z/2\Z}$ (since $\Z/4\Z$ is not a module over
$\mathbb{F}_2$. We have to take a different subscheme structure on
$O_Y$ to get this to work out, and Vojta claims that this can
always
be done. See his solution to (III ex.4.2).\\

Lemma (III 2.10) is better than (III ex.4.1).\\

With induction proof in part (d), need to be sure that the
reduction to $X,Y$ integral is "inside" the induction. Why?
Because a proper closed subscheme of an irreducible scheme might
not be irreducible
(you know, take two points).\\

The proof of (c) was indeed similar to (b), so instead of just
saying that, the proper thing to do is to generalize!!!

\begin{lemma} Let $\I$ be a coherent sheaf of ideals on $X$ and
$\F$ a coherent sheaf on $X$. Let $Y$ be the closed subscheme
corresponding to $\I$. If $H^i(X,\I\F)=0$ and $H^i(Y,\F/\I\F)=0$
(actually, the correspondinig sheaf on $Y$ for some $i$, then
$H^i(X,\F)=0$. \end{lemma} Then phrase (b) and (c) in terms of the
appropriate
coherent ideal sheaves.\\

2nd problem: If $I$ is an injective $A$-module,then
$\widetilde{I}$ is injective in $\mathscr{Q}co$, but not
necessarily in $\mathscr{M}od(X)$.

Generally, if $\mathscr{C}$ is a full abelian subcategory of
$\mathscr{D}$, and if $X \in \mathscr{C}$ is injective in
$\mathscr{D}$, then it is injective also in $\mathscr{C}$. But not
necessairly conversely.\\



 \textbf{Back to III.8}.
 \begin{theorem} Let $f:X\rightarrow
Y$ be a projective morphism of Noetherian schemes, let $\O(1)$ be
a very ample sheaf on $X$ over $Y$, and let $\F$ be a coherent
sheaf on $X$. Then \begin{itemize} \item[(a)] The natural map
$f^*f_*\F(n)\rightarrow\F$ is surjective for $n>>0$. \item[(b)]
$R^if_*\F$ is coherent for every $i$. \item[(c)] $R^if_*\F(n) = 0$
for $i > 0$ and $n >> 0$. \end{itemize}
 \end{theorem}
\begin{proof} The question is local on $Y$ - as $Y$ is quasi
compact (it is Noetherian!) we can find an $n$ for each part of a
finite affine cover, and then take the largest $n$ - and we can
assume that $Y$ is
affine, say $Y = \spec{A}$.\\

(a) By (III 8.5), $\text{R}^0f_*\F(n) = f_*\F(n) =
H^0(X,\F(n))^{\widetilde{}_Y} = M^{\widetilde{}_Y}$. Then for any
open affine $\spec{B}$ in $X$, we have by (II prop.5.2) that
$f^*f_*\F(n) = \widetilde{M\otimes_AB}$ on $X$.

So our map $f^*f_*\F(n)\rightarrow\F(n)$ is defined (at the global
section level) by \[ \xymatrix{ M\otimes_AB \ar[r] &M
\ar[rr]^{\alpha} &&
\Gamma(\spec{B},\F(n))\\
m\otimes b \ar[r]&mb.&& } \]

By prop (II 5.17), for large enough $n$ the sheaf is generated by
global sections, so the map is surjective at the stalk level, and
thus surjective.\\

(b) By (III 8.5),  $\text{R}^if_*\F(n) =
H^i(X,\F(n))^{\widetilde{}_Y}$ is coherent because $H^i(X,\F(n))$
is finitely generated (III 5.2(a))for all $i$
and $n$.\\

(c) By (III 5.2(b)), $\text{R}^if_*\F(n) =
H^i(X,\F(n))^{\widetilde{}_Y} = 0$ for $i > 0$ and large enough
$n$. \end{proof}

\marginpar{\S III.9: Flat Morphisms}

\begin{definition}Let $M$ be an $A$-module. We say that $M$ is
flat (over $A$) if the functor $M\otimes_A -$ is exact, i.e. \[ 0
\rightarrow N' \rightarrow N \rightarrow N'' \rightarrow 0 \]
exact implies \[ 0 \rightarrow M\otimes_AN' \rightarrow
M\otimes_AN\rightarrow M\otimes_AN''\rightarrow0 \] exact.
\end{definition}
 Also, $M$ is faithfully flat if the converse of the above is also true.

\underline{Examples:}\begin{itemize}
 \item A free module is flat,a nd is
faithfully flat iff it's non-zero. In particular, if $A$ is a
field, then everything is flat.
 \item $\Z/n\Z$ is not flat over $\Z$.
Indeed, let $p|n$ and tensor \[ 0 \rightarrow \Z \rightarrow^p \Z
\rightarrow \Z/p\Z \rightarrow 0 \] with $\Z/n\Z$ to get \[ \Z/n\Z
\rightarrow^p \Z/n\Z \rightarrow ? \rightarrow 0 \] \end{itemize}

\underline{Example:} $S^{-1}A$ is flat over $A$ for any
multiplicative system $S$ (i.e. 'localization is exact').

\textbf{Properties of Flatness} \begin{theorem} Let $B$ be an
$A$-algebra, and let $M,N$ be $A$ and $B$ modules, resp. Then
\begin{itemize} \item[(a)]$M$ is flat over $A$ iff the map
$M\otimes_A \mathfrak{a} \rightarrow M$ is injective for every
finitely ideal $\mathfrak{a}$ $A$. \item[(b)] If $M$ is flat over
$A$, then $M\otimes_aB$ is flat over $B$. \item[(c)] Transitivity:
if $N$ is flat over $B$ and $B$ is flat over $A$, thhen $N$ is
flat over $A$. \item[(d)] $M$ is flat over $A$ iff $M_p$ is flat
over $A_p$ for all $p \in \spec{A}$ \item[(e)]Let $0\rightarrow M'
\rightarrow M \rightarrow M'' \rightarrow 0$ be an exact seq of
$A$ modules. Then $M'$ and $M''$ flat imply $M$ flat. $M$ and
$M''$ flat imply $M'$ flat. \item[(f)] If $M$ if a finitely
generated module over a Noetherian local ring $A$, then $M$ is
flat iff its free. \end{itemize} \end{theorem} \begin{proof} (a)
"$M\otimes_A -$" is a right exact functor, and $\mathscr{M}od(X)$
has enough projectives (in fact, every module has a free
resolution). The right derived functors of $M\otimes_A-$ are
called $\text{Tor}_i^A(M,-)$. It is known that $M$ is flat over
$A$ iff $\text{Tor}_i^A(M,N)$ = 0 for all $i>0$ and $N$, iff
$\text{Tor}_i^A(M,\mathfrak{a})$ = 0 for all $i>0$ and finitely
generated ideals $\mathfrak{a}$ of $A$.

(b,c) $(M$ flat over $A \Rightarrow M \otimes_AB$ flat over
$B$)$\Leftarrow N\otimes_B(M\otimes_AB)) \cong N\otimes_AM$.

(d) True because Tor commutes with localization.

(e) Comes from the left exact sequence in Tor: \[\xymatrix{\cdots
\ar[r] & \text{Tor}_1^A(M',N) \ar[r] & \text{Tor}_1^A(M,N) \ar[r]
&\text{Tor}_1^A(M'',N) \ar[r] & \\
 \ar[r] & M'\otimes N \ar[r] & M\otimes N\ar[r]
&M''\otimes N \ar[r] & 0 } \]

We see that if $\text{Tor}_i^A(M',N) = \text{Tor}_i^A(M'',N)= 0$
(i.e. $M'$ and $M''$ are flat), then $\text{Tor}_i^A(M,N) = 0$ and
$M$ is flat too (similarly if $M$ and $M''$ are flat, $M'$ is
also).

(f) \end{proof}

\begin{definition} Let $f:X\rightarrow Y$ be a morphism of schemes
and let $\F$ be a sheaf of $\O_X$-modules. For $x \in X$, we say
that $\F$ is flat over $Y$ at $x$ if $\F_x$ is a flat
$\O_{Y,f(x)}$ module (via the map $f^{\#}: \O_{Y ,f(x)}
\rightarrow \O_{X,x}$. We say that $\F$ is flat over $Y$ if it is
flat over $Y$ at $x$ for all $x \in X$.

We say that $X$ is flat over $Y$ if $\O_X$ is flat over $Y$. We
say that $f$ is flat if $X$ is flat over $Y$. \end{definition}

\begin{proposition}
 \begin{itemize} \item[(a)] An open immersion
is flat. \item[(b)] Flatness is preserved under base change.
\item[(c)] Composition of flat morphisms is flat. \item[(d)] A
morphism $f:X \rightarrow Y$ is flat iff for all open affines,
$\spec{A} \subset Y$ and for all $\spec{B} \subset
f^{-1}(\spec{A})$, $B$ is flat over $A$ (i.e. flatness is local).
\item[(e)] If $X$ is noetherian and $\F$ is coherent, then $\F$ is
flat over $X$ iff it is locally free. \end{itemize}
\end{proposition} \begin{proof} (a) The stalks are the same.

(b)-(e) follow from the algebraic properties above.

Also, flatness is local on the base (because its local upstairs).
If $f:X\rightarrow Y$ and $f':X'\rightarrow Y'$ are flat morphisms
over S, then the product is flat. Indeed, see Vojta's solution of
II ex.4.8 - (a) is not needed for (d). \end{proof}

\underline{Examples:}\begin{itemize}
 \item  Closed immersion are
generall not flat: $\spec{\Z/p\Z} \rightarrow \spec{Z}$ is not
flat. Indeed, by (d) above, it is flat iff $\Z/pZ$ is flat over
$\Z$.
 \item Blowing ups are generally not flat. I'm not going to try to tex
this. Just look at the simplest blow up possible. \end{itemize}
}
 { \stepcounter{lecture}
 \setcounter{lecture}{24}
 \sektion{Lecture 24}


Addendum: why part (b) of flatness lemma is true - $ (M$ flat over
$A \Rightarrow M \otimes_AB$ flat over $B$)$\Leftarrow
N\otimes_B(M\otimes_AB)) \cong N\otimes_AM$.



Cohomology commutes with flat base change:

\begin{proposition} Let

\[
\xymatrix{X' \ar[d]_g\ar[r]^v&X\ar[d]^f\\
Y'\ar[r]^u&Y } \] Be a cartesian square of noetherian schemes,
with $f$ separated and of finite type, and $u$ flat. Also, let
$\F$ be a quasi-coherent sheaf on $X$. Then there is a natural
isomorphism (as sheaves on $Y'$) \[
u^*R^if_*(\F)\cong\text{R}^ig_*(v^*\F) \] \end{proposition}
 \begin{proof} By locality and naturalness, we
may assume $Y$ and $Y'$ are affine, say $Y = \spec{A}$ and $Y'
=\spec{A'}$. Then \[ R^if_*(\F) \cong
\text{H}^i(X,\F)^{\widetilde{}_Y} \] \[u^*R^if_*(\F) \cong
(\text{H}^i(X,\F ))^{\widetilde{}_{Y'}}\otimes_A A' \cong
(H^i(X,\F)\otimes_A A')^{\widetilde{}_{Y'}} \] and \[
\text{R}^ig_*(v^*\F) = \text{H}^i(X',v^*\F)^{\widetilde{}_{Y'}} \]
So we need a natural isomorphism \[ H^i(X,\F)\otimes_AA' \cong
H^i(X',v^*\F) \] Use Cech cohomology: let $\U$ be an open affine
cover of $X$. Say $\U = (U_i)_{i \in I}$ with $U_i = \spec{B_i}$.
Then let $\U' = v^{-1}\U = \U = (v^{-1}(U_i))_{i \in I} =
(\spec{B_i\otimes_A A'})_{i \in I}$. This is an open affine cover
of $X'$.

Then we have \[ H^i(X, \F)\otimes_AA' \cong
h^i(\C^{\cdot}(\U,\F))\otimes_AA' \cong_{flatness}
h^i(\C^{\cdot}(\U,\F)\otimes_AA') = h^i(\C^{\cdot}(\U,v^*F)) =
H^i(X',v^*\F) \] \end{proof}

\textbf{Flat Families} We want to exclude things like blow ups,
where the dimension of fibres is not constant. For flat morphisms,
this cannot happen. \begin{proposition} Let $f:X\rightarrow Y$ be
a flat morphism of schemes of finite type over a field $k$. Let $x
\in X$, let $y = f(x)$, and let $X_y$ be the fibre of $f$ over
$y$. Then $\text{dim}_x X_y$ = $\text{dim}_x X$ - $\text{dim}_y
Y$, where $\text{dim}_x X = dim \O_{X,x}$ \end{proposition}
\begin{proof} \textbf{Step 1}: Reduce to the situation where $Y$
is affine and there is a unique closed point, and $y$ is that
point. \emph{(say a local ring)}. Let $Y' = \spec{\O_{Y,y}}$, and
let $X' = X \times_Y Y'$. Then $X_y$ is unchanged, and so are all
the local rings involved. (Now $X$ and $Y$ are covered by open
affines which are
localizations of $k$-algebras of finite type.\\

\textbf{Step 2}: Reduce to $Y$ reduced. Base change to
$Y_{\text{red}}$. All of the local rings are replaced by quotients
by ideals contained in the nilradical. So essentially, nothing is
changed.\\

\textbf{Step 3}: The rest. Prove by induction on $\text{dim } Y$.
Note that $\text{dim } Y$ = $\text{dim}_y Y$.

\textbf{Base case}: If $\text{dim } Y$ = 0, then $Y$ has a unique
minimal prime, which is also maximal.  Thus $Y$ is spec of an
artin ring (it has a unique maximal ideal, which is also minimal),
and since $Y$ is reduced \emph{(this excludes products of
fields)}, $Y = \spec{E}$ for some field $E$. But then $X_y$ = $X$,
so we're done.

\textbf{Inductive Step}: If $\text{dim } Y > 0$, then the maximal
ideal of $\O_{Y,y}$ contains a non-zero element $t$, which is not
a zero divisor. (Later). Thus, \[ 0 \rightarrow \O_{Y,y}
\rightarrow^t \O_{Y,y} \] is exact. By flatness(?), \[ 0
\rightarrow \O_{X,x} \rightarrow^{f^{\#}t} \O_{X,x} \] is exact.
Let $Y' = \spec{\O_{Y,y}/(t)}$. By the Hauptidealsatz and the fact
that $\O_{Y,y}$ is catenary (?) \[ \text{dim }_y Y' = \text{dim
}_y Y -  1 \] Let $X' = X \times_Y Y'$. Then
$\O_{X',x}=\O_{X,x}(f^{\#}t)$. By flatness, $f^{\#}t$ is not a
zero divisor (or 0), so $\text{dim }_x X' = \text{dim }_x X - 1$.

Finally, $X'_y = X_y$, since $t \in \mathfrak{m}_y$. So we're done
by induction. \end{proof}

\begin{corollary} Let $f:X\rightarrow Y$ be as in the statement of
the proposition. Assume also that $Y$ is irreducible. The the
conditions \begin{itemize} \item[(i)] $X$ is equidimensional of
dimension $\text{dim} Y + n$ \item[(ii)] $X_y$ is equidimensional
of dimension  $n$ for all $y \in Y$ (closed or not). \end{itemize}
are equivalent. \end{corollary} \begin{proof} $(i) \rightarrow
(ii)$. Pick $y \in Y$, let $Z$ be an irreducible component of
$X_y$, and let $x \in Z$ be a closed point not lying in any other
irreducible component. By the proposition, \[\text{dim}_x Z =
\text{dim} Z = \text{dim}_x X_y = \text{dim}_x X - \text{dim}_y
Y\] Also, \[ \text{dim}_x X =  \text{dim} X - \text{dim}
\overline{\{x\}} \] and likewise \[ \text{dim}_y Y =  \text{dim} Y
- \text{dim} \overline{\{y\}}. \] But now, since $x$ is a closed
point of $X_y$, $k(x)$ is a finite (and therefore algebraic)
extension of $k(y)$. Thus, \[\text{dim} \overline{\{x\}} =
\text{tr. deg}(k(x)/k) = \text{tr. deg}(k(y)/k) = \text{dim}
\overline{\{y\}}\] Therefore the right hand side is
$\text{dim}X - \text{dim} Y$, which by (i) is $n = \text{dim} Z$.\\

$(ii) \rightarrow (i)$: Let $Z$ be an irreducible component of
$X$, let $x \in Z$ be a closed point not lying in any other
irreducible component, and let $y = f(x)$. Then $y$ is a closed
point of $Y$, because $k \subset k(y) \subset k(x)$ (which are
algebraic extensions). By the proposition (and assumption (ii)),
\[ n = \text{dim}_x X_y = \text{dim}_x X - \text{dim}_y Y =
\text{dim}_x Z - \text{dim} Y \] \end{proof}

\textbf{Associated Primes}: Assume Noetherian throughout.
\begin{definition} Let $A$ be a ring. An associated prime of $A$
is a prime ideal equal to the annihilator of some element of $A$.
\emph{Not every element of $A$ has its annihilator equal to some
prime. You can even do this more generally for modules, just take
associated primes to be the annihilators of elements of $M$.}
\end{definition} \begin{theorem}(Eisennbud Thm 3.1): Let $A$ be a
noetherian ring. Then \begin{itemize} \item[(a)] it has only
finitely many associated primes \item[(b)] the union of the
associated primes is the set of zero divisors of $A (\cup \{0\})$
\end{itemize} \end{theorem} \begin{remark} All minimal primes of
$A$ are also associated primes. And these associated primes
localize nicely: If $S$ is a multiplicative subset of $A$, then a
prime of $S^{-1}A$ is an associated prime iff the corrsponding
prime of $A$ is an associated prime. In other words: \[
\text{Ass}(S^{-1}A) ``=" (\text{Ass} A)\cup \spec{S^{-1}A} \]
\end{remark} \begin{definition} Let $X$ be a (locally noetherian)
scheme. Then an associated point of $X$ is a point $x \in X$ such
that the maximal ideal of $\O_{X,x}$ is an associated prime of
$\O_{X,x}$.

If $A$ is noetherian, then the set of associated points of
$\spec{A}$ corresponds to the set of associated primes of $A$ (by
this comment about localization). \end{definition}

\emph{Can almost fill in the `later' above, but need one more
thing.
Okay, several more things.}\\

\begin{proposition}\textbf{Primary Decomposition}: If $A$ is a
noetherian rind, then we can write $(0) = \mathfrak{q_1} \cap
\cdots \cap \mathfrak{q_r}$, where the $\mathfrak{q_i}$ are
primary ideals (\emph{this is more natural using Eisenbud's
definition, which we don't know offhand}). A primary ideal
satisfies (for $xy \in \mathfrak{q_i}, x \not \in \mathfrak{q_i}
\Rightarrow y^n \in \mathfrak{q_i}$ for some $n$).
\end{proposition} \begin{remark} Note that if $\mathfrak{q}$ is
primary, then $\sqrt{\mathfrak{q}}$ is prime. If $\mathfrak{q_1}
\cap \cdots \cap \mathfrak{q_r} = (0)$ is a minimal primary
decomposition ($r$ minimal), then $\sqrt{\mathfrak{q_i}}$ are
distinct and are exactly the set of associated primes of $A$.
\end{remark}

Primary decompositions localize well: \begin{definition} An
\textbf{embedded prime} in a noetherian ring $A$ is a non-minimal
associated prime. An \textbf{embedded point} in a noetherian
scheme is an associated point that is not the generic point of an
irrreducible component. \end{definition}

\begin{proposition} A reduced noetherian scheme $X$ has no
embedded points. \end{proposition} \begin{proof} This is a local
question, so we may assume that $X = \spec{A}$ is affine (where by
assumption $A$ is reduced).

Then $0 = \text{nil}(A) = $ intersection of the minimal primes of
$A$. \emph{In fact it is the intersection of all primes of $A$,
but if a prime is not minimal its not needed in the intersection.}
This gives a primary decomposition of $0$ in $A$. So all
associated primes are minimal. \end{proof}

\emph{The union of the minimal primes in not the maximal ideal
because its of dimension greater than one, and by prime something,
if you have the maximal ideal, and you have some primes that are
strictly contained in that, then their union is not the whole
maximal ideal because, its an exercise. But its basically prime
something... ``later"...}\\

Actually, a little more is true:

\begin{remark} (Eisenbud Ex. 11.10) A noetherian ring is reduced
iff it has no embedded primes and its localization at each minimal
prime is a field.

Intuition: Non-zero nilpotents in a ring $A$ correspond to 'fuzz'
in $\spec{A}$. 'Fuzz' on a dense open subset of an irreducible
component occurs iff the localization at the corresponding minimal
prime is not a field. Embedded points correspond to fuzz nt spread
over an irreducible component. Ex.
$\spec{k[x,y]/(xy,y^2)}$\end{remark}

\emph{This embedded stuff might be useful for the homework...}\\

\emph{You'd like to represent a number theory problem as a problem
over $\spec{\Z}$ or spec of a ring of integers in a number field.}\\

\begin{proposition} Let $f:X\rightarrow Y$ be a morphism of
schemes. Assume that $Y$ is integral and regular of dim 1. Then
$f$ is flat iff all associated points of $X$ lie over the generic
point of $Y$. \end{proposition} \begin{proof} ``$\Rightarrow$" Let
$x \in X$ be a point lying over a closed point $y \in Y$. We want
to show that $x$ is not an associated point. Then $\O_{Y,x}$ is a
discrete valuation ring, so its maximal ideal contains a nonzero
element $t$, which is a nonzerodivisor since $Y$ is integral. By
flatness, $f^{\#}t$ is not a zero divisor in
$\O_{X,x}$, so $x$ is not an associated point of $X$.\\

\emph{Because if it was that would mean that the maximal ideal
would be an associated prime. Multiplication by $t$ is injective
in $\O_{Y,y}$, so it is in $\O_{X,x}$ too}.

\emph{Remark: We didn't need $Y$ to be regular, or even dimension
1. We just needed $Y$ to be integral of dimension greater than
zero.} \end{proof}
}
 { \stepcounter{lecture}
 \setcounter{lecture}{25}
 \sektion{Lecture 25}

[[The last proposition of the last lecture was missing some
hypothesis -  like noetherian.]]

\begin{proposition} Let $f:X\rightarrow Y$ be a morphism of
noetherian schemes. Assume that $Y$ is integral and regular of dim
1. Then $f$ is flat iff all associated points of $X$ lie over the
generic point of $Y$.

In particular, if $X$ is reduced, then $f$ is flat iff all
irreducible components of $X$ dominate $Y$ \end{proposition}
\begin{proof}
`$\Rightarrow$' last time.\\

`$\Leftarrow$' Let $x \in X$, and let $y = f(x)$. We need to show
that the local rings are flat, i.e. $\O_{x,X}$ is flat over
$\O_{y,Y}$.

\textbf{Claim}. A module $M$ over a PID $R$ is flat iff it is
torsion free. \\

\textbf{Proof} `$\Rightarrow$' Suppose $M$ has torsion, say $t =
0$ with $x \in R$, $m \in M$, both $\neq 0$. Then $M$ is not flat
because multiplication by $x$ is an injection from $R$ to $R$
(since $R$ is a domain and $x$ is non-zero, but it aquires a
kernel after
tensoring wiht $M$.\\

`$\Leftarrow$' We need to show that $\mathfrak{a} \otimes M
\rightarrow M$ is injective for all ideals $\mathfrak{a}$ [[which
are all finitely generated since $A$ is a PID]]. We may assume
$\mathfrak{a} \neq (0)$. Then $\mathfrak{a} = (x)$ for some $x \in
R$, and so $\mathfrak{a}\otimes M \cong M$, and the map is
multiplication by $x$ on $M$. That's injective because $M$ is
torsion free. $\Box$\\


\textbf{Case 1}. If $y$ is the generic point then $\O_{Y,y}$ is a
field so there's nothing to show.\\

\textbf{Case 2}. here, $y$ is a closed point. Then $\O_{Y,y}$ is a
d.v.r. Suppose $\O_{X,x}$ is not flat over $\O_{Y,y}$. Then it has
torsion: there exists some nonzero $t \in \O_{Y,y}$ s.t.
multiplication by $f^{\#}t$ is not injective in $\O_{X,x}$. Then
$f^{\#}t$ is a zero divisor, so it lies in some associated prime
$\mathfrak{p}$ of $\O_{X,x}$. This gives an associated point $x'
\in X$. But $f(x') = y$ since [[as there are only 2 prime ideals
in a PID]] $\mathfrak{p} \cap \O_{Y,y}$ contains $t \neq 0$.
Therefore $\mathfrak{p}\cap\O_{Y,y} \neq (0)$, so $f(x') \neq $
the generic point, contradicting our assumption that all
associated points lie
over the generic point.\\

\textbf{Example}. $\mathbb{Q}$ is flat over $\Z$, therefore it is
torsion
free. But it's not free.\\
\end{proof}

\textbf{Hard Work - Scheme Theoretic Closure, Schematic Denseness,
and Associated Points}.\\

\underline{Recall (II ex.3.11(d)):} Let $f:Z \rightarrow X$ be a
morphism with $Z$ noetherian. Then there exists a unique closed
subscheme $Y$ of $X$ such that $f$ factors through $Y$, and such
that if $f$ factors through any other closed subscheme $Y'$, then
$Y \subset Y'$ schematically ($\mathscr{I} \supset \mathscr{I}'$).
This is the \textbf{closed scheme theoretic image of $f$}.

\begin{definition} Let $i:Z \rightarrow X$ be a subscheme of a
noetherian scheme $X$. Then the \textbf{scheme theoretic closure}
of $Z$ is the closed scheme-theoretic image of $i$.
\end{definition}

\begin{definition} $Z$ is \textbf{scheme-theoretically dense} if
its scheme-theoretic closure is all of $X$ (as a scheme). An open
set is scheme theoretically dense if the corresponding open
subscheme is. \end{definition}

\underline{Example} $X = \spec{k[x,y]/(x^2,xy)}$ (i.e. the
$y$-axis with an embedded point at (0,0)). Let $U = X -
\{(0,0)\}$. Let $A = k[x,y]/(x^2,xy)$. Then $U = \spec{A_y}$. But
$A_y = (k[x,y]/(x))_y \cong k[y]_y$. Since $U$ is reduced, its
scheme theoretic closure is $X_{\text{red}} = \spec{k[y]} $, so
$U$ is not schematically dense. (Note that $U$ is dense
set-theoretically). (You can compute the scheme-theroetic closure
of $\spec{B}\rightarrow\spec{A}$ by finding $\ker(A\rightarrow
B)$).


\begin{proposition} An open subset of a noetherian scheme is
chematicallly dense iff it contains all of the associated points.
\end{proposition}

\begin{proof} This is a local question, so we may assume $X =
\spec{A}$ is affine. Let $U$ be an open subscheme [[equivalently
an open subset]]. Lets let $\mathfrak{q}_1 \cap \cdots \cap
\mathfrak{q}_n$ be a minimal primary decomposition of $(0)$ in
$A$. [[This is where we use the fact that $A$ is noetherian.]] Let
$\mathfrak{p}_i = \sqrt{\mathfrak{q}_i}$, so
$\mathfrak{p}_1,\cdots,\mathfrak{p}_n$
are the associated primes of $A$.\\

`$\Leftarrow$'. Say $U$ contains
$\mathfrak{p}_1,\cdots,\mathfrak{p}_n$, and let $\mathfrak{a}$ be
the ideal associated to the schematic closure $\overline{U}$ of
$U$. Since $\mathfrak{p}_i \in U$, and $U \subset \overline{U}$,
we have $(A/\mathfrak{a})_{\mathfrak{p}_i/\mathfrak{a}} \cong
A_{\mathfrak{p}_i}$. Therefore, for any $f \in \mathfrak{a}$, $f =
0$ in $A _{\mathfrak{p}_i}$, so $\text{Ann}(f)$ meets
$A/\mathfrak{p}_i$. By prime avoidance [an actual term in
Eisenbud's book, or just by general messing around], there is some
$x \in \text{Ann}(f)$ such that $x \not \in \mathfrak{p}_1 \cap
\cdots \cap \mathfrak{p}_n$. Therefore $x$ is not a zero divisor
(and $x \neq 0$), so $f = 0$. Therefore $\mathfrak{a} = (0)$, so
$\overline{U} = X$ and therefore $U$ is schematically dense.\\

`$\Rightarrow$' Suppose its false. Then $U$ is schematically dense
but without loss of generality $\mathfrak{p}_n \not \in U$. We may
assume $\{i : \mathfrak{p}_i \not \supset \mathfrak{p}_n\}$ =
$\{1,\cdots,r\}$; $r < n$. Let $\mathfrak{a} = \mathfrak{q}_1 \cap
\cdots \cap \mathfrak{q}_r$. By minimality, $\mathfrak{a} \neq
(0)$. So it will suffice to show that $U \subset
\spec{(A/\mathfrak{a})}$.

Let $\p \in U$. Since $\p_n \not \in U$, $\p_n \not \rightarrow
\p$ [make this arrow squiggly], so $\p_n \not \subset \p$.
Therefore $\p \not \supset \p_i$, for all $i > r$. Let $S = A /
\p$; then $S$ meets $\p_i$ for all $i > r$, so $S^{-1}\q_i = (1)$
for all $i > r$. Therefore $(0) = \bigcap_{i=1}^nS^{-1} \q_i =
\bigcap _{i = 1}^r S^{-1}\q_i = S^{-1}\mathfrak{a}$, so
$S^{-1}\mathfrak{a} = (0)$. Therefore $\p \in
\spec{(A/\mathfrak{a})}$. ($f \in \mathfrak{a} \Rightarrow
\text{Ann}(f)$ meets $S \Rightarrow f \in \p$ primeness). also
$(A/\mathfrak{a})_{\p/\mathfrak{a}} \cong A_{\p}$, since LHS $ =
A_{\p}/\mathfrak{a}/{\p} = A_{\p}/(0)$. So $U \subset
\spec{(A/\mathfrak{a})}$ as schemes [[here he drew a big frowny
face with x'ed out eyes]].\end{proof}

\begin{proposition}

Lelt $Y$ be an integral, regular, noetherian scheme of dimension
1, let $P \in Y$ be a closed point, and let $X \subset
\mathbb{P}^n_{Y/P}$ be a closed subscheme which is flat over
$y/P$. Then there exists a unique closed sobscheme $\overline{X}$
of $\mathbb{P}^n_{Y}$ such that $\overline{X} \cap
\mathbb{P}^n_{Y/P} = X$ and $\overline{X}$ is flat over $Y$.
\end{proposition}

\begin{proof} \textbf{Existence} Let $\overline{X}$ be the
schematic closure of $X$ in $\mathbb{P}^n_{Y}$. We didn't add any
associated points, so
its still flat.\\

\textbf{Uniqueness}. Suppose $X'$ also saatisfies the condition.
We have to have $X' \subset_{\text{subschemes}}\overline{X}$ (by
definition of schematic closure). since $\overline{X}$ is not
schematically dense in $X'$, $X'$ contains some associated point
not in $\overline{X}$. Then that point lies over $P$ (because
where else could it lie), contradicting flatness. \end{proof}

[[We didn't really need the fact that we were in
$\mathbb{P}^n_{Y}$. We just wanted to be concrete, and in
applications we only use $\mathbb{P}^n_{Y}$. In practice you could
use any noetherian scheme in place of $\mathbb{P}^n_{Y}$.]]

The last application has to do with bases of arbitrary dimension.

\begin{theorem} Let $T$ be an integral noetherian scheme, and let
$X \subset \mathbb{P}^n_{T}$ be a closed subscheme. Then for each
point $t \in T$ (closed or not), lets let $P_t \in \mathbb{Q}[z]$
be the Hilbert polynomial of the fiber $X_t = X \times_T k(t)$
(which is a closed subscheme of $\mathbb{P}^n_{k(t)}$ - calculate
the Hilbert polynomial by calculating dimensions of things over
$k(t)$).

Then $X$ is flat over $T$ iff $P_t$ is independent of $t$.
\end{theorem}

\begin{proof} Recall the definition of Hilbert Polynomial: $P_t$
is characterized by $P_t(m) = \text{dim}_{k(t)} H^o(X_t,\O(m))$
for all $m \gg 0$. Also, $k(t)$ is not necessarily closed, but by
a flat base change, we can pass to the algebraic closure.

We will prove this theorem by first generalizing it: More
generally let $\F$ be a coherent sheaf on $X$. Then we can write
$\F_t = \F|_{X_t} = i^*\F$, where $i:X_t \rightarrow X$ is the
inclusion map. So we have a Hilbert polynomial of $\F_t$" \[
P_t(m) = h^0(X_t,\F_t(m)), m \gg 0.\] So we may assume $X =
\mathbb{P}^n_{T}$ (let $\F = \O_{\text{old } X}$ [[literally waved
hands and said `usual trick']]).

Also, we may assume $T = \spec{A}$, with $A$ a local noetherian
ring.\\

So the situation is $T = \spec{A}$ as above, $X =
\mathbb{P}^n_{A}$,
and $\F$ is coherent on $X$. [[Now we prove some claims]].\\

\textbf{Claim 1}. $\F$ is flat over $T$ iff $H^0(X,\F(m))$ if a
free $A$ module (of finite rank - this part always holds) for $m
\gg
0$.\\

\textbf{Proof}. `$\Rightarrow$'. Let $\mathscr{U}$ be the standard
open affine cover of $\mathbb{P}^n_{A}$ [[offhand I can't think of
why it has to be the standard one, but why not?]]. Then
$H^i(\U,\F(m)) = H^i(X,\F(m)) = 0$ for all $m \gg 0$ So the
sequence (imagine the kernels and cokernels) \[ 0 \rightarrow
H^0(X, \F(m)) \rightarrow C^0(\U,\F(m)) \rightarrow C^1(\U,\F(m))
\rightarrow \cdots \rightarrow C^{n-1}(\U,\F(m)) \rightarrow
C^{n}(\U,\F(m)) \rightarrow 0 \] (no verb either).

All the $C^{i}(\U,\F(m))$ are flat over $A$. [[Twisting by $m$
doesn't affect flatness - you can check it locally, and it is
unaffected locally. So then you have a short exact sequence on the
right where everything is flat, and working your way back all the
kernels and cokernels are flat too.]] Splitting this into short
exact sequences, you get that $\mathscr{K}_i$ is flat over $A$ for
all $i$, so $H^0(X,\F(m))$ is flat over $A$ by (III 9.1A(e)).
Also, $H^0(X, \F(m))$ is finitely generated, so by (III 9.1A(f)),
its free (of finite rank).

\end{proof}
}
 { \stepcounter{lecture}
 \setcounter{lecture}{26}
 \sektion{Lecture 26}

 We were showing: A projective morphism is flat if and only if the
 Hilbert polynomial is independent of $y\in Y$.

 \underline{The Setup:} $T=\spec A$, where $A$ is local noetherian
 domain.  $X=\P^n_T$, and $\F$ is a coherent sheaf on $X$.  It will suffice
  to show that TFAE:
 \begin{itemize}
 \item[(i)] $\F$ is flasque over $T$.
 \item[(ii)] $H^0(X,\F(m))$ if a free $A$-module (of finite rank)
 for all $m\gg 0$.
 \item[(iii)] The Hilbert polynomial, $P_t$ of $\F_t$ on $X_t$ is
 independent of $t\in T$.
 \end{itemize}
 We have already shown that (i)$\Rightarrow$ (ii)

 (ii)$\Rightarrow$ (i): Choose $m_0$ such that $H^0(X,\F(m))$ is
 free for all $m\ge m_0$.  Let $M=\bigoplus_{m\ge m_0}
 H^0(X,\F(m))$.  Then $M\cong \Gamma_*(\F)_{m\ge m_0}$, so $\tilde
 M \cong \widetilde{\Gamma_*(\F)}_{m\ge m_0} \cong \F$.  Since $M$
 is free, it's flat over $A$, so $\F$ is flat over $A$ (on open
 affines it's $\sim$ of localizations of $M$).

 \begin{lemma}
 Let $T,A,X$ as above, and let $\F$ be coherent on $X$.  Then
 $H^0(X_t,\F_t(m)) \cong H^0(x,\F(m))\otimes_A k(t)$ for all $t\in
 T$ and $m\gg_t 0$.
 \end{lemma}
 \begin{proof}
 We may assume that $t$ is the closed point of $T$ (replace $T$ with $T':=\spec
 \O_{T,t}$ and note that $T'$ is flat over $T$, then LHS unaffected by base
 change and RHS is tensored with $A':=\O_{T,t}$).  Given a finite
 generating set for the maximal ideal of $A$, we have an exact
 sequence
 \[
    A^n\to A\to k(t) \to 0 \tag{$\ast$}
 \]
 so we can get the exact sequence ``$(\ast)\otimes_A \F$'':
 \[
    \F^n\to \F\to \F_t \to 0.
 \]
 Therefore,
 \[\xymatrix{
 H^0(X,\F(m)^n) \ar[r] \ar[d]^{\wr} & H^0(X,\F(m)) \ar[r] \ar@{}[d]|{\parallel} & H^0(X_t,\F_t(m))
 \ar[r] \ar@{.>}[d]^{\wr}_{\exists} & 0\\
H^0(X,\F(m))^n \ar[r] & H^0(X,\F(m)) \ar[r] &
H^0(X,\F(m))\otimes_A k(t) \ar[r] &
 0\\
 }\]
 The top row is exact by some homework exercise for $m\gg 0$, and
 the bottom row is exact for all $m$ because it is $(\ast)\otimes_A
 H^0(X,\F(m))$ (and tensor is right exact).  And the diagram
 commutes. Therefore, you get the isomorphism on the right.
 \end{proof}

 (ii)$\Rightarrow$(iii): The Lemma implies that the rank of
 $H^0(X,\F(m))$ is $\dim_{k(t)} H^0(X-t,\F_t(m))$ for all $t\in T$
 and $m\gg_t 0$.  The RHS is $P_t(m)$ for all $m\gg 0$, and the
 LHS is independent of $t$.

 (iii)$\Rightarrow$(ii): Recall II.8.9: A finitely generated
 module over $A$ (as above) is free if and only if $M\otimes_A K$
 and $M\otimes_A k$ have the same dimensions over the fraction
 field $K$ and the residue field $k$, respectively.  In our case,
 $H^0(X,\F(m))$  free is implied by $\dim_{k(\tau)}
 H^0(X,\F(m)\otimes_A k(\tau)) = \dim_{k(t)} H^0(X,\F(m))\otimes_A
 k(t)$ where $\tau$ and $t$ are the generic and special points of
 $T$, respectively.  By the Lemma, the LHS is $P_t(m)$ for $m\gg
 0$, and the RHS is $P_t(m)$ for $m\gg 0$.

 So we're done.

 \begin{remark}
 To prove flatness, it's enough to compare Hilbert polynomials at
 closed and generic points.
 \end{remark}

 \begin{corollary}
 Let $T$ be a connected noetherian scheme, and let $X$ be a closed
 subscheme of $\P^n_T$, flat over $T$.  Then the degree, dimension, and
 arithmetic genus of $X_t$ is independent of $t\in T$.
 \end{corollary}
 \begin{proof}
 By base change to an irreducible component of $T$ (with reduced induced
 subscheme structure), we may assume
 that $T$ is integral.  Then use the fact that the Hilbert
 polynomial is independent of $t\in T$ (all of these things are determined
 by the Hilbert polynomial).
 \end{proof}

 \underline{Exercise III.9.1:} {\it Let $f:X\to Y$ be a flat morphism
 of finite type of noetherian schemes.  Then $f$ is an
 open\footnote{For all open $U\subseteq X$, $f(U)\subseteq Y$ is open.}
 morphism.}
 \begin{proof}
 Let $U\subseteq X$ be open.  Then $f(U)\subseteq Y$ is
 constructable (Ex. II.3.19), meaning that it is a closed set
 minus a constructable set of lower dimension.  By Ex. II.3.18c, it
 will suffice to show that $f(U)$ is stable under generization.
 That is, if $y\in f(U)$ and $\eta\in Y$ such that $\eta
 \rightsquigarrow y$, then $\eta \in f(U)$.  To see this, let
 $\spec A\subseteq Y$ be an open affine neighborhood of $y$ (thus,
 $\eta\in \spec A$).  Let $x\in f^{-1}(y)$, and let $\spec B$ be an
 open affine neighborhood of $x$ in $U\cap f^{-1}(\spec A)$.  Then
 $B$ is flat over $A$.  Let $\p,\p',\q$ be prime ideals of $A,A,
 B$ (resp.) corresponding to $y,\eta,x$ (resp.)
 \[\xymatrix @C=4mm {
  & \q \ar@{}[r]|{\subseteq} \ar@{-}[d] & B \ar@{-}[d]^{\txt{flat}}\\
  \p'\ar@{}[r]|{\subseteq} & \p\ar@{}[r]|{\subseteq} & A
 }\]
 There is a prime $\q'$ of $B$ lying over $\p'$ such that
 $\q'\subseteq \q$.  let $\xi\in U\subseteq X$ be the
 corresponding point.  Then $\eta = f(\xi) \subseteq f(U)$, as was
 to be shown.
 \end{proof}

 \underline{Exercise III.9.4:} {\it Let $f:X\to Y$ be a morphism
 of finite type of noetherian schemes.  Then $\{x\in X|f \text{ flat at }x\}\subseteq
 X$ is open.}

 \marginpar{\S III.10: Smoothness}
 This is a relative version of non-singularity (\emph{almost}).
 For this section, $k$ is any field (not necessarily algebraically
 closed), and all schemes considered will be assumed to be of
 finite type over $k$ (or locally of finite type).

 \begin{definition}
 Let $f:X\to Y$ be a morphism of schemes over $k$ (as above), then
 $f$ is \emph{smooth} of relative dimension $n$ if
 \begin{itemize}
 \item[(1)] $f$ is flat
 \item[(2)] if $X'$ and $Y'$ are irreducible components of $X$ and
 $Y$ (resp.) such that $f(X')\subseteq Y'$, then $\dim X' = \dim
 Y' +n$.
 \item[(3)] for all $x\in X$ (closed or not), $\dim_{k(x)}
 (\Omega_{X/Y}\otimes k(x))=n$.
 \end{itemize}
 \end{definition}

 \begin{remark}
 \begin{itemize}
 \item[]
 \item[(i)] By $(1)$ and Cor III.9.6, (2) is equivalent to

    (2') for all $y\in Y$ (closed or not), $X_y$ is equidimensional
    of dimension $n$.

    Also, (3) is equivalent to

    (3') for all $x\in X$, $\dim_{k(x)} (\Omega_{X_y/k(y)} \otimes
    k(x))=n$ where $y=f(x)$. $\Omega_{X_y/k(y)}$ is really
    $\Omega_{X/Y}$ pulled back to $X_y$.

    So (2) and (3) just concern the fibers $X_y$.
 \item[(ii)] Smoothness of relative dimension $n$ is local on $X$
 (therefore local on the base) in the sense that $f:X\to Y$ is
 smooth of relative dimension $n$ if and only if there exists an
 open cover $(U_i)_{i\in I}$ of $X$ such that $f_{U_i}$ is smooth
 of relative dimension $n$ for all $i$.  So we can define $f:X\to
 Y$ is smooth of relative dimension $n$ \emph{at $x$} if there is
 an open neighborhood $U\subseteq X$ such that $f|_U$ is smooth of
 relative dimension $n$.  Then $f$ is smooth of relative dimension
 $n$ if and only if it is at each point in $X$.  Also, $\{x\in
 X|\text{$f$ smooth of relative dimension $n$}\}\subseteq X$ is
 open.
 \end{itemize}
 \end{remark}

 \underline{Examples:}
 \begin{itemize}
 \item[(i)] For any $Y$, $\P^n_Y$ and $\A^n_Y$ are smooth over
 $Y$ of relative dimension $n$.  It is enough to show it for $\A^n_Y$ since $\P^n_Y$ is
 covered by open affines isomorphic to $\A^n_Y$.
 \begin{proof}
 (1) Flatness is ok. (2') is easy: $\dim_{k(y)} \A^n_Y = n$.  (3')
 $\Omega_{\A^n_Y/Y}$ is free of rank $n$; ditto for its fibers.
 \end{proof}

 \item[(ii)] Let $Y=\spec k$ with $k$ algebraically closed.  Then
 $X$ is smooth over $Y$ of relative dimension $n$ if and only if
 $X$ is non-singular of pure dimension $n$ if and only if $X$ is
 regular of pure dimension $n$.
 \begin{proof}
 The second equivalence comes from Chapter I.  The first condition
 is equivalent to the third is given by II.8.15.
 \end{proof}
 \underline{Caution:} We do need $k$ algebraically closed here
  (see Ex III.10.1).
 \end{itemize}

 \begin{proposition}
 \begin{itemize}
 \item[]
 \item[(a)] An open immersion is smooth of relative dimension 0.
 \item[(b)] (Base Change) If $f:X\to Y$ is smooth of relative
 dimension $n$, and $Y'\to Y$ is any morphism, then $f':X':=
 X\times_Y Y'\to Y'$ is also smooth of relative dimension $n$.
 \item[(c)] (Composition) If $X\xrightarrow{f} Y\xrightarrow{g} Z$
 are smooth of relative dimension $n$ and $m$, resp., then $g\circ
 f:X\to Z$ is smooth of relative dimension $n+m$.
 \item[(d)] (Products) if $f:X\to Y$ and $f':X'\to Y'$ are smooth
 $S$-morphisms of relative dimensions $n$ and $n'$, resp., then
 $f\times_S f':X\times_S X' \to Y\times_S Y'$ is smooth of
 relative dimension $n+n'$. (Book's special case: $Y=Y'=S$)
 \end{itemize}
 \end{proposition}
 \begin{proof}
 (a) is trivial. (d) follows from (b) and (c) by published proof of Exercise
 II.4.8d.

 (b): Flatness is immediate.  For (2') and (3'), let $y'\in Y'$ and
 let $y\in Y$ be the image of $y'$.  Then $X'_{y'} = X_y
 \times_{k(y)} k(y')$.  $k(y')$ is a finite extension of $k(y)$,
 so it doesn't affect dimension\footnote{see next lecture}.  And it doesn't affect (3')
 \end{proof}
}
 { \stepcounter{lecture}
 \setcounter{lecture}{27}
 \sektion{Lecture 27}

 Comments on the homework:
 \begin{itemize}
 \item $\Gamma(-)$ does \emph{not} commute with $\otimes$ (in
 general).
 \item Exercises 9.3cd give ways in which 9.7 fails when $\dim Y
 >1$.
 \item You \emph{do} need to prove that $x\otimes z \pm y\otimes
 w\not=0$ in $(x,y)\otimes A$
 \item In 9.3c, to show that $I\subseteq k[x,y,z,w]$ is primary,
 you can use the fact that it's homogeneous in $z$ and $w$, but
 you cannot immediately reduce to working with homogeneous
 elements.
 \end{itemize}

 All schemes today are assumed to be of finite type over
 appropriate field.

 Loose end from last time:  Let $X$ be a scheme over $k$, let
 $k'$ be an extension of $k$, and let $X'=X\times_k k'$, then
 $\dim X' = \dim X$
 \begin{proof}
 Let $\spec A$ be an open affine in $X$, and let $A'=A\otimes_k
 k'$, so that $\spec A' = \pi^{-1}(\spec A)$, where $\pi:X'\to X$
 is the projection.  By Noether's normalization lemma, there is a
 subring $B\subseteq A$ with $k\subseteq B$, $B\cong k[x_1,\dots,
 x_r]$, and $A$ is finite over $B$.

 \[\xymatrix{
  A'\ar@{-}[r]\ar@{-}[d] & A\ar@{-}[d] \\
  B'\ar@{-}[d] & B\ar@{-}[d] \\
  k' \ar@{-}[r] & k
 }\]
 Then $B':=B\otimes_k k'$ is $\cong k'[x_1,\dots,x_r]$, and $A'$
 is finite over $B'$.  Then $\dim A'= \dim B' = r = \dim B = \dim
 A$ (Note that $A$ is integral over $B$; then $\dim A \ge \dim B$
 by lying over and going up for integral extensions, and $\dim
 A\le \dim B$ by incomparability for integral extensions).  Then
 $\dim X = \max_A \dim A = \max_A \dim A' = \dim X'$.
 \end{proof}

 \underline{Also}, $\pi:X'\to X$ is onto.  To see this, we may assume that $X=\spec
 A$.  We have that $k'$ is faithfully flat over $k$, so $a'$ is
 faithfully flat over $A$ (by base change: $M\otimes_k k' =
 M\otimes_A(A\otimes_k k') = M\otimes_A A' = 0$, so $M=0$).  Then
 $\spec A'\to \spec A$ is surjective by homework.

 \underline{Also}, all irreducible components of $X'$ dominate irreducible
 components of $X$.  This follows from going down for flat
 extensions.

 Back to smooth morphisms.  We were proving part (c) of the
 proposition: if $X\xrightarrow{f} Y \xrightarrow{g} Z$ are smooth
 of relative dimension $n$ and $m$, resp., then $g\circ f$ is
 smooth of relative dimension $n+m$.
 \begin{proof}
 (1) flatness is obvious.

 (2) Let $X'$, $Y'$, and $Z'$ be irreducible components of
 $X,Y,Z$, respectively, such that $f(X')\subseteq Y'$ and
 $g(Y')\subseteq Z'$.  Then $\dim X' = \dim Z' + n + m$ by (2)

 (3) Use the first exact sequence: let $x\in X$, let $y=f(x)$, and
 $z=g(f(x))$.  Then
 \[
    f^*\Omega_{Y/Z} \to \Omega_{X/Z} \to \Omega_{X/Y} \to 0
 \]
 is exact, so
 \[
    \underbrace{f^*\Omega_{Y/Z}\otimes k(x)}_{(\Omega_{Y/Z}\otimes k(y))\otimes k(x) } \to \Omega_{X/Z}\otimes k(x) \to
    \underbrace{\Omega_{X/Y}\otimes k(x)}_{\dim_{k(x)} = n} \to 0
 \]
 dimension of first term over $k(x)$ is $m$, so the dimension over
 $k(x)$ of the middle is $\le n+m$.

 Show the other inequality.  Note that $x\in X_z$, so
 \[
    \Omega_{X/Z}\otimes k(x) = \Omega_{X_x/k(z)} \otimes k(x).
 \]
 By (2'), for $g\circ f$, $X_z$ has pure dimension $n+m$.  Let
 $X'$ be an irreducible component of $X_z$ containing $x$.  By the
 second exact sequence:
 \[
    i^*\Omega_{X_z/k(z)} \to \Omega_{X'/k(z)} \to 0
 \]
 is exact where $i:X'\to X_z$, so
 \[
    \underbrace{i^*\Omega_{X_z/k(z)}\otimes k(x)}_{\Omega_{X_z/k(z)}\otimes k(x)} \to \Omega_{X'/k(z)}\otimes k(x) \to 0
 \]
 is exact,
 so it suffices to show that $\Omega_{X_z/k(z)}\otimes k(x)$ has
 dimension $\ge n+m$.  We know that $\Omega_{X'/k(z)}$ has rank
 $\ge n+m$ at the generic point, and it is coherent, so the rank
 does not decrease when you specialize to $x$ (use Nakayama's Lemma).
 \end{proof}

 \begin{theorem}
 Let $f:X\to Y$ be a morphism of schemes (of finite type) over
 $k$.  Then $f$ is smooth of relative dimension $n$, if and only
 if
 \begin{itemize}
 \item[(1)] $f$ is flat.
 \item[(2)] for all $y\in Y$, the geometric fiber $X_{\bar y} =
 X_y\times_{k(y)} \overline{k(y)}$ is regular of pure dimension
 $n$.
 \end{itemize}
 \end{theorem}
 \begin{proof}
 ($\Rightarrow$) (1) is obvious. (2): by base change, $X_{\bar y}$
 is smooth over $\overline{k(y)}$, so it is regular by (II.8.8 or
 II.8.15).

 ($\Leftarrow$) (1) implies condition (1) for smoothness.  (2)
 implies (2') for smoothness (all fibers have the same dimension,
 so geometric fibers have the same dimension).  (2) also implies
 (3') for smoothness.  To see this, note that $\Omega_{X_{\bar y}/\overline{k(y)}}$
 is locally free of rank $n$ by II.8.8, which
 implies that $\Omega_{X_{\bar y}/\overline{k(y)}} \otimes\overline{k(y)}$
  has dimension $n$ for all $y$, which implies
 that $\Omega_{X_y/k(y)}\otimes k(x)$ has dimension $n$ for all
 $x\in X_y$ since $\Omega_{X_{\bar y}/\overline{k(y)}} = \Omega_{X_y/k(y)}\otimes_{k(y)} \overline{k(y)}$
 \end{proof}

 \begin{corollary}
 Let $f:X\to Y$ be a morphism of schemes (of finite type) over
 $k$.  Then $f$ is smooth of relative dimension $n$ if and only if
 $f$ is flat and $X_y$ is smooth of relative dimension $n$ over
 $k(y)$ for all $y\in Y$. (second part equivalent to (2))
 \end{corollary}

\underline{Smoothness over a field:}

 \begin{theorem}[EGA IV 6.74a] Let $X$ be a scheme of finite type
 over a field $k$, and let $k'$ be an extension field of $k$, let
 $X'=X\times_k k'$, let $\pi:X' \to X$ be the projection, let
 $x'\in X'$ and let $x=\pi(x')$.  Then
 \begin{itemize}
 \item[(a)] If $X'$ is regular at $x'$, then $X$ is regular at
 $x$.
 \item[(b)] The converse holds if $k'$ is separable over $k$.
 \end{itemize}
 \end{theorem}

 \begin{corollary}
 Since $\pi$ is surjective,
 \begin{itemize}
 \item[(a)] $X'$ is regular implies that $X$ is
 regular, and
 \item[(b)] the converse holds if $k'$ is separable over $k$.
 \end{itemize} Note that $(b)$ is false in general (Ex 10.1).
 \end{corollary}

 \begin{corollary}
 Let $X$ be a scheme of finite type over a perfect field $k$.  If
 $X$ is regular, then it is smooth over $k$.
 \end{corollary}
 \begin{proof}
 $X\times_k \bar k$ is regular, so $X$ is smooth.
 \end{proof}

 \begin{corollary}
 If $X$ is smooth over an arbitrary field $k$, then it is regular by (b) ($\bar k$
 is separable over $k$).
 \end{corollary}
 \begin{proof}
 $X\times_k \bar k$ is smooth over $\bar k$, so $X\times_k \bar k$
 i s regular, so $X$ is regular by (a) (of the first corollary).
 \end{proof}

 A ``more self-contained'' proof:
 \begin{lemma}
 Let $k$ be a field.  Then $\A^n_k$ is regular.
 \end{lemma}
 \begin{proof}
 We need to show that $k[x_1,\dots, x_n]$ is a regular ring.  This
 follows from Matsumura, \underline{Commutative Ring Theory}
 [1986], Theorem 19.5\footnote{If $R$ is regular and noetherian, then so is
 $R[x]$}.  It is also Eisenbud, Exercise 19.3
 \end{proof}

 \begin{proposition}
 Let $X$ be a smooth scheme of finite type over a field $k$.  Then
 $X$ is regular.
 \end{proposition}
 \begin{proof}
 We may assume $X=\spec A$ is affine.  Choose a generating set
 $x_1,\dots ,x_n\in A$, so $k[x_1,\dots, x_n]\twoheadrightarrow
 A$, so $X\hookrightarrow \A^n_k$.  By Matsumura Thm 19.3 (a
 localization of a regular local ring at a prime ideal is
 regular), it suffices to show that $X$ is regular at $x$ for all
 closed points $x\in X$.  Let $\O$ be the local ring
 $\O_{\A^n_k,x}$, let $\m$ be its maximal ideal, and let $I$ be
 the kernel of the surjection $\O\twoheadrightarrow \O_{X,x}$.
 Then $I\subseteq \m$ ($x\in X$).  Let $d=\dim \O_{X,x} = \dim_x
 X$.  Use the second exact sequence:
 \[
    I/I^2 \xrightarrow{\delta} \Omega_{\O/k}\otimes_{\O} \O_{X,x} \xrightarrow{\alpha}
    \Omega_{\O_{X,x}/k} \to 0
 \]
 is exact, so
 \[
    (I/I^2) \otimes k(x) \xrightarrow{\delta'}
    \underbrace{\Omega_{\O/k}\otimes_ k(x)}_{\txt{dim $n$ over $k(x)$}} \xrightarrow{\alpha'}
    \underbrace{\Omega_{\O_{X,x}/k}\otimes k(x)}_{\txt{dim $d$ over $k(x)$}} \to 0
 \]
 is exact.
 Let $r-n-d$, and pick $f_1,\dots, f_r\in I$ such that
 $\delta'((f_i \text{mod } I^2\otimes 1)$ generate $\ker \alpha'$.
 Let $\O' = \O/(f_1,\dots, f_r)$.  It is local with maximal ideal
 $\m' = \m/(f_1,\dots,f_r)$.  Then $\m'/\m'^2 \cong \m/(\m^2+(f_1,\dots, f_r))$
 has dimension $\ge n-r=d$ over $k(x)$. But also $\O'$ maps
 surjectively to $\O_{X,x}$, so
 \[
   \dim_{k(x)}\m'/\m'^2 \ge \dim \O' \ge \dim \O_{X,x} = \text{Stuck}
 \]
 is also exact.
 \end{proof}

 \underline{Next:} \'{E}tale morphisms

 \begin{definition}[Ex 10.3]
 Let $f:X\to Y$ be a morphism of schemes (of finite type) over
 $k$.  Then
 \begin{itemize}
 \item[(a)] $f$ is \emph{\'{e}tale} if it is smooth of relative dimension
 0.
 \item[(b)] $f$ is \emph{unramified} if for all $x\in X$, letting
 $y=f(x)$, we have $\m_y \cdot \O_{X,x} = \m_x$ and $k(x)$ is a
 separable algebraic extension of $k(y)$.
 \end{itemize}
 \end{definition}

 \underline{Example:} Let $R=\Z[\sqrt{2}]$, then $\spec R$ is \'{e}tale
 over $\spec \Z$ except at the prime $(\sqrt{2})\subseteq R$.
 (These are not schemes over a field, but the same principles
 apply.)  At other primes, there may be residue field extensions,
 but they are separable.
}
 { \stepcounter{lecture}
 \setcounter{lecture}{28}
 \sektion{Lecture 28}

 All schemes still of finite type over a field.

 \begin{proof}[Continued Proof]
 $X$ smooth of relative dimension $d$ over $k$ ... we were showing that $X$ is
 regular.  All notation as in last lecture.

 We need $\dim \m'/\m'^2 \le n-r=d$.
 \[\xymatrix{
    (I/I^2)\otimes k(x)\ar[d] \ar[r] & \Omega_{\O/k}\otimes k(x) \ar@{}[d]|{\parallel} \ar[r]^{\alpha'}
    & \Omega_{\O_{X,x}/k}\otimes k(x) \ar@{->>}[d] \ar[r] & 0\\
    \m/\m^2\ar[r] & \Omega_{\O/k}\otimes k(x) \ar[r] &
    \Omega_{k(x)/k} \ar[r] & 0\\
    & & \text{may be 0} &
 }\]

 By arrow chasing, the images of $f_1,\dots, f_r$ in $\m/\m^2$ ar linearly
 independent over $k(x)$.  Thus, the dimension of $\m'/\m'^2$ is $n-r=d$.  Then by
 the list of inequalities from the last lecture, equality holds.  So $\O'$ is a
 regular local ring.  So $\O'$ is entire (II.6.11.1A), so $(0)$ is prime in $\O'$,
 and it is the unique minimal prime, so it doesn't go away in $\O_{X,x}$.  Thus,
 $\O'=\O_{X,x}$.  So $X$ is regular at $x$.
 \end{proof}

 \underline{Next:} Instead of \'{e}taleness, we'll do the jacobian criterion for
 smoothness.

 \begin{lemma}\label{lec28lem1} Let $\xymatrix{Z\ar[r]^j \ar[dr]_{\phi} &
 X\ar[d]^{\psi}\\ & Y}$, be a commutative diagram of schemes (of finite type) over $k$.
 Let $z\in Z$ be a closed point, with $x=j(z)$.  Assume that $j$ is an immersion and
 that $\phi$ and $\psi$ are smooth of relative dimensions $d$ and $n$, respectively
 (at $z$ and $x$, resp.).  Let $\I$ be the sheaf of ideals defining $Z$ in some open
 neighborhood of $x\in X$. Let $r=n-d$ (the codimension of $Z$ in $X$).  Then
 \begin{itemize}
 \item[(a)] There is an open neighborhood $U$ ov $x$ in $X$ and sections $f_1,\dots,
 f_r\in \I(U)$ such that the images $df_1,\dots, df_r$ in $\Omega_{X/Y}\otimes k(x)$
 are linearly independent, and
 \item[(b)] Any such $f_1,\dots, f_r$ generated $\I$ in a (possibly smaller) open
 neighborhood of $x$.
 \end{itemize}
 \end{lemma}
 \begin{proof}
 We may assume that $Z\hookrightarrow X$ is a closed immersion.  Let $\O=\O_{X,x}$,
 $\m =$ maximal ideal in $\O$, $I=\ker (\O_{X,x}\to \O_{Z,z})=\I_x$; note that
 $I\subseteq \m$.  We have a commutative diagram with exact rows:
 \[\xymatrix{
   (I/I^2)\otimes k(x)\ar[d] \ar[r] & \Omega_{X/Y}\otimes k(x)
   \ar@{}[d]|{\parallel}^{\wr}
   \ar[r]^{\alpha'} & \Omega_{Z/Y}\otimes k(x) \ar@{->>}[d] \ar[r] & 0\\
    \bar\m/\bar\m^2\ar[r] & \Omega_{X_y/k(y)}\otimes k(x) \ar[r] & \Omega_{k(x)/k(y)} \ar[r] & 0\\
 }\]
 where $y=\psi(x)=\phi(z)$.  Here, $\bar \O = \O\otimes k(y) = \O_{X_y,x} = \O/\m_y
 \O$, and $\bar m = \m\bar\O$.  For part (a), pick $f_1,\dots, f_r\in I$ so that
 their images form a basis for $\ker \alpha'$.  Let $\bar I = I\bar \O = \ker(\bar O \to
 \O_{Z_y/k(y)})$, and let $\bar f_1,\dots, \bar f_r\in \bar I$ be the images of
 $f_1,\dots,f_r$.  If we put bars on everything in the top row of the diagram, then
 as in the previous proof, $\bar f_1,\dots,\bar f_r$ generate $\bar I$, so by
 Nakayama's Lemma, $f_1,\dots,f_r$ generate $I$, so some open neighborhood $U\ni x$
 in $X$, $f_1,\dots, f_r$ extend to sections of $\I(U)$ (by defn of a stalk), and
 generated $\I(U)$ by coherence (after shrinking $U$).
 \end{proof}

 \begin{corollary}
 In this situation (one smooth thing, $Z$, inside another smooth thing, $X$), $Z$ is
 a locally complete intersection in $X$ of codimension $r$.
 \end{corollary}

 In the proof of the Lemma, we assumed that $z$ is a closed point, but we don't
 really need that.  If $\Omega_{X/Y}$ is locally free (e.g. if $X$ is integral,
 (III.10.0.2), or $X=\A^n_Y$, then the lemma holds for all $z\in Z$.  To see this,
 let $z\in Z$ be any point.  Shrink to open neighborhoods of $z$ and $x$ such that
 $Z\hookrightarrow X$ is a closed immersion, and $\Omega_{X/Y}$ is free.  Then at a
 closed point $z'$ such that $z\rightsquigarrow z'$, the conclusions of the Lemma
 hold.

 But actually, we don't the assumption on $X$ because we just need $z$ to be a closed
 point \emph{in its fiber} $Z_y$, and then use the fact that $\Omega_{X/Y}\otimes
 k(x)$ is a localization of $\Omega_{X/Y}\otimes k(x')$.

 \begin{lemma}\label{lec28lem2} Let $\phi:X\to Y$ be a smooth morphism of schemes
 over $k$ of relative dimension $n$, and let $f\in \Gamma(X,\O_X)$. Let $x_0\in X$ a point such
 that $x_0\in Z(f)$ (i.e. $f\in \m_{x_0}$), and the image of $df\in \Omega{X/Y}\otimes
 k(x_0)$ is nonzero.  Then $Z:=Z(f)$ is smooth of relative dimension $n-1$ over $Y$ at
 $x_0$.  Also, $\Omega_{Z/Y}\otimes k(x_0) \cong (\Omega_{X/Y}\otimes k(x_0))/(\text{image
 of } f)$.
 \end{lemma}
 \begin{proof}
 The condition on $df$ is equivalent to $x\not\in \supp (\ker (\O_X\xrightarrow{\cdot df}
 \Omega_{X/Y}))$, so replace $X$ with an open neighborhood of $x_0$ such that this
 condition (on $df$) holds for all $x\in Z$ (in place of $x_0$).  Also,
 \[
    ((f)/(f)^2)\otimes k(x) \xrightarrow{\delta'} \Omega_{X/Y}\otimes k(x) \to \Omega_{Z/Y}\otimes k(x)
    \to 0
 \]
 is exact and the image of $\delta'$ has rank 1, so (3) holds at $x$ for all $x\in
 Z$, and the last sentence in the lemma is true.

 To show (2'), note that $\bar f$ is non-zero in $\O_{X_y,x}$, which is a regular
 local ring, and therefore entire, so $\bar f$ is not a zero divisor, and therefore
 (by Hauptidealsatz)
 \begin{align*}
 \dim_x Z &= \dim \O_{Z_y,z}\\
 &= \dim \O_{Z_y,z}/(\bar f) \\
 &= \dim \O_{X_y,z} -1\\
 & = \dim_z X_y -1 = n-1
 \end{align*}
 Thus, (2') for $Z$ holds for all $x\in Z$.  here $y=\phi(x)$

 To show that $Z$ is flat over $Y$, pick $x\in Z$, and let $y=\phi(x)$.  Then by
 smoothness, $X_y$ is regular at $x$, so $\O_{X_y,x}$ is a regular local ring, and
 therefore entire.  And again, $f$ is non-zero in this ring, so it is not a zero
 divisor, so
 \[
    0\to \O_{X_y,x} \xrightarrow{f} \O_{X_y,x} \to \O_{Z_y,x} \to 0
 \]
 is exact, so
 \[
    0\to \O_{X,x}\otimes k(y) \to \O_{X,x}\otimes k(y) \to \O_{Z,x}\otimes k(y) \to 0
 \]
 is exact.  Also,
 \[
    \O_{X,x} \xrightarrow{f} \O_{X,x} \to \O_{Z,x} \to 0
 \]
 is exact, and forms part of a flat resolution for $\O_{Z,x}$ over $\O_{Y,y}$.  So
 $\tor_1^{\O_{X,x}}(\O_{Z,x},k(y)) = \ker(\O_{X,x}\otimes k(x) \xrightarrow{\cdot f} \O_{X,x}\otimes
 k(y))/(something)$.  Therefore, $\tor_1^{\O_{X,x}}(\O_{Z,x},k(y))=0$.  Therefore,
 $\O_{Z,x}$ is flat over $\O_{Y,y}$ (Bourbaki; Comm Alg III\S 5 No. 2 Thm 1 and some
 proposition a few pages later).  This holds for all $x\in Z$, so $Z$ is flat over
 $Y$.

  So $Z$ is smooth at $x$ over $Y$ of relative dimension $n-1$.
 \end{proof}

 \marginpar{Jacobian Criterion for Smoothness}
 \begin{theorem}[Jacobian Criterion for Smoothness]
 Let $Z$ be a subscheme of codimension $r$ is a scheme $X$, which is smooth over $Y$
 of relative dimension $n$.  Then $Z$ is smooth over $Y$ at a point $z\in Z$ if and
 only if:
 locally at $z\in X$, the sheaf of ideals defining $Z$ can be generated by $r$
 elements $f_1,\dots, f_r$, and the differentials $df_1,\dots, df_r$ are linearly
 independent over $k(z)$ in $\Omega_{X/Y}\otimes k(z)$.  Moreover, if this hold, then
 $Z$ is smooth of relative dimension $n-r$ over $Y$.
 \end{theorem}
 \begin{proof}
 ($\Rightarrow$) This is given to us by Lemma (\ref{lec28lem1})

 ($\Leftarrow$) This follows from repeated applications of Lemma (\ref{lec28lem2}).
 \end{proof}

 \underline{Next:} \'{E}tale morphisms.

 \begin{lemma}
 Let $k$ be a field, and let $(R,\m)$ be a noetherian local $k$-algebra.  Assume that
 the residue field $k':=R/\m$ is a finite separable extension of $k$, and that
 $\m^2=0$.  Then $\dim_{k'}\Omega_{R/k}\otimes k' = \dim_{k'} \m\quad (=\dim_{k'} \m/\m^2)$.
 \end{lemma}
 \begin{proof}
 Let $a_1,\dots, a_r$ be a basis for $\m$ over $k'$.  Pick $t\in R$ such that $\bar t
 \in R/\m$ is a primitive element for $k'$ over $k$, and let $f\in k[T]$ be its
 irreducible polynomial.  Note that $f\in R[T]$, and $f(t)\in \m$, but $f'(t)\not\in
 \m$ (since $k'$ is separable over $k$).

 We have a map $\phi:k[T,X_1,\dots, X_r] \to R$ defined by $T\mapsto t$, $X_i\mapsto
 a_i$ for all $i$, and $k$ is fixed.  Note that $\phi$ maps $(T)$\footnote{?} onto $R/\m$ and that
 $(X_1,\dots,X_r)$ onto $\m$.  So $\phi$ is onto.  Since $f(t)\in \m$, there are
 $g_1,\dots, g_r\in k[T,X_1,\dots,X_r]$ such that $f-\sum g_i X_i \in \ker \phi$.
 Also, $X_i X_j \in \ker \phi$ for all $i,j$ (including $i=j$).  let $\a$ be the
 ideal generated by these elements.
 \begin{claim}
 $\a = \ker \phi$
 \end{claim}
 \begin{proof}
 $\a\in \ker \phi$ by construction.  Then the other inclusion follows by noting that
 $\dim_k k[T,X_1,\dots,X_r]/\a \le \dim_k R = (r+1)[k':k]$

 \renewcommand{\qedsymbol}{$\square_\text{\tiny Claim}$}
 \end{proof}

 Then $\Omega_{R/k}\otimes k' = \Omega_{R/k}/\m\Omega_{R/k}$ is described by
 generators $dt,da_1,\dots,da_r$ over $k'$ and relations $\underbrace{f'(t)}_{\text{unit}}dt = \sum
 (\underbrace{a_idg_i}_{\in \m}+g_ida_i)$ and $a_ida_j+a_jda_i\in \m$, so
 $\Omega_{R/k}\otimes k'$ has basis $da_1,\dots, da_r$.
 \end{proof}
}
 { \stepcounter{lecture}
 \setcounter{lecture}{29}
 \sektion{Lecture 29}

 Homework(12):  For the last problem, you may
 assume that $k$ is algebraically closed of characteristic
 different from 2 and 3.

 Comment on last homework(11): $Z'$ should be noetherian.

 Comment of homework 10: $X\subseteq \P^n$, then there may be line
 sheaves on $X$ which do not come from line sheaves on $\P^n$.
 For example, take the 2-uple embedding $i:\P^1\to \P^2$.  Then
 $i^*\O(1) = \O(2)$, so you can only get the $\O(2n)$'s on $\P^1$.
 Thus, $\pic \P^n \to \pic X$ may not be onto.

 A reference for what we're doing on smoothness:  Bosch,
 Lutkebohnert, and Raynand, Neron Models \S 2.2.

 \underline{Last time:} If $(R,\m)$ is a local noetherian
 $k$-algebra such that $\m^2=0$ and $k'=R/\m$ is finite separable
 over $k$, then
 \[
 0\to \underbrace{\m/\m^2}_{\m} \to \Omega_{R/k}\otimes k' \to
 \underbrace{\Omega_{k'/k}}_0 \to 0
 \]
 is exact.

 \begin{corollary}
 If $(R,\m)$ is a local noetherian
 $k$-algebra such with $k'=R/\m$ is finite separable
 over $k$, then $\m/\m^2 \to \Omega_{R/k}\otimes k'$ is an
 isomorphism (cf. II.8.7)
 \end{corollary}
 \begin{proof}
   Apply the earlier lemma to $R/\m^2$.  Note that $\m^2/\m^4 \to
   \Omega_{R/k}\otimes (R/\m^2) \to \Omega_{(R/\m)/k} \to 0$ is
   exact.  This gives
   \[
    (\m/\m^2)\otimes k' \xrightarrow{0} \Omega_{R/k}\otimes k'
    \xrightarrow{\sim} \underbrace{\Omega_{(R/\m^2)/k}}_{\m/\m^2}\otimes k'
    \to 0
   \]
 \end{proof}

 For today, all schemes are of finite type over $k$.  Recall that
 \'etale means smooth of relative dimension 0.  $f:X\to Y$ is
 unramified if for all $x\in X$, writing $y=f(x)$, we have $k(x)$
 is separable algebraic over $k(y)$ and $\m_y\cdot \O_{X,x}=\m_x$
 ($X$ of finite type implies $k(x)$ is finite separable over
 $k(y)$).

 \underline{Exercise III.10.3} {\it Let $f:X\to Y$ be a morphism,
 then TFAE:
 \begin{itemize}
 \item[(i)] $f$ is \'etale
 \item[(ii)] $f$ is flat and $\Omega_{X/Y}=0$
 \item[(iii)] $f$ is flat and unramified
 \end{itemize}
 }
 \begin{proof}
 (i$\Rightarrow$ ii) flat is obvious.  Also, (3) in smoothness
 condition implies that $\Omega_{X/Y}\otimes k(x)=0$ for all $x$.,
 so by Nakayama, we have $\Omega_{X/Y}=0$.

 (ii$\Rightarrow$ iii) Let $x\in X$ and $y=f(x)$.  By the second
 exact sequence,
 \[
    \m_x/\m_x^2 \to \Omega_{X/Y}\otimes k(x) \to
    \Omega_{k(x)/k(y)}\to 0
 \]
 and $\Omega_{X/Y}=0$ implies that $\Omega_{k(x)/k(y)}=0$ which
 implies that $k(x)$ is separated and algebraic over $k(y)$
 (II.8.6A)

 Assume $\m_y\cdot \O_{X,x}\not=\m_x$ (i.e. $\subsetneqq$).  We have $\O_{X_y,x}
 = \O_{X,x}\otimes k(y) = \O_{X,x}/\m_y\O_{X,x}$.  This is a
 noetherian local $k$-algebra as in the corollary, with maximal
 ideal not zero.  Thus, $\Omega_{\O_{X_y,x}/k(y)}\otimes k(x)
 \not=0$.  Thus, $0 =_{ii} \Omega_{X/Y}\otimes k(x) =
 \Omega_{X_y/k(y)}\otimes k(x)\not=0$. contradiction.

 (iii$\Rightarrow$ i)Check (1),(2'),and (3') in the definition of
 smoothness.  (1) is obvious.  For (2'), (3'), note that
 $\O_{X_y,x}=\O_{X,x}/\m_y\O_{X,x} =_{iii} \O_{X,x}/\m_x = k(x)$.
 (2') holds because $\dim_x X_y = \dim \O_{X_y,x} = \dim k(x_=0$.
 (3') holds because $\Omega_{X_y/k(y)}\otimes k(x) =
 \Omega_{\O_{X_y,x}/k(y)} \otimes k(x) = \Omega_{k(x)/k(y)}\otimes
 k(x)=_{sep\ alg}0$
 \end{proof}

 \underline{Eisenbud 4.4:}{\it Let $R$ be a ring.  If $n$ elements of
 the $R$-module $R^n$ generate it, then they form a basis.}

 \begin{corollary}
 Let $R$ be a ring, and let
 \[
     M' \xrightarrow{\alpha} M\to M''\to 0
 \]
 be and exact sequence of $R$-modules.  If $M'$ and $M''$ can be
 generated by $n$ and $m$ elements, resp, and if $M$ is free of
 rank $n+m$, then $\alpha$ is injective, and the given generating
 sets are bases of $M'$ and $M''$.
 \end{corollary}
 \begin{proof}
 exercise.
 \end{proof}

 \begin{proposition}
 If $f:X\to Y$ is smooth of relative dimension $n$, then $\Omega_{X/Y}$ is locally
 free of rank $n$.
 \end{proposition}
 \begin{proof}
 The question is local on $X$, so we may assume that $X=\spec A$.
 Pick a closed immersion $i:X\hookrightarrow \A^N_Y$.  The second
 exact sequence gives
 \[
    \I/\I^2 \xrightarrow{\delta} i^*\Omega_{\A^N_Y/Y} \to \Omega_{X/Y}\to 0
 \]
 where $\I$ is the sheaf of ideals defining $X$ in $\A^N_Y$.
 Taking stalks, we have
 \[
    (\I/\I^2)_x \xrightarrow{\delta_x} (i^*\Omega_{\A^N_Y/Y})_x \to (\Omega_{X/Y})_x\to 0
 \]
 The middle thing is free of rank $N-n$, and the two other things
 can be generated by $N-n$ and $n$ elements, respectively.  By the
 corollary, $(\Omega_{X/Y})_x$ is free, so $\Omega_{X/Y}$ is
 locally free of rank $n$.
 \end{proof}

 \begin{lemma}
 Let $f:X\to Y$ be a smooth $S$-morphism of smooth $S$-schemes.
 \marginpar{\xymatrix{X \ar[r]\ar[dr] & Y\ar[d]\\ & S}}
 Then the canonical sequence
 \[
    0\to f^*\Omega_{Y/S} \to \Omega_{X/S} \to \Omega_{X/Y}\to 0
 \]
 is exact and locally split.
 \end{lemma}
 \begin{proof}
 For all $x\in X$, the first exact sequence gives an exact
 sequence
 \[
    (f^*\Omega_{Y/S})_x\xrightarrow{\alpha_x} (\Omega_{X/S})_x \to (\Omega_{X/Y})_x
    \to 0
 \]
 in which the terms are locally free and the ranks add up, so by
 the corollary, $\alpha_x$ is injective, and the short exact
 sequence is split.
 \end{proof}

 \begin{proposition}[Another Jacobian Criterion]
 Let $f:X\to Y$ be an $S$-morphism of smooth $S$-schemes.
 Consider the conditions
 \begin{itemize}
 \item[(i)] $f$ is smooth
 \item[(ii)] the canonical map $f^*\Omega_{Y/S} \to \Omega_{X/S}$
 is locally left invertible
 \item[(iii)] for all $x\in X$, the map $(f^*\Omega_{Y/S})\otimes
 k(x) \to \Omega_{X/S}\otimes k(x)$ is injective
 \end{itemize}
 Then (i)$\Rightarrow$(ii)$\Rightarrow$(iii).  If $Y=\A^N_S$, then
 (iii)$\Rightarrow$(i) (actually, this assumption is unnecessary).
 \end{proposition}
 \begin{proof}
 (i$\Rightarrow$ii) follows from the most recent lemma.

 (ii$\Rightarrow$iii) obvious.

 (iii$\Rightarrow$i) Since $Y=\A^N_S$, $f$ is given by coordinates
 $\bar f_1,\dots, \bar f_N\in \Gamma(X,\O_X)$.  Then (iii) is equivalent to
 the $d\bar f_i$ being linearly independent in $\Omega_{X/S}\otimes
 k(x)$ for all $x\in X$.  We may assume that $X$ is affine.
 Embed $i:X\hookrightarrow \A^m_S$, let $\I$ be the sheaf of
 ideals, and let $r$ be the relative dimension of $X$ over $S$.
 Pick $x\in X$.

 Lift $\bar f_1,\dots,\bar f_N$ to functions $f_1,\dots, f_N$ on a
 neighborhood of $x$ in $\A^m_S$.  Also, let $h_1,\dots, h_{m-r}$
 be a generating set for $\I$ near $x$ (by the Jacobian
 Criterion).

 Consider the composite morphism
 \[\xymatrix{
    X\ar@{^(->}[r]^{\Gamma_f}& X\times_S Y \ar@{^(->}[r] &
    \A^m_S\times_S Y = \A^m_Y
 }\]
 Here $Y$ has relative dimension $N$ over $S$ and $X$ has
 codimension $m-r$ in $\P^m_S$, so the image of the composition
 has codimension $N+m-r$.

 Let $t_1,\dots, t_N$ be the coordinate functions on $Y=\A^N_S$.
 Then (locally) $X\subseteq \A^m_Y$ has local defining equations
 $\underbrace{h_1,\dots, h_{m-r}}_{\text{first factor}}$ and
 $\underbrace{t_1-f_1,\dots,t_N-f_N}_{\text{graph morphism}}$.
 \begin{claim}
 These functions satisfy the earlier Jacobian criterion for
 smoothness at the image, $x'$, of $x$ in $\A^m_Y$.
 \end{claim}
 \begin{proof}
 The images of $dh_1,\dots, dh_{m-r}$ form a basis for the kernel
 of $\Omega_{\A^m_S/S}\otimes k(x)\to \Omega_{X/S}\otimes k(x)$,
 which is isomorphic to the kernel of $\Omega_{\A^m_Y/Y}\otimes
 k(x') \to \Omega_{X\times_S Y/Y}\otimes k(x')$.  So we need to
 know whether $d(t_i-f_i)$ are linearly independent in
 $\Omega_{X\times Y/Y}\otimes k(x')$.  The $dt_i$ are zero ($t_i$
 are functions on $Y$), so we are looking at $-df_i$, which are
 linearly independent in $\Omega_{X/S}\otimes k(x)$ by (iii).
 \renewcommand{\qedsymbol}{$\square_{\text{Claim}}$}
 \end{proof}
 \end{proof}

 \begin{corollary}\label{lec29cor}
 Let $X$ be a smooth scheme over $S$, and let $f:X\to Y$ be an
 $S$-morphism, where $Y$ is smooth over $S$.  If $f$ is \'etale,
 then the natural map
 \[
    f^*\Omega_{Y/S}\to \Omega_{X/S}
 \]
 is an isomorphism, and the converse holds if $Y=\A^N_S$ (actually, it
 always holds).
 \end{corollary}
 \begin{proof}
 If $f$ is \'etale, then it is smooth, so
 \[
 0\to f^*\Omega_{Y/S}\to \Omega_{X/S}\to \Omega_{X/Y}\to 0
 \]
 is exact, but $\Omega_{X/Y}=0$, so we get our isomorphism.

 For the converse, use (ii)$\Rightarrow$(i) to get that $f$ is
 smooth, and then $\Omega_{X/Y}=0$ from the first exact sequence
 to get relative dimension 0.
 \end{proof}

 \begin{theorem}[``\'etale over affine'']\label{lec29etale}
 A morphism $f:X\to Y$ is smooth of relative dimension $n$ at a point
 $x\in X$ if an only if there is an open neighborhood $x\in
 U\subseteq X$ and an \'etale morphism $g:U\to \A^n_Y$ such that
 \[\xymatrix{
  U\ar[r]^g \ar[rd]_{f|_U} & \A^n_Y \ar[d]\\
  & Y
 }\]
 commutes
 \end{theorem}
 We are saying that locally, near $X$, it looks like the affine
 fibers.  previous corollary is like implicit function theorem.
}
 { \stepcounter{lecture}
 \setcounter{lecture}{30}
 \sektion{Lecture 30}

 \underline{Last time:} We did Corollary (\ref{lec29cor}).
 Compare this with the \emph{inverse} function theorem.\\
 We also did Theorem (\ref{lec29etale})
 \begin{proof}
 ($\Leftarrow$) $f|_U$ is a composition of smooth morphisms, so it
 is smooth, of relative dimension $0+n=n$.\\
 ($\Rightarrow$)  Pick local sections $g_1,\dots, g_n$ of $\O_X$
 near $x$ such that $dg_1,\dots,dg_n$ form a basis for
 $\Omega_{X/Y}$ near $x$.  Let $U$ be an open neighborhood of $x$,
 where this happens.  Let $g:U\to \A^n_Y$ be the map given by
 $(g_1,\dots, g_n)$ (think of the $g_i$ as maps $X\to \A^1_Y$).
 Then note that $\Omega_{\A^n_Y/Y}$ is free with basis
 $dx_1,\dots, dx_n$.  Then $g^* \Omega_{\A^n_Y/Y}$ is free with
 basis $g^*dx_1,\dots, g^*dx_n$, but these are just $dg_1,\dots,
 dg_n$ because $g^*dx_i=d(g^*x_i)=dg_i$.  So
 \[
    g^*\Omega_{\A^n_Y/Y} \to \Omega_{X/Y}
 \]
 is an isomorphism, so $g$ is \'etale.
 \end{proof}

 \marginpar{Riemann-Roch (for curves)}

 \underline{Riemann-Roch (for curves):}  For this section,
 \emph{curves} (varieties of dimension 1, thus over an
 algebraically closed field) are assumed to be projective and
 non-singular (= smooth).  If $X$ is such a curve, we have that
 $p_a(X)=p_g(X)$; call this value $g(X)$ or the \emph{genus}.  We
 have that $g\ge 0$ (since one of these is the dimension of some
 $H^1$), and all integers $\ge 0$ occur.

 \underline{Divisors and line sheaves on curves:} Weil divisors
 and Cartier divisors are the same thing.  Recall that a Weil
 divisor is of the form $\sum n_P P$, where the $P$ are closed points,
 and $n_P$ are integers, almost all of which are zero. A Cartier
 divisor is a pull-back for a dominant (i.e. finite) morphism of
 curves.  If $f\in K(X)^{\times}$ and $P\in X$ a closed
 point\footnote{We're usually using closed points ... if I don't say it, I probably meant
 to.}.  Then $\O_{X,P}$ is a dvr with fraction field $K(X)$, so we
 can get $n_P=v(F)\in \mathbb{Z}$.  note that $n_P=0$ for almost
 all $P$.  Thus, we get a principal divisor $(f)=\sum n_P P$.  Two
 divisors are equivalent ($D_1\sim D_2$) if the difference is
 principal.  We get a homomorphism $K(X)^{\times} \to \div X$.
 The cokernel is the group of divisor classes, $\cl X$.  For all
 $D$, we can define a line sheaf $\O(D)$ (or $\L(D)$), defined as
 a subsheaf of the constant sheaf $K(X)$.  Then $1\in K(X)$
 corresponds to a non-zero rational section of $\O(D)$, called
 $1_D$.

 Given a rational section $s\not=0$ of a line sheaf $\L$, form a
 divisor as follows:  Given $P\in X$, let $s_0$ be a local
 generator for $\L$ near $P$, then $s/s_0\in K(X)^{\times}$, so
 let $n_P=v(s/s_0)$.  This is well-defined, so define $(s)=\sum
 n_P P$.  For two rational sections $s,t\not=0$ of $\L$, $(s)-(t)$ is
 a principal divisor, equal to the principal divisor $(s/t)$ (note
 that $s/t$ is a non-zero rational function).  More generally, if
 $s_1,s_2$ are rational sections of $\L_1$ and $\L_2$, resp., then
 $s_1\otimes s_2$ is a non-zero rational section of $\L_1\otimes
 \L_2$, and $(s_1\otimes s_2)=(s_1)+(s_2)$.  Also, recalling $1_D$
 as a rational section of $\O(D)$, we have $(1_D)=D$.  Recall also
 that $\O(D_1+D_2)\cong \O(D_1)\otimes \O(D_2)$, so we get an
 isomorphism $\cl X \xrightarrow{\sim} \pic X$, taking $D$ to
 $\O(D)$.  The reverse map is obtained by taking any non-zero
 rational section, and taking its divisors.  All of this is
 functorial.

 A divisor $D=\sum n_P P$ is \emph{effective} if $n_P\ge 0$ for
 all $P$.  Let $s\not=0$ be a rational section of $\L$, then $(s)$
 is effective if and only if $s\in \Gamma(X,\L)$.

 \begin{definition}
 Let $D$ be a divisor on $X$.  Then the \emph{complete linear
 system}, written $|D|$, is the set of effective divisors linearly
 equivalent to $D$.
 \[
    |D|\buildrel{\leftrightarrow}\over{1-1} (H^0(X,\O(D))\smallsetminus \{0\})/k^*
 \]
 given by $(s)\leftarrow s$, and $E=D+(f)\mapsto f$ (in $K(X)$,
 this is $\{f\in K(X)^{\times}: D+(f)\ge 0\} $).  Then $|D|$ has an
 algebraic structure of dimension
 $\underbrace{h^0(X,\O(D))}_{=:l(D)}-1$  A \emph{linear system}
 associated to $D$ is a subset of a complete linear system
 corresponding to a linear subspace of
 $H^0(X,\O(D))\smallsetminus\{0\}$.  $D_1\sim D_2$ implies that
 $l(D_1)=l(D_2)$.
 \end{definition}
 All of this behaves nicely with respect to pull-backs (via
 dominant morphisms of curves).

 \begin{definition}
 If $D=\sum n_P P$, then $\deg D=\sum n_P$.  If $D$ is principal,
 say $(f)$, then $\deg D=0$.
 \end{definition}
 \begin{proof}
 If $\phi:X\to Y$ is a finite (= dominant) morphism of curves and
 $E$ is a divisor on $Y$, then
 \begin{align*}
    \deg(\phi^* D) &= \underbrace{(\deg \phi)(\deg
    E)}_{[K(X):K(Y)]}\\
    &= k- \text{length of fibers over closed points}
 \end{align*}
 The function $f$ gives a finite map (provided $f\not\in k$; $f\in k\Rightarrow (f)=0$, so $\deg
 D=0$) $\phi:X\to \P^1$ and $f=\phi^*z$ so
 $(f)=(\phi^*z)=\phi^*(z) = \phi^*([0]-[\infty])$, so $\deg D =
 (\deg \phi)\underbrace{(\deg (z))}_0 = 0$.
 \end{proof}
 Thus, the degree gives well-defined group homomorphisms:
 \[\xymatrix{
    \cl X\ar[r]\ar@{}[d]|{\parallel}^{\wr} & \Z\\
    \pic X\ar[r] & \Z
 }\]

 Since $X$ is a curve, $\omega_X=\wedge^1 \Omega_{X/k} =
 \Omega_{X/k}$.  Any divisor of $X$ associated to $\omega_X$ is
 called ``the'' canonical divisor of $X$, written $K$.

 \begin{theorem}[Riemann-Roch for curves]
 Let $X$ be a curve over $k$, let $K$ be a canonical divisor of
 $X$, and let $D$ be any divisor on $X$.  Then
 \[
    l(D)-l(K-D) = \deg D +1-g.
 \]
 where $g=g(X)$ (Note that the LHS is  independent of the choice
 of $K$)
 \end{theorem}
 \begin{proof}
 Note that $l(D)=h^0(X,\O(D))$ and
 \begin{align*}
 l(K-D)&=h^0(X,\O(K-D))\\
 &=h^0(X,\omega_X\otimes \O(D)^{\vee})\\
 &=_{\text{duality}} h^1(X,\O(D))
 \end{align*}
 so LHS $= \chi (\O(D))$

 Now show:
 \begin{itemize}
 \item[(1)] $\chi(\O(D))=\deg D$ is independent of $D$
 \item[(2)] compute $\chi(\O(D))$ for one particular $D$
 \end{itemize}
 Start with (2): if $D=0$, then $\O(D)=\O_X$, and
 $\chi(\O_X)=h^0(X ,\O_X)-h^1(X,\O_X)=1-g$. So $\chi(\O(D))=\deg D
 = 1-g$ if $D=0$.

 Now we'll do (1): $\chi(\O(D+P))=\underbrace{\deg(D+P)}_{\deg
 D+1} = \chi(\O(D)) - \deg D$ for all $D,P$.  That is,
 $\chi(\O(D+P))=\chi(\O(D))+1$

 Regard $P$ as a reduced closed subscheme of $X$; $i:P\to X$.  Its
 sheaf of ideals is $\O(-P)$, so
 \[
    0\to \O(-P) \to \O_X \to \underbrace{k(P)}_{\text{skyscraper}} \to 0
 \]
 is exact.  Tensor with $\O(D+P)$, and note that $k(P)\otimes
 \underbrace{\O(D+P)}_{\txt{locally triv\\ near $P$}}\cong k(P)$
 (non-canonically),so
 \[
    0\to \O(D) \to \O(D+P) \to k(P) \to 0
 \]
 is exact, so
 \[
    \chi(\O(D+P)) = \chi(\O(D))+\chi(k(P)).
 \]
 \begin{align*}
 \chi(k(P) &= h^0(X,i_*\O_P) - h^1(X,i_*\O_P)\\
 &= h^0(P,\O_P)-h^1(P,\O_P)\\
 &= 1-0 = 1
 \end{align*}
 \end{proof}

 What happens if $D=K$?  $\underbrace{l(K)-l(0)}_{-\chi(\O_X)=g-1}=\deg(K)+1-g$,
 so we have that $\deg(K)=2g-2$.

 To compute $K$, pick any $f\in K(X)\smallsetminus k$.  Compute
 $(df)$ (where $df$ is the rational section of $\Omega_{X/k}$).

 \begin{lemma}
 Let $D$ be an effective divisor.  Then $\deg D\ge 0$, and if
 $\deg D=0$, then $D=0$.
 \end{lemma}

 \begin{lemma}
 Let $D$ be a divisor on $X$ with $l(D)\not=0$.  Then $\deg D\ge
 0$, and if $\deg D=0$, then $D$ is principal.
 \end{lemma}
 \begin{proof}
 Apply the earlier lemma to some effective divisor $E$, linearly
 equivalent to $D$ (such an $E$ exists by the assumption).
 \end{proof}

 \begin{proposition}
 If $X$ is a curve of genus 0, then $X\cong \P^1$.
 \end{proposition}
 \begin{proof}
 let $P\not= Q$ be two distinct points on $X$, and apply RR to
 $D=P-Q$.  Then
 $l(P-Q)-l(\underbrace{K-P+Q}_{\deg =
 -2})=\underbrace{\deg(P-Q)}_0+\underbrace{1-g}_1$.  So
 $l(P-Q)=1$, so since $\deg(P-Q)=0$, $P-Q$ is principal.  Then $f$
 gives a morphism $X\to \P^1$, which is thus an isomorphism.
 \end{proof}
}
 { \stepcounter{lecture}
 \setcounter{lecture}{31}
 \sektion{Lecture 31}

 Comments on Homework 12:
 \begin{itemize}
 \item[(i)]Common mistake (\#1) was failing to realize the difference between a closed point and a rational point.  $X$
 defined by $y^2=x^p-t$.  There is a point in $X$ corresponding to
 the point $(t^{1/p},0)$, namely $P= \spec k[x,y]/(y,x^p-t)$. Then
 $k(P)=k(t^{1/p})$.  This is a closed, but not rational point.
 \item[(ii)] (\#2) This can be used to \emph{define} \'etale
 (provided you rephrase it not to require that
 $k(x)\hookrightarrow \hat \O_x$
 \item[(iii)](\#3) This is an example of the general principal
 that irreducible components in $\spec \hat \O_{X,x}$ can be
 separated in an \'etale cover.
 \end{itemize}

 Back to Riemann-Roch.  Again, curves are smooth and projective
 over an algebraically closed field $k$, until we say otherwise.

 Recall that we have $\deg:\pic X\to \Z$ well-defined surjective group
 homomorphism.

 \begin{definition}
 $\pic^0 X =$ the kernel of $\deg$, which is $\{\L\in \pic X|
 \deg(\L)=0\}$.
 \end{definition}
 If $X$ has genus 0, then $X\cong \P^1$, so $\pic X=\Z$ (via
 $n\leftrightarrow \O(n)$, so $\pic^0 X$ is trivial.  In genus
 $>0$, $\pic X$ is non-trivial.  Pick two distinct points $P,Q$.
 Then $\O(P-Q)\not\cong \O_X$ (lest $X\cong \P^1$), so $\pic^0
 X\not=0$.

 Genus 1 curves:  Let $X$ be a curve of genus 1 and fix $P_0\in
 X$.  Then we have a map $\phi:X(k)\to \pic^0 X$ given by
 $P\mapsto \O(P-P_0)$.

 \begin{remark}
   The canonical divisor $K$ is $\sim 0$.  By RR applied to the
   divisor 0, $l(0)-l(K)=1-l(k)=0$.  So $l(K)=1$.  But $\deg
   K=2g-2 =0$, so $K\sim 0$ by an earlier lemma.
 \end{remark}

 \begin{proposition}
   $\phi$ as above is a bijection.
 \end{proposition}
 \begin{proof}
 1-1: $\phi(P)=\phi(Q) \Leftrightarrow P-P_0\sim Q-P_0 \sim P-Q
 \sim 0$, which cannot happen (lest $X\cong \P^1$)

 Onto: Pick $\L\in \pic^0 X$, and let $D$ be a corresponding
 divisor.  Apply RR to $D+P_0$:
 \[
    l(D+P_0)-l(K-D-P_0) = 1+1-g =1.
 \]
 so $l(D+P_0)=1$, so $D+P_0\sim Q$, for an effective divisor $Q$.
 Then $D\sim Q-P_0$, so $\L=\phi(Q)$
 \end{proof}

 \begin{corollary}
 There is a canonical (depending on choice of $P_0$) group
 structure on $X(k)$ given by $\phi$ and the group structure on
 $\pic^0 X$.
 \end{corollary}

 \begin{definition}
 An \emph{elliptic curve} is a smooth curve of genus 1 over a
 field $k$ together with a choice of rational point $P_0\in X(k)$.
 \end{definition}

 Note that $\phi(P_0)=\O(P_0-P_0)=\O_X$, so $P_0$ is the identity
 element.  Let's call it 0 from now on.

 Apply RR to some multiples of $[0]$:
 \[
    l([0])-\underbrace{l(K-[0])}_0 = 1+1-g =1 = \dim
    (H^0(X,\O([0]))=k)
 \]
 where we think of $\O([0])$ as a subsheaf of $K(X)$. Similarly
 \[
    l(2[0])=2
 \]
 A basis of $H^0(X,\O(2[0]))$ is $1,x$.\\
 $l(3[0])=3$ ... basis is $1,x,y$.\\
 $l(4[0])=4$ ... basis is $1,x,y,x^2$.\\
 $l(5[0])=5$ ... basis is $1,x,y,x^2,xy$.\\
 $l(6[0])=6$ ... basis is $1,x,y,x^2,xy$, ($x^2$ or $y^2$).\\

 So $1,x,y,x^2,xy,x^2,y^2$ are linearly dependent over $k$, and the
 resulting function $f$ \emph{does} involve $y^2$ and $x^3$.  By
 Exercise 1.3, $X\smallsetminus\{0\}$ is affine, and $x,y$ generate
 its affine ring: ($H^0(X,\O(n\cdot[0]))$ has basis
 $1,x,y,x^2,xy,x^3,x^2y, ...$), so $k[x,y]$ maps onto the affine
 ring, and the kernel is $(f)$.  So $X$ is the projective closure
 of $\spec k[x,y]/(f)$ in $\P^2$.  So it is a non-singular cubic in
 $\P^2$, and it has only one point at infinity, namely 0.

 If the characteristic of $k$ is not 2 or 3, then we can do a
 linear coordinate change so that $f$ is $y^2=4x^3+ax+b$ with
 $a,b\in k$.  This is called Weierstrass form.

 Let $P=(x_0,y_0)\in X$, and let $P'=(x_0,-y_0)$.  Then
 $[P]+[P']\sim 2\cdot[0]$ because $(x-x_0) = [P]+[P']-2[0]$.  So
 $P'=-P$ in the group.  Also, if $P,Q,R$ are colinear, then
 \[
    [P]+[Q]+[R]-3[0] = \text{the principal divisor } (\text{equation of the line})
 \]
 so $P+Q+R=0$ in the group law, so $Q+R=P'$.

 \marginpar{insert picture around here}

 \emph{Varieties over arbitrary fields:} now $k$ may not be
 algebraically closed.

 \underline{Examples:}
 \begin{itemize}
 \item[(i)] Let $k$ be a field.  On $\A^1_k=\spec k[x]$, you have
 the generic point, plus the closed points;  also, $\{\text{closed
 points}\} \leftrightarrow \{\text{irreducible monics}\}/\text{action of Aut}_k(\bar k)$
 \item[(ii)] $\sqrt{2}\in \A^1_{\Q}$, only it is paired with
 $-\sqrt{2}$. $\spec \Q[x]/(x^2-2)$.  $k{P}=\Q[\sqrt{2}]$.  it is
 a closed point, but not a rational point.
 \item[(iii)] Let $k=\mathbb{F}_p(t)$.  Then $\sqrt[p]{t}\in
 \A^1_k$, realized as $\spec k[x]/(x^p-t)$.  When we base change
 to $\bar k$, we get $\spec \bar k[x]/(x-t^{1/p})^p$, which is not
 reduced.\\
 So the point $t^{1/p}$ in $\A^1_k$ is not geometrically
 reduced.\\
 Likewise $\spec \Q[x]/(x^2-2)$ becomes $\bar
 \Q[x]/(x-\sqrt{2})(x+\sqrt{2})$ which is reducible, so our point
 is not geometrically irreducible. (See Exercise II.3.15)
 \end{itemize}
 What about $\A^1_k$?  This situation is similar:
 \[
    \{\text{closed points}\} ``='' \bar k^n/(\text{action of Aut}_k\bar k)
 \]

 \emph{Grothendieck Topologies:} (reference: Vistoli, Notes on
Grothendieck topologies fibered categories, and descent theory\footnote{This is on the
arXiv.})

 General idea: If you had a smooth map of complex manifolds $f:X\to
 Y$, you can work near $P\in X$ by finding a local section of $f$
 near $f(P)$.  You can't necessarily do this with schemes because
 the Zariski topology is too coarse (exceptions: $\P(\E)$, etc.).
 One way to appoximate the classical topology is to work in $\hat
 \O_{X,x}$. Then you have
 \[\xymatrix{
 \spec \hat \O_{X,x} \ar[d]\\
 \spec \hat \O_{Y,f(x)} \ar@{.>}@/^/[u]^{\exists}
 }\]
 The image is cut out by $f_1,\dots, f_r\in \hat \m_x/\hat\m_x^2$.
 These can be lifted to $\mod \hat\m^2$ to $f_1,\dots, f_r\in
 \m_x\smallsetminus \m_x^2$.  Lift to $f_1,\dots, f_r\in \O_X(U)$
 for some open neighborhood.  Take the closure $Z(f_1,\dots, f_r)$
 to get $Z\subseteq X$ closed subscheme.\\
 $Z$ may have many sheets over $Y$, and you can't keep one and
 throw out the others.  But you do have $f|_Z$ is \'etale at $x$
 ($x=P$).  You can base change to $Z:$
 \[\xymatrix{
 X' = X\times_Y Z\ar[d]\\
 Z\ar@/^/[u]
 }\]
 Of course $f|_Z$ is not \'etale everywhere, but it is in a
 Zariski-open neighborhood of $x$.\\
 This is called working locally in the \'etale topology.

 \def\C{\underline{\mathcal{C}}}

 \begin{definition}
 Let $\C$ be a category.  A \emph{Grothendieck topology} on $\C$
 is, for each object $U$, a collection of covers of $U$, where a
 cover is a set of morphisms $\{U_i\to U\}$ in $\C$ satisfying:
 \begin{itemize}
 \item[(i)] if $V\to U$ is an isomorphism in $\C$, then $V\to U$
 is a cover.
 \item[(ii)] if $\{U_i\to U\}$ is a covering and $V\to U$ is a
 morphism, then $U_i\times_U V$ exists in $\C$, and $\{U_i\times_U
 V\}$ is a covering (of $V$)
 \item[(iii)] if $\{U_i\to U\}$ is a covering, and $\{V_{i,j}\to
 U_i\}_{j\in J}$ is a covering for each $i$, then the composites
 $\{V_{i,j}\to U\}$ form a covering.
 \end{itemize}
 \end{definition}

 \underline{Example:} Let $\C=$ category of schemes.  Given $U\in
 \C$, a covering is a collection $\{U_i\to U\}$ of open immersions
 whose images cover $U$.
 \begin{itemize}
 \item[(i)] ok
 \item[(ii)] if $\{U_i\}$ cover $U$, then $\{f^{-1}(U_i)\}$ cover
 $V$.
 \item[(iii)] trivial
 \end{itemize}

 This is how the Zariski topology is realized as a Grothendieck
 topology.
 \begin{definition}
 The Grothendieck topology on $U$ is the collection of open
 covers.
 \end{definition}
 \begin{remark}
 Grothendieck topologies versus the earlier kind:
 \begin{itemize}
 \item open sets are now morphisms
 \item intersections are now fibered products over $U$
 \item unions are now obsoleted by looking at covers
 \end{itemize}
 \end{remark}

 \begin{definition}
 A \emph{site} is a category with a Grothendieck topology.
 \end{definition}

 Another example of a site: fix a field $k$.  Let $\C=$ category
 of schemes of finite type over $k$.  Given $U\in \C$, a covering
 of $U$ is a finite collection $\{U_i\to U\}$ of \'etale morphisms
 such that the union of the images is all of $U$.  This ``is'' the
 \'etale site.

 Exercise 10.5 says: if a (Zariski) sheaf is locally free as a
 sheaf on the \'etale topology, then it is locally free under the
 Zariski topology.
}

\end{document}
