 \stepcounter{lecture}
 \setcounter{lecture}{3}
 \sektion{Lecture 3}

\def\A{\mathcal{A}}

For effaceable functors, see the book.

Recap of what we've done:
\begin{list}{}{}
 \item[(1)] Given a covariant left exact functor from an abelian
category with enough injectives to an abelian category, there are
right derived functors.
 \item[(2)] A Cohomology should have some desirable properties.
\end{list}

\begin{definition}
An abelian group $A$ is \emph{divisible} if for all $a\in A$ and
non-zero $n\in \Z$, there is some $a'\in A$ such that $na'=a$.
\end{definition}
\marginpar{Injective\\ $A$-modules} For example, 0,
$\mathbb{Q},\mathbb{Q/Z}, \mathbb{R}, \mathbb{R/Z}, \dots$ are
divisible groups.

\begin{lemma} An abelian group is injective if and only if it is
divisible.
\end{lemma}
\begin{proof}
($\Rightarrow$) Let $A$ be an injective group, and let $a\in A,
0\not=n\in \Z$, then
\[\xymatrix{
    0 \ar[r] & \Z \ar[r]^{\cdot n} \ar[d]^{\phi} & \Z \ar@{.>}[dl]^{\psi} \\
    & A
}\]
 where $\phi(1)=a$.  Then by injectivity, $\psi$ exists, and
 $n\cdot \psi(1)=a$.  Thus, $A$ is divisible.

($\Leftarrow$) Let $M'\subseteq M$ be abelian groups, and let
$\phi:M'\to A$ be a homomorphism, where $A$ is divisible.
\[\xymatrix{
    0 \ar[r] & M' \ar[r] \ar[d]^{\phi} & M  \\
    & A
}\]
 Pick $x\in M\smallsetminus M'$, and let $d$ be a non-negative
 integer generator of the ideal $\{n\in \Z| nx\in M'\}$.  If
 $d=0$, then $\langle M',x \rangle \cong M'\oplus \Z$ and we can
 extend $\phi$ to $\langle M',x \rangle \to A$ by setting
 $\phi(x)=0$.  If $d\not=0$, then pick $a\in A$ such that
 $da=\phi(dx)$.  Then we can extend $\phi$ to $\langle M',x
 \rangle$ by setting $\phi(x)=a$.  By the standard Zorn's Lemma argument, we
 can extend $\phi$ to a map $M\to A$, so $A$ is injective.
\end{proof}

\begin{lemma}
The category $\Ab$ of abelian groups has enough injectives.
\end{lemma}
\begin{proof}
For an abelian group $A$, we define the \emph{dual} $\hat
A=\Hom(A,\mathbb{Q/Z})$.  Then a homomorphism $f:A\to B$ has a
dual $\hat f: \hat B\to \hat A$.  We have a natural map $A\to
\hat{\hat A}= \Hom(\Hom(A,\mathbb{Q/Z},\mathbb{Q/Z}), a\mapsto
[\phi\mapsto \phi(a)]$
\begin{itemize}
\item[\underline{Claim:}] This map is one to one.

\noindent Indeed, if it were not, then there would be a non-zero
$a\in A$ such that $\phi(a)=0$ for all $\phi \in \hat A$.  But
since $\mathbb{Q/Z}$ is injective, we have
\[\xymatrix{
 0\ar[r] & \langle a \rangle \ar[r] \ar[d]_{\phi_0} & A\ar@{.>}[dl]\\
 & \mathbb{Q/Z}
}\] For any $\phi_0:\langle a \rangle \to \mathbb{Q/Z}$ we have an
extension $\phi: A\to \mathbb{Q/Z}$ so it suffices to find
$\phi_0:\langle a \rangle \to \mathbb{Q/Z}$ with
$\phi_0(a)\not=0$.  But either $\langle a \rangle \cong \Z$, in
which case we can send $a\mapsto 1/2 \in \mathbb{Q/Z}$, or
$\langle a \rangle \cong \Z/n\Z$, in which case we can send
$a\mapsto 1/n\in \mathbb{Q/Z}$.  Thus, the claim is true.
\end{itemize}
So $A\hookrightarrow \hat{\hat A}$.  Next, there is a surjection
$\bigoplus_{i\in I} \Z \to \hat A \to 0$ for some index set $I$.
By taking duals, we get a map
 \begin{align*}
 \hat{\hat A} \to \widehat{\bigoplus_{i\in I}\Z} &= \Hom(\bigoplus \Z,
 \mathbb{Z/Q})\\
 &= \prod \Hom(\Z,\mathbb{Q/Z})\\
 &= \prod \mathbb{Q/Z}
 \end{align*}
 which is divisible, and therefore injective.  Also, the map $\hat{\hat A} \to
 \widehat{\bigoplus_{i\in I}\Z}$ is one to one because the dual of
 a surjective map is one to one.  Thus, $A\hookrightarrow
 \hat{\hat A} \hookrightarrow \prod \mathbb{Q/Z}$.
\end{proof}

If $A$ is a ring, $T$ is an abelian group, and $X$ is an
$A$-module, then we can put an $A$-module structure on
$\Hom_{\Z}(X,T)$ by setting $a\cdot \phi = (x\mapsto \phi(a\cdot
x))$ for each $a\in A$

\begin{lemma}
If $A$ is a ring, $X$ is an $A$-module, and $T$ is an abelian
group, then
\begin{align*}
\Hom_{\Z}(X,T) & \stackrel{\sim}{\rightarrow}
\Hom_A(X,\Hom(A,T))\\
\phi &\mapsto \ (x\mapsto (a\mapsto \phi(ax)))
\end{align*}
as abelian groups (also as $A$-modules).
\end{lemma}
\begin{proof}
exercise.
\end{proof}

\begin{lemma}
If $T$ is and injective abelian group, then $\Hom_{\Z}(A,T)$ is an
injective $A$-module.
\end{lemma}
\begin{proof}
Given $A$-modules $0\to X\to Y$, we have \marginpar{Exercise: show
that the diagram commutes}\[\xymatrix{
    \Hom_A(Y,\Hom_{\Z}(A,T)) \ar[r] \ar[d]^{\wr} & \Hom_A(X,\Hom_{\Z}(A,T))
    \ar[r] \ar[d]^{\wr} & 0\\
    \Hom_{\Z}(Y,T)\ar[r] & \Hom_{\Z}(X,T) \ar[r] & 0
}\] with the top row exact.
\end{proof}

\begin{theorem}
The category of $A$-modules has enough injectives.
\end{theorem}
\begin{proof}
Let $M$ be an $A$-module.  Embed it into an injective abelian
group $T$, so $f:M\hookrightarrow T$.  Define $g:M\to
\Hom_{\Z}(A,T)$ by $g(m)=(a\mapsto f(am))$.  If $m\not=0$, then
$g(m)(1)=f(1\cdot m)\not=0$, so $g$ is one to one.
\end{proof}

\begin{theorem}
If $(X,\O_X)$ is a ringed space, then $\Mod(X)$ has enough
injectives.
\end{theorem}
\begin{proof}
Let $\F$ be a sheaf of $\O_X$-modules.  For each $x\in X$, $\F_x$
is an $\O_{X,x}$-module, and can therefore be embedded into an
injective module $I_x$.  Define $\J = \prod_{x\in X}j_*\I_x$,
where $\I_x=I_x$ as a sheaf at $x$ and $j:\{x\}\hookrightarrow X$
is the inclusion, so $j_*\I_x$ is a skyscraper sheaf.

Then we have that
\[
 \Hom(\F,\J) = \prod_{x\in X} \Hom(\F,j_*\I_x) = \prod_{x\in
 X}\Hom_{\O_{X,x}}(\F_x,I_x).
\]
By taking an injection from each factor, we get a map $\F\to \J$
which must be an injection because it is an injection on all
stalks.

Finally, we must show injectivity of $\J$.  Given any $\G_1\to
\G_2$, we have that
\begin{align*}
 \Hom_{\O_{X,x}}(\G_{2,x},I_x) &\twoheadrightarrow
 \Hom_{\O_{X,x}}(\G_{1,x},I_x) \\
 \Hom(\G_{2,x},j_*\I_x) &\twoheadrightarrow
 \Hom(\G_{1,x},j_*\I_x) \\
 \Hom(\G_{2,x},\J) &\twoheadrightarrow
 \Hom(\G_{1,x},\J) & \text{(prod of surj is surj)} \\
\end{align*}
\end{proof}

\begin{corollary}
If $X$ is a topological space, then $\Ab(X)$ has enough
injectives.
\end{corollary}
\begin{proof}
Make $X$ into a ringed space by setting $\O_X$ to the constant
sheaf $\Z$.  Then $\Mod((X,\O_X))=\Ab(X)$.
\end{proof}

\begin{definition}
Let $X$ be a topological space, then the cohomology functors
$H^{\cdot}(X,-)$ are the right derived functors of
$\Gamma(X,-):\Ab(X)\to \Ab$. \marginpar{$H^{\cdot}(X,-)$ defined}
\end{definition}

Note: We use injective filtrations in $\Ab(X)$ rather than
$\Mod(X)$ because sometimes we will want to consider sheaves of
abelian groups which are not $\O_X$-modules.  The following
theorem says that we get the same cohomology as we would have if
we used $\Mod(X)$ injectives.

\begin{theorem}[*] If $(X,\O_X)$ is a ringed space (e.g. a
scheme), then the right derived functors $\Mod(X)\to \Ab$
associated to $\Gamma(X,-)$ coincide with $H^{\cdot}(X,-)$,
restricted to $\Mod(X)$.
\end{theorem}

Recall that if $X$ is a topological space and $\F$ is a sheaf (of
abelian groups) on $X$, then we say $\F$ is flasque if
$\rho_{UV}:\F(U)\to \F(V)$ is surjective for all open sets
$V\subseteq U$.

\begin{lemma}
Let $(X,\O_X)$ be a ringed space.  Then any injective element of
$\Mod(X)$ is flasque. \marginpar{Injective sheaves are flasque}
\end{lemma}
\begin{proof}
Let $V\subseteq U$ be open.  Regard $\O_U$ and $\O_V$ as sheaves
on $X$ by extending by zero (see exercise II.1.19b), so $W\mapsto
\O_U(W)$ if $W\subseteq U$ and $W\mapsto 0$ otherwise.

We have a map $\O_V\to \O_U$ as $\O_X$-modules, which is injective
on stalks, and therefore injective.  Now look at
\[
 \Hom_{\O_X}(\O_U,\F) \to \Hom_{\O_X}(\O_V,\F) \to 0.
\]
The sequence is exact by injectivity of $\F$.  But we have a
natural isomorphism $\Hom_{\O_X}(\O_U,\F)\cong \F(U)$ (since a
morphism is determined by the image of $1\in \O_U(U)$), so
$\F(U)\to \F(V)$ is surjective.
\end{proof}

\begin{proposition}
Let $X$ be a topological space and $\F$ a flasque sheaf on $X$,
the $\F$ is acyclic.
\end{proposition}
\begin{proof}
 Embed $\F$ into an injective sheaf and let $\G$ be the quotient.
 \[
    0\to \F\to \I \to G\to 0.
 \]
Then $\F$ and $\I$ are flasque, so $\G$ must also be flasque
(Exercise in chapter II), and we get
\[
    0\to \Gamma(X,\F)\to \Gamma(X,\I)\to \Gamma(X,\G)\to 0
\]
from the same exercise.  Therefore, we get the long exact sequence
\[\begin{tabular}{lllll}
 0 &$\to \Gamma(X,\F) $&$\to \Gamma(X,\I) $&$\twoheadrightarrow
 \Gamma(X,\G) $&$\xrightarrow{0}$\\
 &$\xrightarrow{0} H^1(X,\F) $&$\to 0 $&$\to H^1(X,\G)$&$ \to$ \\
 &$\xrightarrow{0} H^2(X,\F) $&$\to 0 $&$\to H^2(X,\G)$&$ \to \cdots$ \\
\end{tabular}\]
where the middle column is 0 because $\I$ is injective.  So we
have that $H^1(X,F)=0$ and $H^i(X,\F)=H^{i-1}(X,\G)$.  But since
$\G$ is also flasque, we have that $H^{i-1}(X,\G)=0$ by induction.
\end{proof}

So flasque sheaves can be used to compute cohomology.

\begin{proof}[Proof of (*)]
Let $\F\in \Mod(X)$ and let $0\to \F\to \I^{\cdot}$ be an
injective resolution (in $\Mod(X)$).  Then
$h^i(\Gamma(X,\I^{\cdot}))$ computes the right derived functors of
$\Gamma(X,-):\Mod(X)\to \Ab$.  But the $\I^i$ are flasque and
therefore acyclic, and so $h^i(\Gamma(X,\I^{\cdot})) = H^i(X,\F)$.
\end{proof}

For next time, read page 209.
