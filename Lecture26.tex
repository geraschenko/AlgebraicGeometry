 \stepcounter{lecture}
 \setcounter{lecture}{26}
 \sektion{Lecture 26}

 We were showing: A projective morphism is flat if and only if the
 Hilbert polynomial is independent of $y\in Y$.

 \underline{The Setup:} $T=\spec A$, where $A$ is local noetherian
 domain.  $X=\P^n_T$, and $\F$ is a coherent sheaf on $X$.  It will suffice
  to show that TFAE:
 \begin{itemize}
 \item[(i)] $\F$ is flasque over $T$.
 \item[(ii)] $H^0(X,\F(m))$ if a free $A$-module (of finite rank)
 for all $m\gg 0$.
 \item[(iii)] The Hilbert polynomial, $P_t$ of $\F_t$ on $X_t$ is
 independent of $t\in T$.
 \end{itemize}
 We have already shown that (i)$\Rightarrow$ (ii)

 (ii)$\Rightarrow$ (i): Choose $m_0$ such that $H^0(X,\F(m))$ is
 free for all $m\ge m_0$.  Let $M=\bigoplus_{m\ge m_0}
 H^0(X,\F(m))$.  Then $M\cong \Gamma_*(\F)_{m\ge m_0}$, so $\tilde
 M \cong \widetilde{\Gamma_*(\F)}_{m\ge m_0} \cong \F$.  Since $M$
 is free, it's flat over $A$, so $\F$ is flat over $A$ (on open
 affines it's $\sim$ of localizations of $M$).

 \begin{lemma}
 Let $T,A,X$ as above, and let $\F$ be coherent on $X$.  Then
 $H^0(X_t,\F_t(m)) \cong H^0(x,\F(m))\otimes_A k(t)$ for all $t\in
 T$ and $m\gg_t 0$.
 \end{lemma}
 \begin{proof}
 We may assume that $t$ is the closed point of $T$ (replace $T$ with $T':=\spec
 \O_{T,t}$ and note that $T'$ is flat over $T$, then LHS unaffected by base
 change and RHS is tensored with $A':=\O_{T,t}$).  Given a finite
 generating set for the maximal ideal of $A$, we have an exact
 sequence
 \[
    A^n\to A\to k(t) \to 0 \tag{$\ast$}
 \]
 so we can get the exact sequence ``$(\ast)\otimes_A \F$'':
 \[
    \F^n\to \F\to \F_t \to 0.
 \]
 Therefore,
 \[\xymatrix{
 H^0(X,\F(m)^n) \ar[r] \ar[d]^{\wr} & H^0(X,\F(m)) \ar[r] \ar@{}[d]|{\parallel} & H^0(X_t,\F_t(m))
 \ar[r] \ar@{.>}[d]^{\wr}_{\exists} & 0\\
H^0(X,\F(m))^n \ar[r] & H^0(X,\F(m)) \ar[r] &
H^0(X,\F(m))\otimes_A k(t) \ar[r] &
 0\\
 }\]
 The top row is exact by some homework exercise for $m\gg 0$, and
 the bottom row is exact for all $m$ because it is $(\ast)\otimes_A
 H^0(X,\F(m))$ (and tensor is right exact).  And the diagram
 commutes. Therefore, you get the isomorphism on the right.
 \end{proof}

 (ii)$\Rightarrow$(iii): The Lemma implies that the rank of
 $H^0(X,\F(m))$ is $\dim_{k(t)} H^0(X-t,\F_t(m))$ for all $t\in T$
 and $m\gg_t 0$.  The RHS is $P_t(m)$ for all $m\gg 0$, and the
 LHS is independent of $t$.

 (iii)$\Rightarrow$(ii): Recall II.8.9: A finitely generated
 module over $A$ (as above) is free if and only if $M\otimes_A K$
 and $M\otimes_A k$ have the same dimensions over the fraction
 field $K$ and the residue field $k$, respectively.  In our case,
 $H^0(X,\F(m))$  free is implied by $\dim_{k(\tau)}
 H^0(X,\F(m)\otimes_A k(\tau)) = \dim_{k(t)} H^0(X,\F(m))\otimes_A
 k(t)$ where $\tau$ and $t$ are the generic and special points of
 $T$, respectively.  By the Lemma, the LHS is $P_t(m)$ for $m\gg
 0$, and the RHS is $P_t(m)$ for $m\gg 0$.

 So we're done.

 \begin{remark}
 To prove flatness, it's enough to compare Hilbert polynomials at
 closed and generic points.
 \end{remark}

 \begin{corollary}
 Let $T$ be a connected noetherian scheme, and let $X$ be a closed
 subscheme of $\P^n_T$, flat over $T$.  Then the degree, dimension, and
 arithmetic genus of $X_t$ is independent of $t\in T$.
 \end{corollary}
 \begin{proof}
 By base change to an irreducible component of $T$ (with reduced induced
 subscheme structure), we may assume
 that $T$ is integral.  Then use the fact that the Hilbert
 polynomial is independent of $t\in T$ (all of these things are determined
 by the Hilbert polynomial).
 \end{proof}

 \underline{Exercise III.9.1:} {\it Let $f:X\to Y$ be a flat morphism
 of finite type of noetherian schemes.  Then $f$ is an
 open\footnote{For all open $U\subseteq X$, $f(U)\subseteq Y$ is open.}
 morphism.}
 \begin{proof}
 Let $U\subseteq X$ be open.  Then $f(U)\subseteq Y$ is
 constructable (Ex. II.3.19), meaning that it is a closed set
 minus a constructable set of lower dimension.  By Ex. II.3.18c, it
 will suffice to show that $f(U)$ is stable under generization.
 That is, if $y\in f(U)$ and $\eta\in Y$ such that $\eta
 \rightsquigarrow y$, then $\eta \in f(U)$.  To see this, let
 $\spec A\subseteq Y$ be an open affine neighborhood of $y$ (thus,
 $\eta\in \spec A$).  Let $x\in f^{-1}(y)$, and let $\spec B$ be an
 open affine neighborhood of $x$ in $U\cap f^{-1}(\spec A)$.  Then
 $B$ is flat over $A$.  Let $\p,\p',\q$ be prime ideals of $A,A,
 B$ (resp.) corresponding to $y,\eta,x$ (resp.)
 \[\xymatrix @C=4mm {
  & \q \ar@{}[r]|{\subseteq} \ar@{-}[d] & B \ar@{-}[d]^{\txt{flat}}\\
  \p'\ar@{}[r]|{\subseteq} & \p\ar@{}[r]|{\subseteq} & A
 }\]
 There is a prime $\q'$ of $B$ lying over $\p'$ such that
 $\q'\subseteq \q$.  let $\xi\in U\subseteq X$ be the
 corresponding point.  Then $\eta = f(\xi) \subseteq f(U)$, as was
 to be shown.
 \end{proof}

 \underline{Exercise III.9.4:} {\it Let $f:X\to Y$ be a morphism
 of finite type of noetherian schemes.  Then $\{x\in X|f \text{ flat at }x\}\subseteq
 X$ is open.}

 \marginpar{\S III.10: Smoothness}
 This is a relative version of non-singularity (\emph{almost}).
 For this section, $k$ is any field (not necessarily algebraically
 closed), and all schemes considered will be assumed to be of
 finite type over $k$ (or locally of finite type).

 \begin{definition}
 Let $f:X\to Y$ be a morphism of schemes over $k$ (as above), then
 $f$ is \emph{smooth} of relative dimension $n$ if
 \begin{itemize}
 \item[(1)] $f$ is flat
 \item[(2)] if $X'$ and $Y'$ are irreducible components of $X$ and
 $Y$ (resp.) such that $f(X')\subseteq Y'$, then $\dim X' = \dim
 Y' +n$.
 \item[(3)] for all $x\in X$ (closed or not), $\dim_{k(x)}
 (\Omega_{X/Y}\otimes k(x))=n$.
 \end{itemize}
 \end{definition}

 \begin{remark}
 \begin{itemize}
 \item[]
 \item[(i)] By $(1)$ and Cor III.9.6, (2) is equivalent to

    (2') for all $y\in Y$ (closed or not), $X_y$ is equidimensional
    of dimension $n$.

    Also, (3) is equivalent to

    (3') for all $x\in X$, $\dim_{k(x)} (\Omega_{X_y/k(y)} \otimes
    k(x))=n$ where $y=f(x)$. $\Omega_{X_y/k(y)}$ is really
    $\Omega_{X/Y}$ pulled back to $X_y$.

    So (2) and (3) just concern the fibers $X_y$.
 \item[(ii)] Smoothness of relative dimension $n$ is local on $X$
 (therefore local on the base) in the sense that $f:X\to Y$ is
 smooth of relative dimension $n$ if and only if there exists an
 open cover $(U_i)_{i\in I}$ of $X$ such that $f_{U_i}$ is smooth
 of relative dimension $n$ for all $i$.  So we can define $f:X\to
 Y$ is smooth of relative dimension $n$ \emph{at $x$} if there is
 an open neighborhood $U\subseteq X$ such that $f|_U$ is smooth of
 relative dimension $n$.  Then $f$ is smooth of relative dimension
 $n$ if and only if it is at each point in $X$.  Also, $\{x\in
 X|\text{$f$ smooth of relative dimension $n$}\}\subseteq X$ is
 open.
 \end{itemize}
 \end{remark}

 \underline{Examples:}
 \begin{itemize}
 \item[(i)] For any $Y$, $\P^n_Y$ and $\A^n_Y$ are smooth over
 $Y$ of relative dimension $n$.  It is enough to show it for $\A^n_Y$ since $\P^n_Y$ is
 covered by open affines isomorphic to $\A^n_Y$.
 \begin{proof}
 (1) Flatness is ok. (2') is easy: $\dim_{k(y)} \A^n_Y = n$.  (3')
 $\Omega_{\A^n_Y/Y}$ is free of rank $n$; ditto for its fibers.
 \end{proof}

 \item[(ii)] Let $Y=\spec k$ with $k$ algebraically closed.  Then
 $X$ is smooth over $Y$ of relative dimension $n$ if and only if
 $X$ is non-singular of pure dimension $n$ if and only if $X$ is
 regular of pure dimension $n$.
 \begin{proof}
 The second equivalence comes from Chapter I.  The first condition
 is equivalent to the third is given by II.8.15.
 \end{proof}
 \underline{Caution:} We do need $k$ algebraically closed here
  (see Ex III.10.1).
 \end{itemize}

 \begin{proposition}
 \begin{itemize}
 \item[]
 \item[(a)] An open immersion is smooth of relative dimension 0.
 \item[(b)] (Base Change) If $f:X\to Y$ is smooth of relative
 dimension $n$, and $Y'\to Y$ is any morphism, then $f':X':=
 X\times_Y Y'\to Y'$ is also smooth of relative dimension $n$.
 \item[(c)] (Composition) If $X\xrightarrow{f} Y\xrightarrow{g} Z$
 are smooth of relative dimension $n$ and $m$, resp., then $g\circ
 f:X\to Z$ is smooth of relative dimension $n+m$.
 \item[(d)] (Products) if $f:X\to Y$ and $f':X'\to Y'$ are smooth
 $S$-morphisms of relative dimensions $n$ and $n'$, resp., then
 $f\times_S f':X\times_S X' \to Y\times_S Y'$ is smooth of
 relative dimension $n+n'$. (Book's special case: $Y=Y'=S$)
 \end{itemize}
 \end{proposition}
 \begin{proof}
 (a) is trivial. (d) follows from (b) and (c) by published proof of Exercise
 II.4.8d.

 (b): Flatness is immediate.  For (2') and (3'), let $y'\in Y'$ and
 let $y\in Y$ be the image of $y'$.  Then $X'_{y'} = X_y
 \times_{k(y)} k(y')$.  $k(y')$ is a finite extension of $k(y)$,
 so it doesn't affect dimension\footnote{see next lecture}.  And it doesn't affect (3')
 \end{proof}
