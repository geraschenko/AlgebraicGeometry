 \stepcounter{lecture}
 \setcounter{lecture}{4}
 \sektion{Lecture 4}

\def\A{\mathcal{A}}

Last time, we showed that injective objects in $\Mod(X)$ are
acyclic in $H^{\cdot}(X,-)$.  In particular, the right derived
functors of $\Gamma(X,-):\Mod(X)\to \Ab$ coincide with
$H^{\cdot}(X,-)|_{\Mod(X)}$.

\begin{corollary}
Let $(X,\O_X)$ be a ringed space with $B=\Gamma(X,\O_X)$.  Then
for all $\F\in \Mod(X)$ and $i\in \mathbb{N}$, $H^i(X,\F)$ has a
natural $B$-module structure.  Thus, if $X$ is a scheme over
$\spec A$, then the groups $H^i(X,\F)$ are $A$-modules.
\end{corollary}

\begin{remark}
Let $\A$ be an abelian category with enough injectives and
$\mathcal{B}$ and $\mathcal{C}$ abelian categories.  If $F:\A\to
\mathcal{B}$ is left exact and $G:\mathcal{B}\to \mathcal{C}$ is
exact, then the $\delta$-functor $\{R^i(G\circ F)\}$ is isomorphic
to $\{G\circ R^iF\}$.
\end{remark}

\begin{theorem}
Let $X$ be a noetherian topological space of dimension $n$.  Then
$H^i(X,\F)=0$ for all $i>n$ and for all $\F\in \Ab(X)$.
\marginpar{Grothendieck Vanishing Theorem}
\end{theorem}
Note: For the rest of this lecture, $X$ is a noetherian
topological space.

\begin{lemma}
The direct limit of flasque sheaves is flasque.
\end{lemma}
\begin{proof}
Let $\{\F_{\alpha}\}_{\alpha\in A}$ be a directed system of
flasque sheaves and let $V\subseteq U\subseteq X$ be open sets.
Then we have that
\[\xymatrix{
 (\varinjlim \F_{\alpha})(U) \ar[r]\ar[d]_{\wr} & (\varinjlim
 \F_{\alpha})(V) \ar[d]_{\wr}\\
 \varinjlim \F_{\alpha}(U) \ar@{->>}[r] & \varinjlim
 \F_{\alpha}(V)
}\]
 Where the bottom arrow is surjective because the $\F_{\alpha}$
 are flasque and $\varinjlim$ is an exact functor on abelian
 groups, and the vertical arrows are isomorphisms by exercises
 II.1.10 and II.1.11.
\end{proof}

\begin{lemma} \label{L:CohomologyLimit}
Let $\{\F_{\alpha}\}$ be a directed system in $\Ab(X)$.  Then
there are natural isomorphisms for all $i\in \mathbb{N}$
\[
 \varinjlim H^i(X,\F_{\alpha}) \xrightarrow{\sim} H^i(X,\varinjlim
 \F_{\alpha}).
\]
\end{lemma}
\begin{proof}
Let $\mathcal{C}$ be the category of $A$-directed systems in
$\Ab(X)$.  Then $(\F_{\alpha})\in \mathcal{C}$.  For each
$\alpha$, inject $\F_{\alpha}$ into its sheaf of discontinuous
sections $\G^0_{\alpha}$.  Then inject the cokernel of this map
into \emph{its} sheaf of discontinuous sections, $\G^1_{\alpha}$,
etc. This yields a flasque resolution $0\to \F_{\alpha} \to
\G^{\cdot}_{\alpha}$ which is functorial in $\F_{\alpha}$, so we
get an $A$-directed system of complexes.  In particular,
$(\G^i_{\alpha})_{\alpha}\in \mathcal{C}$.

Then we have that \begin{align*}
 \varinjlim_{\alpha} H^i(X,\F_{\alpha}) &\cong \varinjlim_{\alpha}
 h^i(\Gamma(X,\G^{\cdot}_{\alpha})) & \text{(definition)} \\
  &\cong h^i(\varinjlim_{\alpha} \Gamma(X,\G^{\cdot}_{\alpha})) & \text{($\varinjlim$ exact in
  $\Ab$)}\\
  &\cong h^i(\Gamma(X,\underbrace{\varinjlim_{\alpha} \G^{\cdot}_{\alpha}}_{\text{flasque}})) &
  \text{(nontrivial, need $X$ noetherian)}
\end{align*}
If $0\to \varinjlim \F_{\alpha} \to \varinjlim \G^0_{\alpha} \to
\cdots$ is exact, then we are done.  To see that it is exact,
observe that $\varinjlim$ commutes with taking stalks:
\[
    (\varinjlim \F_{\alpha})_P = \varinjlim_{P\in U}
    \varinjlim_{\alpha} \F_{\alpha}(U) = \varinjlim_{\alpha}
    \varinjlim_{P\in U} \F_{\alpha}(U) = \varinjlim
    (\F_{\alpha})_P
\]
\end{proof}

\begin{lemma} \label{L:ClosedImmersion}
Let $j:Y\hookrightarrow X$ be a closed immersion of topological
spaces and let $\F\in \Ab(Y)$.  Then
\[
    H^i(X,j_*\F)\cong H^i(Y,\F)
\]
for all $i$.
\end{lemma}
\begin{proof}
$H^0$ coincides for all $j$ by definition of $j_*$.  If $0\to \F
\to \I^{\cdot}$ is a flasque resolution of $\F$ on $Y$, then $0\to
j_*\F \to j_*\I^{\cdot}$ is exact (look at the stalks), so the
computation is the same
\end{proof}

If $Y\subseteq X$ is a closed subset, $U=X\smallsetminus Y$ and
$\F\in \Ab(X)$, then define
\[\F_Y = j_*(\F|_Y)\]
where $j:Y\hookrightarrow X$ and $\F|_Y = j^{-1}\F$.  We also
define
\[ \F_U = i_!(\F|_U)\]
where $i:U\hookrightarrow X$ is the inclusion.  Then by exercise
II.1.19, we have that
\[
    0 \to \F_U \to \F \to \F_Y \to 0.
\]

\begin{proof}[Proof of Theorem]\def\R{\mathscr{R}}  Use induction on $n=\dim X$.
\begin{itemize}
\item[\underline{Step 1}:] Reduce to the case where $X$ is
irreducible by induction on the number of components.  If the
number of components is 0, then we are done (since vacuously).  If
it is 1, then $X$ is irreducible.  If there is more than one
irreducible component, then let $Y$ be one component and let
$U=X\smallsetminus Y$.  Then
\[
     0 \to \F_U \to \F \to \F_Y \to 0
\]
is exact, so by the long exact sequence in cohomology, we have the
exact sequence
\[
    H^i(X,\F_U)\to H^i(X,\F) \to H^i(X,\F_Y)
\]
for each $i>n$, but $H^i(X,\F_Y) = H^i(Y,\F|_Y)=0$ by the
inductive hypothesis (and Lemma (\ref{L:ClosedImmersion})).  Then
\[
    0\to \underbrace{(\F_U)_{X\smallsetminus \overline{U}}}_0 \to
    \F_U\xrightarrow{\sim} (\F_U)_{\overline{U}}\to 0
\]
Hence $H^i(X,\F_U)\cong H^i(X,(\F_U)_{\overline{U}} =
H^i(\overline{U},(\F_U)_{\overline{U}}) = 0$ by induction.  Thus,
$H^i(X,\F)=0$ for all $i>n$.

\item[\underline{Step 2}:] If $X$ is irreducible of dimension 0,
then the open subsets are $\varnothing$ and $X$, so $\F$ is
flasque and therefore acyclic.  So $H^i(X,\F)=0$ for all $i>0$, as
desired.

\item[\underline{Step 3}:] Assume $X$ is irreducible (with $n=\dim
X>0$).  Let $B$ be a generating set for $\F$ (e.g.
$\coprod_{U\subseteq X} \F(U)$).  Let $A=\{\text{finite subsets of
} B\}$, then $A$ is a directed system.  For $\alpha\in A$, let
$\F_{\alpha}$ be the subsheaf of $\F$ generated by $\alpha$. That
is, $\F_{\alpha}$ is the set of sections that can be obtained from
$\alpha$ by restrictions and gluing.

Since $X$ is noetherian, all open subsets are quasi-compact, so
all gluings are finite. Thus, any section of $\F$ comes from a
finite subset of $B$, so $\F=\varinjlim \F_{\alpha}$.

By Lemma (\ref{L:CohomologyLimit}), it is enough to show the
theorem for the $\F_{\alpha}$.

Finally, we do induction on the number of elements in $\alpha$. If
the number of elements in $\alpha$ is greater than 1, choose a
proper subset $\alpha'$. Then
\[
    0\to \F_{\alpha'} \to \F_{\alpha} \to
    \underbrace{\text{quotient}}_{\text{gen by images of
    $\alpha\smallsetminus \alpha'$}}\to 0.
\]
By the long exact sequence, we have
\[
    H^i(X,\F_{\alpha'})\to H^i(X,\F_{\alpha})\to
    H^i(X,\text{quotient}).
\]
By induction on $|\alpha|$, we have that the side terms are zero,
and so the middle is zero.  Thus, we may assume that $\F$ is
generated by a single element (over some open set $U$).  Then we
have the exact sequence
\[
    0\to \R \to \Z_U\to \F \to 0
\]
which gives us the exact sequence
\[
    H^i(X,\Z_U)\to H^i(X,\F)\to H^{i+1}(X,\R).
\]
So it is enough to show the result for $\Z_U$ and its subsheaves.
\begin{itemize}
\item[\underline{Proof for $\Z_U$}:] Let $Y=X\smallsetminus U$. We
have that
\[
    0\to \Z_U \to \Z \to \Z_Y\to 0,
\]
and $\Z$ is flasque since $X$ is irreducible, so for all $i>n$,
\[
    H^{i-1}(X,\Z_Y)\to H^i(X,\Z_U)\to \underbrace{H^i(X,\Z)}_0.
\]
The leftmost term is $H^{i-1}(X,\Z_Y)\cong H^{i-1}(Y,\Z|_Y)=0$ by
induction on dimension ($\dim Y<n$).  Thus, $H^i(X,\Z_U)=0$ for
all $i>n$.

\item[\underline{Proof for $\R$}:]
%We may assume that for all $x\in U$, $\R_x\not=0$ since
For all $x\in U$, $\R_x$ is a subgroup of $(\Z_U)_x = \Z$.  Let
$d$ be a minimal positive element in $\Z$ such that $d\in \R_x$
for some $x\in U$.  Then $d\in \R(V)$ for some open set $x\in
V\subseteq U$.  Since $\Z$ is a PID, we have that $\R|_V\subseteq
d\cdot \Z_V$.  But we also have that $d\cdot \Z_V\subseteq \R|_V$,
so $\R|_V\cong \Z$ on $V$.  Thus, we have
\[
    0\to \Z_V \to \R \to \text{quotient} \to 0.
\]
We have already shown that $H^i(X,\Z_V)=0$ for $i>n$, and the
quotient is a sheaf on the lower dimensional $X\smallsetminus V$,
so $H^i(X,\text{quotient})=0$ for $i>n$ by induction on dimension.
Therefore, $H^i(X,\R)=0$ for all $i>n$.
\end{itemize}
\end{itemize}
\end{proof}

Example: Exercise III.2.1a.  Let $X=\mathbb{A}^1_k$ for an
infinite field $k$.  Let $P,Q$ be distinct closed points.  Let
$Y=\{P,Q\}$, and let $U=X\smallsetminus Y$.  Then
$H^1(X,\Z_U)\not=0$ (i.e. the bound given by the theorem cannot be
improved).
\begin{proof}
From
\[
    0\to \Z_U\to \Z\to \Z_Y\to 0
\]
we get
\[
    \underbrace{H^0(X,\Z)}_{\Z} \xrightarrow{\text{not onto}}
    \underbrace{H^0(X,\Z_Y)}_{\Z\times\Z} \to H^1(X,\Z_U).
\]
Therefore, $H^1(X,\Z_U)\not=0$.
\end{proof}
Part (b)*: Show for all $n>0$ that if $X=\mathbb{A}^n_k$, $Y=$
union of $n+1$ hyperplanes with empty intersection (in $\P^n_k$)
and $U=X\smallsetminus Y$, then $H^n(X,\Z_U)\not=0$.

In the case $n=2$, we have
\[
 0\to \Z_U\to \underbrace{\Z}_{\text{flasque}}\to \Z_Y\to 0
\]
\marginpar{
 \begin{xy}<1cm,0cm>:
  \ar@{-} (-.7,1.5);(1.5,-.7)
  \ar@{-} (0,1.5);(0,-.7)
  \ar@{-} (-.7,0);(1.5,0)
 \end{xy}
} so
\[
    \underbrace{H^{n-1}(X,\Z)}_0 \to H^{n-1}(X,\Z_Y)
    \xrightarrow{\sim} H^n(X,\Z_U) \to \underbrace{H^n(X,\Z)}_0
\]
so it suffices to show that $H^{n-1}(Y,\Z_Y)\not=0$.  Let $Y'=$
the three points of intersection and let $U'=Y\smallsetminus Y'$.
Then
\[
    0\to \Z_{U'} \to \Z \to \Z_{Y'}\to 0
\]
so
\[
    \underbrace{H^0(Y,\Z)}_{\Z} \to
    \underbrace{H^0(Y,\Z_{Y'})}_{\Z^3} \xrightarrow{\text{not
    onto}} \underbrace{H^1(Y,\Z_{U'})}_{\Z^3} \to H^1(Y,\Z).
\]
To see that $H^1(Y,\Z_{U'})=\Z^3$, observe that $U'=U_1'\cup U_2'
\cup U_3'$, where each $U_i'\cong\mathbb{A}^1_k\smallsetminus $ 2
points.  So $H^1(Y,\Z_{U'}) = \bigoplus_{i=1}^3 H^1(Y,\Z_{U_i'}) =
\Z^3$ by the Mayer-Vietoris sequence (exercise III.2.4).
Therefore, $H^1(Y,\Z)\not=0$, as desired.
