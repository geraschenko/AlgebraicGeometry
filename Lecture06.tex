 \stepcounter{lecture}
 \setcounter{lecture}{6}
 \sektion{Lecture 6}

Application to Corollary (\ref{C:lec5}): Every such sheaf has a
quasi-coherent flasque resolution which can therefore be used to
compute cohomology.

Also (Exercise III.3.6) $\mathfrak{Qco}(X)$ has enough injectives,
and the resulting derived functor cohomology theory agrees with
the usual one.

\begin{theorem}[Serre] Let $X$ be a noetherian scheme.  Then the
following are equivalent:
\begin{itemize}
 \item[(1)] $X$ is affine
 \item[(2)] All quasi-coherent sheaves on $X$ are acyclic
 \item[(3)] For all (quasi-)coherent sheaves of ideals $\I$ on
 $X$, $H^1(X,\I)=0$.
\end{itemize}
\end{theorem}
\begin{proof}
$(1)\Rightarrow (2)$ is Theorem (\ref{T:lec5}).

$(2)\Rightarrow (3)$ is trivial.

$(3)\Rightarrow (1)$:  Assume (3) and let $A=\Gamma(X,\O_X)$.
\vspace{-5mm}
\begin{itemize}
\item[] \begin{claim} Every closed point $P\in X$ has an open
\underline{affine} neighborhood of the form $X_f$ for some $f\in
A$.
\end{claim}
\begin{proof}[Proof of Claim]
Let $U$ be any open affine neighborhood of $P$, and let
$Y=X\smallsetminus U$.  Consider
\[
    0\to \I_{Y\cup \{P\}} \to \I_Y \to k(P) \to 0
\]
where $k(P)$ is the residue field skyscraper sheaf at $P$, and $Y$
and $Y\cup \{P\}$ are taken as reduced subschemes of $X$.  By the
long exact sequence, we have
\[
    H^0(X,\I_Y)\to \underbrace{H^0(X,k(P))}_{k(P)} \to
    \underbrace{H^1(X,\I_{Y\cup\{P\}})}_{0 \text{ by }(3)}.
\]
Therefore, there is some $f\in \Gamma(X,\I_Y)$ mapping to $1\in
k(P)$.  In particular, $f\not\in \mathfrak{m}_P$, so $P\in X_f$
[$f\in H^0(X,\I_Y)\subseteq H^0(X,\O_X)=A$].  Also, for all $Q\in
Y$, $f\in \I_Y \subseteq \mathfrak{m}_Q$, so $Q\not\in X_f$. Thus,
$X_f\subseteq U$.  Let $B=\Gamma(U,\O_X)$ and $\bar f=f|_U\in B$.
Then $X_f=X_f\cap U=D(\bar f) = \spec B_{\bar f}$ (in $U$).  So
$X_f$ is affine.
\renewcommand{\qedsymbol}{$\square_{\text{Claim}}$}
\end{proof}
\end{itemize}
Since $X$ is noetherian, the sets $X_f$ cover $X$.  Take a finite
subcover, $X_{f_1},\dots,X_{f_r}$.  By the handout (Exercises
II.2.16 and II.2.17), it will suffice to check that the ideal
$(f_1,\dots,f_n)\subseteq A$ is all of $A$.  Define a map of
sheaves $\O_X^r\to \O_X$ by $(a_1,\dots,a_r)\mapsto \sum a_if_i$.
We may assume it is onto, because if $P\in X$, then $P\in X_{f_i}$
for some $i$ and $e_i = (0,\dots, 1,\dots, 0)\mapsto f_i\in \O_X$,
and $f_i\not\in \mathfrak{m}_P$.

Now blah blah \marginpar{Start:Fast forward}
\end{proof}

Application: Exercise III.3.1 (neat filtration trick)

Exercise III.3.8: Lemma (\ref{L:lec5lemma1}) (Hartshorne Lemma
III.3.3) and Lemma (\ref{L:lec5keylemma}) (Hartshorne Proposition
III.3.4) are false without the noetherian hypothesis.

\marginpar{Chapter II, \S 4:\\ Separated and Proper morphisms}

\begin{definition} Let $f:X\to Y$ be a morphism of schemes.  Then
the diagonal map is the morphism $\Delta:X\to X\times_Y X$ defined
by the diagram
\[\xymatrix{
 X\ar@/_4ex/[ddr]_{\text{id}_X}\ar@{.>}[dr]^{\Delta}
 \ar@/^4ex/[drr]_{\text{id}_X} \\
 & X\times_Y X \ar[r] \ar[d] & X\ar[d]^f\\
 & X\ar[r]^f & Y
}\]
 We say that $f$ is \emph{separated} if $\Delta$ is a closed
 immersion.  In this case, we say that $X$ is \emph{separated
 over} $Y$.  A \emph{separated scheme} is a scheme $X$ which is
 separated over $\spec \Z$.
\end{definition}

Notation:
\begin{itemize}
 \item[1)] In EGA, prescheme is the same as scheme in Hartshorne,
and scheme means separated scheme.
 \item[2)] If $f_1:X\to Y_1$ and $f_2:X\to Y_2$ are morphisms in
 $\mathfrak{Sch}(S)$, then $(f_1,f_2)$ is the $S$-morphism defined
 by the diagram
\[\xymatrix{
 X\ar@/_4ex/[ddr]_{f_1}\ar@{.>}[dr]^{(f_1,f_2)}
 \ar@/^4ex/[drr]_(.6){f_2} \\
 & Y_1\times_S Y_2 \ar[r] \ar[d] & Y_2\ar[d]\\
 & Y_1\ar[r] & S
}\] Thus, $\Delta=(\text{id}_X,\text{id}_X)$.

 On the other hand, if $f_i:X_i\to Y_i$ are morphisms for $i=1,2$,
 then $f_1\times f_2 = (f_1\circ pr_1, f_2\circ pr_2)$.
\end{itemize}

Example: The affine line with two origins is not separated over
$k$ because $\Delta\subseteq X\times_k X=\mathbb{A}^2$ with double
axes and quadruple origin contains only two of the origins, but
$\overline{\Delta}$ contains all four.  Thus, $\Delta$ is not a
closed immersion.
