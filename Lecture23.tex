 \stepcounter{lecture}
 \setcounter{lecture}{23}
 \sektion{Lecture 23}

\textbf{Homework comments:}\\

1st problem: Ampleness. Given a coherent sheaf $\F$ on a
Noetherian scheme $X$, Supp $\F$ is defined as a \textbf{set}, and
is closed. Then $\F$ can be viewed as a sheaf on $Y$, but it is
not a sheaf of $\O_Y$ modules (therefore not coherent).

Example: $X$ = $\spec{\Z}$, $\F = \widetilde{\Z/4\Z}$. Then $Y = $
Supp $\F$ = $\{2\}$, but $\F$ is not a sheaf of $O_Y$ modules if
we take $Y = \spec{\Z/2\Z}$ (since $\Z/4\Z$ is not a module over
$\mathbb{F}_2$. We have to take a different subscheme structure on
$O_Y$ to get this to work out, and Vojta claims that this can
always
be done. See his solution to (III ex.4.2).\\

Lemma (III 2.10) is better than (III ex.4.1).\\

With induction proof in part (d), need to be sure that the
reduction to $X,Y$ integral is "inside" the induction. Why?
Because a proper closed subscheme of an irreducible scheme might
not be irreducible
(you know, take two points).\\

The proof of (c) was indeed similar to (b), so instead of just
saying that, the proper thing to do is to generalize!!!

\begin{lemma} Let $\I$ be a coherent sheaf of ideals on $X$ and
$\F$ a coherent sheaf on $X$. Let $Y$ be the closed subscheme
corresponding to $\I$. If $H^i(X,\I\F)=0$ and $H^i(Y,\F/\I\F)=0$
(actually, the correspondinig sheaf on $Y$ for some $i$, then
$H^i(X,\F)=0$. \end{lemma} Then phrase (b) and (c) in terms of the
appropriate
coherent ideal sheaves.\\

2nd problem: If $I$ is an injective $A$-module,then
$\widetilde{I}$ is injective in $\mathscr{Q}co$, but not
necessarily in $\mathscr{M}od(X)$.

Generally, if $\mathscr{C}$ is a full abelian subcategory of
$\mathscr{D}$, and if $X \in \mathscr{C}$ is injective in
$\mathscr{D}$, then it is injective also in $\mathscr{C}$. But not
necessairly conversely.\\



 \textbf{Back to III.8}.
 \begin{theorem} Let $f:X\rightarrow
Y$ be a projective morphism of Noetherian schemes, let $\O(1)$ be
a very ample sheaf on $X$ over $Y$, and let $\F$ be a coherent
sheaf on $X$. Then \begin{itemize} \item[(a)] The natural map
$f^*f_*\F(n)\rightarrow\F$ is surjective for $n>>0$. \item[(b)]
$R^if_*\F$ is coherent for every $i$. \item[(c)] $R^if_*\F(n) = 0$
for $i > 0$ and $n >> 0$. \end{itemize}
 \end{theorem}
\begin{proof} The question is local on $Y$ - as $Y$ is quasi
compact (it is Noetherian!) we can find an $n$ for each part of a
finite affine cover, and then take the largest $n$ - and we can
assume that $Y$ is
affine, say $Y = \spec{A}$.\\

(a) By (III 8.5), $\text{R}^0f_*\F(n) = f_*\F(n) =
H^0(X,\F(n))^{\widetilde{}_Y} = M^{\widetilde{}_Y}$. Then for any
open affine $\spec{B}$ in $X$, we have by (II prop.5.2) that
$f^*f_*\F(n) = \widetilde{M\otimes_AB}$ on $X$.

So our map $f^*f_*\F(n)\rightarrow\F(n)$ is defined (at the global
section level) by \[ \xymatrix{ M\otimes_AB \ar[r] &M
\ar[rr]^{\alpha} &&
\Gamma(\spec{B},\F(n))\\
m\otimes b \ar[r]&mb.&& } \]

By prop (II 5.17), for large enough $n$ the sheaf is generated by
global sections, so the map is surjective at the stalk level, and
thus surjective.\\

(b) By (III 8.5),  $\text{R}^if_*\F(n) =
H^i(X,\F(n))^{\widetilde{}_Y}$ is coherent because $H^i(X,\F(n))$
is finitely generated (III 5.2(a))for all $i$
and $n$.\\

(c) By (III 5.2(b)), $\text{R}^if_*\F(n) =
H^i(X,\F(n))^{\widetilde{}_Y} = 0$ for $i > 0$ and large enough
$n$. \end{proof}

\marginpar{\S III.9: Flat Morphisms}

\begin{definition}Let $M$ be an $A$-module. We say that $M$ is
flat (over $A$) if the functor $M\otimes_A -$ is exact, i.e. \[ 0
\rightarrow N' \rightarrow N \rightarrow N'' \rightarrow 0 \]
exact implies \[ 0 \rightarrow M\otimes_AN' \rightarrow
M\otimes_AN\rightarrow M\otimes_AN''\rightarrow0 \] exact.
\end{definition}
 Also, $M$ is faithfully flat if the converse of the above is also true.

\underline{Examples:}\begin{itemize}
 \item A free module is flat,a nd is
faithfully flat iff it's non-zero. In particular, if $A$ is a
field, then everything is flat.
 \item $\Z/n\Z$ is not flat over $\Z$.
Indeed, let $p|n$ and tensor \[ 0 \rightarrow \Z \rightarrow^p \Z
\rightarrow \Z/p\Z \rightarrow 0 \] with $\Z/n\Z$ to get \[ \Z/n\Z
\rightarrow^p \Z/n\Z \rightarrow ? \rightarrow 0 \] \end{itemize}

\underline{Example:} $S^{-1}A$ is flat over $A$ for any
multiplicative system $S$ (i.e. 'localization is exact').

\textbf{Properties of Flatness} \begin{theorem} Let $B$ be an
$A$-algebra, and let $M,N$ be $A$ and $B$ modules, resp. Then
\begin{itemize} \item[(a)]$M$ is flat over $A$ iff the map
$M\otimes_A \mathfrak{a} \rightarrow M$ is injective for every
finitely ideal $\mathfrak{a}$ $A$. \item[(b)] If $M$ is flat over
$A$, then $M\otimes_aB$ is flat over $B$. \item[(c)] Transitivity:
if $N$ is flat over $B$ and $B$ is flat over $A$, thhen $N$ is
flat over $A$. \item[(d)] $M$ is flat over $A$ iff $M_p$ is flat
over $A_p$ for all $p \in \spec{A}$ \item[(e)]Let $0\rightarrow M'
\rightarrow M \rightarrow M'' \rightarrow 0$ be an exact seq of
$A$ modules. Then $M'$ and $M''$ flat imply $M$ flat. $M$ and
$M''$ flat imply $M'$ flat. \item[(f)] If $M$ if a finitely
generated module over a Noetherian local ring $A$, then $M$ is
flat iff its free. \end{itemize} \end{theorem} \begin{proof} (a)
"$M\otimes_A -$" is a right exact functor, and $\mathscr{M}od(X)$
has enough projectives (in fact, every module has a free
resolution). The right derived functors of $M\otimes_A-$ are
called $\text{Tor}_i^A(M,-)$. It is known that $M$ is flat over
$A$ iff $\text{Tor}_i^A(M,N)$ = 0 for all $i>0$ and $N$, iff
$\text{Tor}_i^A(M,\mathfrak{a})$ = 0 for all $i>0$ and finitely
generated ideals $\mathfrak{a}$ of $A$.

(b,c) $(M$ flat over $A \Rightarrow M \otimes_AB$ flat over
$B$)$\Leftarrow N\otimes_B(M\otimes_AB)) \cong N\otimes_AM$.

(d) True because Tor commutes with localization.

(e) Comes from the left exact sequence in Tor: \[\xymatrix{\cdots
\ar[r] & \text{Tor}_1^A(M',N) \ar[r] & \text{Tor}_1^A(M,N) \ar[r]
&\text{Tor}_1^A(M'',N) \ar[r] & \\
 \ar[r] & M'\otimes N \ar[r] & M\otimes N\ar[r]
&M''\otimes N \ar[r] & 0 } \]

We see that if $\text{Tor}_i^A(M',N) = \text{Tor}_i^A(M'',N)= 0$
(i.e. $M'$ and $M''$ are flat), then $\text{Tor}_i^A(M,N) = 0$ and
$M$ is flat too (similarly if $M$ and $M''$ are flat, $M'$ is
also).

(f) \end{proof}

\begin{definition} Let $f:X\rightarrow Y$ be a morphism of schemes
and let $\F$ be a sheaf of $\O_X$-modules. For $x \in X$, we say
that $\F$ is flat over $Y$ at $x$ if $\F_x$ is a flat
$\O_{Y,f(x)}$ module (via the map $f^{\#}: \O_{Y ,f(x)}
\rightarrow \O_{X,x}$. We say that $\F$ is flat over $Y$ if it is
flat over $Y$ at $x$ for all $x \in X$.

We say that $X$ is flat over $Y$ if $\O_X$ is flat over $Y$. We
say that $f$ is flat if $X$ is flat over $Y$. \end{definition}

\begin{proposition}
 \begin{itemize} \item[(a)] An open immersion
is flat. \item[(b)] Flatness is preserved under base change.
\item[(c)] Composition of flat morphisms is flat. \item[(d)] A
morphism $f:X \rightarrow Y$ is flat iff for all open affines,
$\spec{A} \subset Y$ and for all $\spec{B} \subset
f^{-1}(\spec{A})$, $B$ is flat over $A$ (i.e. flatness is local).
\item[(e)] If $X$ is noetherian and $\F$ is coherent, then $\F$ is
flat over $X$ iff it is locally free. \end{itemize}
\end{proposition} \begin{proof} (a) The stalks are the same.

(b)-(e) follow from the algebraic properties above.

Also, flatness is local on the base (because its local upstairs).
If $f:X\rightarrow Y$ and $f':X'\rightarrow Y'$ are flat morphisms
over S, then the product is flat. Indeed, see Vojta's solution of
II ex.4.8 - (a) is not needed for (d). \end{proof}

\underline{Examples:}\begin{itemize}
 \item  Closed immersion are
generall not flat: $\spec{\Z/p\Z} \rightarrow \spec{Z}$ is not
flat. Indeed, by (d) above, it is flat iff $\Z/pZ$ is flat over
$\Z$.
 \item Blowing ups are generally not flat. I'm not going to try to tex
this. Just look at the simplest blow up possible. \end{itemize}
