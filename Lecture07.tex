 \stepcounter{lecture}
 \setcounter{lecture}{7}
 \sektion{Lecture 7}

Recall that $f:X\to Y$ is separated if $\Delta:X\to X\times_Y X$ is a closed
immersion.

\begin{proposition}
If $f:X\to Y$ is a morphism of affine schemes, then it is
separated.
\end{proposition}
\begin{proof}
Let $X=\spec A$ and $Y=\spec B$.  Then $X\times_Y X = \spec
(A\otimes_B A)$ and $\Delta$ corresponds to the morphism
$A\otimes_B A\to A$, $a\otimes a'\mapsto aa'$, which is onto.
Thus, $\Delta$ is a closed immersion.
\end{proof}

\begin{corollary}
All affine schemes are separated.
\end{corollary}
\begin{corollary}
A morphism $f:X\to Y$ is separated if and only if $\Delta(X)$ is a
closed subset of $X\times_Y X$
\end{corollary}
\begin{proof}
($\Rightarrow$) obvious.

($\Leftarrow$) We need to show that
\begin{itemize}
 \item[(1)] $\Delta$ is a homeomorphism onto $\Delta(X)$
 \item[(2)] $\O_{X\times_Y X} \to \Delta_* \O_X$ is onto
\end{itemize}

(1) We have
\[\xymatrix{
 X\ar@/_4ex/[ddr]_{\text{id}_X}\ar@{.>}[dr]^{\Delta}
 \ar@/^4ex/[drr]_{\text{id}_X} \\
 & X\times_Y X \ar[r]^(.6){\pi_2} \ar[d]^{\pi_1} & X\ar[d]^f\\
 & X\ar[r]^f & Y
}\]
 Since $\pi_1\circ \Delta:X\to \Delta(X)\to X$ is the identity,
 $\Delta$ must be one to one, and $\pi_1$ is a continuous inverse,
 so $\Delta$ is a homeomorphism $X\to \Delta(X)$.

(2) It is enough to look at stalks.  Let $P=\Delta(P')\in
\Delta(X)$ for some $P'\in X$.  Restrict to an open affine
neighborhood $V$ of $f(P')$ in $Y$, and an open affine
neighborhood $U$ of $P'$ in $f^{-1}(V)$.  Then we've reduced to
the affine case, where we know the result holds.
\end{proof}

\begin{theorem}[Valuative Criterion of Separatedness]
\marginpar{Valuative Criterion for Separatedness}
A morphism $f:X\to Y$ of noetherian schemes is separated if and
only if the following criterion holds: for any field $K$ and any
valuation ring $R$ of $K$ (Frac($R)=K$), and for any diagram
\[\xymatrix{
 U:= \spec K \ar[r] \ar[d]^i & X\ar[d]^f\\
 T:= \spec R \ar[r] \ar@{.>}[ur]^{\exists \le 1} & Y
}\] there exists at most one $h:T\to X$ such that the diagram
commutes.
\end{theorem}

\begin{remark}
\begin{itemize}
 \item[(1)] You really only need $X$ noetherian.
 \item[(2)] Criterion fails for the affine line with the doubled
 origin.
 \item[(3)] Have to use valuation rings rather than curves.
\end{itemize}
\end{remark}

\begin{lemma}
Let $R,K,U,T$ be as above, and let $X$ be a scheme, then
\begin{itemize}
 \item[(a)] To give a map $U\to X$ is equivalent to giving a point
 $x\in X$ and a field extension $k(x)\hookrightarrow K$, where
 $k(x)=\O_x/\mathfrak{m}_x$.
 \item[(b)] Giving a map $T=\spec R \to X$ is equivalent to giving
 points $x_0,x_1\in X$ with $x_1\rightsquigarrow x_0$
 and\footnote{$x_1\rightsquigarrow x_0$ means that $x_0$ is a
 specialization of $x_1$.  i.e. $x_0$ is in the closure of
 $\{x_1\}$} an inclusion $k(x_1)\hookrightarrow K$ such that if
 you let $z=\overline{\{x_1\}}\subseteq X$ (with reduced induced
 subscheme structure), then $R$ dominates the local ring
 $\O_{X,z}$
\end{itemize}
\end{lemma}
\begin{proof}
in the works.
\end{proof}

\begin{lemma}
Let $f:X\to Y$ be a quasi-compact morphism of schemes, and let
$Z=f(X)\subseteq Y$.  Then $Z$ is closed if and only if it is
stable under specialization.
\end{lemma}
\begin{proof}
in the works.
\end{proof}

\begin{remark}
Since $X$ is noetherian, $\Delta: X\to X\times_Y X$ is
quasi-compact.
\end{remark}

\begin{proof}[Proof of Valuative Criterion]
in the works

\renewcommand{\qedsymbol}{\text{\tiny proof continued in next lecture}}
\end{proof}
