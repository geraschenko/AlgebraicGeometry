 \stepcounter{lecture}
 \setcounter{lecture}{16}
 \sektion{Lecture 16}

 \marginpar{\S III.6: Ext Groups and Sheaves}
 Let $(X,\O_X)$ be a ringed space.  We'll be working with the
 category of $\O_X$-modules, so $\hom$ means $\hom_{\O_X}$ and
 $\Hom$ means $\Hom_{\O_X}$.  Recall that $\Hom(\F,\G):U\mapsto
 \hom_{\O_U}(\F|_U,\G|_U)$ is a sheaf (Ex. II.1.15).

 For us, $(X,\O_X)$ will be one of
 \begin{itemize}
 \item[(i)] a scheme, or
 \item[(ii)] $X=\{point\}$, $\O_X=$ some ring $A$, so $\Mod(X)=Mod(A)$.
 \end{itemize}

 \begin{definition}
 Let $(X,\O_X)$ be a ringed space, and let $\F$ be a sheaf of
 $\O_X$-modules.  Then \[\ext^i(\F,-)\] are the right derived
 fuctors of $\hom(\F,-)$, and \[\Ext^i(\F,-)\] are the right
 derived functors of $\Hom(\F,-)$. (Note that $\Mod(X)$ has enough
 injectives)
 \end{definition}

 \underline{Motivation:}
 \begin{itemize}
 \item[(a)] $\hom$ and $\Hom$ are basic functors, so it makes sense to
 look at thier derived functors.
 \item[(b)] Used in duality.
 \item[(c)] (Ex. III.6.1) $\ext^1(\F,\G)$ parameterizes \emph{extensions} of
 $\F$ by $\G$.  An extension is a sheaf $\F'$ such that
 \[
    0\to \G\to \F'\to \F\to 0.
 \]
 \end{itemize}

 \begin{lemma}
 If $\I$ is an injective $\O_X$-module and $U\subseteq X$ is open,
 then $\I|_U$ is an injective $\O_U$-module.
 \end{lemma}
 \begin{proof}
 Say we have a diagram of $\O_U$-modules with the top row exact:
 \[\xymatrix{
 0 \ar[r] & \F \ar[r] \ar[d] & \G\\ & \I|_U
 }\]
 Then we have
 \[\xymatrix{
 0 \ar[r] & j_!\F \ar[r] \ar[d] & j_!\G\\ & j_!(\I|_U)
 \ar[r]^-{\tiny \txt{ Ex.\\
 II.1.19}} & \I
 }\]
 where $j:U\to X$ is the inclusion and $j_!$ is as in (Ex.
 II.1.19).  By injectivity of $\I$, there is a map $\phi:j_!\G \to
 \I$ extending this diagram.  Restricting to $U$, we have
 $\phi|_U:(j_!\G)|_U=\G \to \I_U$, as desired.
 \end{proof}

 \begin{proposition} For any $U\subseteq X$, $\Ext^i(\F,\G)|_U =
 \Ext^i(\F|_U,\G|_U)$ naturally.
 \end{proposition}
 \begin{proof}
 Let $0\to \G \to \I^{\cdot}$ be an injective resolution of $\G$.
 Then, by the lemma, $0\to \G|_U\to \I^{\cdot}|_U$ is an injective
 resolution of $\G|_U$, so
 \begin{align*}
    \Ext^i(\F,\G)|_U &= h^i(\Hom(\F,\G))|_U \\
        &= h^i(\Hom(\F|_U,\G|_U))\\
        &= \Ext(\F|_U,\G|_U).
 \end{align*}
 \end{proof}

 \begin{proposition}[III.6.3]\label{P:III.6.3}\begin{itemize}
 \item[]
 \item[(a)] $\Ext^i(\O_X,\G) = \left\{ \begin{tabular}{ll}
  $\G$ & if $i=0$\\
  0 & if $i\not=0$
  \end{tabular} \right.$
 \item[(b)] $\ext^i(\O_x,\G) = H^i(X,\G)$ for all  $i$.
 \end{itemize}
 \end{proposition}
 \begin{proof}
 (a) $\Ext^i(\O_X,-)$ are the right derived functors of
 $\Hom(\O_X,-)$, which is the identity functor, which is exact.

 (b) $\ext^i(\O_X,-)$ are the right derived functors of
 $\hom(\O_X,-)$, which is the functor $\Gamma(X,-)$, whose right
 derived functors are $H^i(X,-)$.
 \end{proof}

 \begin{proposition}[III.6.4]
 If $0\to \F'\to \F\to \F''\to 0$ is a short exact sequence of
 $\O_X$-modules, then we have long exact sequences
 \[\xymatrix{
 0\to \hom(\F'',\G)\to \hom(\F,\G) \to \hom(\F',\G) \to
 \ext^1(\F'',\G)\to \cdots
 }\]
 and
 \[\xymatrix{
 0\to \Hom(\F'',\G)\to \Hom(\F,\G) \to \Hom(\F',\G) \to
 \Ext^1(\F'',\G)\to \cdots
 }\]
 \end{proposition}
 \begin{proof}
 Let $0\to \G\to \I^{\cdot}$ be an injective resolution.  Then by
 injectivity, we the exact sequences
 \[
    0\to \hom(\F'',\I^{\cdot})\to \hom(\F,\I^{\cdot}) \to
    \hom(\F',\I^{\cdot}) \to 0.
 \]
 Applying the Snake Lemma gives the result.  Similarly for the
 second long exact sequence.
 \end{proof}

 \begin{lemma}[III.6.6]
 If $\I$ is an injective $\O_X$-module and $\L$ is locally free of
 finite rank, then $\I\otimes \L$ is also injective.
 \end{lemma}
 \begin{proof}
 Recall from Ex II.5.1 that $\check \L = \Hom(\L,\O_X)$ and
 \begin{itemize}
  \item[(a)] $\check{\check \L} \cong \L$
  \item[(b)]$\Hom(\L,\F) \cong \F\otimes \check \L$ for all
  sheaves $\F$
  \item[(c)]  $\hom(\L\otimes \F,\G) \cong
  \hom(\F,\Hom(\L,\G))$ for all $\F,\G$.  More generally, it is
  true that
  \[\hom(\F_1\otimes \F_2,\G) \cong \hom(\F_1,\Hom(\F_2,\L))\]
  for all $\F_1,\F_2,\G$.
 \end{itemize}
  Thus, $\hom(-,\I\otimes\L)$ is equal to $\hom(-\otimes \check
  \L,\I)$, which is the composite of two exact functors, and is
  therefore exact.
 \end{proof}

 Note that
 \[
    \hom(\F\otimes\L,\G)\stackrel{(b),(c)}{\cong}
    \hom(\F,\G\otimes\check \L) \cong \Hom(\F,\G)\otimes \check
    \L \footnote{why is the second isormophism true?}
 \]

 \begin{proposition}
 If $\L$ is locally free of finite rank and $\F,\G\in \Mod(X)$,
 then
 \begin{itemize}
  \item[(a)] $\ext^i(\F\otimes \L,\G) \cong \ext^i(\F,\G\otimes
  \check \L)$ and
  \item[(b)] $\Ext^i(\F\otimes\L,\G) \cong \Ext^i(\F,\G\otimes
  \check \L) \cong \Ext^i(\F,\G)\otimes \check \L$.
 \end{itemize}
 \end{proposition}
 \begin{proof}
 Let $0\to \G \to \I^{\cdot}$ be an injective resolution. Then
 (a):
 \begin{align*}
 \ext^i(\F\otimes \L,\G) &= h^i(\hom(\F\otimes \L,\I^{\cdot}))\\
    &= h^i(\hom(\F,\I^{\cdot}\otimes \check \L))\\
    &= \ext^i(\F,\G\otimes \check \L) & (\I^{\cdot}\otimes \check \L
        \text{ inj res of } \G\otimes \check \L)
 \end{align*}
 And for (b), the first isomorphism is ``the same''.  We also have
 that
 \begin{align*}
 \Ext^i(\F,\G\otimes\check \L) &=
 h^i(\Hom(\F,\I^{\cdot}\otimes\check \L) \\
 &= h^i(\Hom(\F,\I^{\cdot}))\otimes \check \L\\
 &= \Ext^i(\F,\G)\otimes \check \L
 \end{align*}
 \end{proof}

 \begin{corollary}\label{C:lec16}
 We know $\ext^i$ and $\Ext^i$ when the first argument is locally
 free of finite rank:
 \begin{align*}
    \ext^i(\L,\G) &= \ext^i(\O_X\otimes \L,\G)\\
        &= \ext^i(\O_X,\G\otimes \check \L)\\
        &= H^i(X,\G\otimes \check \L)
 \end{align*}
 and
 \begin{align*}
    \Ext^i(\L,\G) &= \Ext^i(\O_X,\G)\otimes \check \L\\
        &=\left\{\begin{tabular}{ll}
            $\G\otimes \check \L$ & if $i=0$\\
            0 & if $i\not=0$\end{tabular} \right.
 \end{align*}
 \end{corollary}

 \begin{proposition}[III.6.5]
 Suppose we have a locally free resolution of $\F$ (i.e. an exact
 sequence $\cdots\to \L_1\to \L_0\to \F\to 0$, where $\L_i$ are
 locally free of finite rank for all $i$).  Then for all $\G$,
 \[
    \Ext^i(\F,\G)\cong h^i(\Hom(\L_{\cdot},\G)).
 \]
 \end{proposition}
 \begin{proof}
 in the works
 \end{proof}

 \begin{proposition}[III.6.8]
 Let $X$ be a noetherian scheme.  Let $\F$ be a coherent sheaf on
 $X$, and let $\G$ be any sheaf of $\O_X$-modules.  Then for all
 $x\in X$, and $i\in \mathbb{N}$,
 \[
    \Ext^i(\F,\G)_x = \ext^i_{\O_{X,x}}(\F_x,\G_x).
 \]
 \end{proposition}
 \begin{proof}
 This is a local question, so we may assume that $X$ is affine,
 say $X=\spec A$ with $A$ noetherian, and $\F=\tilde M$ where $M$
 is a finitely generated $A$-module.  Then ther is a free
 resolution
 \[
    \cdots \to L_1\to L_0\to M\to 0
 \]
 giving the locally free resolution $\tilde L_{\cdot}\to \F \to
 0$.  So
 \begin{align*}
  \Ext^i_X(\F,\G)_x &= h^i(\Hom(\tilde L_{\cdot},\G))_x\\
    &= h^i(\Hom(\tilde L_{\cdot},\G)_x)\\
    &= h^i((\check{\tilde {L_{\cdot}}}\otimes \G)_x)\\
    &= h^i(((\check{\tilde {L_{\cdot}}})_x\otimes_{\O_{X,x}} \G_x))\\
    &= h^i(\hom_{\O_{X,x}}((L_{\cdot})_x,\G_x))\\
    &= \ext^i_{\O_{X,x}}(\F_x,\G_x).
 \end{align*}
 \end{proof}

 \begin{proposition}[III.6.9] Let $X$ be a projective scheme over a
 noetherian ring $A$, let $\O_X(1)$ be a very ample line sheaf on
 $X$ over $A$; let $\F$ and $\G$ be coherent sheaves on $X$, and
 let $i\in \mathbb{N}$.  Then
 \[
    \Gamma(X,\Ext^i(\F,\G(n))) = \ext^i(\F,\G(n))
 \]
 for all $n\gg 0$ (depending on $i$).
 \end{proposition}
 \begin{proof}
 If $i=0$, then its true for all $\F,\G,n$ by definition of
 $\Hom$, so assume $i>0$.

 If $\F$ is locally free of finite rank, then we compute that the
 left hand side is 0 for all $n$ (Prop \ref{P:III.6.3}), and the right hand side
 is zero for $n\gg 0$ (Cor \ref{C:lec16} and Thm \ref{T:III.5.2}).
  In general, by Corollary (II.5.18), there is a short exact sequence
  \[
    0\to \R \to \E\to \F\to 0
  \]
  where $\E$ is a finite direct sum of twisted structure sheaves.
  Then for $n\gg 0$, $\check \E \otimes \G(n)$ has no higher
  cohomology (by Thm \ref{T:III.5.2}), so $\ext^i(\E,\G(n))=0$ for all $i>0$.
 \end{proof}
