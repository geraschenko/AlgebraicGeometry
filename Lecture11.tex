 \stepcounter{lecture}
 \setcounter{lecture}{11}
 \sektion{Lecture 11}

We will show that $\check H^p(\U,\F)\xrightarrow{\sim} H^p(X,\F)$
whenever $\U$ is such that $\F|_{U_{i_0\dots i_p}}$ is acyclic
with respect to derived functor cohomology.

Define a sheaf version of $C^{\cdot}(\U,\F)$ in the following way:
\[
    \C^p(\U,\F) = \prod_{i_0<\cdots <i_p} f_*(\F|_{U_{i_0\dots
    i_p}})
\]
where $f$ has components $f_{i_0\dots i_p}:U_{i_0\dots
i_p}\hookrightarrow X$.  Then we have that
$\Gamma(X,\C^p(\U,\F))=C^p(\U,\F)$.  Also, define
$d:\C^p(\U,\F)\to \C^{p+1}(\U,\F)$ in the obvious way.

\begin{lemma}\label{L:lec11cechexact}
For all $X,\U,\F$, the complex $\C^{\cdot}(\U,\F)$ is a resolution
of $\F$.
\end{lemma}
\begin{proof}
Define $\epsilon: \F \to \C^0(\U,\F)=\prod_i f_*(\F|_{U_i})$ by
$\F(V)\ni s\mapsto (s|_{U_i\cap V})_i$.  Then we need to show that
the sequence
\[
    0\to \F \xrightarrow{\epsilon} \C^0(\U,\F)\xrightarrow{d}
    C^1(\U,\F) \xrightarrow{d} \cdots
\]
is exact.

First, $\epsilon$ is injective because it is injective on $\F(V)$
for all $V$ (since $\F$ is a sheaf).  So view $\F$ as a subsheaf
of $\C^0(\U,\F)$.  Then exactness at $p=0$ is equivalent to
\[
    \ker(\C^0(\U,\F)\to \C^1(\U,\F)) = \F.
\]
But that follows from
\begin{align*}
    \ker(\C^0(\U,\F)(V)\to \C^1(\U,\F)(V)) &= \ker(C^0(\U|_V,\F|_V)\to
    C^1(\U|_V,\F|_V))\\
     &= \check H^0(\U|_V,\F|_V) = \Gamma(V,\F|_V)\\
     &= \F(V)
\end{align*}

To show exactness everywhere else, fix $x\in X$ and $j$ such that
$x\in U_j$.  For each $p\ge 1$, define
\[
    k:\C^p(\U,\F)_x\to \C^{p-1}(\U,\F)_x
\]
in the following way: given $\alpha_x\in \C^p(\U,\F)_x$, lift it
to $\C^p(\U,\F)(V)$ for some open neighborhood $x\in V\subseteq
U_j$.  Then for any $i_0<\cdots<i_{p-1}$, let
\[
    (k\alpha)_{i_0\dots i_{p-1}} = \alpha_{j,i_0\dots i_{p-1}}.
\]
Then $k(\alpha_x) = (k\alpha)_x$.  This is well defined
(independent of $V$).
\[\xymatrix{
    \C^0(\U,\F)\ar[r]^d \ar@<-2pt>[d]_{\text{id}} \ar@<2pt>[d]^0 &
    \C^1(\U,\F)\ar[r]^d \ar[dl]_k \ar@<-2pt>[d]_{\text{id}} \ar@<2pt>[d]^0 &
    \C^1(\U,\F)\ar[r]^d \ar[dl]_k \ar@<-2pt>[d]_{\text{id}} \ar@<2pt>[d]^0 &
    \cdots \ar[dl]_k \\
    \C^0(\U,\F)\ar[r]^d & \C^1(\U,\F)\ar[r]^d & \C^1(\U,\F)\ar[r]^d & \cdots\\
}\]
 \vspace{-5mm}
\begin{itemize}
 \item[] \begin{claim} $(kd+dk)(\alpha_x) = \alpha_x$ for all
 $\alpha_x\in\C^p(\U,\F)_x$.  That is, $k$ is a homotopy between
 the identity map on the complex $\C^{\cdot}(\U,\F)$ and the zero
 map. \end{claim}
 \begin{proof}[Proof of Claim]
  Compute away:
  \begin{align*}
   (d(k(\alpha_x)))_{i_0\dots i_p} &= \sum_{l=0}^p (-1)^l
   (k(\alpha_x))_{i_0\dots \hat i_l \dots i_p}|_{U_{i_0\dots
   i_p}}\\
   &= \sum_{l=0}^p (-1)^l \alpha_{j,i_0\dots \hat i_l \dots
   i_p}|_{U_{i_0\dots i_p}}\\
   (k(d(\alpha_x)))_{i_0\dots i_p} &= (d\alpha)_{j,i_0\dots i_p}\\
   &= \alpha_{i_0\dots i_p} +
   \sum_{l=0}^p (-1)^{l+1}\alpha_{j,i_0\dots \hat i_l \dots
   i_p}|_{U_{i_0\dots i_p}}
  \end{align*}
  Now add the two and you get $\alpha_{i_0\dots i_p}$.
 \renewcommand{\qedsymbol}{$\square_{\text{Claim}}$}
 \end{proof}
\end{itemize}

Since the identity map on the complex $\C^{\cdot}(\U,\F)$ is
homotopic to the identity, the induced maps on homologies are the
same, so the homologies are zero\footnote{Another way to say this:
if $\alpha$ is a cocycle, then we have that $\alpha =
(kd+dk)\alpha = d(k\alpha)$, so it is a coboundary.}. That is, the
sequence is exact for $p\ge 1$, as desired.
\end{proof}

\begin{lemma}\label{L:lec11cohommap}
Let $X,\U,\F$ be as usual.  Then there is a map
\[
    \check H^p(\U,\F) \to H^p(X,\F)
\]
for all $p\in \mathbb{N}$ which is natural in $\F$.
\end{lemma}
\begin{proof}
Let $0\to \F \to \I^{\cdot}$ be an injective resolution of $\F$.
Then by Lemma (\ref{L:lec2lemma1}), there are maps $f^p$ for all
$p\in \mathbb{N}$ such that
\[\xymatrix{
 0 \ar[r] & \F \ar[r]^(.4){\epsilon}\ar@{}[d]|{\parallel} &
 \C^0(\U,\F)\ar[r]^d \ar[d]^{f^0} & \C^1(\U,\F)
 \ar[r]^d\ar[d]^{f^1} & \cdots\\
 0 \ar[r] & \F \ar[r]_{\epsilon} & \I^0\ar[r] & \I^1\ar[r] & \cdots
}\] and the system of maps $f^{\cdot}$ is unique up to homotopy.
Eliminating the first column and taking global sections, we have
\[\xymatrix{
 0\ar[r] & C^0(\U,\F) \ar[r]^d \ar[d] & C^1(\U,\F)\ar[r]^(.6)d \ar[d] & \cdots \\
 0\ar[r] & \Gamma(X,\I^0)\ar[r] & \Gamma(X,\I^1)\ar[r] & \cdots
}\] Taking homologies, we get well defined maps
\[
    \check H^p(\U,\F)\to H^p(X,\F)
\]
for all $p\in \mathbb{N}$.

Finally, we need to show naturality.  That is, we need to show
that for all maps $\F\to \G$, the box
\[\xymatrix{
 \check H^p(\U,\F) \ar[r]\ar[d] & H^p(X,\F)\ar[d]\\
 \check H^p(\U,\G) \ar[r] & H^p(X,\G)
}\] commutes.  To see this, let $\J^{\cdot}$ be an injective
resolution of $\G$, then observe that the diagram of complexes
\[\xymatrix{
 C^{\cdot}(\U,\F) \ar[r]\ar[d] & \I^{\cdot}\ar[d]\\
 C^{\cdot}(\U,\G) \ar[r] & \J^{\cdot}
}\] commutes up to homotopy\footnote{I didn't actually check
this.}. Thus, when we take homologies, we get a commutative
diagram.
\end{proof}

\begin{lemma}
Let $X,\U,\F$ be as usual.  If $\F$ is flasque, then
\[ \check H^p(\U,\F)=0 \] for all $p>0$.
\end{lemma}
\begin{proof}
For all $V\subseteq X$ open and for all $p$, we have
\[\xymatrix{
 \C^p(\U,\F)(X)\ar[r] \ar@{}[d]|{\parallel} &
 \C^p(\U,\F)(V)\ar@{}[d]|{\parallel}\\
 \prod \F(U_{i_0\dots i_p}) \ar[r] & \prod \F(U_{i_0\dots i_p}\cap
 V)
}\]where the bottom arrow is surjective because it is surjective
componentwise.  Thus, the top arrow is surjective.  This shows
that $\C^p(\U,\F)$ is flasque.  Thus, $\C^{\cdot}(\U,\F)$ is a
flasque resolution of $\F$, so we can use it to compute derived
functor cohomology.
\begin{align*}
 H^p(X,\F) &= h^p(\Gamma(X,\C^{\cdot}(\U,\F)))\\
    &= h^p(C^{\cdot}(\U,\F))\\
    &= \check H^p(\U,\F)
\end{align*}
But $H^p(X,\F)=0$ for all $p>0$ because $\F$ is flasque.
\end{proof}

\begin{theorem}[Exercise III.4.11] \label{T:lec11} \marginpar{\v{C}ech Cohomology
agrees with Derived Functor Cohomology for the right $\U$} Let
$X,\U,\F$ be as usual. Assume that $H^l(U_{i_0\dots
i_p},\F|_{U_{i_0\dots i_p}})=0$ for all $p\in \mathbb{N}$,
$i_0<\cdots i_p$, and $l>0$.  Then the maps from Lemma
(\ref{L:lec11cohommap}) are isomorphisms:
\[
    \check H^p(\U,\F) \xrightarrow{\sim} H^p(X,\F).
\]
\end{theorem}
\begin{proof}
We apply induction on $p$.  If $p=0$, then both cohomologies are
isomorphic to $\Gamma(X,\F$.

For $p>0$, assume the result up to $p-1$.  Embed $\F$ into a
flasque sheaf $\G$, and let $\R$ be the quotient:
\[
    0\to \F \to \G \to \R \to 0
\]
Then we get the exact sequence
\[
    0\to \F(U_{i_0\dots i_p}) \to \G(U_{i_0\dots i_p}) \to \R(U_{i_0\dots
    i_p}) \to \underbrace{H^1(U_{i_0\dots i_p},\F|_{U_{i_0\dots
    i_p}})}_{0\text{ by assumption}}
\]
That is,
\begin{equation}\label{S:lec11seq}
    0\to C^{\cdot}(\U,\F)\to C^{\cdot}(\U,\G) \to C^{\cdot}(\U,\R) \to 0
\end{equation}
is exact, so we get a long exact sequence in \v{C}ech cohomology:
\[
    0\to \check H^0(\U,\F) \to  \check H^0(\U,\G) \to  \check H^0(\U,\R)
    \to  \check H^1(\U,\F) \to  \underbrace{\check H^1(\U,\G)}_{0\text{ since $\G$ flasque}}
\]

Let $0\to \F \to \I^{\cdot}$ and $0\to \R \to \J^{\cdot}$ be
injective resolutions.  Then, as in Theorem (\ref{T:lec2thm}), we
see that $\I^{\cdot}\oplus \J^{\cdot}$ is an injective resolution
for $\G$. such that
 \[\xymatrix{
 & 0\ar[d] & 0\ar[d] & 0\ar[d]\\
 0\ar[r] &\F\ar[r]\ar[d] & \G\ar[r]\ar[d] & \R\ar[r]\ar[d] & 0\\
 0\ar[r]& \I^0\ar[r]\ar[d] & \I^0\times \J^0 \ar[r] \ar[d] & \J^0 \ar[r]
 \ar[d] & 0\\
 & \vdots & \vdots & \vdots
} \] commutes.  Then we will get a diagram
\[\xymatrix{
 0\ar[r] & C^{\cdot}(\U,\F)\ar[r] \ar[d] & C^{\cdot}(\U,\G) \ar[r] \ar[d] &
 C^{\cdot}(\U,\R) \ar[r] \ar[d] & 0\\
 0\ar[r] & \Gamma(X,\I^{\cdot}) \ar[r] & \Gamma(X,\I^{\cdot}\oplus
 \J^{\cdot}) \ar[r] & \Gamma(X,\J^{\cdot}) \ar[r] & 0
}\tag{loose end 1}\] such that the rows are exact and it commutes.

Taking long exact sequences, we get
\[\xymatrix{
    0\ar[r]& \check H^0(\U,\F) \ar[r] \ar[d]^{\wr} &  \check H^0(\U,\G) \ar[r] \ar[d]^{\wr}
    &  \check H^0(\U,\R) \ar[r] \ar[d]^{\wr} &  \check H^1(\U,\F) \ar[r] \ar[d] &  0 \\
    0\ar[r]& H^0(\U,\F) \ar[r]& H^0(\U,\G) \ar[r]&  H^0(\U,\R)
    \ar[r]& H^1(\U,\F) \ar[r]&  0
}\] So the map on the right must also be an isomorphism (though we
still need to show it is the same as the map obtained in Lemma
(\ref{L:lec11cohommap}) ... this is loose end 2).

For all $p>1$, the long exact sequence obtained from
(\ref{S:lec11seq}) give us that
\[\xymatrix{
 0\ar@{}[r]|(.3){=} & \check H^{p-1}(\U,\G) \ar[r] &
 \check H^{p-1}(\U,\R) \ar[d]^{\wr\text{ induction}} \ar[r]^{\sim}& \check H^p(\U,\F)\ar[r]\ar[d] & \check H^p(\U,\G)\ar@{}[r]|(.7){=} & 0 \\
 0\ar@{}[r]|(.3){=} & H^{p-1}(X,\G) \ar[r] & H^{p-1}(X,\R) \ar[r]^{\sim} & H^p(X,\F)\ar[r] &
 H^p(X,\G) \ar@{}[r]|(.7){=} & 0
}\]
 To use induction, we need $\R$ to satisfy the hypothesis.  That
 is, we need to check that $H^l(U_{i_0\dots i_p},\R|_{U_{i_0\dots
 i_p}})=0$ for all $l>0$, $p\in \mathbb{N}$, and $i_0<\cdots<i_p$.
 To see this, note that we have
 \[
    0\to \F|_{U_{i_0\dots i_p}} \to \G|_{U_{i_0\dots i_p}} \to
    \R|_ {U_{i_0\dots i_p}}\to 0
 \]
 exact, so the long exact sequence in cohomology tells us that
 \[
    \underbrace{H^l(U_{i_0\dots i_p},\G|_{U_{i_0\dots i_p}})}_{0\text{ since $\G$ flasque}}
    \to H^l(U_{i_0\dots i_p},\R|_{U_{i_0\dots i_p}}) \to \underbrace{H^{l+1}(U_{i_0\dots i_p},\F|_{U_{i_0\dots
    i_p}})}_{0\text{ by assumption}}
 \]
 for all $l>0$, so $H^l(U_{i_0\dots i_p},\R|_{U_{i_0\dots
 i_p}})=0$, as required.

Now only two loose ends remain.
\renewcommand{\qedsymbol}{\text{\tiny proof continued in next lecture}}
\end{proof}
