 \stepcounter{lecture}
 \setcounter{lecture}{18}
 \sektion{Lecture 18}

 Last time: If $Y=\spec A$ and $X=\P^n_Y$, then there is an exact
 sequence
 \[
    0\to \Omega_{X/Y} \to \O(-1)^{n+1} \to \O_X\to 0.
 \]
  By base change from $\spec \Z$, this is true for arbitrary $Y$.
  By Exercise II.5.16b, we have that $\wedge^{n+1}\O(-1)^{n+1} =
  \O(-n-1)\cong \wedge^1 \O(-1)\otimes \wedge^n
  \Omega_{X/Y}$.

  \begin{definition}
  Let $X$ be a non-singular variety over an algebraically closed
  field $k$.  Then the \emph{canonical sheaf} is $\omega_X =
  \wedge^n\Omega_{X/k}$, where $n=\dim X$.
  \end{definition}

 So if $X=\P^n_k$, then $\omega_X=\O(-n-1)$.  Concretely, say
 $X=\proj k[x_0,\dots,x_n]$.  Then on $D_+(x_0)\cong \spec
 k[y_1,\dots,y_n]$ for $y_i=x_i/x_0$, $\Omega_{X/k}|_{D_+(x_0)}$ is
 free with generators $dy_1,\dots, dy_n$.  Thus,
 $\omega_X|_{D_+(x_0)}$ is free of rank 1, generated by
 $dy_1\wedge\cdots\wedge dy_n$.

 On $D_+(x_n) = \spec k[x_0/x_n,\dots,x_{n-1}/x_n]$,
 $\Omega_{X/k}|_{D_+(x_n)}$ is generated by $d(x_0/x_n),\dots,
 d(x_{n-1}/x_n)$, so $\omega_{X}|_{D_+(x_n)}$ is free, generated
 by $d(x_0/x_n)\wedge \cdots \wedge d(x_{n-1}/x_n)$.

 On $D_+(x_0)\cap D_+(x_n)$, this generator is
 \begin{align*}
  d(1/y_n)\wedge &d(y_1/y_n)\wedge\cdots \wedge d(y_{n-1}/y_n) =\\
   &=
    (-y_n^{-2}dy_n)\wedge
    (y_n^{-1}dy_1-y_n^{-2}y_1dy_n)\wedge\cdots\wedge(y_n^{-1}dy_{n-1}-y_n^{-2}y_{n-1}dy_n)\\
   &= -y_n^{-n-1} dy_n\wedge dy_1\wedge \cdots \wedge dy_{n-1}
  \end{align*}
  which has a pole of order $n+1$ at $y_n=0$.  So the divisor of $d(x_0/x_n),\dots,
 d(x_{n-1}/x_n)$ is $-(n+1)\{x_n=0\}$, and
 \[
    \L(-(n+1)\{x_n=0\}) = \O(-n-1).
 \]

 \vspace{3mm}

 \marginpar{\S III.7: Serre Duality}
 In our computations on $\P^n$, we computed $H^i(\P^n_A,\O(q))$
 and came up with the perfect pairing
 \[
    \underbrace{\hom(\O_X,\O(r))}_{\hom(\O(-r-n-1),\omega)}\times
    H^n(\P^n_A what???
 \]
 in the works

 \begin{theorem}[III.7.1]\label{T:III.7.1} Let $k$ be a field and $X=\P^n_k$.  Then
 \begin{itemize}
  \item[(a)] $H^n(X,\omega_X) \cong k$
  \item[(b)] Fix such an isomorphism.  For all $\F\in
  \mathfrak{Coh}(X)$, the pairing
  \[
    \hom(\F,\omega)\times H^n(X,\F) \to H^n(X,\omega)
    \xrightarrow{\sim} k
  \]
  is a perfect pairing of finite dimensional vector spaces over
  $k$, and
  \item[(c)] For all $i\ge 0$, there is a natural isomorphism
  \[
    \ext^i(\F,\omega) \xrightarrow{\sim} H^{n-i}(X,\F)'
  \]
  which for $i=0$ is the isomorphism comming from the pairing (b)
  (and isn't canonical).
 \end{itemize}
 \end{theorem}

 \begin{proof}
 in the works
 \end{proof}

 \begin{definition} \marginpar{dualizing sheaf!}
 Let $X$ be a proper scheme over a field $k$, of
 dimension $n$.  A \emph{dualizing sheaf} for $X$ (over $k$) is a
 coherent sheaf $\omega_X^{\circ}$ on $X$ which represents the
 contravariant functor
 \[
    \mathfrak{Coh}(X)\to \Mod(k)
 \]
 given by $\F\mapsto H^n(X,\F)'$.  That is,
 \[
    H^n(X,-)'\cong \hom (-,\omega_X^{\circ}).
 \]
 \end{definition}

 Given $\alpha: \hom(-,\omega_X^{\circ})\xrightarrow{\sim}
 H^n(X,-)'$, we have
 \begin{align*}
  \alpha(\omega_X^{\circ}):
  \hom(\omega_X^{\circ},\omega_X^{\circ})&\to
  H^n(X,\omega_X^{\circ})'\\
  \id_{\omega_X^{\circ}} &\mapsto t
 \end{align*}
 So $\alpha$ gives us a $t:H^n(X,\omega_X^{\circ})\to k$.
 Conversely, given such a $t$, there is \emph{at most} one
 $\alpha$ inducing it because for all $\F$ and for all $\phi:\F\to
 \omega_X^{\circ}$, the diagram
 \[\xymatrix{
 \omega_X^{\circ}& & \hom(\omega_X^{\circ},\omega_X^{\circ})\ar[r]\ar[d]_{\hom(\phi,\omega_X^{\circ})}
 & H^n(X,\omega_X^{\circ})' \ar[d]^{H^n(X,\phi)} \\
 \F\ar[u]^{\phi}& & \hom(\F,\omega_X^{\circ}) \ar[r]^{\alpha(\F)} & H^n(X,\F)'
 }\]
 commutes.  So
 \[
 \alpha(\F)(\phi) =
 [H^n(X,\F)\xrightarrow{H^n(X,\phi)}H^n(X,\omega_X^{\circ})
 \xrightarrow{t} k]\in H^n(X,\F)' \tag{$\ast$}.
 \]
 If $\alpha$ exists, then it gives an isomorphism
 \[
 \alpha(\F):\hom(\F,\omega_X^{\circ})\xrightarrow{\sim} H^n(X,\F)'
 \]
 for all $\F$, and by $(\ast)$ it is the map associated with the
 pairing
 \[
    \hom(\F,\omega_X^{\circ})\times H^n(X,\F) \to
    H^n(X,\omega_X^{\circ})\xrightarrow{t} k
 \]
 and conversely\footnote{what?}.

 \begin{corollary} The dualizing sheaf, if it exists, is unique up
 to unique isomorphism.
 \end{corollary}

 \begin{lemma} Let $k$ be a field, $P=\P^N_k$, and let $X$ be a
 closed subscheme of $P$ of codimension $r$ (i.e. $r = \inf_{Z\subseteq X\
 irr} \codim Z$).  Then
 \[
    \Ext^i_P(\O_X,\omega_P)=0
 \]
 for all $i<r$.
 \end{lemma}
 \begin{proof}
 in the works
 \end{proof}
 Note that $X$ doesn't have to be equidimensional.
