 \stepcounter{lecture}
 \setcounter{lecture}{27}
 \sektion{Lecture 27}

 Comments on the homework:
 \begin{itemize}
 \item $\Gamma(-)$ does \emph{not} commute with $\otimes$ (in
 general).
 \item Exercises 9.3cd give ways in which 9.7 fails when $\dim Y
 >1$.
 \item You \emph{do} need to prove that $x\otimes z \pm y\otimes
 w\not=0$ in $(x,y)\otimes A$
 \item In 9.3c, to show that $I\subseteq k[x,y,z,w]$ is primary,
 you can use the fact that it's homogeneous in $z$ and $w$, but
 you cannot immediately reduce to working with homogeneous
 elements.
 \end{itemize}

 All schemes today are assumed to be of finite type over
 appropriate field.

 Loose end from last time:  Let $X$ be a scheme over $k$, let
 $k'$ be an extension of $k$, and let $X'=X\times_k k'$, then
 $\dim X' = \dim X$
 \begin{proof}
 Let $\spec A$ be an open affine in $X$, and let $A'=A\otimes_k
 k'$, so that $\spec A' = \pi^{-1}(\spec A)$, where $\pi:X'\to X$
 is the projection.  By Noether's normalization lemma, there is a
 subring $B\subseteq A$ with $k\subseteq B$, $B\cong k[x_1,\dots,
 x_r]$, and $A$ is finite over $B$.

 \[\xymatrix{
  A'\ar@{-}[r]\ar@{-}[d] & A\ar@{-}[d] \\
  B'\ar@{-}[d] & B\ar@{-}[d] \\
  k' \ar@{-}[r] & k
 }\]
 Then $B':=B\otimes_k k'$ is $\cong k'[x_1,\dots,x_r]$, and $A'$
 is finite over $B'$.  Then $\dim A'= \dim B' = r = \dim B = \dim
 A$ (Note that $A$ is integral over $B$; then $\dim A \ge \dim B$
 by lying over and going up for integral extensions, and $\dim
 A\le \dim B$ by incomparability for integral extensions).  Then
 $\dim X = \max_A \dim A = \max_A \dim A' = \dim X'$.
 \end{proof}

 \underline{Also}, $\pi:X'\to X$ is onto.  To see this, we may assume that $X=\spec
 A$.  We have that $k'$ is faithfully flat over $k$, so $a'$ is
 faithfully flat over $A$ (by base change: $M\otimes_k k' =
 M\otimes_A(A\otimes_k k') = M\otimes_A A' = 0$, so $M=0$).  Then
 $\spec A'\to \spec A$ is surjective by homework.

 \underline{Also}, all irreducible components of $X'$ dominate irreducible
 components of $X$.  This follows from going down for flat
 extensions.

 Back to smooth morphisms.  We were proving part (c) of the
 proposition: if $X\xrightarrow{f} Y \xrightarrow{g} Z$ are smooth
 of relative dimension $n$ and $m$, resp., then $g\circ f$ is
 smooth of relative dimension $n+m$.
 \begin{proof}
 (1) flatness is obvious.

 (2) Let $X'$, $Y'$, and $Z'$ be irreducible components of
 $X,Y,Z$, respectively, such that $f(X')\subseteq Y'$ and
 $g(Y')\subseteq Z'$.  Then $\dim X' = \dim Z' + n + m$ by (2)

 (3) Use the first exact sequence: let $x\in X$, let $y=f(x)$, and
 $z=g(f(x))$.  Then
 \[
    f^*\Omega_{Y/Z} \to \Omega_{X/Z} \to \Omega_{X/Y} \to 0
 \]
 is exact, so
 \[
    \underbrace{f^*\Omega_{Y/Z}\otimes k(x)}_{(\Omega_{Y/Z}\otimes k(y))\otimes k(x) } \to \Omega_{X/Z}\otimes k(x) \to
    \underbrace{\Omega_{X/Y}\otimes k(x)}_{\dim_{k(x)} = n} \to 0
 \]
 dimension of first term over $k(x)$ is $m$, so the dimension over
 $k(x)$ of the middle is $\le n+m$.

 Show the other inequality.  Note that $x\in X_z$, so
 \[
    \Omega_{X/Z}\otimes k(x) = \Omega_{X_x/k(z)} \otimes k(x).
 \]
 By (2'), for $g\circ f$, $X_z$ has pure dimension $n+m$.  Let
 $X'$ be an irreducible component of $X_z$ containing $x$.  By the
 second exact sequence:
 \[
    i^*\Omega_{X_z/k(z)} \to \Omega_{X'/k(z)} \to 0
 \]
 is exact where $i:X'\to X_z$, so
 \[
    \underbrace{i^*\Omega_{X_z/k(z)}\otimes k(x)}_{\Omega_{X_z/k(z)}\otimes k(x)} \to \Omega_{X'/k(z)}\otimes k(x) \to 0
 \]
 is exact,
 so it suffices to show that $\Omega_{X_z/k(z)}\otimes k(x)$ has
 dimension $\ge n+m$.  We know that $\Omega_{X'/k(z)}$ has rank
 $\ge n+m$ at the generic point, and it is coherent, so the rank
 does not decrease when you specialize to $x$ (use Nakayama's Lemma).
 \end{proof}

 \begin{theorem}
 Let $f:X\to Y$ be a morphism of schemes (of finite type) over
 $k$.  Then $f$ is smooth of relative dimension $n$, if and only
 if
 \begin{itemize}
 \item[(1)] $f$ is flat.
 \item[(2)] for all $y\in Y$, the geometric fiber $X_{\bar y} =
 X_y\times_{k(y)} \overline{k(y)}$ is regular of pure dimension
 $n$.
 \end{itemize}
 \end{theorem}
 \begin{proof}
 ($\Rightarrow$) (1) is obvious. (2): by base change, $X_{\bar y}$
 is smooth over $\overline{k(y)}$, so it is regular by (II.8.8 or
 II.8.15).

 ($\Leftarrow$) (1) implies condition (1) for smoothness.  (2)
 implies (2') for smoothness (all fibers have the same dimension,
 so geometric fibers have the same dimension).  (2) also implies
 (3') for smoothness.  To see this, note that $\Omega_{X_{\bar y}/\overline{k(y)}}$
 is locally free of rank $n$ by II.8.8, which
 implies that $\Omega_{X_{\bar y}/\overline{k(y)}} \otimes\overline{k(y)}$
  has dimension $n$ for all $y$, which implies
 that $\Omega_{X_y/k(y)}\otimes k(x)$ has dimension $n$ for all
 $x\in X_y$ since $\Omega_{X_{\bar y}/\overline{k(y)}} = \Omega_{X_y/k(y)}\otimes_{k(y)} \overline{k(y)}$
 \end{proof}

 \begin{corollary}
 Let $f:X\to Y$ be a morphism of schemes (of finite type) over
 $k$.  Then $f$ is smooth of relative dimension $n$ if and only if
 $f$ is flat and $X_y$ is smooth of relative dimension $n$ over
 $k(y)$ for all $y\in Y$. (second part equivalent to (2))
 \end{corollary}

\underline{Smoothness over a field:}

 \begin{theorem}[EGA IV 6.74a] Let $X$ be a scheme of finite type
 over a field $k$, and let $k'$ be an extension field of $k$, let
 $X'=X\times_k k'$, let $\pi:X' \to X$ be the projection, let
 $x'\in X'$ and let $x=\pi(x')$.  Then
 \begin{itemize}
 \item[(a)] If $X'$ is regular at $x'$, then $X$ is regular at
 $x$.
 \item[(b)] The converse holds if $k'$ is separable over $k$.
 \end{itemize}
 \end{theorem}

 \begin{corollary}
 Since $\pi$ is surjective,
 \begin{itemize}
 \item[(a)] $X'$ is regular implies that $X$ is
 regular, and
 \item[(b)] the converse holds if $k'$ is separable over $k$.
 \end{itemize} Note that $(b)$ is false in general (Ex 10.1).
 \end{corollary}

 \begin{corollary}
 Let $X$ be a scheme of finite type over a perfect field $k$.  If
 $X$ is regular, then it is smooth over $k$.
 \end{corollary}
 \begin{proof}
 $X\times_k \bar k$ is regular, so $X$ is smooth.
 \end{proof}

 \begin{corollary}
 If $X$ is smooth over an arbitrary field $k$, then it is regular by (b) ($\bar k$
 is separable over $k$).
 \end{corollary}
 \begin{proof}
 $X\times_k \bar k$ is smooth over $\bar k$, so $X\times_k \bar k$
 i s regular, so $X$ is regular by (a) (of the first corollary).
 \end{proof}

 A ``more self-contained'' proof:
 \begin{lemma}
 Let $k$ be a field.  Then $\A^n_k$ is regular.
 \end{lemma}
 \begin{proof}
 We need to show that $k[x_1,\dots, x_n]$ is a regular ring.  This
 follows from Matsumura, \underline{Commutative Ring Theory}
 [1986], Theorem 19.5\footnote{If $R$ is regular and noetherian, then so is
 $R[x]$}.  It is also Eisenbud, Exercise 19.3
 \end{proof}

 \begin{proposition}
 Let $X$ be a smooth scheme of finite type over a field $k$.  Then
 $X$ is regular.
 \end{proposition}
 \begin{proof}
 We may assume $X=\spec A$ is affine.  Choose a generating set
 $x_1,\dots ,x_n\in A$, so $k[x_1,\dots, x_n]\twoheadrightarrow
 A$, so $X\hookrightarrow \A^n_k$.  By Matsumura Thm 19.3 (a
 localization of a regular local ring at a prime ideal is
 regular), it suffices to show that $X$ is regular at $x$ for all
 closed points $x\in X$.  Let $\O$ be the local ring
 $\O_{\A^n_k,x}$, let $\m$ be its maximal ideal, and let $I$ be
 the kernel of the surjection $\O\twoheadrightarrow \O_{X,x}$.
 Then $I\subseteq \m$ ($x\in X$).  Let $d=\dim \O_{X,x} = \dim_x
 X$.  Use the second exact sequence:
 \[
    I/I^2 \xrightarrow{\delta} \Omega_{\O/k}\otimes_{\O} \O_{X,x} \xrightarrow{\alpha}
    \Omega_{\O_{X,x}/k} \to 0
 \]
 is exact, so
 \[
    (I/I^2) \otimes k(x) \xrightarrow{\delta'}
    \underbrace{\Omega_{\O/k}\otimes_ k(x)}_{\txt{dim $n$ over $k(x)$}} \xrightarrow{\alpha'}
    \underbrace{\Omega_{\O_{X,x}/k}\otimes k(x)}_{\txt{dim $d$ over $k(x)$}} \to 0
 \]
 is exact.
 Let $r-n-d$, and pick $f_1,\dots, f_r\in I$ such that
 $\delta'((f_i \text{mod } I^2\otimes 1)$ generate $\ker \alpha'$.
 Let $\O' = \O/(f_1,\dots, f_r)$.  It is local with maximal ideal
 $\m' = \m/(f_1,\dots,f_r)$.  Then $\m'/\m'^2 \cong \m/(\m^2+(f_1,\dots, f_r))$
 has dimension $\ge n-r=d$ over $k(x)$. But also $\O'$ maps
 surjectively to $\O_{X,x}$, so
 \[
   \dim_{k(x)}\m'/\m'^2 \ge \dim \O' \ge \dim \O_{X,x} = \text{Stuck}
 \]
 is also exact.
 \end{proof}

 \underline{Next:} \'{E}tale morphisms

 \begin{definition}[Ex 10.3]
 Let $f:X\to Y$ be a morphism of schemes (of finite type) over
 $k$.  Then
 \begin{itemize}
 \item[(a)] $f$ is \emph{\'{e}tale} if it is smooth of relative dimension
 0.
 \item[(b)] $f$ is \emph{unramified} if for all $x\in X$, letting
 $y=f(x)$, we have $\m_y \cdot \O_{X,x} = \m_x$ and $k(x)$ is a
 separable algebraic extension of $k(y)$.
 \end{itemize}
 \end{definition}

 \underline{Example:} Let $R=\Z[\sqrt{2}]$, then $\spec R$ is \'{e}tale
 over $\spec \Z$ except at the prime $(\sqrt{2})\subseteq R$.
 (These are not schemes over a field, but the same principles
 apply.)  At other primes, there may be residue field extensions,
 but they are separable.
