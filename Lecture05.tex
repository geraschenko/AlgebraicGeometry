 \stepcounter{lecture}
 \setcounter{lecture}{5}
 \sektion{Lecture 5}

Note on Freyd's Theorem:  every abelian category can be embedded
as a full subcategory of $\Ab$.  This does \emph{not} imply that
$\varinjlim$ is always exact (just because it is exact in $\Ab$).
For example, $\varprojlim$ (which is the $\varinjlim$ in
$\Ab^{\text{op}}$) is not exact.

\marginpar{III \S 3}

\begin{theorem}[Goal]\label{T:lec5}
Let $\F$ be a quasi-coherent sheaf on a noetherian affine scheme
$X=\spec A$, then $\F$ is acyclic. \marginpar{Qco sheaves on
noetherian affine schemes are acyclic}
\end{theorem}

Note: $A$ is noetherian (Proposition II.3.2) and $\F= \tilde{M}$
for some $A$-module $M$.

\begin{lemma}[Key Lemma]\label{L:lec5keylemma}
If $I$ is an injective module over a noetherian ring $A$, then the
sheaf $\tilde I$ on $\spec A$ is flasque.
\end{lemma}

\begin{proof}[Proof assuming Key Lemma]
Let $0\to M\to I^{\cdot}$ be an injective resolution of $M$ (in
$\Mod{A}$.  Then $0\to \tilde M\to \tilde I^{\cdot}$ is exact
($\tilde{}$ is exact).  By the Key Lemma, this is an acyclic
resolution, so we have $H^i(X,\F) = h^i(\Gamma(X,I^{\cdot})) =
h^i(I^{\cdot})=0$ for all $i>0$.
\end{proof}

\begin{lemma}\label{L:lec5IdealsEnoughInjectivity}
Let $A$ be a ring and $J$ an $A$-module.  Then $J$ is injective if
and only if for all ideals $\a\subseteq A$, all maps $\phi:\a\to
J$ extend to $A\to J$.
\end{lemma}
\begin{proof}
($\Rightarrow$) Trivial from definition of injectivity:$\xymatrix
@C=4mm @R=4mm{0\ar[r] & \a \ar[d]_(.4){\phi} \ar[r] &
A\ar@{.>}[dl]^(.4){\text{$J$ inj}} \\ & J}$

($\Leftarrow$) Let $M'\subseteq M$ be $A$-modules and let
$\phi:M'\to J$.  By the usual Zorn's Lemma argument, it suffices
to extend $\phi$ to a map $\langle M',x\rangle \to J$ for some
$x\in M\smallsetminus M'$.  We have the obvious surjection
$M'\oplus A \to \langle M',x\rangle$.  Let $\a=\{a\in A|ax\in
M'\}$ be the kernel
\begin{align*}
    0\to &\a \to M'\oplus A \to
    \langle M',x\rangle \to 0\\
    &a\mapsto (ax,-a)
\end{align*}
To extend $\phi$, we need a map $\psi:A\to J$ such that $\phi+
\psi: M'\oplus A \to J$ vanishes on $\a$.  This is exactly what we
get from the assumption.
\end{proof}

\begin{lemma}\label{L:lec5lemma1}
Let $I$ be an injective module over a noetherian ring $A$ and let
$f\in A$.  Then the localization map $\theta:I\to I_f$ is
surjective.
\end{lemma}
\begin{proof}
For all $i\in \mathbb{N}$, let $\b_i=\ann(f^i)$.  Then
$0=\b_0\subseteq \b_1\subseteq \cdots$.  By the noetherian
hypothesis, there is some $r$ such that $\b_r=\b_{r+1}=\cdots$.
Pick $x\in I_f$ and write $x=y/f^n$ with $y\in I$ and $n\in
\mathbb{N}$.  Define $\phi:(f^{n+r})\to I$ by $f^{n+r}\mapsto
f^ry$ [if $af^{n+r}=0$, then $a\in \b_{n+r}=\b_r$, so $af^r=0$ and
so the map is well defined]. \marginpar{$\xymatrix @C=4mm
@R=4mm{0\ar[r] & (f^{n+r}) \ar[r]\ar[d]_{\phi} &
A\ar@{.>}[dl]^{\psi} \\ & I}$}By injectivity, there is some
$\psi:A\to I$ such that $\psi|_{(f^{n+r})}=\phi$. Let $z=\psi(1)$,
then $\theta(z)=f^{-n}y=x$.
\end{proof}

\begin{definition}[Exercise II.5.6] Let $A$ be a ring, let $\a$ be a
principal ideal in $A$, and let $M$ be an $A$-module. Then
$J=\Gamma_{\a}(M) = \{x\in M|\a^nx=0 \text{ for some } n\in
\mathbb{N}\}$.
\end{definition}

\begin{lemma}
Let $X=\spec A$, $I$ be an $A$-module, $\a=(f)\subseteq A$ a
principal ideal, $J=\Gamma_{\a}(I)$, $U\subseteq X$ open, and
$Z=Z(f)=V_f$.  Then
\[
    J=\Gamma_Z(X,\tilde I) = \{s\in \Gamma(X,\tilde I)|s_P=0 \text{ for all } P\not\in Z\}
\]
and
\[
 \Gamma(U,\tilde J) = \Gamma_Z(U,\tilde I).
\]
\end{lemma}
\begin{proof}
\begin{align*}
\Gamma_Z(X,\tilde I) &= \{x\in I: x\mapsto 0 \in I_P \text{ for
all } P\not\in Z\}\\
    &= \{x\in I| x\mapsto 0\in I_f\} \\
    &= \ker(I\to I_f) \\
    &=\{x\in I| f^nx=0 \text{ for some } n\in \mathbb{N}\}\\
    &=\{x\in I| \a^nx=0 \text{ for some } n\in \mathbb{N}\}\\
    &= \Gamma_{\a}(I) = J.
\end{align*}

For the other claim, start with the special case $U=D(g)$, so
$\Gamma(U,\tilde J)=J_g=(\ker(I\to I_f))_g$.  So $x\in
\Gamma(U,\tilde J)$ if and only if there are $n,m\in \mathbb{N}$
such that $x=g^{-n}y$, $y\in I$ and $f^my=0$. On the other hand,
$\Gamma_Z(U,\tilde I)=\ker(I_g\to (I_g)_f)$,  So $x\in
\Gamma_Z(U,\tilde I)$ if and only if there are $n,n',m\in
\mathbb{N}$ such that $x=g^{-n}y$, $y\in I$ and $g^{n'}f^my=0$.
Thus, $\Gamma_Z(U,\tilde I)=\Gamma(U,\tilde J)$ in this case. For
the general case, cover $U$ with open affine sets an glue.
\end{proof}

\begin{lemma}\label{L:lec5lemma2}
 Let $A$ be a noetherian ring, $\a\in A$ an ideal, $I$
 an $A$-module, and $J=\Gamma_{\a}(I)$.  If $I$ is
 injective, then so is $J$.
\end{lemma}
\begin{proof}
By Lemma (\ref{L:lec5IdealsEnoughInjectivity}), it suffices to
complete the diagram
\[\xymatrix{
 0\ar[r] & \b \ar[r] \ar[d]_{\phi} & A \ar@{.>}[dl]\\
 & J
 }\]
for any ideal $\b$ and any $\phi: \b\to J$. For all $b\in \b$,
there is some $n$ such that $\phi(\a^nb) = \a^n\phi(b)=0$.  Since
$\b$ is finitely generated, we may choose one $n$ that works for
all $b$. Thus, $\a^n\b\subseteq \ker \phi$.

Recall \underline{Krull's Theorem}: Let $\a\subseteq A$ be an
ideal in a noetherian ring, and let $M\subseteq N$ be finitely
generated $A$-modules.  Then the $\a$-adic topology on $M$ is
induced by the $\a$-adic topology on $N$.  That is, for all $n$,
there is some $n'$ such that $\a^nM\supseteq M\cap \a^{n'}N$.
Thus, taking $M=\b$ and $N=A$, $\a^n\b \supseteq \b\cap \a^{n'}$.
So we have
\[\xymatrix@C=13mm{
 A\ar@{->>}[r] & A/\a^{n'}\ar@{.>}@/^/[dr] \ar@{.>}@/^/[drr]^{\psi}\\
 \b\ar@{^(->}[u]\ar[r] \ar@/_3ex/[rr]_{\phi} & \b/\b\cap \a^{n'}
 \ar@{^(->}[u] \ar[r]^(.6){\text{Krull}} & J \ar@{^(->}[r] & I
 }\]
By injectivity of $I$, $\psi$ exists such that the diagram
commutes.  If we can show $\im \psi \subseteq J$, then we're done.
This is true because the image is killed by $\a^{n'}$.
\end{proof}

\begin{proof}[Proof of Key Lemma]
Statement:if $X=\spec A$ is noetherian, $I\in \Mod(A)$ injective,
then $\tilde I$ is flasque.

We want to show that for all $U\subseteq V$, $\Gamma(V,\tilde
I)\to \Gamma(U,\tilde I)$ is
surjective.\marginpar{$\xymatrix@C=3mm
@R=3mm{\Gamma(V,\tilde I)\ar[r] & \Gamma(U,\tilde I)\\
\Gamma(X,\tilde I)\ar[u]\ar[ur]}$}  It suffices to check the case
where $V=X$.  Let $Y=\supp \tilde I = \{P\in X| I_P\not=0\}$.  Now
we apply noetherian induction on $Y$.  If $Y\cap U=\varnothing$,
then $\Gamma(U,\tilde I)=0$ and there is nothing to show.  So we
may assume that $Y$ meets $U$.  Then there is some $f\in A$ such
that $D(f)\subseteq U$ and $D(f)\cap Y\not=\varnothing$.  Let
$Z=X\smallsetminus D(f) = Z(f)$ and consider the commutative
diagram
\[\xymatrix{
 I=\Gamma(X,\tilde I) \ar[r] & \Gamma(U,\tilde I) \ar[r] & \Gamma(D(f),\tilde
 I)=I_f & \text{(row \emph{not} exact)}\\
 \Gamma_Z(X,\tilde I)\ar@{^(->}[u] \ar[r] & \Gamma_Z(U,\tilde
 I)\ar@{^(->}[u]\\
 J=\Gamma(X,\tilde J) \ar@{}[u]|{\parallel} \ar[r] & \Gamma(U,\tilde J)\ar@{}[u]|{\parallel}
 }\]
Let $s\in \Gamma(U,\tilde I)$, and let $s'\in \Gamma(D(f),\tilde
I)$ be its image.  By Lemma (\ref{L:lec5lemma1}) \marginpar{Lem
\ref{L:lec5lemma1}: $I\to I_f$ surjective} there is some $t\in
\Gamma(X,\tilde I)$ mapping to $s'$. Let $t'=t|_U\in
\Gamma(U,\tilde I)$, so that $s-t'\in \Gamma_Z(U,\tilde I)$ [$s$
and $t'$ agree on $D(f)=Z^c$].  So it suffices to check that the
bottom map is surjective.

By Lemma (\ref{L:lec5lemma2}), $J$ is injective \marginpar{Lem
\ref{L:lec5lemma2}: $I$ inj $\Rightarrow
J=\Gamma_{\mathfrak{a}}(I)$ inj}.  Also, $\supp J \subseteq Y\cap
Z \subsetneq Y$, so we can apply induction on the dimension of
$Y=\supp I$.
\end{proof}

\begin{corollary}[to the Theorem]\label{C:lec5}
Let $\F$ be a quasi-coherent sheaf on a noetherian scheme $X$.
Then $\F$ can be embedded into a flasque, quasi-coherent sheaf.
\end{corollary}
\begin{proof}
Cover $X$ with finitely many open affine sets $U_i=\spec A_i$,
$i=1\dots n$.  For each $i$, $\F|_{U_i} \cong \tilde M_i$ for some
$A_i$-module $M_i$.  Embed $M_i$ into an injective $A_i$-module
$I_i$, and let
\[
    \G = \bigoplus_{i=1}^n f_{i*}(\tilde I_i)
\]
where $f_i:U_i\to X$ is the inclusion map.  We have $\F|_{U_i}\to
\tilde I_i$, which is injective for each $i$, and we have
\[\F\to
\underbrace{f_{i*}(\F|_{U_i})}_{V\mapsto \F(V\cap U_i)} \to
f_{i*}(\tilde I_i).\] Composing and adding these maps, we get an
injection $\F\to \G$ [check on stalks that $\F\to f_{i*}(\tilde
I_i)$ is injective].  By the Key Lemma (\ref{L:lec5keylemma}),
$\tilde I_i$ is flasque for each $i$, so $f_{i*}(\tilde I_i)$ is
flasque by exercise II.1.16d, so $\G$ is flasque.  Similarly, $\G$
is quasi-coherent.
\end{proof}
