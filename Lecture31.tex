 \stepcounter{lecture}
 \setcounter{lecture}{31}
 \sektion{Lecture 31}

 Comments on Homework 12:
 \begin{itemize}
 \item[(i)]Common mistake (\#1) was failing to realize the difference between a closed point and a rational point.  $X$
 defined by $y^2=x^p-t$.  There is a point in $X$ corresponding to
 the point $(t^{1/p},0)$, namely $P= \spec k[x,y]/(y,x^p-t)$. Then
 $k(P)=k(t^{1/p})$.  This is a closed, but not rational point.
 \item[(ii)] (\#2) This can be used to \emph{define} \'etale
 (provided you rephrase it not to require that
 $k(x)\hookrightarrow \hat \O_x$
 \item[(iii)](\#3) This is an example of the general principal
 that irreducible components in $\spec \hat \O_{X,x}$ can be
 separated in an \'etale cover.
 \end{itemize}

 Back to Riemann-Roch.  Again, curves are smooth and projective
 over an algebraically closed field $k$, until we say otherwise.

 Recall that we have $\deg:\pic X\to \Z$ well-defined surjective group
 homomorphism.

 \begin{definition}
 $\pic^0 X =$ the kernel of $\deg$, which is $\{\L\in \pic X|
 \deg(\L)=0\}$.
 \end{definition}
 If $X$ has genus 0, then $X\cong \P^1$, so $\pic X=\Z$ (via
 $n\leftrightarrow \O(n)$, so $\pic^0 X$ is trivial.  In genus
 $>0$, $\pic X$ is non-trivial.  Pick two distinct points $P,Q$.
 Then $\O(P-Q)\not\cong \O_X$ (lest $X\cong \P^1$), so $\pic^0
 X\not=0$.

 Genus 1 curves:  Let $X$ be a curve of genus 1 and fix $P_0\in
 X$.  Then we have a map $\phi:X(k)\to \pic^0 X$ given by
 $P\mapsto \O(P-P_0)$.

 \begin{remark}
   The canonical divisor $K$ is $\sim 0$.  By RR applied to the
   divisor 0, $l(0)-l(K)=1-l(k)=0$.  So $l(K)=1$.  But $\deg
   K=2g-2 =0$, so $K\sim 0$ by an earlier lemma.
 \end{remark}

 \begin{proposition}
   $\phi$ as above is a bijection.
 \end{proposition}
 \begin{proof}
 1-1: $\phi(P)=\phi(Q) \Leftrightarrow P-P_0\sim Q-P_0 \sim P-Q
 \sim 0$, which cannot happen (lest $X\cong \P^1$)

 Onto: Pick $\L\in \pic^0 X$, and let $D$ be a corresponding
 divisor.  Apply RR to $D+P_0$:
 \[
    l(D+P_0)-l(K-D-P_0) = 1+1-g =1.
 \]
 so $l(D+P_0)=1$, so $D+P_0\sim Q$, for an effective divisor $Q$.
 Then $D\sim Q-P_0$, so $\L=\phi(Q)$
 \end{proof}

 \begin{corollary}
 There is a canonical (depending on choice of $P_0$) group
 structure on $X(k)$ given by $\phi$ and the group structure on
 $\pic^0 X$.
 \end{corollary}

 \begin{definition}
 An \emph{elliptic curve} is a smooth curve of genus 1 over a
 field $k$ together with a choice of rational point $P_0\in X(k)$.
 \end{definition}

 Note that $\phi(P_0)=\O(P_0-P_0)=\O_X$, so $P_0$ is the identity
 element.  Let's call it 0 from now on.

 Apply RR to some multiples of $[0]$:
 \[
    l([0])-\underbrace{l(K-[0])}_0 = 1+1-g =1 = \dim
    (H^0(X,\O([0]))=k)
 \]
 where we think of $\O([0])$ as a subsheaf of $K(X)$. Similarly
 \[
    l(2[0])=2
 \]
 A basis of $H^0(X,\O(2[0]))$ is $1,x$.\\
 $l(3[0])=3$ ... basis is $1,x,y$.\\
 $l(4[0])=4$ ... basis is $1,x,y,x^2$.\\
 $l(5[0])=5$ ... basis is $1,x,y,x^2,xy$.\\
 $l(6[0])=6$ ... basis is $1,x,y,x^2,xy$, ($x^2$ or $y^2$).\\

 So $1,x,y,x^2,xy,x^2,y^2$ are linearly dependent over $k$, and the
 resulting function $f$ \emph{does} involve $y^2$ and $x^3$.  By
 Exercise 1.3, $X\smallsetminus\{0\}$ is affine, and $x,y$ generate
 its affine ring: ($H^0(X,\O(n\cdot[0]))$ has basis
 $1,x,y,x^2,xy,x^3,x^2y, ...$), so $k[x,y]$ maps onto the affine
 ring, and the kernel is $(f)$.  So $X$ is the projective closure
 of $\spec k[x,y]/(f)$ in $\P^2$.  So it is a non-singular cubic in
 $\P^2$, and it has only one point at infinity, namely 0.

 If the characteristic of $k$ is not 2 or 3, then we can do a
 linear coordinate change so that $f$ is $y^2=4x^3+ax+b$ with
 $a,b\in k$.  This is called Weierstrass form.

 Let $P=(x_0,y_0)\in X$, and let $P'=(x_0,-y_0)$.  Then
 $[P]+[P']\sim 2\cdot[0]$ because $(x-x_0) = [P]+[P']-2[0]$.  So
 $P'=-P$ in the group.  Also, if $P,Q,R$ are colinear, then
 \[
    [P]+[Q]+[R]-3[0] = \text{the principal divisor } (\text{equation of the line})
 \]
 so $P+Q+R=0$ in the group law, so $Q+R=P'$.

 \marginpar{insert picture around here}

 \emph{Varieties over arbitrary fields:} now $k$ may not be
 algebraically closed.

 \underline{Examples:}
 \begin{itemize}
 \item[(i)] Let $k$ be a field.  On $\A^1_k=\spec k[x]$, you have
 the generic point, plus the closed points;  also, $\{\text{closed
 points}\} \leftrightarrow \{\text{irreducible monics}\}/\text{action of Aut}_k(\bar k)$
 \item[(ii)] $\sqrt{2}\in \A^1_{\Q}$, only it is paired with
 $-\sqrt{2}$. $\spec \Q[x]/(x^2-2)$.  $k{P}=\Q[\sqrt{2}]$.  it is
 a closed point, but not a rational point.
 \item[(iii)] Let $k=\mathbb{F}_p(t)$.  Then $\sqrt[p]{t}\in
 \A^1_k$, realized as $\spec k[x]/(x^p-t)$.  When we base change
 to $\bar k$, we get $\spec \bar k[x]/(x-t^{1/p})^p$, which is not
 reduced.\\
 So the point $t^{1/p}$ in $\A^1_k$ is not geometrically
 reduced.\\
 Likewise $\spec \Q[x]/(x^2-2)$ becomes $\bar
 \Q[x]/(x-\sqrt{2})(x+\sqrt{2})$ which is reducible, so our point
 is not geometrically irreducible. (See Exercise II.3.15)
 \end{itemize}
 What about $\A^1_k$?  This situation is similar:
 \[
    \{\text{closed points}\} ``='' \bar k^n/(\text{action of Aut}_k\bar k)
 \]

 \emph{Grothendieck Topologies:} (reference: Vistoli, Notes on
Grothendieck topologies fibered categories, and descent theory\footnote{This is on the
arXiv.})

 General idea: If you had a smooth map of complex manifolds $f:X\to
 Y$, you can work near $P\in X$ by finding a local section of $f$
 near $f(P)$.  You can't necessarily do this with schemes because
 the Zariski topology is too coarse (exceptions: $\P(\E)$, etc.).
 One way to appoximate the classical topology is to work in $\hat
 \O_{X,x}$. Then you have
 \[\xymatrix{
 \spec \hat \O_{X,x} \ar[d]\\
 \spec \hat \O_{Y,f(x)} \ar@{.>}@/^/[u]^{\exists}
 }\]
 The image is cut out by $f_1,\dots, f_r\in \hat \m_x/\hat\m_x^2$.
 These can be lifted to $\mod \hat\m^2$ to $f_1,\dots, f_r\in
 \m_x\smallsetminus \m_x^2$.  Lift to $f_1,\dots, f_r\in \O_X(U)$
 for some open neighborhood.  Take the closure $Z(f_1,\dots, f_r)$
 to get $Z\subseteq X$ closed subscheme.\\
 $Z$ may have many sheets over $Y$, and you can't keep one and
 throw out the others.  But you do have $f|_Z$ is \'etale at $x$
 ($x=P$).  You can base change to $Z:$
 \[\xymatrix{
 X' = X\times_Y Z\ar[d]\\
 Z\ar@/^/[u]
 }\]
 Of course $f|_Z$ is not \'etale everywhere, but it is in a
 Zariski-open neighborhood of $x$.\\
 This is called working locally in the \'etale topology.

 \def\C{\underline{\mathcal{C}}}

 \begin{definition}
 Let $\C$ be a category.  A \emph{Grothendieck topology} on $\C$
 is, for each object $U$, a collection of covers of $U$, where a
 cover is a set of morphisms $\{U_i\to U\}$ in $\C$ satisfying:
 \begin{itemize}
 \item[(i)] if $V\to U$ is an isomorphism in $\C$, then $V\to U$
 is a cover.
 \item[(ii)] if $\{U_i\to U\}$ is a covering and $V\to U$ is a
 morphism, then $U_i\times_U V$ exists in $\C$, and $\{U_i\times_U
 V\}$ is a covering (of $V$)
 \item[(iii)] if $\{U_i\to U\}$ is a covering, and $\{V_{i,j}\to
 U_i\}_{j\in J}$ is a covering for each $i$, then the composites
 $\{V_{i,j}\to U\}$ form a covering.
 \end{itemize}
 \end{definition}

 \underline{Example:} Let $\C=$ category of schemes.  Given $U\in
 \C$, a covering is a collection $\{U_i\to U\}$ of open immersions
 whose images cover $U$.
 \begin{itemize}
 \item[(i)] ok
 \item[(ii)] if $\{U_i\}$ cover $U$, then $\{f^{-1}(U_i)\}$ cover
 $V$.
 \item[(iii)] trivial
 \end{itemize}

 This is how the Zariski topology is realized as a Grothendieck
 topology.
 \begin{definition}
 The Grothendieck topology on $U$ is the collection of open
 covers.
 \end{definition}
 \begin{remark}
 Grothendieck topologies versus the earlier kind:
 \begin{itemize}
 \item open sets are now morphisms
 \item intersections are now fibered products over $U$
 \item unions are now obsoleted by looking at covers
 \end{itemize}
 \end{remark}

 \begin{definition}
 A \emph{site} is a category with a Grothendieck topology.
 \end{definition}

 Another example of a site: fix a field $k$.  Let $\C=$ category
 of schemes of finite type over $k$.  Given $U\in \C$, a covering
 of $U$ is a finite collection $\{U_i\to U\}$ of \'etale morphisms
 such that the union of the images is all of $U$.  This ``is'' the
 \'etale site.

 Exercise 10.5 says: if a (Zariski) sheaf is locally free as a
 sheaf on the \'etale topology, then it is locally free under the
 Zariski topology.
