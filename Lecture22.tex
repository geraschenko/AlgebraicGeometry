 \stepcounter{lecture}
 \setcounter{lecture}{22}
 \sektion{Lecture 22}

\begin{corollary}[Adunction formula] In this situation,
 \[
 \omega_Y \cong \omega_X\otimes \wedge^r (\I/\I^2)\check{}
 \]
 \end{corollary}
 \begin{proof}
 By Exercise II.5.16d, $\wedge^n(\Omega_{X/k} \otimes
 \wedge^r(\I/\I^2)$.  The left hand side is $(\wedge^n
 \Omega_{X/k}) \otimes \O_Y = \omega_X\otimes \O_Y$.  Here $n=\dim
 X$ and $q=\dim Y$.  Also, $\wedge^q\Omega_{Y/k} = \omega_Y$, so
 \[
    \omega_Y \cong \omega_X\otimes \O_Y\otimes
    \wedge^r(\I/\I^2)\check{} \cong \omega_X \otimes
    \wedge^r(\I/\I^2)\check{}.
 \] \end{proof}

\begin{corollary} If $X$ is a non-singular projective variety, the
$\omega_X^{\circ} \cong \omega_X$. \end{corollary}
 \begin{proof}
 Embed $X$ into $P=\P^N_k$ and let $\I$ be the sheaf of ideals.
 Then
 \[
    \omega_X^{\circ} \cong \omega_P \otimes
    \wedge^r(\I/\I^2)\check{} \cong \omega_X.
 \]
 \end{proof}

 \begin{corollary}
 If $Y\subseteq X$ are non-singular varieties over an
 algebraically closed field $k$, and $\I$ is the ideal sheaf of
 $Y$, with $r=\codim Y$ then
 \[
    \omega_Y^{\circ} \cong \omega_X^{\circ} \otimes \wedge^r
    (\I/\I ^2)\check{}.
 \]
 \end{corollary}
 \begin{proof}
 $\omega_X^{\circ} \cong \omega_Y^{\circ}$, $\omega_Y^{\circ}
 \cong \omega_Y$.  Actually, if $X$ is a non-singular projective
 variety over $k$ ($k=\bar k$ for now) and $Y$ is a locally
 complete intersection closed subvariety with ideal sheaf $\I$.
 Then $\omega_Y\cong \omega_X\otimes \wedge^r(\I/\I^2)\check{}$.
 Same proof as before.
 \end{proof}

in the works

%\stepcounter{lecture} \setcounter{lecture}{23} \section*{Lecture
%23}
%
%\begin{theorem}[III.8.8] Let $f:X\to Y$ be a projective morphism
%of noetherian schemes, let $\O(1)$ be a very ample sheaf on $X$
%over $Y$, and let $\F$ be a coherent sheaf on $X$.  Then
% \begin{itemize}
%  \item[(a)] The natural map $f^*f_*\F(n)\to \F(n)$ is surjective
%  for $n\gg 0$
%  \item[(b)] $R^if_*\F$ is coherent for all $i$
%  \item[(c)] $r^if_*\F(n) = 0$ for all $i>0$ and $n\gg 0$.
% \end{itemize}
%\end{theorem}
% \begin{proof}
% The question is local on $Y$ (since $Y$ is quasi-compact), so we
% may assume that $Y$ is affine, say $Y=\spec A$.
%
% (a)
% \end{proof}
