 \stepcounter{lecture}
 \setcounter{lecture}{25}
 \sektion{Lecture 25}

[[The last proposition of the last lecture was missing some
hypothesis -  like noetherian.]]

\begin{proposition} Let $f:X\rightarrow Y$ be a morphism of
noetherian schemes. Assume that $Y$ is integral and regular of dim
1. Then $f$ is flat iff all associated points of $X$ lie over the
generic point of $Y$.

In particular, if $X$ is reduced, then $f$ is flat iff all
irreducible components of $X$ dominate $Y$ \end{proposition}
\begin{proof}
`$\Rightarrow$' last time.\\

`$\Leftarrow$' Let $x \in X$, and let $y = f(x)$. We need to show
that the local rings are flat, i.e. $\O_{x,X}$ is flat over
$\O_{y,Y}$.

\textbf{Claim}. A module $M$ over a PID $R$ is flat iff it is
torsion free. \\

\textbf{Proof} `$\Rightarrow$' Suppose $M$ has torsion, say $t =
0$ with $x \in R$, $m \in M$, both $\neq 0$. Then $M$ is not flat
because multiplication by $x$ is an injection from $R$ to $R$
(since $R$ is a domain and $x$ is non-zero, but it aquires a
kernel after
tensoring wiht $M$.\\

`$\Leftarrow$' We need to show that $\mathfrak{a} \otimes M
\rightarrow M$ is injective for all ideals $\mathfrak{a}$ [[which
are all finitely generated since $A$ is a PID]]. We may assume
$\mathfrak{a} \neq (0)$. Then $\mathfrak{a} = (x)$ for some $x \in
R$, and so $\mathfrak{a}\otimes M \cong M$, and the map is
multiplication by $x$ on $M$. That's injective because $M$ is
torsion free. $\Box$\\


\textbf{Case 1}. If $y$ is the generic point then $\O_{Y,y}$ is a
field so there's nothing to show.\\

\textbf{Case 2}. here, $y$ is a closed point. Then $\O_{Y,y}$ is a
d.v.r. Suppose $\O_{X,x}$ is not flat over $\O_{Y,y}$. Then it has
torsion: there exists some nonzero $t \in \O_{Y,y}$ s.t.
multiplication by $f^{\#}t$ is not injective in $\O_{X,x}$. Then
$f^{\#}t$ is a zero divisor, so it lies in some associated prime
$\mathfrak{p}$ of $\O_{X,x}$. This gives an associated point $x'
\in X$. But $f(x') = y$ since [[as there are only 2 prime ideals
in a PID]] $\mathfrak{p} \cap \O_{Y,y}$ contains $t \neq 0$.
Therefore $\mathfrak{p}\cap\O_{Y,y} \neq (0)$, so $f(x') \neq $
the generic point, contradicting our assumption that all
associated points lie
over the generic point.\\

\textbf{Example}. $\mathbb{Q}$ is flat over $\Z$, therefore it is
torsion
free. But it's not free.\\
\end{proof}

\textbf{Hard Work - Scheme Theoretic Closure, Schematic Denseness,
and Associated Points}.\\

\underline{Recall (II ex.3.11(d)):} Let $f:Z \rightarrow X$ be a
morphism with $Z$ noetherian. Then there exists a unique closed
subscheme $Y$ of $X$ such that $f$ factors through $Y$, and such
that if $f$ factors through any other closed subscheme $Y'$, then
$Y \subset Y'$ schematically ($\mathscr{I} \supset \mathscr{I}'$).
This is the \textbf{closed scheme theoretic image of $f$}.

\begin{definition} Let $i:Z \rightarrow X$ be a subscheme of a
noetherian scheme $X$. Then the \textbf{scheme theoretic closure}
of $Z$ is the closed scheme-theoretic image of $i$.
\end{definition}

\begin{definition} $Z$ is \textbf{scheme-theoretically dense} if
its scheme-theoretic closure is all of $X$ (as a scheme). An open
set is scheme theoretically dense if the corresponding open
subscheme is. \end{definition}

\underline{Example} $X = \spec{k[x,y]/(x^2,xy)}$ (i.e. the
$y$-axis with an embedded point at (0,0)). Let $U = X -
\{(0,0)\}$. Let $A = k[x,y]/(x^2,xy)$. Then $U = \spec{A_y}$. But
$A_y = (k[x,y]/(x))_y \cong k[y]_y$. Since $U$ is reduced, its
scheme theoretic closure is $X_{\text{red}} = \spec{k[y]} $, so
$U$ is not schematically dense. (Note that $U$ is dense
set-theoretically). (You can compute the scheme-theroetic closure
of $\spec{B}\rightarrow\spec{A}$ by finding $\ker(A\rightarrow
B)$).


\begin{proposition} An open subset of a noetherian scheme is
chematicallly dense iff it contains all of the associated points.
\end{proposition}

\begin{proof} This is a local question, so we may assume $X =
\spec{A}$ is affine. Let $U$ be an open subscheme [[equivalently
an open subset]]. Lets let $\mathfrak{q}_1 \cap \cdots \cap
\mathfrak{q}_n$ be a minimal primary decomposition of $(0)$ in
$A$. [[This is where we use the fact that $A$ is noetherian.]] Let
$\mathfrak{p}_i = \sqrt{\mathfrak{q}_i}$, so
$\mathfrak{p}_1,\cdots,\mathfrak{p}_n$
are the associated primes of $A$.\\

`$\Leftarrow$'. Say $U$ contains
$\mathfrak{p}_1,\cdots,\mathfrak{p}_n$, and let $\mathfrak{a}$ be
the ideal associated to the schematic closure $\overline{U}$ of
$U$. Since $\mathfrak{p}_i \in U$, and $U \subset \overline{U}$,
we have $(A/\mathfrak{a})_{\mathfrak{p}_i/\mathfrak{a}} \cong
A_{\mathfrak{p}_i}$. Therefore, for any $f \in \mathfrak{a}$, $f =
0$ in $A _{\mathfrak{p}_i}$, so $\text{Ann}(f)$ meets
$A/\mathfrak{p}_i$. By prime avoidance [an actual term in
Eisenbud's book, or just by general messing around], there is some
$x \in \text{Ann}(f)$ such that $x \not \in \mathfrak{p}_1 \cap
\cdots \cap \mathfrak{p}_n$. Therefore $x$ is not a zero divisor
(and $x \neq 0$), so $f = 0$. Therefore $\mathfrak{a} = (0)$, so
$\overline{U} = X$ and therefore $U$ is schematically dense.\\

`$\Rightarrow$' Suppose its false. Then $U$ is schematically dense
but without loss of generality $\mathfrak{p}_n \not \in U$. We may
assume $\{i : \mathfrak{p}_i \not \supset \mathfrak{p}_n\}$ =
$\{1,\cdots,r\}$; $r < n$. Let $\mathfrak{a} = \mathfrak{q}_1 \cap
\cdots \cap \mathfrak{q}_r$. By minimality, $\mathfrak{a} \neq
(0)$. So it will suffice to show that $U \subset
\spec{(A/\mathfrak{a})}$.

Let $\p \in U$. Since $\p_n \not \in U$, $\p_n \not \rightarrow
\p$ [make this arrow squiggly], so $\p_n \not \subset \p$.
Therefore $\p \not \supset \p_i$, for all $i > r$. Let $S = A /
\p$; then $S$ meets $\p_i$ for all $i > r$, so $S^{-1}\q_i = (1)$
for all $i > r$. Therefore $(0) = \bigcap_{i=1}^nS^{-1} \q_i =
\bigcap _{i = 1}^r S^{-1}\q_i = S^{-1}\mathfrak{a}$, so
$S^{-1}\mathfrak{a} = (0)$. Therefore $\p \in
\spec{(A/\mathfrak{a})}$. ($f \in \mathfrak{a} \Rightarrow
\text{Ann}(f)$ meets $S \Rightarrow f \in \p$ primeness). also
$(A/\mathfrak{a})_{\p/\mathfrak{a}} \cong A_{\p}$, since LHS $ =
A_{\p}/\mathfrak{a}/{\p} = A_{\p}/(0)$. So $U \subset
\spec{(A/\mathfrak{a})}$ as schemes [[here he drew a big frowny
face with x'ed out eyes]].\end{proof}

\begin{proposition}

Lelt $Y$ be an integral, regular, noetherian scheme of dimension
1, let $P \in Y$ be a closed point, and let $X \subset
\mathbb{P}^n_{Y/P}$ be a closed subscheme which is flat over
$y/P$. Then there exists a unique closed sobscheme $\overline{X}$
of $\mathbb{P}^n_{Y}$ such that $\overline{X} \cap
\mathbb{P}^n_{Y/P} = X$ and $\overline{X}$ is flat over $Y$.
\end{proposition}

\begin{proof} \textbf{Existence} Let $\overline{X}$ be the
schematic closure of $X$ in $\mathbb{P}^n_{Y}$. We didn't add any
associated points, so
its still flat.\\

\textbf{Uniqueness}. Suppose $X'$ also saatisfies the condition.
We have to have $X' \subset_{\text{subschemes}}\overline{X}$ (by
definition of schematic closure). since $\overline{X}$ is not
schematically dense in $X'$, $X'$ contains some associated point
not in $\overline{X}$. Then that point lies over $P$ (because
where else could it lie), contradicting flatness. \end{proof}

[[We didn't really need the fact that we were in
$\mathbb{P}^n_{Y}$. We just wanted to be concrete, and in
applications we only use $\mathbb{P}^n_{Y}$. In practice you could
use any noetherian scheme in place of $\mathbb{P}^n_{Y}$.]]

The last application has to do with bases of arbitrary dimension.

\begin{theorem} Let $T$ be an integral noetherian scheme, and let
$X \subset \mathbb{P}^n_{T}$ be a closed subscheme. Then for each
point $t \in T$ (closed or not), lets let $P_t \in \mathbb{Q}[z]$
be the Hilbert polynomial of the fiber $X_t = X \times_T k(t)$
(which is a closed subscheme of $\mathbb{P}^n_{k(t)}$ - calculate
the Hilbert polynomial by calculating dimensions of things over
$k(t)$).

Then $X$ is flat over $T$ iff $P_t$ is independent of $t$.
\end{theorem}

\begin{proof} Recall the definition of Hilbert Polynomial: $P_t$
is characterized by $P_t(m) = \text{dim}_{k(t)} H^o(X_t,\O(m))$
for all $m \gg 0$. Also, $k(t)$ is not necessarily closed, but by
a flat base change, we can pass to the algebraic closure.

We will prove this theorem by first generalizing it: More
generally let $\F$ be a coherent sheaf on $X$. Then we can write
$\F_t = \F|_{X_t} = i^*\F$, where $i:X_t \rightarrow X$ is the
inclusion map. So we have a Hilbert polynomial of $\F_t$" \[
P_t(m) = h^0(X_t,\F_t(m)), m \gg 0.\] So we may assume $X =
\mathbb{P}^n_{T}$ (let $\F = \O_{\text{old } X}$ [[literally waved
hands and said `usual trick']]).

Also, we may assume $T = \spec{A}$, with $A$ a local noetherian
ring.\\

So the situation is $T = \spec{A}$ as above, $X =
\mathbb{P}^n_{A}$,
and $\F$ is coherent on $X$. [[Now we prove some claims]].\\

\textbf{Claim 1}. $\F$ is flat over $T$ iff $H^0(X,\F(m))$ if a
free $A$ module (of finite rank - this part always holds) for $m
\gg
0$.\\

\textbf{Proof}. `$\Rightarrow$'. Let $\mathscr{U}$ be the standard
open affine cover of $\mathbb{P}^n_{A}$ [[offhand I can't think of
why it has to be the standard one, but why not?]]. Then
$H^i(\U,\F(m)) = H^i(X,\F(m)) = 0$ for all $m \gg 0$ So the
sequence (imagine the kernels and cokernels) \[ 0 \rightarrow
H^0(X, \F(m)) \rightarrow C^0(\U,\F(m)) \rightarrow C^1(\U,\F(m))
\rightarrow \cdots \rightarrow C^{n-1}(\U,\F(m)) \rightarrow
C^{n}(\U,\F(m)) \rightarrow 0 \] (no verb either).

All the $C^{i}(\U,\F(m))$ are flat over $A$. [[Twisting by $m$
doesn't affect flatness - you can check it locally, and it is
unaffected locally. So then you have a short exact sequence on the
right where everything is flat, and working your way back all the
kernels and cokernels are flat too.]] Splitting this into short
exact sequences, you get that $\mathscr{K}_i$ is flat over $A$ for
all $i$, so $H^0(X,\F(m))$ is flat over $A$ by (III 9.1A(e)).
Also, $H^0(X, \F(m))$ is finitely generated, so by (III 9.1A(f)),
its free (of finite rank).

\end{proof}
