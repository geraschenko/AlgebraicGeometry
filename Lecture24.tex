 \stepcounter{lecture}
 \setcounter{lecture}{24}
 \sektion{Lecture 24}


Addendum: why part (b) of flatness lemma is true - $ (M$ flat over
$A \Rightarrow M \otimes_AB$ flat over $B$)$\Leftarrow
N\otimes_B(M\otimes_AB)) \cong N\otimes_AM$.



Cohomology commutes with flat base change:

\begin{proposition} Let

\[
\xymatrix{X' \ar[d]_g\ar[r]^v&X\ar[d]^f\\
Y'\ar[r]^u&Y } \] Be a cartesian square of noetherian schemes,
with $f$ separated and of finite type, and $u$ flat. Also, let
$\F$ be a quasi-coherent sheaf on $X$. Then there is a natural
isomorphism (as sheaves on $Y'$) \[
u^*R^if_*(\F)\cong\text{R}^ig_*(v^*\F) \] \end{proposition}
 \begin{proof} By locality and naturalness, we
may assume $Y$ and $Y'$ are affine, say $Y = \spec{A}$ and $Y'
=\spec{A'}$. Then \[ R^if_*(\F) \cong
\text{H}^i(X,\F)^{\widetilde{}_Y} \] \[u^*R^if_*(\F) \cong
(\text{H}^i(X,\F ))^{\widetilde{}_{Y'}}\otimes_A A' \cong
(H^i(X,\F)\otimes_A A')^{\widetilde{}_{Y'}} \] and \[
\text{R}^ig_*(v^*\F) = \text{H}^i(X',v^*\F)^{\widetilde{}_{Y'}} \]
So we need a natural isomorphism \[ H^i(X,\F)\otimes_AA' \cong
H^i(X',v^*\F) \] Use Cech cohomology: let $\U$ be an open affine
cover of $X$. Say $\U = (U_i)_{i \in I}$ with $U_i = \spec{B_i}$.
Then let $\U' = v^{-1}\U = \U = (v^{-1}(U_i))_{i \in I} =
(\spec{B_i\otimes_A A'})_{i \in I}$. This is an open affine cover
of $X'$.

Then we have \[ H^i(X, \F)\otimes_AA' \cong
h^i(\C^{\cdot}(\U,\F))\otimes_AA' \cong_{flatness}
h^i(\C^{\cdot}(\U,\F)\otimes_AA') = h^i(\C^{\cdot}(\U,v^*F)) =
H^i(X',v^*\F) \] \end{proof}

\textbf{Flat Families} We want to exclude things like blow ups,
where the dimension of fibres is not constant. For flat morphisms,
this cannot happen. \begin{proposition} Let $f:X\rightarrow Y$ be
a flat morphism of schemes of finite type over a field $k$. Let $x
\in X$, let $y = f(x)$, and let $X_y$ be the fibre of $f$ over
$y$. Then $\text{dim}_x X_y$ = $\text{dim}_x X$ - $\text{dim}_y
Y$, where $\text{dim}_x X = dim \O_{X,x}$ \end{proposition}
\begin{proof} \textbf{Step 1}: Reduce to the situation where $Y$
is affine and there is a unique closed point, and $y$ is that
point. \emph{(say a local ring)}. Let $Y' = \spec{\O_{Y,y}}$, and
let $X' = X \times_Y Y'$. Then $X_y$ is unchanged, and so are all
the local rings involved. (Now $X$ and $Y$ are covered by open
affines which are
localizations of $k$-algebras of finite type.\\

\textbf{Step 2}: Reduce to $Y$ reduced. Base change to
$Y_{\text{red}}$. All of the local rings are replaced by quotients
by ideals contained in the nilradical. So essentially, nothing is
changed.\\

\textbf{Step 3}: The rest. Prove by induction on $\text{dim } Y$.
Note that $\text{dim } Y$ = $\text{dim}_y Y$.

\textbf{Base case}: If $\text{dim } Y$ = 0, then $Y$ has a unique
minimal prime, which is also maximal.  Thus $Y$ is spec of an
artin ring (it has a unique maximal ideal, which is also minimal),
and since $Y$ is reduced \emph{(this excludes products of
fields)}, $Y = \spec{E}$ for some field $E$. But then $X_y$ = $X$,
so we're done.

\textbf{Inductive Step}: If $\text{dim } Y > 0$, then the maximal
ideal of $\O_{Y,y}$ contains a non-zero element $t$, which is not
a zero divisor. (Later). Thus, \[ 0 \rightarrow \O_{Y,y}
\rightarrow^t \O_{Y,y} \] is exact. By flatness(?), \[ 0
\rightarrow \O_{X,x} \rightarrow^{f^{\#}t} \O_{X,x} \] is exact.
Let $Y' = \spec{\O_{Y,y}/(t)}$. By the Hauptidealsatz and the fact
that $\O_{Y,y}$ is catenary (?) \[ \text{dim }_y Y' = \text{dim
}_y Y -  1 \] Let $X' = X \times_Y Y'$. Then
$\O_{X',x}=\O_{X,x}(f^{\#}t)$. By flatness, $f^{\#}t$ is not a
zero divisor (or 0), so $\text{dim }_x X' = \text{dim }_x X - 1$.

Finally, $X'_y = X_y$, since $t \in \mathfrak{m}_y$. So we're done
by induction. \end{proof}

\begin{corollary} Let $f:X\rightarrow Y$ be as in the statement of
the proposition. Assume also that $Y$ is irreducible. The the
conditions \begin{itemize} \item[(i)] $X$ is equidimensional of
dimension $\text{dim} Y + n$ \item[(ii)] $X_y$ is equidimensional
of dimension  $n$ for all $y \in Y$ (closed or not). \end{itemize}
are equivalent. \end{corollary} \begin{proof} $(i) \rightarrow
(ii)$. Pick $y \in Y$, let $Z$ be an irreducible component of
$X_y$, and let $x \in Z$ be a closed point not lying in any other
irreducible component. By the proposition, \[\text{dim}_x Z =
\text{dim} Z = \text{dim}_x X_y = \text{dim}_x X - \text{dim}_y
Y\] Also, \[ \text{dim}_x X =  \text{dim} X - \text{dim}
\overline{\{x\}} \] and likewise \[ \text{dim}_y Y =  \text{dim} Y
- \text{dim} \overline{\{y\}}. \] But now, since $x$ is a closed
point of $X_y$, $k(x)$ is a finite (and therefore algebraic)
extension of $k(y)$. Thus, \[\text{dim} \overline{\{x\}} =
\text{tr. deg}(k(x)/k) = \text{tr. deg}(k(y)/k) = \text{dim}
\overline{\{y\}}\] Therefore the right hand side is
$\text{dim}X - \text{dim} Y$, which by (i) is $n = \text{dim} Z$.\\

$(ii) \rightarrow (i)$: Let $Z$ be an irreducible component of
$X$, let $x \in Z$ be a closed point not lying in any other
irreducible component, and let $y = f(x)$. Then $y$ is a closed
point of $Y$, because $k \subset k(y) \subset k(x)$ (which are
algebraic extensions). By the proposition (and assumption (ii)),
\[ n = \text{dim}_x X_y = \text{dim}_x X - \text{dim}_y Y =
\text{dim}_x Z - \text{dim} Y \] \end{proof}

\textbf{Associated Primes}: Assume Noetherian throughout.
\begin{definition} Let $A$ be a ring. An associated prime of $A$
is a prime ideal equal to the annihilator of some element of $A$.
\emph{Not every element of $A$ has its annihilator equal to some
prime. You can even do this more generally for modules, just take
associated primes to be the annihilators of elements of $M$.}
\end{definition} \begin{theorem}(Eisennbud Thm 3.1): Let $A$ be a
noetherian ring. Then \begin{itemize} \item[(a)] it has only
finitely many associated primes \item[(b)] the union of the
associated primes is the set of zero divisors of $A (\cup \{0\})$
\end{itemize} \end{theorem} \begin{remark} All minimal primes of
$A$ are also associated primes. And these associated primes
localize nicely: If $S$ is a multiplicative subset of $A$, then a
prime of $S^{-1}A$ is an associated prime iff the corrsponding
prime of $A$ is an associated prime. In other words: \[
\text{Ass}(S^{-1}A) ``=" (\text{Ass} A)\cup \spec{S^{-1}A} \]
\end{remark} \begin{definition} Let $X$ be a (locally noetherian)
scheme. Then an associated point of $X$ is a point $x \in X$ such
that the maximal ideal of $\O_{X,x}$ is an associated prime of
$\O_{X,x}$.

If $A$ is noetherian, then the set of associated points of
$\spec{A}$ corresponds to the set of associated primes of $A$ (by
this comment about localization). \end{definition}

\emph{Can almost fill in the `later' above, but need one more
thing.
Okay, several more things.}\\

\begin{proposition}\textbf{Primary Decomposition}: If $A$ is a
noetherian rind, then we can write $(0) = \mathfrak{q_1} \cap
\cdots \cap \mathfrak{q_r}$, where the $\mathfrak{q_i}$ are
primary ideals (\emph{this is more natural using Eisenbud's
definition, which we don't know offhand}). A primary ideal
satisfies (for $xy \in \mathfrak{q_i}, x \not \in \mathfrak{q_i}
\Rightarrow y^n \in \mathfrak{q_i}$ for some $n$).
\end{proposition} \begin{remark} Note that if $\mathfrak{q}$ is
primary, then $\sqrt{\mathfrak{q}}$ is prime. If $\mathfrak{q_1}
\cap \cdots \cap \mathfrak{q_r} = (0)$ is a minimal primary
decomposition ($r$ minimal), then $\sqrt{\mathfrak{q_i}}$ are
distinct and are exactly the set of associated primes of $A$.
\end{remark}

Primary decompositions localize well: \begin{definition} An
\textbf{embedded prime} in a noetherian ring $A$ is a non-minimal
associated prime. An \textbf{embedded point} in a noetherian
scheme is an associated point that is not the generic point of an
irrreducible component. \end{definition}

\begin{proposition} A reduced noetherian scheme $X$ has no
embedded points. \end{proposition} \begin{proof} This is a local
question, so we may assume that $X = \spec{A}$ is affine (where by
assumption $A$ is reduced).

Then $0 = \text{nil}(A) = $ intersection of the minimal primes of
$A$. \emph{In fact it is the intersection of all primes of $A$,
but if a prime is not minimal its not needed in the intersection.}
This gives a primary decomposition of $0$ in $A$. So all
associated primes are minimal. \end{proof}

\emph{The union of the minimal primes in not the maximal ideal
because its of dimension greater than one, and by prime something,
if you have the maximal ideal, and you have some primes that are
strictly contained in that, then their union is not the whole
maximal ideal because, its an exercise. But its basically prime
something... ``later"...}\\

Actually, a little more is true:

\begin{remark} (Eisenbud Ex. 11.10) A noetherian ring is reduced
iff it has no embedded primes and its localization at each minimal
prime is a field.

Intuition: Non-zero nilpotents in a ring $A$ correspond to 'fuzz'
in $\spec{A}$. 'Fuzz' on a dense open subset of an irreducible
component occurs iff the localization at the corresponding minimal
prime is not a field. Embedded points correspond to fuzz nt spread
over an irreducible component. Ex.
$\spec{k[x,y]/(xy,y^2)}$\end{remark}

\emph{This embedded stuff might be useful for the homework...}\\

\emph{You'd like to represent a number theory problem as a problem
over $\spec{\Z}$ or spec of a ring of integers in a number field.}\\

\begin{proposition} Let $f:X\rightarrow Y$ be a morphism of
schemes. Assume that $Y$ is integral and regular of dim 1. Then
$f$ is flat iff all associated points of $X$ lie over the generic
point of $Y$. \end{proposition} \begin{proof} ``$\Rightarrow$" Let
$x \in X$ be a point lying over a closed point $y \in Y$. We want
to show that $x$ is not an associated point. Then $\O_{Y,x}$ is a
discrete valuation ring, so its maximal ideal contains a nonzero
element $t$, which is a nonzerodivisor since $Y$ is integral. By
flatness, $f^{\#}t$ is not a zero divisor in
$\O_{X,x}$, so $x$ is not an associated point of $X$.\\

\emph{Because if it was that would mean that the maximal ideal
would be an associated prime. Multiplication by $t$ is injective
in $\O_{Y,y}$, so it is in $\O_{X,x}$ too}.

\emph{Remark: We didn't need $Y$ to be regular, or even dimension
1. We just needed $Y$ to be integral of dimension greater than
zero.} \end{proof}
