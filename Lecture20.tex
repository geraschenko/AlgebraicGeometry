 \stepcounter{lecture}
 \setcounter{lecture}{20}
 \sektion{Lecture 20}

Last time we showed that if $Y$ is a locally complete intersection
in a regular (or Cohen-Macauley) scheme $X$, then $Y$ is
Cohen-Macaulay.

To find an example of a non-Cohen-Macaulay scheme, we look to
contradict facts we know about Cohen-Macaulay rings, like

\begin{theorem}[Eisenbud Cor 18.10] In a Cohen-Macaulay ring, all
associated primes are minimal. \end{theorem}

If $k$ is a field and $A=k[x,y]/(x^2,xy)$, then is has an embedded
point, so $\spec A$ is not Cohen-Macaulay.  To see this directly,
let $\m=(x,y)\subseteq A$.  Then $A_{\m}$ is not Cohen-Macaulay
since $\dim A_{\m}=1$, but $\depth A_{\m}=0$ since all elements of
the maximal ideal are zero divisors.

By II.8.21Ab, if $A$ is a local Cohen-Macaulay ring, then any
localization of $A$ at a prime ideal is also Cohen-Macaulay. Thus,
a scheme $X$ is Cohen-Macaulay if and only if all of its local
rings \emph{at closed points} are Cohen-Macaulay.

 \begin{remark}
 Let $Y$ be a locally complete intersection in a Cohen-Macaulay
 scheme $X$, and let $\I$ be its ideal sheaf.  Assume that $Y$ is
 equicodimensional (all irreducible components are of the same
 codimension, $r$).  Then $\I/\I^2$ is a locally free sheaf on $Y$
 of rank $r$.
 \begin{proof}
 We may assume $Y$ is globally a complete intersection in $X$, cut
 out by $x_1,\dots, x_r$, and that $X$ is affine, say $X=\spec A$.
  Then $\I=\tilde I$ for $I=(x_1,\dots, x_r)$.  We have that
  \[
    (A/I)^r\to I/I^2\quad,\quad (a_1,\dots,a_r)\mapsto \sum a_ix_i
  \]
  is well defined and onto.  We need to show that it is injective.

  Suppose not.  Then there is a prime $P\in A$ such that
  $(\text{kernel})_P \not =0$.  Clearly $I\subseteq P$.  Replace
  $A$ with $A_P$.  Then $A$ is Cohen-Macaulay and local, and
  $x_1,\dots,x_r$ is a regular sequence.  Let $(b_1,\dots, b_r)$
  be a nonzero element of the kernel in question.  We may assume
  $b_r\not\in I$.  Then $\sum b_ix_i \in I^2$, so $\sum b_ix_i =
  \sum c_ix_i$ with $c_i\in I$ for all $i$.  Then we have that
  $\sum (b_i-c_i)x_i = 0$.  Since $b_r\not\in I=(x_1,\dots, x_r)$,
  it is non-zero in $A/(x_1,\dots, x_{r-1})$.  Thus, we have that
  $x_r$ is a zero divisor (or 0) in $A/(x_1,\dots, x_{r-1})$.
  Contradiction.
  \end{proof}
  \end{remark}

  \begin{theorem}
  Let $X$ be a non-empty projective scheme of dimension $n$ over
  an algebraically closed field $k$, and let $\omega_X^{\circ}$ be
  a dualizing sheaf for $X$ (over $k$).  Then TFAE:
  \begin{itemize}
   \item[(i)] $X$ is equidimensional and Cohen-Macaulay
   \item[(ii)] for all locally free sheaves $\F$ on $X$,
   $H^{n-i}(X,\F(-q))=0$ for all $i>0$ and $q\gg 0$ (depending on
   $\F$).
   \item[(iii)] the maps $\theta^i:\ext^i(\F,\omega_X^{\circ})\to
   H^{n-i}(X,\F)'$ are isomorphisms for all $i$ and for all
   coherent sheaves $\F$.
  \end{itemize}
 \end{theorem}
 \begin{proof}
 in the works
 \end{proof}

 \begin{corollary}
 Let $X$ be an equidimensional Cohen-Macaulay projective scheme of
 dimension $n$ over an algebraically closed field $k$ (e.g. a
 non-singular variety of dimension $n$).  Let $\F$ be a locally
 free sheaf on $X$ and let $\omega_X^{\circ}$ be the dualizing
 sheaf.  Then there is a natural isomorphism for all $i$:
 \[
    H^i(X,\F) \cong H^{n-i}(X,\check \F \otimes
    \omega_X^{\circ})'.
 \]
 \end{corollary}
 \begin{proof}
 \[
    H^{n-i}(X,\F)' \cong \ext_X^i(\F,\omega_X^{\circ}) \cong
    \ext_X^i(\O_X,\check \F\otimes \omega_X^{\circ}) \cong
    H^i(X,\check \F\otimes \omega_X^{\circ}).
 \]
 \end{proof}

 \begin{remark}
 In proving Lemmas III.7.3 and III.7.4, we only used duality for
 $P$, so the proofs actually give
 \begin{theorem} Let $P$ be an equidimensional Cohen-Macaulay
 projective scheme of dimension $N$ over an algebraically closed
 field $k$, and let $X$ be a non-empty closed subscheme of $P$ of
 dimension $n$.  Then
 \[
    \omega_X^{\circ} = \Ext^{N-n}_P(\O_X,\omega_P^{\circ})
 \]
 \end{theorem}
 \end{remark}

 \begin{definition} Let $A$ be a ring, $f_1,\dots, f_r\in A$.
 Then the \emph{Kozul complex} of $A$ is the complex $K_{\cdot} =
 K_{\cdot}(f_1,\dots,f_r)$ defined by $K=$ free $A$-module of rank
 $r$, and $K_i:=\wedge^i K$, and $d:K_p\to K_{p-1}$ is given by
 \[
    e_{i_1}\wedge \cdots \wedge e_{i_p} \mapsto \sum_{j=1}^p
    (-1)^{j-1} f_{i_j} e_{i_1}\wedge\cdots \wedge \hat e_{i_j}
    \wedge \cdots \wedge e_{i_p}
 \]
 for all $i_1<\cdots < i_p$, where $\{e_1,\dots, e_r\}$ is the
 standard basis for $A^r$.  Note that $d^2=0$.  If $M$ is an
 $A$-module, then se define
 \[
    K_{\cdot}(f_1,\dots, f_r, M) = K_{\cdot}(f_1,\dots,
    f_r)\otimes_A M.
 \]
 \end{definition}
