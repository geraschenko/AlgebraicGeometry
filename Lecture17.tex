 \stepcounter{lecture}
 \setcounter{lecture}{17}
 \sektion{Lecture 17}

 \marginpar{\S II.8: Differentials}
 \begin{definition}[An alternative definition] Let $B$ be an
 $A$-algebra ($A\xrightarrow{f} B$), then $\Omega_{B/A}$ is the
 $B$-module described by generators $db$ for all $b\in B$ and
 relations
 \begin{itemize}
  \item[] $da = 0$
  \item[] $d(b_1+b_2)=db_1+db_2$
  \item[] $d(b_1b_2)=b_1db_2+b_2db_1$
 \end{itemize}
 for all $a\in A$ and $b_i\in B$.  Equivalently, $d$ is $A$-linear
 and satisfies the Leibniz condition\footnote{$\Rightarrow$ trivial. $\Leftarrow:\
 da = ad1 = ad(1\cdot 1) = 2ad1$, so $da=0$.}.
 \end{definition}

 \begin{proposition}[II.8.3A: First Exact Sequence]\label{P:II.8.3A} If $A\to B \to C$
 are ring homomorphisms, then
 \[
 \Omega_{B/A}\otimes_B C   \longrightarrow \Omega_{C/A}   \longrightarrow \Omega_{C/B} \to
 0\]
 \[ db\otimes c  \mapsto cdb,\ dc  \mapsto dc \]
 is an exact sequence of $C$-modules.
 \end{proposition}
 \begin{proof}
 Surjectivity is obvious (you are simply imposing more relations).
 The kernel of the second map is generated by the ``new''
 relations, $\{ db=0| b\in B\}$, which is clearly the image of the
 first term.
 \end{proof}

 \underline{Example:} If $B$ is the polynomial ring $A[x_i]_{i\in
 I}$, then $\Omega_{B/A}$ is the free $B$-module generated by
 $dx_i$ for all $i\in I$\footnote{Anton and Dave say $I$ has to be finite.}.
 \begin{proof}
 By the universal property of $\Omega_{B/A}$, we get
 \[\xymatrix{
  B \ar[r]^d \ar[dr]_{\partial/\partial x_i} & \Omega_{B/A}\ar@{-->}[d]^{\exists\ dx_i\mapsto 1}\\
  & B
 }\]
 This gives a map $\Omega_{B/A}\to \prod_I B$.  The kernel of this
 map is zero (anything mapping to zero must be $d($something w/ no
 $x$'s$) = 0$.  And for any element of the product, it is easy to
 construct an inverse image\footnote{Not if $I$ is infinite.}.
 \end{proof}

 \begin{proposition}[Second Exact Sequence]
 If $A\to B \to C$ is a sequence of ring homomorphisms with $B\to
 C$ surjective, and with kernel $I$, then
 \[
 I/I^2 \xrightarrow{b\mapsto db\otimes 1} \underbrace{\Omega_{B/A} \otimes_B
 C}_{\Omega_{B/A}/I\Omega_{B/A}\footnote{in the works}}
 \xrightarrow{\text{same}} \Omega_{C/A}\to 0
 \]
 is an exact sequence of $C$-modules.
 \end{proposition}
 \begin{proof}
 Since $B\to C$ is surjective, $\Omega_{B/C}=0$, so by the first
 exact sequence, the second map is surjective.  The first map is
 well-defined since
 \[
    d(b_1b_2)\otimes 1 = db_1\otimes b_2 + db_2\otimes b_1 = 0.
 \]
 Now we show exactness in the middle.  It is clear that the
 composition of the two maps is zero.  Conversely, if $\sum
 c_idb_i=0$, then it is a $C$-linear

 in the works
 \end{proof}

 \begin{corollary}
 If $C= A[x_i]_{i\in I}/\a$, then $\Omega_{C/A}$ is the $C$-module
 described by generators $dx_i$ and relations $df=0$ for all $f\in
 \a$ (or for all $f$ in a generating set for $\a$).
 \end{corollary}
 \begin{corollary}[of the Corollary]
 If $A\to B$ and $A\to A'$ are $A$-algebras and $B'=B\otimes_A
 A'$, then
 \[
    \Omega_{B'/A'} \cong \Omega_{B/A}\otimes_B B' \quad ( = \Omega_{B/A}\otimes_A
    A')
 \]
 \end{corollary}
 \begin{proof}
 in the works (for Anton).
 \end{proof}

 \begin{corollary}
  If $S$ is a multiplicative subset of $B$, then
  \[\Omega_{S^{-1}B/A} \cong S^{-1} \Omega_{B/A}.\]
 \end{corollary}
 \begin{proof}
 $\Omega_{S^{-1}B/A}$ has the generators $d(s^-1 b)$.  Then
 $db = d(s\cdot s^{-1} \cdot b) = s^{-1}b\cdot ds + s\cdot d(s^{-1}b)
 $, so
 \[
    d(s^{-1}b) = s^{-1}db - s^{-2}b ds
 \]
 so you don't really get any new relations (exercise).
 \end{proof}

 \marginpar{Sheaves of Differentials} Now we pass to sheaves over
 schemes.  If $X$ is a scheme over $\spec A$, then there is a
 unique sheaf $\Omega_{X/A}$
